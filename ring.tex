\documentclass[autodetect-engine,dvi=dvipdfmx,ja=standard,japaram={units}]{bxjsarticle} %%%every thing OK!  
%\setpagelayout{showframe}	%%%margin可視化
\usepackage{fancyhdr}
\usepackage{graphicx,float}
\usepackage{tikz,tikz-cd}
\usepackage[hyperfigures]{hyperref}
\usepackage[links,thmnumthrice,dottedtoc,sectionmark]{askw}
\input{adjustmentforme}
\usepackage{mathrsfs}
\usepackage{makeidx}
\usepackage[longnamesfirst]{natbib}
\makeindex
\let\bar\overline

\begin{document}
	\nocite{*}
	\mytitle{可換環論}
	\newpage
	\part*{はじめに}
	代数幾何学を学ぶには,多様な前提知識が必要になる.特に可換環論,スキーム論,複素多様体(Riemann面)論が大切である.そこで,筆者のこれらの分野の学習ノートという形も兼ねて,各分野についてのPDFを作成することにした.これはそのうち可換環論についての分冊で,松村英之「復刊~可換環論」\cite{matsu},~堀田良之「可換環と体」\cite{hota2}などの内容をまとめる.とはいえ,最低限の基礎的なこと,例えばテンソル積や中山の補題などについては主に結果のみを第1章にまとめておくことにする.ただし,ホモロジー代数において活躍する射影,単射加群については証明付きでまとめてある.
	
	\textbf{このノートでは,環といえば単位元を持つ可換環のこととする.}
	
	\hspace*{\fill}最終更新日~\dots~\today
	
	\part*{記号}
	
	\symlist{$\N$}{自然数全体の集合(本書では0を含む).}
	\symlist{$\Z,\Q,\R,\Co$}{それぞれ整数,有理数,実数,複素数全体の集合.}
	\symlist{$\Z_+$}{正の整数全体の集合.$\Q,\R$などに対しても同様.}
	\symlist{$\#A$}{集合$A$に対し,$A$の元の個数またはその濃度.}
	\symlist{$\subset$}{本書では等号の可能性を除外しない.真の包含関係は$\subsetneq$を用いる.}
	
	また,写像$f:A\to B$が全射のとき,~$f:A\twoheadrightarrow B$とかくことがある.また,単射のときは$f:A\hookrightarrow B$とすることがある.
	\newpage
	
	\tableofcontents
	
	\part[Preliminaries]{準備}
この章では基礎的なことを集める.

\section{加群}

線形空間の拡張として\textbf{加群}というものを考えると,環$A$の構造だけを観るのではなく加群も合わせて考えることで表現論的な考察が可能になる.

\begin{defi}[加群]\index{かぐん@加群}
	$A$を環とし, $M$をAbel群とする. $A$の作用$A\times M\to M;(a,x)\mapsto ax$が存在して;
	\begin{defiterm}{M}
		\item $1x=x$
		\item $a(bx)=(ab)x$
		\item $(a+b)x=ax+bx$
		\item $a(x+y)=ax+ay$
	\end{defiterm}
	をみたすとき, $M$(と作用の組)を$A$\textbf{加群}(module)という.
\end{defi}

環$R$が非可換の時,作用が左(右)作用のとき左(右)加群という. $A$が可換のときは単に$A$加群という.右加群では(M2)の代わりに
\[\textrm{(M2)}'\quad a(bx)=b(ax)\]
を要請する必要がある(最初に注意しておいたとおり,以後すべて$A$は可換環として進める.). また, $A$が体のときは線形空間に他ならない. $A$の作用のことを\textbf{スカラー}(scalar)ということもある.

以後線形代数の模倣としていくつかの性質と定義を述べる.例えば任意の$a\in A$と$x\in M$に対して;
\[a0=0,0x=0,(-a)x=-ax\]
が成り立つ.

\begin{defi}[準同型]
	$M,N$を$A$加群とする. $\varphi:M\to N$が,任意の$a,b\in A$と$x,y\in M$に対し;
	\[\varphi(ax+by)=a\varphi(x)+b\varphi(y)\]
	を満たす時, $\varphi$を$A$準同型という.
\end{defi}

もちろん, $\varphi$が全単射の時$A$同型という.

\begin{defi}[部分加群]
	$A$加群$M$の部分集合$N$がAbel群として$M$の部分群であり,任意の$a\in A$と$x\in N$に対して$ax\in N, $すなわち$A$の作用で閉じているとき$N$を部分加群であるという.
\end{defi}

$A$自身を,環の積を作用として$A$加群とみなすとき, $A$の部分加群とはまさに$A$のイデアルにほかならない.このとき, $A$準同型は環の準同型とは異なることに注意しよう(単位元の行き先を定めることと$A$準同型を定めることは同値).

\begin{defi}[剰余加群]\index{じょうよかぐん@剰余加群}
	$M$を$A$加群とし, $N$をその部分加群とする.Abel群としての剰余群$M/N$に$A$作用を次のようにして定義する;
	\[a(x+y)=ax+N\]
	これを剰余加群という.
\end{defi}

\begin{thm}[準同型定理]
	$\varphi:M\to N$を$A$加群の$A$準同型とすると, $\ker\varphi,\im\varphi$はそれぞれ$M,N$の部分加群で,同型;
	\[M/\ker\varphi\cong\im\varphi\]
	が成立する.
\end{thm}
\begin{proof}
	群の準同型定理により引き起こされる同型写像に対して, $A$の作用のそれぞれの定義から明らか.
\end{proof}

\begin{defi}[加群の直積,直和]\label{defi:直和,直積の存在}\index{ちょくせき@直積}\index{ちょくわ@直和}
	$A$加群の族$\{M_\lambda\}_{\lambda\in\Lambda}$に対して,直積$\prod M_\lambda$に$A$の作用を;
	\[a(x_\lambda)_{\lambda\in\Lambda}=(ax_\lambda)_{\lambda\in\Lambda}\]
	と定めると$\prod M_\lambda$は$A$加群になる.同様の作用によって;
	\[\bigoplus_{\lambda\in\Lambda}M_{\lambda}=\mkset{(x_\lambda)_{\lambda\in\Lambda}\in\prod M_\lambda}{\text{有限個の}\lambda\text{を除いて}x_\lambda=0\text{である}.}\]
	も$A$加群になる.特に$\Lambda$が有限のとき$\prod M_\lambda$と一致する.これらをそれぞれ$\{M_\lambda\}$の\textbf{直積}(direct product), \textbf{直和}(direct sum)という.
\end{defi}

線形空間の\quo{基底}に対応して,加群の基底について触れる.以下, $M$を$A$加群とし, $(u_\lambda)_{\lambda\in\Lambda}$(しばしば略して$(u_\lambda)$とかく)を$M$の元の族とする.また, $\Lambda$が無限集合である場合, $\Lambda$を走る和$\sum_{\lambda\in\Lambda} u_\lambda$は次の条件;
\[\text{有限個の}\lambda\text{を除いて}u_\lambda\text{が0である.}\tag{$\ast$}\]
を満たす場合に限って定義される.

\begin{defi}[生成系]\index{せいせいけい@生成系}
	$M$の任意の元が$\sum_{\lambda\in\Lambda}a_\lambda u_\lambda ~(a_\lambda\in A)$とかけるとき, $(u_\lambda)$を$M$の\textbf{生成系}{system of generator}という.
\end{defi}
\begin{defi}[一次独立]\index{いちじどくりつ@一次独立}
	$\sum a_\lambda u_\lambda=0$なら,任意の$\lambda\in\Lambda$に対し$a_\lambda=0$となるとき, $(u_\lambda)$は\textbf{一次独立}(linearly independent)であるという. $(x_\lambda)$について$(\ast)$を満たさない場合, $(x_\lambda)$の任意の有限部分集合が一次独立のとき$(x_\lambda)$が一次独立であるという.
\end{defi}
\begin{defi}[基底]\index{きてい@基底}
	$M$の任意の元が$\sum a_\lambda u_\lambda$の形に一意に書けるとき, $(u_\lambda)$は$M$の\textbf{基底}(basis)であるという.	
\end{defi}

それぞれ,準同型;
\[\bigoplus_{\lambda\in\Lambda} A\to M;(a_\lambda)\mapsto\sum_{\lambda\in\Lambda} a_\lambda u_\lambda\]
が,全射,単射,全単射であることとそれぞれ同値.特に, $(u_\lambda)$が基底であることは$(u_\lambda)$が一次独立な生成系であることと同値.また, $A$加群$M$について,有限個の$A$の直和からの全射が存在するとき, $M$は\textbf{有限生成}(finitely generated)であるという.\index{ゆうげんせいせい@有限生成}

\begin{defi}[自由加群]\index{じゆうかぐん@自由加群}
	$M$が基底を持つとき, $M$を\textbf{自由加群}(free module)という.
\end{defi}

線形代数の復習として,線形空間基底の存在は保証されていること,その濃度は一意的であることを思い出そう(証明はしない).
\begin{thm}
	体$K$上の加群$V$は自由加群であり,その基底の濃度は一定である.
\end{thm}

可換環上の自由加群についても,濃度は一定である(非可換環については,基底が有限なとき一般には成立しない).

\begin{thm}\label{thm:可換環上の自由加群のrankは一定}
	可換環上の自由加群の基底の濃度は一定である.
\end{thm}

\begin{proof}
	$A$を可換環とし, $M$を$A$上の自由加群とする.その基底を$\{u_\lambda\}_{\lambda\in\Lambda},\{v_\omega\}_{\omega\in\Omega}$とする.すると
	\[M=\bigoplus_{\lambda\in\Lambda}Au_{\lambda}=\bigoplus_{\omega\in\Omega}Av_\omega\]
	とかける. Krullの定理より, $A$は極大イデアル$\ideal{m}$を持つ.また, $k=A/\ideal{m}$は体である.ここで;
	\[\ideal{m}M=\mkset{\sum_{\text{有限和}}a_k x_k}{a_k\in\ideal{m},x_k\in M}\]
	とすると,これは$M$の部分加群で, $\{u_\lambda\}_{\lambda\in\Lambda}$が$M$の基底なので;
	\[\begin{aligned}
	\ideal{m}M&=\mkset{\sum_{\lambda\in\Lambda} c_\lambda u_\lambda}{c_i\in\ideal{m}\text{は有限個を除いて}0}\\
	&=\bigoplus_{\lambda\in\Lambda}\ideal{m}u_\lambda
	\end{aligned}\]
	となる.よって$M/\ideal{m}M\cong\middleoplus_{\lambda\in\Lambda} Au_\lambda/\ideal{m} u_\lambda$となる.また$Au_\lambda/\ideal{m}u_\lambda$の元を考えると, $au_\lambda-bu_\lambda\in\ideal{m}u_\lambda$であることは$a-b\in\ideal{m}$であることと同値だから, $Au_\lambda/\ideal{m}u_\lambda=(A/\ideal{m})u_\lambda$とかける.ここで$M/\ideal{m}M$は$k$上の加群,すなわち$k$線形空間とみなすことができる.実際,作用;
	\[k\times M/\ideal{m}\to M/\ideal{m}M;(a+\ideal{m},x+\ideal{m}M)\mapsto ax+\ideal{m}M\]
	はwell-definedである(確かめよ).以上より$k$線形空間として;
	\[M/\ideal{m}M\cong\bigoplus_{\lambda\in\Lambda}(A/\ideal{m})u_\lambda\]
	となり, $\{u_\lambda\}_{\lambda\in\Lambda}$を$k$線形空間として基底に持つことがわかる.同様に$\{v_\omega\}_{\omega\in\Omega}$も$M/\ideal{m}M$の基底となっていて,線形空間の基底の濃度は一定であるので$\#\Lambda=\#\Omega$である.
\end{proof}

\begin{defi}[イデアル商]\index{いであるしょう@イデアル商}
	$M$を$A$加群, $L,N$をその部分加群とする.
	\[(L:N)=\mkset{a\in A}{aN\subset L}\]
	は$A$のイデアルになり,これを$L$と$N$の\textbf{イデアル商}(ideal quotient)という.
\end{defi}

また,イデアル$(0:M)$を$M$の\textbf{零化イデアル}(anihilator)といい$\ann(M)$とかく. $\ann(M)=0$となる加群を\textbf{忠実}(faithful)であるという. $A$の部分加群はイデアルであるから,これはイデアルについて考えることもできる.イデアル商の計算は次が基本的である(証明は簡単なので省略する).

\begin{prop}\label{prop:加群商}
	$I,J,K$を$A$のイデアルとし, $L,N$は$A$加群$M$の部分加群であるとすると,次が成り立つ.
	\begin{sakura}
		\item $I\subset(I:J)$
		\item $(I:J)J\subset I$
		\item $((I:J):K)=(I:JK)=((I:K):J)$
		\item $(\bigcap_\lambda I_\lambda:J)=\bigcap_\lambda (I_\lambda:J)$
		\item $(I:\sum_\lambda J_\lambda)=\bigcap_\lambda (I:J_\lambda)$
		\item $\ann (N+L)=\ann (N)\cap\ann (L)$
		\item $(L:N)=\ann((L+N)/L)$
	\end{sakura}
\end{prop}

\section{Noether環}

一般の環ではその構造が抽象的なため扱いが難しいことが多々あるが,ある種の\quo{有限性}を持つ環はそれを手がかりにいろいろな考察が進められている.そのなかで最も頻出するものが次に述べるNoether\textbf{環}である.

\begin{defi}[Noether環]\index{#Noetherかん@Noether環}\label{defi:Noether環}
		環$A$の任意のイデアルが有限生成であるとき, $A$をNoether\textbf{環}という.
\end{defi}
	
Noether環は次の同値条件を持つので,どれを定義にしても良い.

\begin{prop}
	次は同値である.
	\begin{sakura}
		\item $A$はNoether環である.
		\item $A$のイデアルの任意の増大列は有限個で停止する(昇鎖条件).
		\item $A$のイデアルの空でない任意の族は極大元を持つ(極大条件).
	\end{sakura}
\end{prop}
	
\begin{proof}
	\begin{eqv}[3]
		\item 
		$A$のイデアルの増大列を
		\[I_1\subset I_2\subset\dots\subset I_i\subset\cdots\]
		とする.ここで,~$I=\bigcup_{i=1}^\infty I_i$とおくと,これはイデアルである.仮定から有限生成であるので,~$I=(\nitem[r]{x})$とおく.任意の$i$に対して,定義からある$n_i$が存在して$x_i\in I_{n_i}$である.よって,~$i$は有限個であるから,~$n$を十分大きく取ればすべての$i$について$x_i\in I_n$すなわち$I\subset I_{n}$となる.これは$n\leq k$に対して$I_{n}=I_k$であることにほかならない.
		\item
		$I_{\lambda(\lambda\in\Lambda)}$を$A$のイデアルの空でない族とする.これが極大元を持たないとすると,任意の$\lambda_1\in\Lambda$に対して,ある$\lambda_2\in\Lambda$が存在して$I_{\lambda_1}\subsetneq I_{\lambda_2}$となる.以下同様に,真の増大列
		\[I_{\lambda_1}\subsetneq I_{\lambda_2}\subsetneq\dots\subsetneq I_{\lambda_i}\subsetneq\cdots\]
		がとれて,これは(ii)に矛盾.
		\item 
		$I$を$A$のイデアルとする.~$I$の有限部分集合が生成する$A$のイデアル全体の集合を$\Im$とする.~$0\in\Im$より$\Im\neq\emptyset$であるので,これは極大元$I_0$を持つ.ここで,~$I\neq I_0$とすると$x\in I\setminus I_0$に対し$I_0\subsetneq I_0+(x)\in\Im$となり,極大性に反する.よって$I_0=I$である.よって(i)が言える.
	\end{eqv}
\end{proof}

増大列(昇鎖)を減少列(降鎖)に置き換えて包含関係を逆にしたものがArtin\textbf{性}と呼ばれるものである.

\begin{defi}[Artin環]\index{#Artinかん@Artin環}
	次の同値な条件;
	\begin{sakura}
		\item  $A$のイデアルの任意の減少列は有限個で停止する(降鎖条件).
		\item $A$のイデアルの空でない任意の族は極小元を持つ(極小条件).
	\end{sakura}
	を満たす環をArtin\textbf{環}という.
\end{defi}

昇鎖条件,降鎖条件はそれぞれACC (Ascending Chain Condition), DCC (Dscending Chain Condition)と略される.

\begin{prop}
	$A$がNoether (Artin)なら,任意のイデアル$I$について$A/I$もNoether (Artin)である.
\end{prop}
\begin{proof}
	イデアルの対応を考えればわかる.
\end{proof}

Noether環の部分環が必ずしもNoetherではないことに注意しよう.例えば, Noetherでない整域は商体に含まれる(\ref{defi:商環}をみよ).

次の定理から,有限生成を確かめるのは素イデアルだけでよいことがわかる.
\begin{thm}[I.S.Cohen]
	$A$の素イデアルが有限生成なら, $A$はNoether的.
\end{thm}
\begin{proof}
	$A$のイデアルで,有限生成でないもの全体を$\Sigma$とする. $\Sigma\neq\emptyset$と仮定すると, Zornの補題から極大元$I$が存在する.仮定から$I$は素イデアルでないので, $x,y\in A$で$xy\in I,x,y\not\in I$を満たすものが存在する.すると, $I+Ax$は$I$より真に大きいから,有限生成で,ある$\nitem{u}\in I$を;
	\[I+Ax=(\nitem{u},x)\]
	となるようにとれる. $y\in (I:x)=\mkset{a\in A}{ax\in I}$より$I\subsetneq(I:x)$だからこれも有限生成で, $(I:x)=(\nitem[m]{v})$とできる.
	
	よって, $I=(\nitem{u},v_1x,\dots,v_mx)$となり, $I\in\Sigma$に矛盾.よって$\Sigma=\emptyset$である.
\end{proof}

\section{$A$代数と次数付き環}
\begin{defi}[$A$代数]\index{だいすう@($A$)代数}
	環$A,B$に対し,環準同型$f:A\to B$が定まっているとき$B$は$A$\textbf{代数}(algebra)であるという.
\end{defi}

このとき$B$は$a\cdot b=f(a)b$により$A$加群とみなせる.

\begin{defi}[有限型]\index{ゆうげんがた@有限型}
	$f:A\to B$により$B$を$A$代数とみる.ある$n\in\N$が存在して,全射準同型$\varphi:A[\nitem{X}]\to B$が存在するとき, $f$は\textbf{有限型}(finite type)であるという.
\end{defi}

このとき, $S=\{\varphi(X_1),\dots,\varphi(X_n)\}$を$B$の代数としての生成元といい, $B$は$A$\textbf{代数として有限生成}(finitely generated as $A$ algebla)という. $X_i$の$B$における像を$\alpha_i$としたとき, $B=A[\nitem{\alpha}]$とかく. $B$が$A$加群として有限生成であることとは異なる定義であるので,注意が必要である.なお,注意として単にfinite $A$-algebraと書かれた場合,これはfinite type を\textbf{意味しない.} これは$A$代数として自然に入る加群構造で有限生成であることを意味する. 

Hilbertの基底定理について述べておこう.
\begin{thm}[Hilbertの基底定理]\index{#Hilbertのきていていり@Hilbertの基底定理}\label{thm:Hilbertの基底定理}
		Noether環上の有限生成代数はNoether環である.
	\end{thm}
	\begin{proof}
		$A$がNoether環のとき, $A[X]$がNoether環なら帰納的に$A[\nitem{X}]$がNoether環となり,その剰余環である有限生成代数$B$はNoether環である.
		
		よって, $A[X]$がNoetherであることを示せば良い. $I\neq0$を$A[X]$のイデアルとする.これが有限生成であることを示す. $f_1\neq0$を, $I$の最小次数の多項式とする. $(f_1)\subsetneq I$ならば, $f_2$を$I\setminus(f_1)$の最小次数の元とする.同様に$(\nitem[i]{f})\subsetneq I$ならば, $f_{i+1}$を$I\setminus(\nitem[i]{f})$の最小次数の多項式とする.ここで,各$f_i$に対し, $\lt(f_i)=a_iX^{r_i}$とし, $A$のイデアルの増大列
		\[(a_1)\subset(a_1,a_2)\subset\dots\subset(\nitem[i]{a})\subset\cdots\]
		を考えると, $A$はNoether環なのでこれは停まる.すなわち,ある$n$があって, $j\geq n$に対して$a_j\in(\nitem{a})$となる.この$n$に対して$(\nitem{f})=I$であることを示す.
		
		背理法を用いる. $I\setminus(\nitem{f})\neq\emptyset$とすると, $f_{n+1}$を次数最小のものとしてとれる.さて, $a_{n+1}\in(\nitem{a})$より, $a_{n+1}=\sum_{i=1}^n c_ia_i~(c_i\in A)$とかける.いま$\deg{f_i}=r_i$であり,作り方から$r_i\leq r_{n+1}$なので
		\[g=f_{n+1}-\sum c_iX^{r_{n+1}-r_i}f_i\]
		とおくと, $\deg g<r_{n+1}$である. $f$の次数最小性より$g\not\in I\setminus(\nitem{f})$すなわち$g\in(\nitem{f})$となり, $f_{n+1}\in(\nitem{f})$が従うが,これは矛盾.よって$(\nitem{f})=I$である.
\end{proof}

\begin{cor}
	$A$がNoether環ならば$A[X]$はNoether環である.
\end{cor}

\begin{exer}
	$A[X]$がNoetherのとき, $A$はNoetherとなるか?
\end{exer}

\begin{exer}
	$A$がNoetherのとき, $n$変数多項式環$A[X_1,\dots,X_n]$は必ずNoetherか?変数が加算個な多項式環$A[X_1,\dots,X_n,\cdots]$はどうか?
\end{exer}

\begin{defi}[次数付き環]\index{じすうつきかん@次数付き環}
	$S_0$は環, $d>0$について$S_d$は$S_0$加群になっているとする. $s=\middleoplus_{i=0}^\infty S_i$に積が定義でき, $S_0$の元による積はスカラーの作用と一致し,かつ$S_nS_m\subset S_{n+m}$が成立するとき$S$を\textbf{次数付き環}(graded ring)という.
\end{defi}

$S_d$の元を$S$の斉次元という.わかりやすい例として,多項式環は次数付き環である.1変数多項式環$A[X]$は$S_d=AX^d$とすればよく, $n$変数については$d$次元斉次多項式,すなわち各項をなす単項式の総次数がすべて等しいもの(たとえば$X^2+2XY+Y^2$)全体を$S_d$とすると次数付き環になる.また, $S_+=\oplus_{d>0}S_d$は$S$のイデアルとなる.これを$S$の\textbf{無縁イデアル}(irrelevant ideal)という.\index{むえんいである@無縁イデアル}

%\begin{defi}[斉次イデアル]\index{せいじいである@斉次イデアル}
%	$S$を次数付き環, $I$を$S$のイデアルとする. $I=\middleoplus(S_d\cap I)$が成り立つとき, $I$を\textbf{斉次イデアル}(homogeneous ideal)という.
%\end{defi}
%\begin{prop}
%	次数付き環$S$のイデアル$I$が斉次であることと, $I$が$S$の斉次元たちで生成されることは同値である.
%\end{prop}
%\begin{prop}
%	$I$を次数付き環$S$の斉次イデアルとする.このとき$I$が素であること,任意の\textbf{斉次元}$x,y$について$xy\in I$なら$x\in I$または$y\in I$が成り立つことは同値である.
%\end{prop}


\section{中山の補題}


線形代数で習ったCayley-Hamiltonの定理を加群に対して拡張できる(\textbf{行列式のトリック}とも呼ばれる).
\begin{thm}[Cayley-Hamilton]\index{#Cayler-Hamiltonのていり@Cayley-Hamiltonの定理}
	\label{thm:Cayley-Hamilton}
	$M$を$n$個の元で生成される有限生成な$A$加群とし, $\varphi\in\text{End}_A(M)$が,ある$A$のイデアル$I$に対して$\varphi(M)\subset IM$であると仮定する.このとき, $\nitem{a}\in I$が存在して
	\[\varphi^n+a_1\varphi^{n-1}+\dots+a_n=0\]
	を満たす.
\end{thm}
\begin{proof}
	$M$の生成系を$\{\nitem{u}\}$とすると,~$\varphi(M)\subset IM$より
	\[\varphi(u_i)=\sum_{j=1}^n a_{ij}u_j\quad(a_{ij}\in I)\]
	とできる.よってKroneckerのデルタ$\delta_{ij}$を用いると$\sum_j(\delta_{ij}\varphi-a_{ij})u_j=0$である.行列$(\delta_{ij}\varphi-a_{ij})_{i,j}$に対し余因子行列をかけて, $\det(\delta_{ij}\varphi-a_{ij})$は$M$の自己準同型となるが,これはすべての$x_i$を消すので零射にほかならない.行列式を展開すれば求める式が得られる.
\end{proof}

これを利用して,\textbf{中山の補題}と呼ばれる強力な定理を証明できる.	

\begin{thm}[中山の補題]\index{なかやまのほだい@中山の補題(NAK)}\label{thm:NAK}
	$M$を有限生成$A$加群, $I$を$A$のイデアルとする. $M=IM$であるとき, $aM=0$かつ$a\equiv1\pmod{I}$を満たす$a\in A$が存在する.とくに$I\subset\rad(A)$ならば$M=0$である.
\end{thm}

\begin{proof}
	Cayley-Hamiltonの定理で$\varphi=\id_{M}$とすれば
	\[a=1+a_1+\dots a_n\]
	がそれを満たす.また, $I\subset\rad(A)$ならば, $a$は可逆なので$M=0$である.
\end{proof}

\begin{cor}\label{cor:NAK}
	$I$を$\rad(A)$に含まれるイデアルとする. $A$加群$M$と,その部分加群$N$が$M/N$が有限生成かつ$M=N+IM$であるとすると, $M=N$である.
\end{cor}

\begin{proof}
	$M/N\cong I(M/N)$より$M/N=0$である.
\end{proof}

$(A,\ideal{m})$を局所環とする.剰余環$k=A/\ideal{m}$は体になる. $A$加群$\ideal{m}$について$M/\ideal{m}M$は自然な演算(\ref{thm:可換環上の自由加群のrankは一定}の証明中で定義したもの)で$k=A/\ideal{m}$加群,すなわち$k$線形空間になる. $M$が有限生成$A$加群なら中山の補題を使うことで次が言える.

\begin{prop}\label{prop:Atimac_prop_2.8}
	$(A,\ideal{m})$を局所環とし, $M$を有限生成$A$加群とする. $\dim_k M/\ideal{m}M=r$ならば$M$は$r$個の元で生成される.
\end{prop}

\begin{proof}
	$x_1,\dots,x_r\in M$を$M/\ideal{m}M$での像が$k$線形空間としての基底になる元とする.ここで$x_1,\dots,x_r$が生成する$A$加群を$N$とおくと, $N$は$M$の部分加群であって$N+\ideal{m}M=M$が成り立つ.よって\ref{cor:NAK}より$N=M$である.
\end{proof}
\section{準同型と普遍性}

代数学(加群の理論)においては,いろいろな概念についてその構成に\textbf{自然に付随する準同型}が重要な働きをする.まずは準同型そのものに関連する定義について紹介しておこう.
\begin{defi}[Hom加群]
	環$A$上の加群$M,N$において;
	\[\hom_A(M,N)=\mkset{\varphi:M\to N}{\varphi:\text{準同型}}\]
	は$f,g\in\hom_A(M,N)$に対し$f+g$を$x\mapsto f(x)+g(x)$で定めることで加法群をなす.また, $A$によるスカラーを$af$は$x\mapsto f(ax)$となる準同型として定めることで定義できる.
\end{defi}

例えば,自然に$\hom_A(A,M)\cong M$である.加群の間の準同型を考えることで,圏論的に言えば関手的な取り扱いが可能になる.とはいえ,この節では圏論の知識を仮定せずとも良いように配慮した.圏論を学んでから,具体例として検討してもらいたい.その手助けとなるように,断った上で圏論的記述を加えたところもある.

準同型を考えると,自然に出てくるものが完全列である.これについて復習しておこう.
\begin{defi}[完全列]\index{かんぜんれつ@完全列}
	$M_i$を加群とし, $\varphi_i:M_i\to M_{i+1}$を準同型とする.そのとき,列;
	\[\begin{tikzcd}
	\cdots\nxcell M_{i-1}\nxcell[\varphi_{i-1}]M_i\nxcell[\varphi_i]M_{i+1}\nxcell\cdots
	\end{tikzcd}\]
	は任意の$i$に対し$\im\varphi_{i-1}=\ker{\varphi_i}$となるとき\textbf{完全列}(exact sequence)であるという.
\end{defi}

特に;
\[\ses[\varphi][\psi]{M_1}{M_2}{M_3}\]
が完全であることと, $\varphi$が単射, $\psi$が全射であることは同値である.この完全列を特に\textbf{短完全列}(short exact sequence)という.

また,次の完全列;
\[\begin{tikzcd}
M_1\nxcell[\varphi]M_2\nxcell[\psi]M_3\nxcell0
\end{tikzcd}\]
\[\begin{tikzcd}
0\nxcell M_1\nxcell[\varphi]M_2\nxcell[\psi]M_3
\end{tikzcd}\]
はそれぞれ(短)右完全列,左完全列という.

さて,自然に付随する準同型とはどういうものかを説明しよう.まず$A$加群$M$とその部分加群$N$について,剰余加群$M/N$を考えよう.このとき,次が成り立つ.

\begin{prop}[剰余加群の普遍性]
\textbf{自然な}全射$\varphi:M\to M/N$と,包含$\iota:N\to M$が存在する.このとき,任意の$A$加群$L$について,準同型$\psi:M\to L$で$N\subset\ker\psi$となるものに対し, $f:M/N\to L$で$f\circ\varphi=\psi$となるものが\textbf{一意的に}存在する.
\end{prop}

\begin{figure}[H]
	\centering
	\begin{tikzcd}
		N\arrow[bend left=30,rr,"0"]\nxcell[\iota]M\darrow[\varphi]\nxcell[\psi]L\\
		&M/N\arrow[ru,dashed,"f"]
	\end{tikzcd}
	\caption{剰余加群の普遍性}
\end{figure}

このように,自然な準同型$\varphi$について,ある条件を満たした準同型(この例では$\psi$)に対して可換になるような\textbf{$f$が一意に存在する}という性質を\textbf{普遍性}(universality)という.以後紹介していくテンソル積や極限といった概念では,普遍性が重要な働きをする.それどころか圏論では普遍性に完全に依存した議論をすることも珍しくない.具体的には準同型(射)の一意性が大切である.そこに着目した上で,\textbf{ある概念の普遍性を満たすものをそれ自身と定義する}という特徴づけを考える.具体的に説明しよう.ある対象が普遍性を持つことがわかっている(あるいは持ってほしい)場合は,次のように定義するのである.

\begin{defi}[普遍性による剰余加群の定義]
	$A$加群$M$とその部分加群$N$に対して,ある$A$加群$K$と$\varphi:M\to K$で$N\subset\ker\varphi$となるものが存在して,任意の$A$加群$L$と準同型$\psi:M\to L$で$N\subset\ker\psi$となるものについて$f:K\to L$が一意に存在するとき, $(K,\varphi)$を$M$の$N$による\textbf{剰余加群}といい, $M/N$とかく.
\end{defi}

普遍性から$K$は(同型を除いて)一意的に定まることが保証されるので,実際に普遍性をみたす$M/N=\mkset{x+N}{x\in M}$という加群で$K$を表すことが正当化されている.一意に定まることを見よう.剰余加群の普遍性を満たす$K,K'$を考える.次の図式のように,それぞれの普遍性から$f:M\to K',f':M\to K$が存在する($\iota$と合成すると$0$になることを表す$0$は省略した);
\begin{figure}[H]
	\centering
	\begin{tikzcd}
		&K\\
		N\nxcell[\iota]M\darrow[\varphi]\arrow[u,"\varphi"]\nxcell[\varphi']K'\arrow[lu,dashed,"f'",swap]\\
		&K\arrow[ru,dashed,"f",swap]\arrow[uu,bend left=50,"\id_K",crossing over,near end]
	\end{tikzcd}
	\caption{}
\end{figure}

このとき, $f'\circ f=\id_K$となり,\textbf{準同型の一意性から}$f'\circ f=\id_K$となる.同様に$f\circ f'=\id_{K'}$が確かめられ, $f,f'$によって$K$と$K'$は同型である.

この議論の本質は,準同型の一意性から普遍性で得られる準同型の合成と恒等写像が等しくなる、というところにある.よって,まったく同様の証明で普遍性を持つ対象は必ず(同型を除いて)一意に定まることがわかる.このようにある概念を普遍性を満たすもの,と定義して実際にこの加群(対象)が普遍性を満たす,という主張をすることでwell-definedに定義を行うことができる.

\ref{defi:直和,直積の存在}で定義した直積,直和についても普遍性を用いた定義が可能である.

\begin{defi}[普遍性を用いた直積の定義]
	$A$加群の族$\{M_\lambda\}$について,ある加群$L$と準同型の族$p_\lambda:L\to M_\lambda$が存在して,任意の$A$加群$N$と準同型の族$q_\lambda:N\to M_\lambda$に対し, $f:N\to L$で$p_\lambda\circ f=q_\lambda$となるものが一意的に存在するとき, $(L,p_\lambda)$を$\{M_\lambda\}$の\textbf{直積}といい, $\prod_\lambda M_\lambda$とかく.
\end{defi}

\begin{defi}[普遍性を用いた直和の定義]
	$A$加群の族$\{M_\lambda\}$について,ある加群$L$と準同型の族$\iota_\lambda: M_\lambda\to L$が存在して,任意の$A$加群$N$と準同型の族$\kappa_\lambda: M_\lambda\to N$に対し, $f:L\to N$で$f\circ\iota_\lambda=\kappa_\lambda$となるものが一意的に存在するとき, $(L,\iota_\lambda)$を$\{M_\lambda\}$の\textbf{直和}といい, $\bigoplus_\lambda M_\lambda$とかく.
\end{defi}

\begin{minipage}{.45\textwidth}
	\begin{figure}[H]
		\centering
		\begin{tikzcd}
			&\prod M_\lambda\arrow[dr,"p_\lambda"]\\
			N\arrow[ur,dashed,"f"]\arrow[rr,"q_\lambda"]&&M_\lambda
		\end{tikzcd}
		\caption{直積の普遍性}
	\end{figure}
\end{minipage}
\hfill
\begin{minipage}{.45\textwidth}
	\begin{figure}[H]
		\centering
		\begin{tikzcd}
			&\bigoplus M_\lambda\arrow[rd,"f",dashed]\\
			M_\lambda\arrow[ur,"\iota_\lambda"]\arrow[rr,"\kappa_\lambda"]&&N
		\end{tikzcd}
		\caption{直和の普遍性}
	\end{figure}
\end{minipage}

誤解の恐れがない限り添字が省略されることはいつもと同じである.また$p_\lambda$は\textbf{標準的射影}, $\iota_\lambda$は\textbf{標準的単射}と呼ばれる.

\begin{exer}
\ref{defi:直和,直積の存在}で定義された加群たちがそれぞれ直積,直和の普遍性を満たすことを確認せよ.また,普遍性の標準的な結果から同型を除いて一意的に定まることも確かめよ.
\end{exer}

早速次節では普遍性を使って\textbf{テンソル積}を定義していこう.

\section{テンソル積}

\begin{defi}[双線型写像]\index{そうせんけいしゃぞう@双線型写像}
	$M,N,L$を$A$加群とする. $\varphi:M\times N\to L$が次の3つ;
	\begin{defiterm}{BM}
		\item $\varphi(x_1+x_2,y)=\varphi(x_1,y)+\varphi(x_2,y)$
		\item $\varphi(x,y_1+y_2)=\varphi(x,y_1)+\varphi(x,y_2)$
		\item $\varphi(ax,y)=\varphi(x,ay)=a\varphi(x,y)$
	\end{defiterm}
	を満たすとき, $\varphi$を$A$双線型写像という.
\end{defi}

双線型写像$M\times N\to L$の全体を$\bil_A(M,N;L)$で表すとすると,ある$A$加群$K$と$\tau\in\bil_A(M,N;L)$が存在して$\hom_A(K,L)\to\bil_A(M,N;L);f\mapsto f\circ\tau$を同型となるようにできることが知られている.この$K$と$\tau$を, $N$と$M$の$A$上の\textbf{テンソル積}という.

\begin{defi}[テンソル積]\index{テンソル積}
	$M,N$を$A$加群とする.ある$A$加群$K$と$\tau\in\bil_A(M,N;K)$が存在して,任意の$A$加群$L$と$\varphi\in\bil(M,N;L)$に対して$f\circ\tau=\varphi$となる$f$が一意的に存在する. $(K,\tau)$を$M\otimes_A N$とかき, $M,N$の$A$上の\textbf{テンソル積}(tonsor product)という.
\end{defi}

\begin{figure}[H]
	\centering
	\begin{tikzcd}
	M\times N\arrow[r,"\varphi"]\arrow[d,"\tau"]&L\\
	M\otimes_A N\arrow[ur,"f",dashed]
	\end{tikzcd}
	\caption{テンソル積の普遍性}\label{fig:tensor}
\end{figure}

\begin{proof}[\textbf{テンソル積の存在証明}.]
		直積$M\times N$が生成する$A$加群;
		\[\mathcal{T}=\mkset{\sum_{\text{有限和}}a_i(x_i,y_i)}{a_i\in A,(x_i,y_i)\in M\times N}\]
		を考える.ここで,~$a\in A,x,x_1,x_2\in M,y,y_1,y_2\in N$として,次の形の元
		\[(x_1+x_2,y)-(x_1,y)-(x_2,y),\quad (x,y_1+y_2)-(x,y_1)-(x,y_2)\]
		\[(ax,y)-a(x,y),\quad (x,ay)-a(x,y)\]
		で生成される$\mathcal{T}$の部分加群を$\mathcal{I}$とする.すると,~$\mathcal{T}/\mathcal{I}$が$\tau:M\times N\to\mathcal{T}/\mathcal{I};(x,y)\mapsto(x,y)+\mathcal{I}$によりテンソル積となる.実際,~$\varphi:M\times N\to L$に対し;
		\[\widetilde{\varphi}:\mathcal{T}\to L;\sum a_i(x_i,y_i)\mapsto\sum a_i\varphi(x_i,y_i)\]
		を考えると,~$\varphi$の双線形性より$\mathcal{I}\subset\ker\widetilde{\varphi}$がわかる.よって;
		\[\mathcal{T}/\mathcal{I}\to\mathcal{T}/\ker\widetilde{\varphi};(x,y)+\mathcal{I}\mapsto(x,y)+\ker\widetilde{\varphi}\]
		がwell-definedであることがわかるので,同型$\mathcal{T}/\ker\widetilde{\varphi}\to L$と合成して,準同型;
		\[f:\mathcal{T}/\mathcal{I}\to L;(x,y)+\mathcal{I}\mapsto\varphi(x,y)\]
		を得る.これは$\tau$の全射性から一意に定まる.
		
		同型を除いて一意であることは普遍性の標準的な結果である.

\end{proof}
$x\in M,y\in N$に対し$\tau(x,y)=x\otimes y$ともかき,これを\textbf{元のテンソル積}という.
テンソル積は$M\otimes_A N$は$x\otimes y$を生成元とし;
\[(x_1+x_2)\otimes y=x_1\otimes y+x_2\otimes y\]
\[x\otimes(y_1+y_2)=x\otimes y_1+x\otimes y_2\]
\[(ax)\otimes y=x\otimes (ay)=a(x\otimes y)\]
を満たす$A$加群と解釈できる.また,ここから任意の$M\otimes N$の元は$x\otimes y (x\in M,y\in N)$の有限和でかけることもわかる.

これらのことが頭に入っていれば,煩雑な構成の証明は忘れても構わないが,いくつかの注意が必要である.
ここで, $x\otimes y$という表示はどの加群のテンソル積かを決定しないと意味がないことに注意しておこう.例えば$\Z$加群で考えると $\Z\otimes \Z/2\Z$において$2\otimes\bar{1}=1\otimes\bar{2}=1\otimes\bar{0}=0$だが, $\Z$の部分加群$2\Z$を考えると$2\Z\otimes\Z/2\Z$において$2\otimes\bar{1}\neq0$である.

しかし,次の事実はテンソル積の構成から即座に従うもので,有用である.

\begin{prop}\label{prop:テンソルの有限生成への制限}
	$A$加群$M,N$について, $0=\sum x_i\otimes y_i\in M\otimes N$とする.このとき,それぞれ有限生成な部分$A$加群$M_0,N_0$が存在して, $M_0\otimes N_0$において$\sum x_i\otimes y_i=0$である.
\end{prop}

\begin{proof}
	構成の証明中の記号を用いる. $\sum x_i\otimes y_i\in\mathcal{I}$なので,これは$\mathcal{I}$の生成系の有限和である.それらの各項を$x_j\otimes y_j$とし, $x_i,x_j$で生成される$M$の有限生成部分加群を$M_0, y_i,y_j$で生成される$N$の部分加群を$N_0$とすればよい.
\end{proof}

次に準同型のテンソルについて定義しておく.

\begin{prop}
	$f\in\hom(M,M')$と$g\in\hom(N,N')$に対し,準同型$f\otimes g:M\otimes N\to M'\otimes N'$で, $(f\otimes g)(x\otimes y)=f(x)\otimes g(x)$を満たすものが一意的に存在する.
\end{prop}
\begin{proof}
	直積の間の準同型$f\times g:M\times N\to M'\times N'$と$\tau':M'\times N'\to M'\otimes N'$の合成$(x,y)\mapsto f(x)\otimes g(x)$に対し,~$M\otimes N$の普遍性から$(f\otimes g)\circ\tau=\tau'\circ(f\times g)$となる$f\otimes g$が一意に定まる.このとき,~$f\otimes g:x\otimes y\mapsto f(x)\otimes g(x)$であるから,~$f\otimes g$は求める準同型であり,一意性は普遍性から従う.
\end{proof}
これは$f\times g:M\times N\to M'\times N';(x,y)\mapsto (f(x),g(x))$をテンソル積に誘導したものにほかならない.

テンソル積を演算と見ると次が基本的である.
\begin{prop}\label{prop:テンソル積は直和と可換}
	テンソル積は可換なモノイドをなす.つまり次の3つが成り立つ.
	\[A\otimes M\cong M, M\otimes N\cong N\otimes M, (M\otimes N)\otimes L\cong M\otimes (N\otimes L)\]
	また,テンソル積は直和と可換である.すなわち;
	\[\left(\bigoplus_{\lambda\in\Lambda} M_\lambda\right)\otimes N\cong \bigoplus_{\lambda\in\Lambda}(M_\lambda\otimes N)\]
	である.
\end{prop}
\begin{proof}
	\begin{step}
		\item モノイドをなすこと.
	\begin{sakura}
		\item $A\otimes M\to M;a\otimes x\mapsto ax$と,~$M\to A\otimes M;x\mapsto 1\otimes x$が互いに逆の関係となる.
		\item $\varphi:M\times N\to N\otimes M;(x,y)\mapsto y\otimes x$に対し, $f:M\otimes N\to N\otimes M;x\otimes y\mapsto y\otimes x$がとれる.同様に$\psi:N\times M\to M\otimes N;(y,x)\mapsto x\otimes y$に対して$g:N\otimes M\to M\otimes N;y\otimes x\mapsto x\otimes y$とでき,これが$f$の逆となる.
		\item 各$z\in L$について, $\varphi_z:M\times N\to M\otimes(N\otimes L);(x,y)\mapsto x\otimes(y\otimes z)$と定義することで,普遍性から$f_z:M\otimes N\to M\otimes(N\otimes L);x\otimes y\mapsto x\otimes(y\otimes z)$が定まる.これを用いて,双線形写像$\psi:(M\otimes N)\times L\to M\otimes(N\otimes L);(x\otimes y,z)\mapsto f_z(x\otimes y)$が定義できる.これをさらにテンソル積に落とすことで;
		\[f:(x\otimes y)\otimes z\mapsto x\otimes(y\otimes z)\]
		とできる.同様に
		\[g:x\otimes (y\otimes z)\mapsto (x\otimes y)\otimes z\]
		がとれて,~$f$と$g$は逆の関係である.
	\end{sakura}
	\item 直和と可換であること.
	
	\begin{mrkw}
		\item \[\varphi:\left(\bigoplus_\lambda M_\lambda\right)\times N\to\bigoplus_\lambda(M_\lambda\otimes N);((x_\lambda)_\lambda,y)\mapsto(x_\lambda\otimes y)_\lambda\]
		に対し,テンソル積の普遍性から
		\[f:\left(\bigoplus_\lambda M_\lambda\right)\otimes N\to \bigoplus_\lambda(M_\lambda\otimes N);(x_\lambda)_\lambda\otimes y\mapsto(x_\lambda\otimes y)_\lambda\]
		が存在する.
		\item 任意の$\lambda'\in\Lambda$に対し
		\[\psi_\lambda':M_\lambda'\times N\to\left(\bigoplus_\lambda M_\lambda\right)\otimes N;(x_\lambda',y)\mapsto(\tilde{x}_\lambda)_\lambda\otimes y\]
		を,~$\lambda=\lambda'$のとき$\tilde{x}_\lambda=x_\lambda',~\lambda\neq\lambda'$のとき$\tilde{x}_\lambda=0$として定める.テンソル積の普遍性から
		\[g_\lambda':M_\lambda'\otimes N\to\left(\bigoplus_\lambda M_\lambda\right)\otimes N;(x_\lambda'\otimes y\mapsto(\tilde{x}_\lambda)_\lambda\otimes y\]
		がとれて,直和の普遍性から
		\[g:\bigoplus_\lambda M_\lambda\otimes N\to \left(\bigoplus_\lambda M_\lambda\right)\otimes N;(x_\lambda\otimes y)_\lambda\to(x_\lambda)_\lambda\otimes y\]
		とできる.作り方から$f$と$g$は逆の関係である.
	\end{mrkw}
	\end{step}
\end{proof}

ある加群からある加群を作り出す\quo{操作}(圏論では\textbf{関手}(functor)的であるという)があるときには,完全列にどのような影響を与えるかをみることは常套手段である.

\begin{prop}[テンソル積の右完全性]
	$A$加群の完全列;
	\[\begin{tikzcd}
	M_1\nxcell[f]M_2\nxcell[g]M_3\nxcell0
	\end{tikzcd}\]
	に対し,任意の$A$加群$N$は;
	\[\begin{tikzcd}
	M_1\otimes N\nxcell[f\otimes\id_{N}]M_2\otimes N\nxcell[g\otimes\id_{N}]M_3\otimes N\nxcell0
	\end{tikzcd}\]
	を完全にする.
\end{prop}
\begin{proof}
	$g\otimes\id_{N}$の全射性は明らか.~$\im(f\otimes\id_{N})=\ker(g\otimes\id_{N})$を示そう. $\subset$は明らかなので$\supset$を示す.任意の$\sum_i x_i\otimes y_i\in\ker(g\otimes\id_{N})$をとる.~$M_3\cong M_2/f(M_1)$より
	\[\varphi:M_3\times N\to (M_2\otimes N)/(f(M_1)\otimes N);(g(x),y)\mapsto x\otimes y+f(M_1)\otimes N\]
	はwell-definedである.実際$g(x)=g(x')$とすると$x-x'\in\ker g=\im f$より$x\otimes y\in f(M_1)\otimes N$となる.よって,~$\varphi$は普遍性から
	\[h:M_3\otimes N\to(M_2\otimes N)/(f(M_1)\otimes N)\]
	を引き起こす.さて$g\otimes\id_{N}(\sum x_i\otimes y_i)=\sum g(x_i)\otimes y_i=0$より$0=h(\sum g(x_i)\otimes y_i)=\sum \varphi(g(x_i),y_i)=\sum x_i\otimes y_i+f(M_1)\otimes N$であるので,~$\sum x_i\otimes y_i\in\im(f\otimes\id_{N})$である.
\end{proof}
もちろん短完全列に対してこの命題を適用すると,右完全列が得られる.この状況は,圏論的には関手$-\otimes N$は右完全である,ということになる.これが完全関手になるような$N$のことを平坦であるという.

\begin{defi}[平坦加群]\index{へいたんかぐん@平坦加群}
	任意の短完全列;
	\[\ses{M_1}{M_2}{M_3}\]
	に対して;
	\[\ses{M_1\otimes N}{M_2\otimes N}{M_3\otimes N}\]
	が完全であるとき, $N$を\textbf{平坦}(flat)な加群であるという.
\end{defi}

これは単射な$\eta:M_1\to M_2$に対して, $\eta\otimes\id_{N}$もまた単射になることと同値である.

平坦性を確かめるには,実は有限生成な加群についてのみ確かめればよい.

\begin{prop}\label{prop:平坦性は有限生成を調べれば良い}
	$A$加群$N$が平坦であることと, $f:M_1\to M_2$が単射で$M_1,M_2$が有限生成ならば$f\otimes\id$が単射になることは同値である.
\end{prop}

\begin{proof}
	前者から後者が従うことは明らかである. $M_1,M_2$を(有限生成とは限らない)$A$加群とし,単射$f:M_1\to M_2$を考える. $u=\sum_{i=1}^s x_i\otimes y_i\in\ker(f\otimes\id)$をとる. $u=0$を示せばよい. $M_1'$を$x_1,\dots,x_s$によって生成される$M_1$の部分加群とする. $u'$を$M_1'\otimes N$における$\sum x_i\otimes y_i$を表すものとする.ここで$0=\sum f(x_i)\otimes y_i\in M_2\otimes N$であり, \ref{prop:テンソルの有限生成への制限}により有限生成部分加群$M_2'$が存在して, $M_2'\otimes N$において$\sum f(x_i)\otimes y_i=0$である.また, $M_1'$の構成から$f(M_1')\subset M_2'$である.すると, $f$の$M_1'$への制限$f':M_1'\to M_2'$が定義され,先の議論からこの記号のもとで$f'\otimes\id(u')=0$である.仮定から$f'\otimes\id$は単射なので$u'=0$であり,これは$u=0$を導く.
\end{proof}

実際の平坦な加群の例は,次節以降紹介する\textbf{局所化}や\textbf{射影加群}が与える.

\section{素イデアルと局所化}

\begin{defi}[積閉集合]\index{せきへいしゅうごう@積閉集合}
	環$A$の部分集合$S$について$1\in S,x,y\in S$ならば$xy\in S$が成り立つとき, $S$は\textbf{積閉}(multiplicatively closed)であるという.
\end{defi}

\begin{defi}[局所化]\index{きょくしょか@局所化}
	$A$を環とし, $S$を$A$の積閉な部分集合とする. $S$の元を分母に許すような環$S^{-1}A$を, $A$の$S$による\textbf{局所化}(localization)または\textbf{分数環}(fractional ring)という.
\end{defi}

$S^{-1}A$の正確な定義を与えておこう.直積$A\times S$に次の関係を入れる.
\[(a,s)\sim(a',s')\Longleftrightarrow t(sa'-s'a)=0\text{となる}t\in S\text{が存在する.}\]
これによる同値類を$a/s$とかき,その集合に自然な加法と乗法を定めたものを$S^{-1}A$とかく.写像$\varphi:A\to S^{-1}A;a\mapsto a/1$により, $S^{-1}A$には自然な$A$代数としての構造が入る.また;
\[\ker\varphi=\mkset{a}{sa=0\text{となる}s\in S\text{がある}}\]
であるから, $S$が零因子を持たなければ$\varphi$が単射となり, $A$を$S^{-1}A$に埋め込める.

\begin{defi}[全商環]\label{defi:商環}\index{しょうかん@商環}
$S$を$A$の非零因子全体の集合とすると,上の$\varphi$は単射であって, $S^{-1}A$を$A$の\textbf{全商環}(total fractional ring)\index{ぜんしょうかん@全商環}という. 
\end{defi}

$A$が整域のとき,これは商体にほかならない.
\begin{prop}[分数環の普遍性]
	$S$を$A$の積閉集合とする.このとき$f:A\to B$で$f(S)\subset B^\times$となる$A$代数$B$に対し,準同型$g:S^{-1}A\to B$で$g\circ\varphi=f$となるものが同型を除いて一意的に存在する.
\end{prop}
\begin{proof}
	$g:S^{-1}A\to B;x/s\mapsto f(x)f(s)^{-1}$により与えられる.
\end{proof}

次の命題は実際の計算によく用いられる.
\begin{prop}\label{prop:Spec S^-1Aの引き戻し}
	$S$を環$A$の積閉集合とする. $S^{-1}A$の素イデアルは$P\cap S=\emptyset$となる$P\in\spec A$と一対一に対応する.特に, $S^{-1}A$の素イデアルは;
	\[S^{-1}P=\mkset{a/s}{a\in P,s\in S}\]
	という形をしている.
\end{prop}
\begin{proof}
	$P\in\spec A$について, $S^{-1}P$は$S^{-1}A$の素イデアルとなる.
	
	また,一般に環準同型$\varphi:A\to B$と$P\in\spec B$について$\varphi^{-1}(P)\in\spec A$である.ここで$P'\in\spec S^{-1}A$について$\varphi^{-1}(P')\in\spec A$であって, $\widetilde{P'}=\mkset{x/s\in S^{-1}A}{x\in\varphi^{-1}(P'),s\in S}$とするとこれは$P'$に一致する.
	
	以上より, $P$と$\varphi^{-1}(P)$は一対一に対応する.また$Q\in\spec A$が$Q\cap S\neq\emptyset$ならば$S^{-1}Q$は単元を含み,$\varphi^{-1}(P)\cap S=\emptyset$となることもわかる.
\end{proof}

可換環論においては素イデアルが非常に重要な働きをする.環$A$の素イデアル全体を$\spec A$とかき,そこには位相構造が入ることが知られている.

\begin{defi}[Zariski位相]\label{defi:Zariski位相}\index{#Zariskiいそう@Zariski位相}
	環$A$のイデアル$I\subset A$に対し, $V(I)=\mkset{ P\in\spec A}{I\subset P}$は$\spec A$の閉集合系としての位相を定める.これを$A$のZariski位相という.
\end{defi}

\begin{exer}
	$\mathscr{V}=\mkset{V(I)}{I:A\text{のイデアル}}$が閉集合系をなすこと,すなわち;
\begin{sakura}
	\item $\emptyset,\spec A\in\mathscr{V}$である.
	\item $V(I)\cup V(J)=V(I\cap J)$である.
	\item $\bigcap_{\lambda}V(I_\lambda)=V(\sum_\lambda I_\lambda)$である.
\end{sakura}
	を確認せよ.
\end{exer}

この定義の動機は, week nullstellensatz(\ref{thm:week nullstellensatz})により代数閉体の組$K^n$と$K[\nitem{X}]$の極大イデアル全体が1対1対応を持つということにある.極大でない素イデアルをそこに合わせることで$K^n$の点以上の構造をもった$\spec K[\nitem{X}]$を考えよう,としたのがGrothendieckのスキーム論の始まりである(可換環論と幾何の繋がり).

\begin{lem}\label{lem:イデアルの根基と素イデアル}
	$A$のイデアル$I$に対し $\sqrt{I}=\bigcap_{P\in V(I)} P$である.
\end{lem}
\begin{proof}
	$\subset $は簡単にわかる.逆に$x\in\bigcap_{P\in V(I)}P$をとる. $x\not\in\sqrt{I}$であるとしよう.すると$S=\{x^n\}_{n\in\N}$は$I$と交わらない積閉集合となる.すると$I$は$S^{-1}A$の真のイデアルとなる.すると$S^{-1}A$の極大イデアル$\ideal{m}$がとれて$I$を含む.すると$\ideal{m}$は極大であるので,特に素イデアル.一方で$P\cap S\neq\emptyset$であるので$I$を含む素イデアルは存在しない.これは矛盾である.
\end{proof}

この補題を用いると閉集合の包含関係は次で判定できる.
\begin{prop}
	$A$のイデアル$I,J$に対し, $V(I)\subset V(J)$であることと$\sqrt{J}\subset\sqrt{I}$であることは同値である.
\end{prop}

\begin{defi}
	$ P\in\spec A$に対し, $S=A\setminus P$は素イデアルの定義から積閉で,これによる局所化を$A_P$とかいて$A$の$P$による局所化という.
\end{defi}

局所化という名前の通り$A_P$は局所環である(\ref{prop:local ring equiv}を用いる).しかしながら一般の積閉集合$S$による局所化は局所環になるとは限らないことに注意しなければならない(分数化,と呼べばよいだろうか?).

局所化$S^{-1}A$は自然に$A$代数になっていることに注意しよう.さきは$A_P$について素イデアルを考えたが,一般に$A$代数$B$があったときに(素)イデアルの対応を考えることは大切である. たとえば$\varphi:A\to B$があったとき, $P\in\spec A$について必ずしも$\varphi(P)$は$B$のイデアルになるかどうかすらわからない.そこで, $A$のイデアル$I$に対応する$B$のイデアルについては$\varphi(I)$が$B$で生成するイデアルを考えることが自然である.また, $Q\in\spec B$について$\varphi^{-1}(Q)$は必ず素イデアルになることを確かめることができる.これは自然な射$\spec B\to\spec A$
に他ならないが,それについては代数幾何についての教科書を見てほしい. $A\to S^{-1}A$などの自然な準同型による$Q\in\spec S^{-1}A$の引き戻しは$Q\cap A$と書いたりする.

$A$について, $P$による局所化を施すと$P$に含まれていない素イデアルは取り除かれる.また, $P$による剰余を施すと$P$を含んでいない素イデアルは取り除かれていた.これを組み合わせることによって, $A$の$P$以外の素イデアルを取り除くことができる.
\begin{defi}[剰余体]\index{じょうよたい@剰余体}\index{じょうよたい@剰余体}\label{defi:剰余体}
	環$A$と素イデアル$P$について, 体$A_P/PA_P$を$K(P)$とかいて, $A$の$P$における\textbf{剰余体}(residue field)という.
\end{defi}


\begin{prop}\label{prop:上にイデアルがあることの同値条件}
	$\varphi:A\to B$を準同型とし, $P\in\spec A$とする. $\varphi(P)$が$B$で生成するイデアルを$B'$とかく.ある$Q\in\spec B$が存在して$\varphi^{-1}(Q)=P$となることと, $\varphi^{-1}(P')=P$となることは同値である.
\end{prop}

\begin{proof}
	$(\Longrightarrow)$は明らか. $\varphi^{-1}(P')=P$と仮定する. $S=\varphi(A-P)$とおくと, $P'\cap S=\emptyset$である.すると$S^{-1}B$において$P'$は真のイデアルとなり,それを含む極大イデアル$\ideal{m}$がとれる. $Q=\ideal{m}\cap B$とおくと, $\varphi^{-1}(Q)=P$である.実際, $P'\subset Q$であるから$P\subset\varphi^{-1}(Q)$は明らかで, $x\in\varphi^{-1}(Q)$をとると$Q\cap S=\emptyset$より$x\not\in P$ならば$\varphi(x)=Q\cap S$となってしまうので$x\in P$でなければならない.
\end{proof}

素イデアルによる局所化のほかに,元による局所化(局所環になるとは限らない)があり,こちらも代数幾何で特に大切である(すでに\ref{lem:イデアルの根基と素イデアル}で登場している).
\begin{defi}
	$f\in A$に対し, $S=\mkset{f^n}{n\in\N}$は積閉集合である.ただし$f^0=1$と定義する.このとき, $S^{-1}A$を$A_f$と書いて$A$の$f$による局所化という.	
\end{defi}

いままでは環の局所化を考えていたが, $A$加群$M$について, $A$の積閉集合による局所化$S^{-1}M$を考えることができる.また, $A$加群の準同型$\varphi:M\to N$について,次の局所化の間の準同型;
\[S^{-1}\varphi:S^{-1}M\to S^{-1}N;x/s\mapsto \varphi(x)/s\]
が誘導されることに注意しよう.よって,加群の列;
\[\begin{tikzcd}
	M_1\nxcell[f]M_2\nxcell[g]M_3
\end{tikzcd}\]
について,誘導された列;
\[\begin{tikzcd}
S^{-1}(M_1)\nxcell[S^{-1}f]S^{-1}M_2\nxcell[S^{-1}g]S^{-1}M_3
\end{tikzcd}\]
が得られる.ここでこの操作によって完全性が保たれる(すなわち完全関手になっている)ことが大切である.

\begin{prop}\label{prop:局所化は完全関手}
	$\begin{tikzcd}
	M_1\nxcell[f]M_2\nxcell[g]M_3
	\end{tikzcd}$が完全ならば, $\begin{tikzcd}
	S^{-1}(M_1)\nxcell[S^{-1}f]S^{-1}M_2\nxcell[S^{-1}g]S^{-1}M_3
	\end{tikzcd}$も完全である.
\end{prop}
\begin{proof}
	$\im S^{-1}f\subset\ker S^{-1}g$は明らかなので,逆を示す.任意の$x/s\in\ker S^{-1}g$をとる.よって$g(x)/s=0$であるので,ある$h\in S$が存在して$hg(x)=0$である.よって$hx\in\ker g=\im f$であるから,ある$y\in M_1$がとれて$f(y)=hx$とかける.すると$S^{-1}(y/hs)=hx/hs=x/s$となる.
\end{proof}

これによって次の命題が示せる(証明はかんたんである).
\begin{prop}\label{prop:局所化はいろんな操作と可換}
	局所化は有限和,有限個の共通部分,剰余環をとる操作と可換である.すなわち, $A$加群$M$とその部分加群$N,P$について;
	\begin{sakura}
		\item $S^{-1}(N+P)=S^{-1}(N)+S^{-1}(P)$
		\item $S^{-1}(N\cap P)=S^{-1}(N)\cap S^{-1}(P)$
		\item $S^{-1}(M/N)\cong (S^{-1}M)/(S^{-1}N)$
	\end{sakura}
	が成り立つ.
\end{prop}

\ref{defi:剰余体}で説明した素イデアルの取り除きについて, $Q\subset P$を素イデアルとしたときに, $P$による局所化と$Q$による剰余を組み合わせることで$A$の$Q$と$P$の間にある素イデアル以外をすべて取り除くことができることがわかるが,この命題によって$A_P/QA_P$と$(A/Q)_P$が同型であるからその順序が関係ないことがわかる.

また,局所化は平坦な加群の例を与える.
\begin{prop}\label{prop:局所化は平坦}
	$S^{-1}A$は平坦である.とくに$S^{-1}A\otimes M\cong S^{-1}M$である.
\end{prop}

\begin{proof}
	\ref{prop:局所化は完全関手}より$S^{-1}A\otimes M\cong S^{-1}M$を示せばよいが,	
	\[f:S^{-1}A\otimes M\to S^{-1}M;a/s\otimes x\mapsto ax/s,\quad g:S^{-1}M\to S^{-1}M\otimes A;x/s\mapsto 1/s\otimes x\]
	が互いに逆写像となる.
\end{proof}

ところで,位相的性質,すなわち同相で変化しない性質のように(特に素イデアルによる)局所化で変化しないものを\textbf{局所的性質}(local properties)という.\index{きょくしょてきせいしつ@局所的性質} その例をいくつか見ておこう.

\begin{prop}\label{prop:局所化したら0は局所的}
	$M$を$A$加群とすると,次の3つ;
	\begin{sakura}
		\item $M=0$である.
		\item 任意の$P\in \spec A$について, $M_P=0$である.
		\item 任意の$A$の極大イデアル$\ideal{m}$について, $M_{\ideal{m}}=0$である.
	\end{sakura}
	は同値である.
\end{prop}

\begin{proof}
	$(\text{i})\Longrightarrow(\text{ii})\Longrightarrow(\text{iii})$は明らか. $M\neq0$と仮定する.任意の$0\neq x\in M$をとる. $\ann x$は真のイデアルである. $\ann x$を含む$A$の極大イデアル$\ideal{m}$をとる.このとき$M_{\ideal{m}}=0$であるので$x/1\in M_{\ideal{m}}=0$である.すると,ある$h\not\in\ideal{m}$が存在して$hx=0$である.これは$\ann x$が$\ideal{m}$に含まれることに矛盾.
\end{proof}

\begin{prop}\label{prop:局所的性質}
	$\varphi:M\to N$を$A$加群の準同型とする.このとき;
	\begin{sakura}
		\item $\varphi$は単射である.
		\item 任意の$P\in\spec A$について, $\varphi_P:M_P\to N_P$は単射である.
		\item 任意の$A$の極大イデアル$\ideal{m}$について, 
		$\varphi_{\ideal{m}}:M_{\ideal{m}}\to N_{\ideal{m}}$は単射である.
	\end{sakura}
	これは単射を全射に言い換えても成り立つ.
\end{prop}

\begin{proof}
	局所化は平坦であることから(i) $\Longrightarrow$ (ii)が従う. (ii) $\Longrightarrow$ (iii)は明らか. (iii)を仮定する.完全列;
	\[\begin{tikzcd}
	0\nxcell\ker\varphi\nxcell M\nxcell[\varphi]N
	\end{tikzcd}\]
	に対して,任意の極大イデアル$\ideal{m}$について;
	\[\begin{tikzcd}
	0\nxcell(\ker\varphi)_{\ideal{m}}\nxcell M_{\ideal{m}}\nxcell[\varphi_{\ideal{m}}]N_{\ideal{m}}
	\end{tikzcd}\]
	は完全.ここで$(\ker\varphi)_{\ideal{m}}=\ker\varphi_{\ideal{m}}=0$であるので, \ref{prop:局所化したら0は局所的}より$\ker\varphi=0$すなわち$\varphi$は単射である.
\end{proof}

\section{射影,入射加群}

準同型$\varphi:N_1\to N_2$があったとする.この準同型はHom加群の間の準同型;
\[\varphi_\ast :\hom_A(M,N_1)\to\hom_A(M,N_2);f\mapsto \varphi\circ f\]
\[\varphi^\ast :\hom_A(N_2,M)\to\hom_A(N_1,M);f\mapsto f\circ \varphi\]
を引き起こす.それぞれ次のような状況である.

\begin{minipage}{.45\hsize}
	\begin{figure}[H]
		\centering
		\begin{tikzcd}
			&N_2\\
			M\arrow[ur,"\varphi\circ f",dashed]\arrow[r,"f"]&N_1\arrow[u,"\varphi"]
		\end{tikzcd}
		\caption{}
	\end{figure}
\end{minipage}
\hfill
\begin{minipage}{.45\hsize}
	\begin{figure}[H]
		\centering
		\begin{tikzcd}
			&N_1\arrow[d,dashed,"f\circ\varphi"]\arrow[ld,"\varphi",swap]\\
			N_2\arrow[r,"f"]&M
		\end{tikzcd}
		\caption{}
	\end{figure}
\end{minipage}

下付きの$\ast$は写像の合成について共変的であることを,上付きのものは反変的であることを意味している.すなわち,加群の列;
\[\begin{tikzcd}
M_1\nxcell[\varphi]M_2\nxcell[\psi]M_3
\end{tikzcd}\]
について,下付きの$\ast$を考えると;
\[\begin{tikzcd}
\hom_A(N,M_1)\nxcell[\varphi_\ast]\hom_A(N,M_2)\nxcell[\psi_\ast]\hom_A(N,M_3)
\end{tikzcd}\]
が得られ,上付きを考えると;
\[\begin{tikzcd}
\hom_A(M_3,N)\nxcell[\varphi^\ast]\hom_A(M_2,N)\nxcell[\psi^\ast]\hom_A(M_1,N)
\end{tikzcd}\]
が得られる.

これは圏の言葉を用いれば$\hom(N,-)$は共変関手, $\hom(-,N)$は反変関手であると表現できる.これらは左完全になることが知られている.
\begin{prop}
$A$加群の完全列;
\[\ses[\varphi][\psi]{M_1}{M_2}{M_3}\]
と,任意の$A$加群$N$に対して;
\[\begin{tikzcd}
0\nxcell\hom_A(N,M_1)\nxcell[\varphi_\ast]\hom_A(N,M_2)\nxcell[\psi_\ast]\hom_A(N,M_3)
\end{tikzcd}\]
は完全である.
\end{prop}

同様に完全列;
\[\begin{tikzcd}
0\nxcell\hom_A(M_3,N)\nxcell[\varphi^\ast]\hom_A(M_2,N)\nxcell[\psi^\ast]\hom_A(M_1,N)
\end{tikzcd}\]
も完全である.証明はほぼ同じであるから,前者のみ示す.
\begin{proof}
	まず, $\varphi{}_\ast $が単射であることを確かめよう. $\varphi\circ g=\varphi\circ g'$とする.任意の$\in N$に対し$\varphi(g(x))=\varphi(g'(x))$となり, $\varphi$が単射なので$g(x)=g'(x)$すなわち$g=g'$である.
	
	さて, $\im\varphi{}_\ast =\ker\psi_\ast $を示せば良い.
	\begin{mrkw}
		\item 
		任意の$f\in\im\varphi_\ast$を1つとる.ある$g\in\hom_A(N,M_1)$が存在して$f=\varphi\circ g$とかけるので, $\psi_\ast (f)=\psi\circ\varphi\circ g$であり, $\im\varphi=\ker\psi$だからこれは消える.よって$f\in\ker\psi_\ast $である.
		\item 
		任意の$f\in\ker\psi_\ast $を1つとる. $\psi\circ f=0$だから任意の$x\in N$に対し$f(x)\in\ker\psi=\im\varphi$となり, $\varphi$が単射だから$f(x)=\varphi(y_x)$となる$y_x\in M_1$が一意に定まる.ゆえに$g:N\to M_1;x\mapsto y_x$がwell-definedであることがわかる.このとき$\varphi{}_\ast (g)=f$である.
	\end{mrkw}
	以上より,完全である.
\end{proof}

ここで$\psi_\ast $が全射であるとは限らないことに注意しよう.すなわち,上の命題について$\psi$が全射である仮定は不要である.同様に$\hom(-,N)$については$\varphi$が単射である仮定は不要である.では, $\psi_\ast(\psi^\ast)$が全射になる場合(すなわち$\hom$が完全関手になるとき)について考えよう.
\begin{defi}[射影加群,入射加群]
	]\index{しゃえいかぐん@射影加群}\index{にゅうしゃかぐん@入射加群}
	任意の$A$加群の短完全列;
	\[\ses{M_1}{M_2}{M_3}\]
	に対し;
	\[\ses{\hom_A(M,M_1)}{\hom_A(M,M_2)}{\hom_A(M,M_3)}\]
	が完全となるような$A$加群$M$を\textbf{射影加群}(projective module)という.双対的に;
	\[\ses{\hom_A(M_3,M)}{\hom_A(M_3,M)}{\hom_A(M_1,M)}\]
	が完全になるような$M$を\textbf{入射加群}(injective module)という.
\end{defi}

先の議論より, $M$が射影加群であることは,任意の全射な$\psi\in\hom_A(M_2,M_3)$と任意の$f\in\hom_A(M,M_3)$に対し$\varphi{}_\ast $が全射,すなわち$\varphi\circ\widetilde{f}=f$となる$\widetilde{f}\in\hom_A(M,M_2)$の存在と同値.この$\widetilde{f}$を$f$の\textbf{持ち上げ}(lifting)という.同様に, $M$が入射加群であることは任意の単射な$\varphi\in\hom_A(M_1,M_2)$と任意の$f\in\hom_A(M_1,N)$に対し$\varphi^\ast $が全射,すなわち$\widetilde{f}\circ\varphi=f$となる$\widetilde{f}$の存在と同値.これを$f$の\textbf{拡張}(expantion)という.それぞれ,下の図式が可換になる$\widetilde{f}$の存在,ということに要約される.

圏論的に言えば,左完全関手$\hom(M,-)$を完全にするものを射影加群,反変左完全関手$\hom(-,M)$を完全にするものを入射加群という,ということになる.

\begin{minipage}{.45\hsize}
	\begin{figure}[H]
		\centering
		\begin{tikzcd}
			M_2\nxcell[\psi]M_3\nxcell0\\
			P\arrow[ur,"f",swap]\arrow[u,dashed,"\widetilde{f}"]
		\end{tikzcd}
		\caption{射影加群$P$}
	\end{figure}
\end{minipage}
\hfill
\begin{minipage}{.45\hsize}
	\begin{figure}[H]
		\centering
		\begin{tikzcd}
			&&I\\
			0\nxcell M_1\arrow[ur,"f"]\nxcell[\varphi]M_2\arrow[u,dashed,"\widetilde{f}",swap]
		\end{tikzcd}
		\caption{入射加群$I$}
	\end{figure}
\end{minipage}

\begin{prop}\label{prop:自由加群は射影加群}
	自由加群は射影加群である.
\end{prop}
\begin{proof}
	$F=\middleoplus_{i\in I}A_{x_i}$とする. $\psi\in\hom_A(M_2,M_3),f\in\hom_A(F,M_2)$とすると, $\psi$が全射なので$y_i\in\psi^{-1}(f(x_i))$が存在し,それを適当に選んで$\widetilde{f}(x_i)=y_i$とすると,基底の送り先を定めれば十分だから$\widetilde{f}\in\hom_A(F,M_1)$を得る.
\end{proof}

これは選択公理と同値である. $\psi^{-1}(f(x_i))$の元を選んで$\widetilde{f}$を構成してするときに選択公理を使っている.
\begin{thm}
	$A$加群$M$が射影的であることと,ある$A$加群$N$に対して$M\oplus N$が自由であることは同値.
\end{thm}
\begin{proof}
	\begin{eqv}
		\item 
		$M$の生成系をとることで,自由$A$加群$F$からの全射$\varphi:F\to M$が定まる. $M$が射影的なので$\id:M\to M$の持ち上げ$f:M\to F$が存在する.すなわち$\varphi\circ f=\id_M$である. $\id$が単射なので$f$も単射となり,これによって$M$を$F$の部分加群とみなす.次の準同型;
		\[\psi:M\oplus \ker\varphi\to F;(x,y)\mapsto f(x)+y\]
		が同型を与えることを示す.
		
		\begin{mrkw}
			\item 単射であること.
			
			$\psi(x,y)=f(x)+y=0$とする.これを$\varphi$で送ると定義から$x$となるが, 0の像は0なので$x=0$である.すると$\psi(x,y)=y=0$となり, $(x,y)=0$となる.
			\item 全射であること.
			
			任意の$u\in F$をとる.すると定義から$u-f(\varphi(u))\in\ker\varphi$なので, $\psi(\varphi(u),u-f(\varphi(u)))=u$となる.
		\end{mrkw}
		よって全単射となり,同型を与える.
		\item 
		$F=M\oplus N$とおく.全射な$\psi:M_2\to M_3$について$f:M\to M_3$の持ち上げがあればよい.
		\[g:F\to M_3;(x,y)\mapsto f(x)\]
		を考えると, $F$は射影的なので$g$の持ち上げ$\widetilde{g}:F\to M_2$が定まる.このとき$\widetilde{g}|_M$が$f$の持ち上げとなる.実際, $\psi\circ\widetilde{g}|_M(x)=\psi\circ\widetilde{g}(x)=g(x)=f(x)$となる.
	\end{eqv}
\end{proof}

\begin{thm}
	射影加群は平坦である.
\end{thm}
\begin{proof}
	$P$を射影加群とする. $P$はある自由加群の直和因子だから, $F$を自由として$F=P\otimes N$とする. $F\cong\oplus_\lambda Au_\lambda$とすると
	\[F\otimes M_i\cong \bigoplus_\lambda (Au_\lambda\otimes M_i)\]
	より, $\id_{F}\otimes\eta:F\otimes M_i\to F\otimes M_2$も単射.また$F\otimes M_i\cong (P\otimes M_i)\oplus (N\otimes M_i)$より$P\otimes M_1\to P\otimes M_2$に制限しても単射.
\end{proof}

\section{環の直積}

\begin{defi}[直積環]\index{ちょくせきかん@直積環}
	$\{A_i\}_{i\in I}$を環の族とする. 集合としての直積$\prod_{i\in I} A_i$には,各成分ごとの和,積を考えることで環構造が入る.これを\textbf{直積環}(product ring)という. $\prod A_i$の元で,第$i$成分が1であり,それ以外の成分は0であるものを$e_i$とかく.
\end{defi}

単位元はすべての成分が1である元であり,各$A_i$が可換ならば直積も可換になる.また, $e_i$たちはベキ等元であることに注意しよう.第$i$成分が$a_i\in A_i$であり,それ以外の成分が1である元を$a_i$と同一視することで,自然な単射$A_i\to\prod A_i$が存在する.

\begin{lem}
	直積環$\prod A_i$は整域になり得ない.
\end{lem}

\begin{proof}
	$x=(1,0,\dots,0),y=(0,1,0,\dots,0)$とするとどちらも0でなく, $xy=0$である.
\end{proof}

主に有限直積,とくに2つの環の直積であるときを考えよう(3つ以上のときは$A_1\times A_2\times A_3=(A_1\times A_2)\times A_3$であるから同様に議論できる).


\begin{prop}
	$A$を環とすると,次は同値である.
	\begin{sakura}
		\item $A$は環の直積$A_1\times A_2$と同型である.
		\item $e_1,e_2\in A$が存在して,それぞれ$e_i^2=e_i$であり, $e_1+e_2=1,e_1e_2=0$を満たす.
	\end{sakura}
\end{prop}

\begin{proof}
	$(\Longrightarrow)$は明らか. (ii)を仮定しよう.次の準同型;
	\[\varphi:A\to A/(e_1)\times A/(e_2);x\mapsto (x+(e_1),x+(e_2))\]
	が環同型となる.実際$x\in\ker\varphi$とすると$x\in (e_1)$かつ$x\in (e_2)$であるので,ある$a_1,a_2\in A$が存在して$x=a_1e_1=a_2e_2$とかける.すると,各辺に$e_1$を掛けることで$x=a_1e_1=0$が従う.よって$\varphi$は単射であり,また任意の$(x+(e_1),y+(e_2))$について, $xe_2+ye_1$を考えると$xe_2-x=x(1-e_2)=xe_1$より, $xe_2+ye_1-x\in (e_1)$が従う.同様に$xe_2+ye_1-y\in (e_2)$より, $\varphi(xe_2+ye_1)=(x+(e_1),y+(e_2))$であることがわかる.以上より$\varphi$は環同型を与える.
\end{proof}

また, $A=A_1\times A_2$の素イデアルによる局所化を考えよう. $P\times A_2\in\spec A$について, $A_{P\times A_2}$は${A_1}_{P}$と同型である.これはその局所化が整域でも,局所化する前の環が整域でない例を与えている.

さて,環の直積$\prod A_i$があったとき,自然な射影$\pi_i:\prod A_i\to A_i$が存在する.これは環の全準同型であり,スキームの閉移入$\pi_i^\ast:\spec A_i\to\spec\prod A_i$を与える.これは$P\in\spec A_i$について$\pi_i^{-1}(P)$を与える写像である.これによって有限直積については$\spec$の構造を決定できる.素イデアルの直積は素イデアルにならないことに注意しよう.
\begin{prop}
	$A_1,A_2$を環とする. $\spec (A_1\times A_2)$は, $P_1\in\spec A_1$と$P\times \spec A_2$を, $P_2\in\spec A_2$と$A_1\times P_2$を同一視することで$\spec A_1\sqcup\spec A_2$と一致する.
\end{prop}

\begin{proof}
	$A=A_1\times A_2$とおく. $P\in\spec A$について, $e_1e_2=0\in P$より, $e_1\in P$または$e_2\in P$である. $e_1+e_2=1$であるので, $e_1,e_2\in P$となることはない. ここでは$e_1\in P$と仮定する.このとき$\pi_2^{-1}(\pi_2(P))=P$であり, $\pi_2(P)\in\spec A_2$であることを示そう.このとき$\pi_2(P)=P_2$とおけば$P=A_1\times P_2$とかける.
	
	さて, $P\subset\pi_2^{-1}(\pi_2(P))$は明らかなので,逆を示す. $(a_1,a_2)
	\in\pi_2^{-1}(\pi_2(P))$とすると, $\pi_2(a_1,a_2)=a_2\in\pi_2(P)$なので,ある$a_1'\in A_1$が存在して$(a_1',a_2)\in P$である.すると$(a_1',0)=a_1'e_1\in P$であるので, $(0,a_2)\in P$である.これは$(a_1,a_2)=a_1e_1+(0,a_2)\in P$を導く.よって$\pi_2^{-1}(\pi_2(P))=P$である.また,次の環の同型;
	\[A/P=A/\pi_2^{-1}(\pi_2(P))=A_2/\pi_2(P)\]
	より, $A_2/\pi_2(P)$は整域となり$\pi_2(P)\in\spec A_2$である.
	
	$e_2\in P$のときは,同様に$P_1\in\spec A$を用いて$P=P_1\times A_2$とかける.
\end{proof}

これは有限個の場合に拡張できる.すなわち次が成り立つ(証明は略).

\begin{prop}
	環の有限直積$A=A_1\times A_2\times\dots\times A_n$について,任意の $P'\in\spec A$は,ある$1\leq i\leq n$と$P\in\spec A_i$により$P'=\pi_i^{-1}(P)$とかける.すなわち;
	\[\spec A=\spec A_1\sqcup\spec A_2\sqcup\dots\sqcup\spec A_n\]
	である.
\end{prop}

では,無限直積の場合はどうだろうか?実は成り立たないことが知られている.そのために補題を考えよう.

\begin{lem}
	環の無限直積$A=\prod_{i\in I}A_i$について,各$i\in I$に対し, $\{e_i\}_{i\in I}$の生成する$A$のイデアル$I_0$は$A$よりも真に小さい.
\end{lem}

\begin{proof}
	任意の$x\in I_0$をとると,有限個の$i_1,\dots,i_s\in I$が存在して;
	\[x=a_1e_{i_1}+a_2e_{i_2}+\dots+a_se_{i_s}\]
	とかける.これに$e_{i_j}$をかけると, $x$の$i_j$成分は$a_j$であることがわかる.また, $i_{s+1}$を$i_1,\dots,i_s$のどれとも違うものとすれば, $x$の$i_{s+1}$成分は0である.よって, $A$の単位元$1$は$I_0$に含まれない.
\end{proof}

\begin{prop}
	環の無限直積$A=\prod_{i\in I}A_i$について, $P\in\spec A$が存在して, $P$は$\pi^{-1}_i(P_i), P_i\in\spec A_i$の形で表せない.
\end{prop}

\begin{proof}
	補題と同じ記号を用いる.イデアル$I_0$を含む極大イデアル$\ideal{m}$をとる.すると,すべての$i\in I$に対し$e_i\in\ideal{m}$である.ここで,任意の$j\in I$に対して,任意の$P_j\in\spec A_j$をとると, $P_j$は$A_j$の単位元を含まないから$e_j\not\in\pi^{-1}_j(P_j)$である.よって$\ideal{m}\neq\pi_j^{-1}(P_j)$となる.
\end{proof}

\section{GCD整域と原始多項式}
体上の多変数多項式はPIDではなかったが, UFDであることを示すことができる.そのために次の定義を導入しよう.

\begin{defi}[GCD整域]\index{#GCD整域@GCD整域}\label{defi:GCDdomain}
	整域$A$であって,任意の2つの$x,y\in A$が最大公約元を必ず持つとき, $A$をGCD\textbf{整域}(GCD domain)という.
\end{defi}

既約分解することでUFDはGCD整域であることがわかる.

\begin{defi}[内容,原始多項式]\index{げんしたこうしき@原始多項式}\index{ないよう@内容}
	$A$をGCD整域とする. $f(X)=a_nX^n+\dots+a_0\in A[X]$について$a_n,\dots,a_0$の最大公約元を$c(f)$とかき, $f$の\textbf{内容}(content)という. $c(f)$が単元であるとき, $f$は\textbf{原始的}(primitive)であるという.
\end{defi}

$x,y\in A$について,ある$u\in A^\times$が存在して$x=uy$とかけるとき$x,y$は同伴である,と定義したことを思い出そう.この節ではこのことを$x\sim y$と書くことにする.
\begin{lem}[Gaussの補題]
	$A$をGCD整域とする. $f,g\in A[X]$について$c(fg)\sim c(f)c(g)$が成り立つ.
\end{lem}

\begin{proof}
	$f=c(f)f_0, g=c(g)g_0$と分解すると$f_0,g_0$は原始的である.また$c(fg)\sim c(c(f)c(g)f_0g_0)\sim c(f)c(g)c(f_0g_0)$であるので$f,g$は原始的であると仮定してよい. $f=a_nX^n+\dots+a_0,g=b_mX^m+\dots+b_0$とおく. $fg=c_{n+m}X^{n+m}+\dots+c_0$とおき, $n+m$についての帰納法で示す. $c(fg)=\gcd(c_{n+m},\dots,c_0)$であるが,これは;
	\[\gcd(a_n,c_{n+m-1},\dots, c_0)\gcd(b_n,c_{n+m-1},\dots,c_0)\]
	を割り切る.ここで, GCD整域において$\gcd(x,y_1,\dots,y_n)\sim\gcd(x,y_1+zx_1,\dots y_n+zx_n)$であることから;
	\[\gcd(a_n,c_{n+m-1},\dots,c_0)\sim\gcd(a_n,c_{n+m-1}-a_nb_{m-1},\dots,c_n-a_nb_0,c_{n-1},\dots,c_0)\sim\gcd(a_n, c((f-a_nX^n)g))\]
	である. $\deg(f-a_nX^n)g<n+m$であるから,帰納法の仮定より;
	\[c((f-a_nX^n)g)\sim c(f-a_nX^n)c(g)\sim c(f-a_nX^n)=\gcd(a_{n-1},\dots,a_0)\]
	であるので, $\gcd(a_n,c_{n+m-1},\dots,c_0)\sim c(f)$である.同様に$\gcd(b_m,c_{n+m-1},\dots,c_0)\sim c(g)$であるので, $c(fg)$は単元の約元である.よって$c(fg)$も単元である.
\end{proof}

%\begin{cor}
%	$A$をGCD整域, $A$の商体を$K$とする.定数でない$f\in A[X]$に対し, $f$が$A[X]$の既約元であることと, $f$は原始的かつ$K[X]$において既約であるこは同値.
%\end{cor}
%
%\begin{proof}
%	$f$は$A[X]$の既約元とする. $f=c(f)f_0$とすると, $c(f)$または$f_0$は単元である. $f$は定数でないので$c(f)$は単元,すなわち$f$は原始的である.次に, $K[X]$において$f=gh$とかけたとする.分母を払うように0でない$a,b\in A$をとることで$ag,bh\in A[X]$とできる. このとき$abf=c(ag)c(bh)g_0h_0~(g_0,h_0$は原始的)であり, Gaussの補題から$ab\sim c(ag)c(bh)$である.よって$A[X]$は整域だから$f\sim g_0h_0$である. $f$は既約なので$g_0$または$h_0$が単元である. $(A[X])^\times=A^\times$なので$ag$または$bh$は単元である.よって$ag$または$bh$が$A$の元である.よって$g\in K$または$h\in K$である.逆は明らかであろう.
%\end{proof}

\begin{prop}
	$A$はGCD整域とする. $p\in A$が$A$で素元であることと, $A[X]$において素元であることは同値.
\end{prop}

\begin{proof}
	$p$は$A$の素元とする. $f,g\in A[X]$をとり, $fg\in pA[X]$であるとする.ある$h\in A[X]$が存在して$fg=ph$となるので, Gaussの補題より$pc(h)\sim c(f)c(g)$となる. $p$は$A$の素元なので$c(f)\in pA$または$c(g)\in pA$であり,これは$f\in pA[X]$または$g\in pA[X]$を意味する.
\end{proof}

\begin{lem}
	$A$をGCD整域とし, $K$をその商体とする. $f\in A[X]$が原始的であるとき, $f$が$A[X]$で素元であることと$K[X]$において素元であることは同値.
\end{lem}
\begin{proof}
	\begin{eqv}
		\item $g,h\in K[X], gh\in fK[X]$とする.ある$q\in K[X]$が存在して$gh=fq$である. $g,h,q$の分母を払い$ag,bh,dg\in A[X]$とする. $ag=c(ag)g_0,bh=c(bh)h_0,dq=c(dq)q_0$とすると, Gaussの補題から$d c(ag)c(bh)\sim ab c(dq)$であるので, $g_0h_0\sim q_0f$である. $f$は$A[X]$で素元なので$g_0\in fA[X]$または$h_0\in fA[X]$である.よって$g\in fK[X]$または$h\in fK[X]$が従う.
		
		\item $g,h\in A[X], gh\in fA[X]$とする. $K[X]$の元とみなせば素元であるので, $g\in fK[X]$または$h\in fK[X]$が成り立つ. $g\in fK[X]$としよう.ある$\varphi\in K[X]$が存在して$g=\varphi f$となる.分母を払い$a\varphi\in A[X]$とすると,内容をとって$ag=c(a\varphi)\varphi_0f$となる. Gaussの補題より$c(ag)=ac(g)\sim c(a\varphi)$となる.よって$a\varphi=c(a\varphi)\varphi_0\sim ac(g)\varphi_0$より$\varphi\sim c(g)\varphi_0$であるので, $\varphi\in A[X]$である.よって$g\in fA[X]$である. $h\in fk[X]$のときも同様.
	\end{eqv}
\end{proof}

これらの準備によって次が示される.

\begin{thm}
	$A$がUFDであることと$A[X]$がUFDであることは同値である.
\end{thm}

\begin{proof}
	$A$がUFDなら$A[X]$もそうであることを示せばよい. $f=c(f)f_0\in A[X]$をとる. $A$の商体を$K$とおき, $K[X]$で$f_1=p_1\dots p_n$と素元分解する.分母を払って$a_ip_i=c(a_ip_i)p_{i,0}$とかける.ここで$p_{i,0}$は原始的で, $K[X]$において$p_i$と同伴なので素元である.よって補題から$A[X]$でも素元.積をとって$a_1\dots a_n f_1=c(a_1p_1)\dots c(a_np_n)p_{1,0}\dots p_{n,0}$となるから,内容をとって$f_1\sim p_{1,0}\dots p_{n,0}$となる.これは$A[X]$における$f_1$の素元分解を与える. $A$はUFDだから$c(f)$も素元分解でき,よって$f$を分解できる.よって$A[X]$はUFDである.
\end{proof}

\begin{cor}\label{cor:多変数もUFD}
	UDF上の$n$変数多項式環はUFDである.
\end{cor}


\section{素イデアル避け(Prime avoidance)}

\begin{lem}[Prime avoidance]\index{#Prime avoidance@Prime avoidance}\label{lem:Prime avoidance}
	環$A$のイデアル$P_1,\dots, P_n$で,素イデアルでないものは高々2つしかないとする. $A$のイデアル$I$が$I\subset\bigcup_{i=1}^n P_i$を満たすならば,ある$i$について$I\subset P_i$である.
\end{lem}

Prime avoidance, 素イデアル避けという名前の由来は対偶;
\[\{P_i\}\text{に対して,すべての}i\text{について}I\not\subset P_i\text{ならば}I\not\subset\bigcup_{i=1}^n P_i\]
に由来する.
\begin{proof}
	反例$I,P_1,\dots, P_n$があるとする.そのなかでも$n$が最小なものを取ろう. $n=1$ではありえないので$m\geq 2$である.
	
	\begin{sakura}
		\item $n=2$のとき.
		
		$I\not\subset P_1,P_2$より$a_1,a_2\in I$を$a_2\not\in P_1,a_1\not\in P_2$となるようにとれる.このとき$I\subset P_1\cup P_2$なので, $a_1\in P_1,a_2\in P_2$である. $a=a_1+a_2$とおくと,もし$a\in P_1$ならば$a_2=a-a_1\in P_1$となり矛盾. $a\in P_2$のときも同様.よって$a\not\in P_1\cup P_2$であるが,これも矛盾である.
		
		\item $n\geq3$のとき.
		
		$P_i$たちの中に素であるものが少なくとも1つ存在するので,それを並べ替えて$P_1\in\spec A$とする.ここで,各$i$について$I,\{P_i\}_{i\neq j}$は反例になりえないので, $I\not\in\bigcup_{j\neq i}P_j$が成り立つ.よって,ある$a_i\in I$をとって, $a_i\not\in\bigcup_{j\neq i}P_i,a_i\in P_i$となるようにできる. $a=a_1+a_2a_3\dots a_n$とおくと, $a\in I$であって, $a\not\in\bigcup_{i=1}^n P_i$であることを示そう.
		
		$a\in P_1$ならば$a_2\dots a_n\in P_1$だが,これは$P_i$が素なので, $i\geq2$について$a_i\not\in P_1$であることに矛盾.また$i\geq2$について$a\in P_i$ならばやはり$a_1\in P_1$となり矛盾する.よって$I\not\subset\bigcup P_i$となり,仮定に反する.
	\end{sakura}
\end{proof}

\begin{thm}[Davisの補題]\index{#Davisのほだい@Davisの補題}\label{thm:Davisの補題}
	環$A$の素イデアル$P_1,\dots,P_n$に対して,ある$a\in A$とイデアル$I$が存在して$(a)+I\not\subset\bigcup_{i=1}^n P_i$ならば,ある$x\in I$を選んで$a+x\not\in\bigcup_{i=1}^n P_i$であるようにできる.
\end{thm}

定義から,ある$c\in A,x\in I$を選べば$ca+x\not\in\bigcup P_i$とでき, $c=0$で$a$が消えてしまうこともありえるが,この定理は$c=1$とすることができる,と主張しているところが強力である.

\begin{proof}
	$n$についての帰納法で示す.
	
	\begin{step}
		\item $n=1$のとき.
		
		対偶を考える.任意の$x\in I$について$a+x\in P_i$ならば, $x=0$とすると$a\in P_i$となり, $(a)+I\subset P_1$である.
		
		\item $n-1$まで正しいとする.
		
		任意の$1\leq i\leq n-1$について, $P_i\not\subset P_n$としてよい.よって$\prod_{i=1}^{n-1}P_i\not\subset P_n$である.さて, $(a)+I\not\subset\bigcup_{i=1}^{n-1}P_i$より,帰納法の仮定からある$y\in I$をとって$a+y\not\in\bigcup_{i=1}^{n-1}P_i$とできる.もし$a+y\not\in P_n$ならば$y$が求める元となり証明が終了する. $a+y\in P_n$だったとき, $a+x\not\in \bigcup_{i=1}^n P_i$となる$x\in I$を構成しよう.ここで, $I\not\subset P_n$である.もし$I\subset P_n$ならば$a+y\in P_n$より$a\in P_n$tなり$(a)+I\subset P_n$となるので,これは仮定に反する.また, $\prod_{i=1}^{n-1}P_i\not\subset P_n$であったので, $P_n$は素だから$I\prod_{i=1}^{n-1}P_i\not\subset P_n$である.そこで, $z\in I\prod_{i=1}^{n-1}P_i$を$z\not\in P_n$であるようにとれる. $x=y+z$とおくと, $a+x\not\in\bigcup_{i=1}^n P_i$である.実際,任意の$1\leq i\leq n-1$について$a+y\not\in P_i$であって, $z\in P_i$なので, $a+x\not\in P_i$である.また$a+y\in P_n$で$z\not\in P_n$なので$a+x\in P_n$である.	
	\end{step}
\end{proof}

Davisの補題を指してPrime avoidanceということもある.応用例として,次の事実;
\[A\text{が整域なら}(x)=(y)\text{であることと}x\text{と}y\text{が同伴であることは同値.}\]
について, $A$について整域以外の条件を課して成り立つかどうかを考えてみよう.端的に言えば半局所環で成り立つことを示せる.

\begin{prop}\label{prop:半局所環と同伴関係}
	$A$を半局所環とすると, $(x)=(y)$であることと$x$と$y$が同伴であることは同値である.
\end{prop}

\begin{proof}
	ある$a\in A$が存在して$y=ax$とかける.また$A$が半局所環なので$\specm A=\{\nitem{\ideal{m}}\}$とおくことができる. $(a)\subset\ideal{m}_i$となる極大イデアルについて,局所環$A_{\ideal{m}_i}$を考える. $A$のイデアルとして$(x)=(a)(x)$であり, $(x)A_{\ideal{m}_i}=(a)A_{\ideal{m}_i}(x)A_{\ideal{m}_i}$かつ$(x)A_{\ideal{m}_i}$は有限生成$A_{\ideal{m}_i}$加群である.また$\rad A_{\ideal{m}_i}=\ideal{m}_iA_{\ideal{m}_i}$であるので, $(a)A_{\ideal{m}_i}$は$A_{\ideal{m}_i}$のJacobson根基に含まれるイデアルである.よって中山の補題(\ref{thm:NAK})より$(x)A_{\ideal{m}_i}=0$である.よって$\ann (x)\not\subset\ideal{m}_i$である.ゆえに$(a)+\ann (x)\not\subset\ideal{m}_i$が成り立つ.また, $(a)\not\subset\ideal{m}_j$となる$\ideal{m}_j$に対して明らかに$(a)+\ann (x)\not\subset\ideal{m}_j$である.よってDavisの補題(\ref{thm:Davisの補題})から,ある$b\in\ann(x)$が存在して$a+b\not\in\bigcup_{i=1}^n\ideal{m}_i$である.よって$a+b$は$A$の単元である.また, $x(a+b)=ax=y$であるので, $x$と$y$は同伴である.
\end{proof}

 %準備
	\newpage
\part[Noetherian properties]{Noether性}
\section{極大条件と極小条件}
Noether環の定義は先に述べたとおりだが,まずはそれを加群についても考えてみよう.
\begin{prop}
	$A$加群$M$について,次は同値である.
	\begin{sakura}
		\item $M$の任意の部分加群は有限生成である.
		\item $M$の任意の部分加群の増大列
		\[N_1\subset N_2\subset\dots\subset N_i\subset\cdots\]
		は必ず停まる.
		\item $M$の部分加群からなる空でない族は,包含に関する極大元を持つ.
	\end{sakura}
\end{prop} 

証明は環の場合を適切に修正すれば良いので省略する.これらの条件を満たす加群をNoether加群という.環$A$を$A$加群とみなすと,~$A$がNoether加群なら$A$はNoether環である.

(ii)の条件を\textbf{昇鎖条件}(ascending chain condition)といい,~ACCと略す.包含の大小を逆にして
\[N_1\supset N_2\supset\dots\supset N_i\supset\cdots\]
が停まるような$M$をArtin加群という.これを\textbf{降鎖条件}(descending chain condition, DCC)といい,~$M$の部分加群の空でない族は包含に関する極小元を持つことと同値.\index{#Artinかぐん@Artin加群}

Noether性,~Artin性は環とは違って部分加群に遺伝する.これは明らかであろう.

\begin{prop}
	短完全列
	\[0\longrightarrow M_1\overset{f}{\longrightarrow}M_2\overset{g}{\longrightarrow}M_3\longrightarrow0\]
	について,~$M_2$がNoether加群であることと,~$M_1$と$M_3$がNoether加群であることは同値.
\end{prop}
\begin{proof}
	\begin{eqv}
		\item 準同型は包含関係を保存するからわかる.
		\item $M_2$の部分加群による増大列
		\[N_1\subset N_2\subset\dots\subset N_i\subset\cdots\]
		を考える.単射$f$によって$M_1$を$M_2$の部分加群とみなし,制限の列
		\[M_1\cap N_1\subset M_1\cap N_2\subset\dots\subset M_1\cap N_i\subset\cdots\]
		と,全射$g$による像の列
		\[g(N_1)\subset g(N_2)\subset\dots\subset g(N_i)\subset\dots\]
		を考えると,~$M_1,M_3$のACCから,ある共通の$n$がとれて,~$i\geq n$に対し
		\begin{equation}\label{eq:短完全列とNoether加群}
			M_1\cap N_n=M_1\cap N_i,g(N_n)=g(N_i)
		\end{equation}
		となる.また,~$M_2/M_1\cong M_3$であり,~$g$がその射影となっていることから
		\[N_i/(M_1\cap N_i)=g(N_i)\]
		であり,~\fref{eq:短完全列とNoether加群}と組み合わせて
		\[N_i/(M_1\cap N_n)=N_i/(M_1\cap N_i)=g(N_i)=g(N_n)=N_n/(M_1\cap N_n)\]
		となり,~$N_i=N_n$である.すなわち$M_2$のACCが導かれる.
	\end{eqv}
\end{proof}

適切に置き換えることでArtin環についても同様の性質が成り立つ.
\begin{prop}
	$A$がNoether環であることと,任意の有限生成$A$加群がNoether的であることは同値.
\end{prop}
\begin{proof}
	($\Longleftarrow$)は明らか.~($\Longrightarrow$)を示す.~$M$が$A$上有限生成な加群であるとすると,ある$n\in\N$に対し,完全列$A^n\overset{f}{\longrightarrow}M\longrightarrow0$がとれる.すると
	\begin{equation}\label{eq:有限生成A加群がNoether的}
		0\longrightarrow\ker f\longrightarrow A^n\overset{f}{\longrightarrow}M\longrightarrow 0
	\end{equation}
	も完全であり,~$A$がNoether加群なので
	\[0\longrightarrow A\longrightarrow A^n\longrightarrow A^{n-1}\longrightarrow0\]
	を帰納的に用いることで$A^n$はNoether加群であることがわかり,~\fref{eq:有限生成A加群がNoether的}と合わせて$M$がNoether加群であることがわかる.
\end{proof}

Noether加群とArtin加群について例を見てみよう.

\begin{ex}
	$\Z$加群として$\Z$はNoether的だが,~Artin的でない.
	
	実際,~$\Z$はPIDだからNoether環なので,~$\Z$加群としてもNoether的.また
	\[2\Z\supset4\Z\supset\dots\supset2^i\Z\supset\cdots\]
	は停止しない減少列をなす.
\end{ex}	
\begin{ex}
	$\Z[1/p]=\mkset{x/p^n}{x\in\Z,n\in\Z_+}$に自然な演算を入れて$\Z$加群とみる.ここで$\Z[1/p]/\Z$を考えると,これはArtin的だがNoether的でない.~$\Z[1/p]$の部分加群の列を考えよう;
	\[\Z\subset\frac{\Z}{p}\subset\frac{\Z}{p^2}\subset\dots\subset\frac{\Z}{p^n}\subset\cdots\tag{$\ast$}\]
	このとき$x/p^n-y/p^n\in\Z$であることは$x-y\in p^n\Z$と同値なので$(\Z/p^n)/\Z=(\Z/p^n\Z)/p^n$となる.このことから$\Z[1/p]/\Z$の部分加群の列;
	\[0\subset\frac{\Z/p\Z}{p}\subset\frac{\Z/p^2\Z}{p^2}\subset\dots\subset\frac{\Z/p^n\Z}{p^n}\subset\cdots\]
	ができ,これは停まらない増大列をなす.よってNoetherではない.
	
	ここで$N$を$\Z[1/p]$の部分加群とする.すると,列($\ast$)のどれか隣り合う項の間に$N$が存在する.ここで$\Z/p^n\subset N\subset\Z/p^{n+1}$としよう.~$\Z$の部分加群は$k\Z$に限るので,~$N=k\Z/p^{n+1}$とかける.もし$k$と$p$が互いに素でないなら分母の次数が退化するので$(k,p)=1$のときに考えれば十分である.
	
	このとき$(k\Z/p^{n+1})/\Z=(\Z/p^{n+1}\Z)/p^{n+1}$を示す.先と同様に$(k\Z/p^{n+1})/\Z=(k\cdot\Z/p^{n+1}\Z)/p^{n+1}$が言える.ここで,次の準同型;
	\[\varphi:\Z/p^{n+1}\Z\to k\cdot\Z/p^{n+1}\Z;\bar{i}\mapsto\bar{ki}\]
	を考えると$(k,p)=1$よりこれは全単射である.よって$k\cdot\Z/p^{n+1}\Z=\Z/p^{n+1}\Z$が言えるので,~$\Z[1/p]/\Z$の部分加群はすべて$(\Z/p^{n+1}\Z)/p^{n+1}$の形をしていることがわかった.ゆえに部分加群の減少列は必ず停止するのでArtin加群である.
\end{ex}	
逆に,環についてはArtin環はNoether環となる.そのことは\ref{sec:Noether環続論}で示している.

Artin環の特殊性を見よう.
\begin{prop}\label{prop:Artinの素イデアル}
	Artin環の素イデアルは極大イデアルであり,それは有限個しかない.
\end{prop}
\begin{proof}
	$A$をArtin環とし,~$ P$を$A$の素イデアルとする.~$\bar{A}=A/ P$とおき,これが体であることを示す.~$0\neq x\in\bar{A}$をとる.すると
	\[0\neq(x)\supset(x^2)\supset\cdots\]
	はイデアルの降鎖列をなし,~$\bar{A}$はArtin的なので,これは停まる.すると,ある$n$が存在して$(x^n)=(x^{n+1})$であるから,ある$y\in\bar{A}$があって$x^n=yx^{n+1}$である.よって$x^n(1-xy)=0$であり,~$\bar{A}$は整域で$x\neq0$なので$xy=1$となり,~$x$は可逆.よって示された.
	
	次に有限個であることを見る.~$\operatorname{Spm}A$を$A$の極大イデアル全体のなす集合とする.
	\[S=\mkset{\ideal{m}_1\cap\dots\cap\ideal{m}_n}{n\in\Z_+,\ideal{m}\in\operatorname{Specm}A}\]
	とおくと,これは$A$のイデアルの族となるので,極小元$I=\ideal{m}_1\cap\dots\cap\ideal{m}_r$を持つ.このとき,任意の$\ideal{m}\in\operatorname{Spm}A$に対し$\ideal{m}\cap I\in S$であるので,~$M\cap I=I$となり$I\subset\ideal{m}$である.ここで,各$1\leq i\leq r$に対し$\ideal{m}_i\not\subset\ideal{m}$であると仮定すると,それぞれ$x_i\in\ideal{m}_i\cap\ideal{m}^c$が存在して,~$\ideal{m}$が素イデアルなので$x_1x_2\dots x_r\not\in\ideal{m}$であるが,~$x_1x_2\dots x_r\in I\subset\ideal{m}$なので,矛盾.よって,少なくとも1つの$1\leq j\leq r$が存在して$\ideal{m}_j\subset\ideal{m}$である.すると$\ideal{m}_j$は極大なので$\ideal{m}=\ideal{m}_j$である.よって,~$\operatorname{Spm}A$は$I$の成分に現れる有限個の素イデアルのみからなる.
\end{proof}

\section{加群の素因子}

\textbf{以下,この章の終わりまで環$A$はNoether環とする.} 素因子,準素イデアルの議論においてはNoether性が本質的に効いてくるため,これらの概念を使った議論にはNoether性が必要である.これはNoether環の仮定が外しにくいことの一因となっている.

\begin{lem}
	有限生成$A$加群$M$に対し, $S^{-1}(\ann M))=\ann(S^{-1}M)$が成り立つ. 
\end{lem}

\begin{proof}
	$M=A^{\oplus n}$とし, $n$についての帰納法を用いる.
	\begin{step}
		\item $n=1$のとき.
		
		$I=\ann M$とおくと, $A/I\cong M$である. \ref{prop:局所化はいろんな操作と可換}から$S^{-1}M=(S^{-1}A)/(S^{-1}I)$だから, $\ann (S^{-1}M)=S^{-1}(I)=S^{-1}(\ann M)$である.
		
		\item $n-1$まで正しいとする.
		
		$M=A^{\oplus n-1}\oplus A$なので,帰納法の仮定と\ref{prop:局所化はいろんな操作と可換}から;
		
		\[\begin{aligned}
		S^{-1}(\ann (M))&=S^{-1}(\ann(A^{\oplus n-1})\cap\ann A)\\
		&=S^{-1}(\ann (A^{\oplus n-1}))\cap S^{-1}(\ann A)\\
		&=\ann (S^{-1}A^{\oplus n-1})\cap\ann(S^{-1}A)\\
		&=\ann (S^{-1}A^{\oplus n-1}+S^{-1}A)\\
		&=\ann (S^{-1}M)  
		\end{aligned}\]
		となり,主張が従う.
	\end{step} 
\end{proof}

\begin{cor}\label{prop:イデアル商は局所化と可換}
	$N,P$を$A$加群$M$の部分加群で$P$が有限生成であるとすると,積閉集合$S$について $S^{-1}(N:P)=(S^{-1}N:S^{-1}P)$が成り立つ.
\end{cor}
\begin{proof}	
	\ref{prop:加群商}より$(N:P)=\ann((N+P)/N)$であって,補題から$S^{-1}(N:P)=\ann ((S^{-1}N+S^{-1}P)/S^{-1}N)=(S^{-1}N:S^{-1}P)$である.	
\end{proof}

\begin{defi}[素因子]\index{そいんし@素因子(加群)}
	$A$加群$M$に対し,~$\ann x=\mkset{a\in A}{ax=0}$が$A$の素イデアルであるとき,これを$M$の\textbf{素因子}(prime ideal associated to $M$)または,~$M$に\textbf{属する素イデアル}(associated prime ideal)という.
	
	それらの全体を
	\[\ass M(=\ass_A M)=\mkset{ P\in\spec A}{\text{ある}x\in M\text{に対して} P=\ann x\text{を満たす.}}\]
	とかく.
\end{defi}

$M$の素因子がしっかり存在することを確認しておこう.

\begin{prop}\label{prop:素因子の存在}
	$\mkset{\ann x}{0\neq x\in M}$の極大元は素イデアル,すなわち$\ass M$の元である.\footnotemark
\end{prop}\footnotetext{$A$はNoether環なので極大条件を満たす.}

\begin{proof}
	$\ann x$が極大元であるとする.~$ab\in\ann x,a\not\in\ann x$と仮定すると,~$ax\neq0$かつ$abx=b(ax)=0$なので$b\in\ann(ax)$である.すると,定義から$\ann x\subset\ann(ax)$であるが,~$\ann x$の極大性より$\ann x=\ann(ax)$である.よって,~$b\in\ann x$となることがわかり,~$\ann x$は素イデアルである.
\end{proof}

\begin{cor}\label{cor:Mneq0ならann Mは空でない}
	$M\neq0$ならば$\ann M\neq\emptyset$である.
\end{cor}

\begin{defi}[台]\index{だい@台(加群)}
	$A$加群$M$の素イデアル$ P$による局所化$M_ P$が0にならない$ P$の集まりを
	\[\supp M=\mkset{ P\in\spec A}{M_ P\neq 0}\]
	とかき,~$M$の\textbf{台}(support)という.
\end{defi}

\begin{lem}\label{lem:M_p=0との同値条件}
	$M_ P=0$であることと,任意の$x\in M$に対して,ある$a\not\in P(a\in S_ P)$が存在して$ax=0$となることは同値.
\end{lem}

\begin{proof}
	明らか.
\end{proof}

\begin{thm}\label{thm:assとsuppの極小元}
	$\ass M\subset\supp M$であり,それぞれの極小元のなす集合は一致する.
\end{thm}

\begin{proof}
	$ P=\ann x$とする.すると,任意の$x\not\in P$に対して$sx\neq0$である.よって$x=x/1$は$M_ P$の零元でない.よって$M_ P\neq0$,すなわち$ P\in\supp M$である.
	
	次に$ P$を$\supp M$の極小元とする.~$ PA_ P$が$\supp  M_ P$の極小元であることを見る.~$ q\subsetneq  PA_ P=\mkset{a/x}{a\in P,x\not\in P}$とする.自然な準同型$f:A\twoheadrightarrow A_ P$による引き戻し$f^{-1}( q)$は$ P$の部分イデアルで,~$M_{f^{-1}( q)}=0$である.ここで,任意の$a/s\in M_ P$に対し,~$a\in M$であるので,~\ref{lem:M_p=0との同値条件}よりある$t\not\in f^{-1}( q)$が存在して$ta=0$である.このとき$f(t)\not\in q$で,~$f(t)\cdot a/s=0/1$となるので,再び補題から$(M_ P)_ q=0$である.
	
	また,~$ PA_ P$は$\spec A_ P$の極大元なので,~$\supp M_ P=\{ PA_ P\}$である.~$M_ P\neq\emptyset$なので
	\[\emptyset\neq\ass_{A_ P}M_ P\subset\supp M_ P=\{ PA_ P\}\]
	となり,~$\ass_{A_ P}M_ P=\{ PA_ P\}$がわかる.ここで$ PA_ P=\ann_{A_ P}x/s$とかこう.すると,明らかに$ P=\ann_A x/s$である.これは$ P\in\ass_A M_ P$を意味する.~$ P\in\ass_A M$を示せば証明は完了である.いま$A$はNoether環なので$ P$は有限生成である.~$ P=(a_1,\dots,a_n)$とすると,各$a_i$に対して$a_i\cdot x/s=0$なので,~$h_i\not\in P$がとれて$a_ih_ix=0$である.ここで$h=h_1h_2\cdots h_n$とおくと,各$i$について$a_ihx=0$であるので$ P\subset\ann_A hx$である.また,~$h_i\not\in P$より$h\not\in P$であるから$\ann_Ahx\subset\ann_Ax/s$が従う.よって$ P=\ann_Ahx\in\ass_AM$となり,証明が完了した.
\end{proof}

Zariski位相(\ref{defi:Zariski位相})を思い出そう.
\begin{prop}\label{prop:Zariskiの閉集合とsupp}
	$A$加群$M$が$A$上有限生成ならば,~$V(\ann M)=\supp M$である.
\end{prop}
\begin{proof}
	\ref{lem:M_p=0との同値条件}より,~$M_ P\neq0$と,ある$x\in M$が存在して,任意の$a\not\in P$に対し$ax\neq0$となることが同値なので,これと$ P\in V(\ann M)$が同値であることを示す.
	
	$ P\in V(\ann M)$ならば$\ann M\subset P$なので,~$a\not\in P$ならば$a\not\in\ann M$,~すなわち任意の$x\in M$に対して$ax\neq0$である.逆も全く同様であるので,示された.
\end{proof}
\begin{cor}\label{cor:supp A/I}
	$A$のイデアル$I$に対して,~$\supp A/I=V(I)$である.
\end{cor}

\begin{defi}\index{うめこまれたそいんし@埋め込まれた素因子}
	極小な$\ass M$の元を\textbf{孤立}(isolated)素因子といい,そうでないものを\textbf{埋め込まれた}(embedded)素因子という.
\end{defi}

ここで,埋め込まれた素因子のイメージをつかもう.

\[\sqrt{I}=\bigcap_{P\in V(I)} P=\bigcap_{I\subset P:\text{極小}} P\]
であるから
\[V(I)=V(\sqrt{I})=\bigcup_{I\subset P:\text{極小}} V( P)\]
となる.いずれも包含は簡単に確かめられる.これを既約分解という.ここで,このような$ P$は\ref{cor:supp A/I}より$\supp A/I$の極小元であり,それは$\ass A/I$の極小元である.よって$\ass A/I$の埋め込まれた素因子$ q\supseteq P$は$V( q)\subset V( P)$なので,~$V(I)$の分解には現れない.これが埋め込まれていることのイメージである.

\begin{prop}\label{prop:完全列とass}
	$A$加群の完全列
	\[0\longrightarrow M_1\longrightarrow M_2\longrightarrow M_3\longrightarrow0\]
	に対して,~$\ass M_1\subset\ass M_2$であり,~$\ass M_2\subset \ass M_1\cup\ass M_3$が成立する.
\end{prop}
\begin{proof}
	$ P\in\ass M_1$とすると,単射$A/ P\hookrightarrow M_1\hookrightarrow M_2$が存在するから,~$ P\in\ass M_2$がわかる.
	
	次に,~$ P\in\ass M_2$とし,~$N$を$A/ P$と同型な$M_2$の部分加群とする.~$x_0\in M_2$をとり,~$ P=\ann x_0$とすると,~$N$は単射$A/ P\hookrightarrow M_2;a+\ann x_0\longmapsto ax_0$の像だから,~$N=\mkset{ax_0}{a\in A}$とかける.~$M_1$から$M_2$への単射を$\varphi$としよう.
	
	\begin{sakura}
		\item $\varphi(M_1)\cap N\neq0$のとき.
		
		0でない任意の$x\in \varphi(M_1)\cap N$を1つとる.~$x\in N$より,ある$a\in A$が存在して$x=ax_0$とかける.ここで$\ann ax_0=\ann x_0= P$を示す.~$\ann x_0\subset\ann ax_0$は明らかである.~$b\in\ann ax_0$とすると,~$ab\in\ann x_0$であり,~$ax_0=x\neq0$なので$a\not\in\ann x_0$だから$\ann x_0$は素イデアルなので$b\in\ann x_0$である.よって$ P=\ann x_0$である.また$x\in \varphi(M_1)$なので$x_1\in M_1$を用いて$x=\varphi(x_1)$とかける.すると明らかに$\ann \varphi(x_1)=\ann x_1$であるので$p\in\ass M_1$である.
		
		\item $\varphi(M_1)\cap N=0$のとき.
		
		$N\subset M_2/M_1=M_3$より,~$ P\in\ann M_3$である.
	\end{sakura}
\end{proof}
\begin{thm}
	有限生成$A$加群の素因子の集合$\ass M$は有限である.
\end{thm}
\begin{proof}
	$M=0$ならば$\ann M=\emptyset$なので$M\neq0$の時を考えればよい.~$ P_1\in\ann M$
	とすると,~$A/ P_1$と同型な$M$の部分加群$M_1$が存在する.~$M_1\neq M$のとき,~$M/M_1\neq0$だから,~$ P_2\in\ass M/M_1$が存在して,~$A/ P_2\cong\bar{M_2}\subset M/M_1$とできる.~$M_2$を$M_2/M_1=\bar{M_2}$なる$M$の部分加群とすると,~$M_1\subset M_2\subset M$となる.同様に,~$M_2\neq M$ならば操作を続けると,部分加群の増大列
	\[0\neq M_1\subset M_2\subset\dots\subset M_i\subset\cdots\]
	と$M_i/M_{i-1}\cong A/ P_i$となる素イデアル$ P_i$がとれる.~$M$はNoether環上有限生成なので,この増大列は停まる.よって,ある$n$が存在して$M_n=M$としてよい.すると,短完全列
	\[0\longrightarrow M_1\longrightarrow M\longrightarrow M/M_1\longrightarrow0\]
	に対し,~\ref{prop:完全列とass}から$\ass M\subset \ass M_1\cup \ass M/M_1$となる.また
	\[0\longrightarrow \bar{M_2}=M_2/M_1\longrightarrow M/M_1\longrightarrow (M/M_1)/(M_2/M_1)=M/M_2\longrightarrow0\]
	に対して\ref{prop:完全列とass}を用いて,~$\ass M/M_1\subset \ass M_2/M_1\cup \ass M/M_2$となる.以下同様に続けると
	\[\begin{aligned}
		\ass M&\subset\ass M_1\cup \ass M/M_1\\
		&\subset \ass M_1\cup \ass M_2/M_1\cup \ass M/M_2\\
		&\subset\dots\\
		&\subset\ass M_1\cup\ass M_2/M_1\cup\dots\cup\ass M_n/M_{n-1}\\
		&=\bigcup_{i=1}^n\ass A/ P_i\\
		&=\{\nitem{ P}\}
	\end{aligned}\]
	である.最後の等号は,任意の0でない$x\in A/ P_i$に対し$\ann x= P_i$であることを用いた.
\end{proof}
\section{準素イデアル}
\begin{defi}[準素イデアル]
	環$A$のイデアル$ q\neq0$に対し,~$ab\in q$かつ$a\not\in q$のとき,ある$n\in\Z_+$が存在して$b^n\in q$となるとき,~$ q$を\textbf{準素イデアル}(primary ideal)という.\index{じゅんそいである@準素イデアル}
\end{defi}

明らかに素イデアルは準素である.この条件は;
\[A/q\text{のすべての零因子は冪零である.}\]
と同値であることに注意しよう.

\begin{prop}
	準素イデアル$ q$の根基は$ q$を含む最小の素イデアルである.
\end{prop}

\begin{proof}
	イデアルの根基はそのイデアルを含む全ての素イデアルの共通部分であるから,素イデアルであることを示せば十分. $ab\in\sqrt{ q},a\not\in\sqrt{ q}$とする.~$ab\in\sqrt{ q}$なので,ある$n$が存在して$a^nb^n\in q$である.~$a\not\in\sqrt{ q}$だから,この$n$に対して$a^n\not\in q$である.~$ q$が準素なので,ある$m$が存在して$b^{nm}\in q$である.よって$b\in\sqrt{ q}$である.
\end{proof}

$\sqrt{ q}= P$であるとき, $ q$は$ P$準素であるという.この記号は準素加群の定義(\ref{defi:準素加群})と整合性がある(\ref{cor:準素イデアルと準素加群の整合性}).

\begin{prop}\label{prop:sqrt{I}が極大なら準素}
	環$A$のイデアル$I$について, $\sqrt{I}$が極大なら$I$は準素である.
\end{prop}

\begin{proof}
	$I$が素のときは示すことはないので,素でないとしてよい. $\sqrt{I}=\ideal{m}$とおく. $I$を含む$A$の素イデアルは$\ideal{m}$のみなので, $\spec A/I=\{\bar{\ideal{m}}\}$となる.すると$\nil A/I=\bar{\ideal{m}}$となり,任意の$x\in A/I$について$x\in\nil A/I$または$x\in (A/I)^\times$が成り立つ.よって$I$は準素である.
\end{proof}

\begin{prop}
	$ q$が$ P$準素イデアルなら, $ qA_ P\cap A= q$である.
\end{prop}

\begin{proof}
	$a/s\in  qA_{ P}$が$x\in A$を用いて$a/s=x/1$となっているとすると,ある$h\not\in P$がとれて$ha=xhs$が成り立つ.ここで$a\in q$より$xhs\in q$で, $hs\not\in P=\sqrt{q}$なので$x\in q$でなければならない.
\end{proof}

\begin{defi}\label{defi:準素加群}
	$A$加群$M$の部分加群$N$について,~$\ass M/N$が一点集合$\{ P\}$になっているとき,~$N$を$ P$\textbf{準素}($ P$primary)という.
\end{defi}

\begin{prop}\label{prop:準素加群の同値条件}
	Noether環$A$上の加群$M$について,次は同値.
	\begin{sakura}
		\item 0が$M$の準素部分加群である,すなわち$\ass M$は一点である.
		\item $M\neq0$で,~$a$が$M$の零因子ならば,任意の$y\in M$に対して$a^ny=0$となる$n\in\Z_+$がある.~(このことを$a$は\textbf{局所的に冪零}であるという.\index{きょくしょてきにべきれい@局所的に冪零}
	\end{sakura}
\end{prop}
\begin{proof}
	\begin{eqv}
		\item $\ass M=\{ P_0\}$とおく.すると,~\ref{prop:素因子の存在}より,~$ P_0$は任意の$x\in M$に対し$\ann x$の極大元であるので
		\[\bigcup_{x\neq0}\ann x= P_0\]である.また,~$\bigcup_{x\neq0}\ann x$は$M$の零因子全体のなる集合と等しいので,~$a\in P_0$とできる.ここで,零でない$y\in M$を任意にとり,~$Ay\neq0$を部分$A$加群としてみると,~$\ass Ay\neq\emptyset$であり$Ay\subset M$なので\ref{prop:完全列とass}から$\ass Ay=\{ P_0\}$である.すると,~$ P_0$は\ref{thm:assとsuppの極小元}より$\supp Ay$の唯一の極小元である.また,~\ref{prop:Zariskiの閉集合とsupp}より$V(\ann Ay)=\supp Ay$であるので,~$V(\ann Ay)$の極小元は$ P_0$のみだから
		\[\sqrt{\ann Ay}=\bigcap_{ P\in V(\ann Ay)} P= P_0\]
		である.よって,~$a\in P_0$だったから,ある$n\in\Z_+$に対し$a^n\in\ann Ay$である.よって,~$a^n y=0$であることがわかる.
		\item $M$上局所冪零な元全体を$I$とおく.仮定より$I$は$M$の零因子全体の集合と一致する.すると$I=\bigcup_{x\neq0}\ann x$であるから,~$I$は$M$の素因子を含む.~$ P$を$M$の素因子とする.~$I\subset  P$を示せばよい.~$ P=\ann x_0$とする.~$a\in I$ならば,ある$n$が存在して$a^nx_0=0$である.よって~$a^n\in P$であり,~$ P$が素なので$a\in P$である.すなわち$I\subset P$であり,~$I$のみが$M$の素因子である.
	\end{eqv}
\end{proof}
\begin{cor}\label{cor:準素イデアルと準素加群の整合性}
	$A$のイデアル$ q$について,~$q$が準素イデアルであることと,~$q$が$A$の準素部分加群であることは同値.
\end{cor}
\begin{proof}
	\begin{eqv}
		\item $a$を$M$の零因子とする.ある$0\neq x+ q\in A/ q$に対して$a(a+ q)=0,~$すなわち$ax\in q$である.~$x\not\in q$より,~$ q$が準素かつ$x\not\in q$だから,ある$n\in\Z_+$があって$a^n\in q$である.ゆえに任意の$y\in A$に対し$a^n y\in q$である.よって$a^n(y+ q)=0$だから\ref{prop:準素加群の同値条件}より,~$\ass A/ q$は一点である.すなわち$ q$は$A$の準素部分加群である.
		\item $ab\in q$とし,~$b\not\in q$とする.すると$A/ q$において$b+ q\neq0$であって$a$は$b+ q$を零化するので$a$は$A/ q$の零因子となる.\ref{prop:準素加群の同値条件}から$a$は局所冪零で,特に$1+ q\in A/ q$に対して$a^n(1+ q)=0$となる$n\in\Z_+$がある.よって$a^n\in q$である.
	\end{eqv}
\end{proof}

\begin{defi}[準素分解]\index{じゅんそぶんかい@準素分解}
	$M$を$A$加群とする.~$M$の部分加群$N$を,~$M$の有限個の部分加群の交わりとして
	\[N=N_1\cap\dots\cap N_r\]
	と表すことを$N$の\textbf{分解}(decomposition)という.各$N_i$が既約なら\textbf{既約分解},準素なら\textbf{準素分解}という.
\end{defi}

分解$N=\bigcap_{i}^r N_i$に対して,各$1\leq j\leq r$について$N\neq \bigcap_{i\neq j}N_i$であるとき,この分解は\textbf{むだがない}(irredundant)という.\index{むだがない}

次の定理を当面の目標としよう.その後,分解の一意性について考察していく.

\begin{thm}[Laker-Noetherの分解定理]\index{#Laker-Noehterのぶんかいていり@Laker-Noetherの分解定理}\label{thm:Laker-Noetherの分解定理}
	Noether環$A$上の有限生成加群$M$の任意の部分加群$N$は準素分解を持つ.特に,任意の$A$のイデアル$I$は準素分解を持つ.
\end{thm}

\begin{defi}
	$A$加群$M$の部分加群$N$について,~$N_1,N_2$を部分加群として$N=N_1\cap N_2$ならば$N=N_1$または$N=N_2$が成り立つとき,~$N$を\textbf{既約}といい,そうでないときに\textbf{可約}という.~(\cite{matsu},~p.51)\index{きやくぶぶんかぐん@既約(部分加群)}\index{かやくぶぶんかぐん@可約(部分加群)}
\end{defi}

\begin{lem}\label{lem:既約分解できる}
	$M$がNoether加群ならば,任意の部分加群は既約部分加群の有限個の交わりとしてかける.
\end{lem}
\begin{proof}
	そのように分解できない部分加群の集まりを$S$とする.~$M$のNoether性より$S$の極大元がとれるので,それを$N$とする.~$N$は可約なので,~$N_1,N_2\neq N$を用いて$N=N_1\cap N_2$とできる.すると$N\subset N
	_1,N_2$なので,~$N$の極大性から$N_1,N_2\not\in S$であるから,~$N_1$と$N_2$は既約な部分加群の交わりでかける.よって,~$N$も既約な部分加群の有限個の交わりとなり,矛盾.
\end{proof}
\begin{lem}\label{lem:既約なら準素}
	既約な真部分加群は準素である.
\end{lem}
\begin{proof}
	対偶である,~$N\subsetneq M$が準素でなければ可約であることを示す.~$\ass M/N$は少なくとも2つの異なる素因子を持つので,それを$ P_1\neq P_2$とする.~$A/ P_i\cong\bar{N_i}\subset M/N$とすると,~$0\neq x\in\bar{N_i}$ならば$\ann x= P_i$となるので,~$\bar{N_1}\cap\bar{N_2}=0$でなければならない.さて,~$N_i$を$N_i/N=\bar{N_i}$となるようにとると,自然な全射$\pi:M\twoheadrightarrow M/N$に対し$N_1\cap N_2=\pi^{-1}(\bar{N_1}\cap\bar{N_2})=\pi^{-1}(0)=N$であって,~$N\subsetneq N_i$なので,~$N$は可約である.
\end{proof}

2つの補題により\ref{thm:Laker-Noetherの分解定理}が示された.

では,分解の一意性について見ていこう.素イデアル分解のように綺麗には行かず,埋め込まれた素因子がややこしい影響を及ぼしてくる.
\begin{lem}
	$N_1$と$N_2$が$M$の$ P-$準素部分加群ならば,~$N_1\cap N_2$も$ P-$準素である.
\end{lem}
\begin{proof}
	\[\iota:M/(N_1\cap N_2)\longrightarrow M/N_1\oplus M/N_2;x+N_1\cap N_2\longmapsto(x+N_1,x+N_2)\]
	は単射であるので,短完全列
	\[0\longrightarrow M/N_1\longrightarrow M/N_1\oplus M/N_2\longrightarrow M/N_2\longrightarrow0\]
	とあわせて,~\ref{prop:完全列とass}から$\emptyset\neq\ass M/(N_1\cap N_2)\subset\ass M/N_1\cup\ann M/N_2=\{ P\}$であるから$N_1\cap N_2$は$ P-$準素.
\end{proof}

これより,準素分解$N=\bigcap_i^r N_i$がむだのない分解であるとき,~$N_{i_1}$と$N_{i_2}$がともに$ P-$準素なら$N_{i_1}\cap N_{i_2}$も$ P-$準素なので,~$N_j=N_{i_1}\cap N_{i_2}$とおくと,分解の長さを短くできる.このように,すべての$i$に対し$\ass M/N_i$が異なるようにすることで\textbf{最短準素分解}が得られる.\index{さいたんじゅんそぶんかい@最短準素分解}
\begin{thm}
	Noether環上の加群$M$の真部分加群$N$について$N=N_1\cap\dots\cap N_r$を無駄のない準素分解とし,~$ P_i$を$N_i$の素因子とすると$\ass M/N=\{\nitem[r]{ P}\}$となる.
\end{thm}
\begin{proof}
	埋め込み$M/N\subset\midoplus_{i=1}^rM/N_i$と,\ref{prop:完全列とass}により$\ass M/N\subset \bigcup_{i=1}^r\ass M/N_i=\{\nitem[r]{ P}\}$を得る.また,むだのないことから$0\neq\bigcap_{i=2}^r N_i/N$であり
	\[\iota:\bigcap_{i=2}^rN_i/N\longrightarrow M/N_i;x+N\longmapsto x+N_i\]
	は単射.実際,~$\iota(x+N)=\iota(y+N)$とすると,~$x-y\in N_1$である.一方,~$x,y\in\bigcap_{i=2}^rN_i$より$x-y\in\bigcap_{i=1}^rN_i=N$となり,~$x+N=y+N$がわかる.よって,~$\emptyset\neq\ass(\bigcap_{i=2}^r N_i/N)\subset\ass M/N_1=\{ P_1\}$すなわち$\ass(\bigcap_{i=2}^r N_i/N)=\{ P_1\}$である.また,~$\bigcap_{i=2}^r N_i/N\subset M/N$でもあるので,~$ P_1\in\ass M/N$である.他の$ P_i$についても同様.
\end{proof}
\begin{thm}
	$M$を有限生成なNoether環上の加群とする.~$N=N_1\cap\dots\cap N_r$をむだのない最短準素分解とする.このとき,~$ P_i$が孤立素因子なら,~$f_{ P_i}:M\longrightarrow M_{ P_i}$を$ P_i$における局所化とすると,~$N_i=f^{-1}_{ P_i}(N_{ P_i})$となり,~$N_{ P_i}$は$N$と$ P_i$から一意に定まる.
\end{thm}
\begin{proof}
	$ P_1$が極小であるときに示す.~$ P_1= P$とする.すると,~$i\neq1$に対して$N_i$が準素だから,~$\ass M/N_i=\{ P_i\}$であり,~$M$が有限生成なので$M/N_i$もそうである.このとき$\sqrt{\ann M/N_i}= P_i$が成立する.まずこれを示す.
	\begin{step}
		\item $\sqrt{\ann M/N_i}\subset P_i$であること.任意の$x\in\sqrt{\ann M/N_i}$を1つとる.すると,ある$n\in\Z_+$があって$x^n\in\ann M/N_i$であるので,~$x^n$は$M/N_i$の零因子であるので$x^n\in P_i$である.これは素なので,~$x\in P_i$である.
		
		\item その逆を示す.任意の$x\in P_i$を1つとり,~$M/N_i$の生成系を$(\nitem[k]{u})$とする.~$x$は零因子だから局所冪零なので,各$u_i$に対し$x^{n_i}u_i=0$となる$n_i\in\Z_+$がある.よって,それらの最大元を$n$とすると,~$x^n(M/N_i)=0$であるので,~$x\in\sqrt{\ann M/N_i}$である.
	\end{step}
	
	さて,~$ P$が極小なので$ P_i\not\subset P$である.よって,~$y\not\in P$かつ$y\in P_i$となる$y\in A$が存在する.すると$(M/N_i)_ P$は$A_ P$加群で,~$y\in A_ P$は可逆.また,~$y\in P_i=\sqrt{\ann M/N_i}$だから,ある$n$があって$y^n(M/N_i)=0$である.よって$y^n(M/N_i)_ P=0$で,~$y$は可逆なので$(M/N_i)_ P=0$である.すると,局所化は平坦だから,完全列
	\[0\longrightarrow N_i\longrightarrow M\longrightarrow M/N_i\longrightarrow0\]
	に対し
	\[0\longrightarrow (N_i)_ P\longrightarrow M_ P\longrightarrow (M/N_i)_ P=0\longrightarrow0\]
	も完全であるので$(N_i)_ P=M_ P$である.ここで$N_ P=\bigcap_i^r (N_i)_ P$であるから,~$N_ P=(N_1)_ P$であるので$f^{-1}_ P(N_ P)=f^{-1}_ P((N_1)_ P)=N_1$であることがわかった.
\end{proof}

これをイデアルに関して述べ返すと,次のようになる.
\begin{cor}\label{cor:イデアルの準素分解}
	Noether環$A$上のイデアル$I$は,有限個の準素分解$I= q_1\cap\dots\cap q_r$を持つ.この分解に無駄がなければ$P_i=\sqrt{ q_i}$は素であって,~$\ass A/I=\{\nitem[r]{P}\}$である.さらに,これが最短ならば極小素因子$P_i$に対応する準素イデアル$ q_i$は$I$と$ P_i$から一意に定まる.
\end{cor}

これより, $P\in\ass A/I$であることと, $I$の無駄のない準素分解$I=\bigcap q_i$において$\sqrt{q_i}=P$となる$i$が存在することは同値である.そこで,次のように定義しておく.

\begin{defi}[随伴素イデアル]\index{ずいはんそいである@随伴素イデアル}
	$P\in\spec A$について$P\in\ass A/I$であるとき, $P$は$I$の\textbf{随伴素イデアル}(associated prime ideal)という. $I$に\textbf{属する素イデアル}\index{ぞくするそいである@属する素イデアル}(prime ideal belong to $I$)ともいうことがある.
\end{defi}

\ref{thm:assとsuppの極小元}より,随伴素イデアルのなかで極小なものは$\supp A/I=V(I)$で極小なものなので, $I$の極小素イデアルは$I$の随伴素イデアルであることに注意する.そこで$A$加群$M$については$\Min _AM=\mkset{P\in\supp_A M}{P\text{は}\supp_A M\text{で極小である}}$と定義すると;
\[\Min_A A/I=\mkset{P\in V(I)}{P\text{は}V(I)\text{で極小である}}=\mkset{P\in\spec A}{P\text{は}I\text{の随伴素イデアルのなかで極小}}\]
となる.

これらから次が従う.

\begin{prop}\label{prop:Noether環の極小素イデアルは有限個}
	$A$をNoether環とすると,イデアル$I$の極小素イデアルは有限個である.
\end{prop}
\section{Noether環続論}

\label{sec:Noether環続論}

Noether環がそれなりに広いクラスをなすこと,~Artin環は特殊なNoether環であることなどを見よう.
\begin{defi}[単純]\index{たんじゅん@単純(加群)}
	$A$加群$M\neq0$が,~$M$と0以外に部分加群を持たないとき,\textbf{単純}(simple)であるという.
\end{defi}

加群を群,部分加群を正規部分群でおきかえたとき,そのような群を単純群という.
\begin{defi}
	$A$加群$M$の部分加群の列
	\[M=M_0\supset M_1\supset\dots\supset M_r=0\]
	が,各$M_i/M_{i+1}$が単純であるとき,この列を$M$の組成列という.
\end{defi}

組成列については,~Jordan-H\uml{o}lderの定理という大定理が知られている.
\begin{thm}[Jordan-H\uml{o}lder]\index{#Jordan-Holder@Jordan-H\uml{o}lderの定理}
	加群$M$の任意の2つの組成列は長さが等しく,各成分も順序と同系の違いを除いて等しい.
\end{thm}

ここでは,その一部分である長さについてのみ示す.
\begin{prop}
	$M$が組成列を持てば,その長さは一定である.
\end{prop}
\begin{proof}
	$M$の組成列の長さの最小値を$\ell(M)$とする.もし,~$M$が組成列を持たないなら$\ell(M)=\infty$とする.
	\begin{step}
		\item $M$の真の部分加群$N$に対し,~$\ell(N)<\ell(M)$であること.
		
		$M$の組成列$M=M_0\supset M_1\supset\dots\supset M_r=0$に対し,~$N_i=M_i\cap N$とおく.
		\[N_i/N_{i+1}=(M_i\cap N_i)/(M_{i+1}\cap N)\subset M_i/M_{i+1}\]
		より,~$M_i/M_{i+1}$が単純だから$N_i/N_{i+1}=0$または$N_i/N_{i+1}=M_i/M_{i+1}$である.すなわち$N_i$の重複を除けば,それは$N$の組成列となる.よって,~$\ell(N)\leq\ell(M)$である.また,~$\ell(N)=\ell(M)$とすると,任意の$i$に対し$N_i\neq N_{i+1}$であるから$N_i/N_{i+1}=M_i/M_{i+1}$となるので,帰納的に$N_i=M_i$を得る.よって$N=M$である.
		
		\item 
		$M={M_0}'\supset{M_1}'\supset\dots\supset{M_k}'=0$を$M$の組成列とする.~Step1より$\ell(M)>\ell({M_1}')>\dots>\ell({M_k}')=0$より$\ell(M)\geq k$である.よって,~$\ell(M)$の定義から,~$\ell(M)=k$となるので,任意の組成列の長さは等しい.
	\end{step}
\end{proof}
\begin{prop}\label{prop:有限な組成列の同値条件}
	$\ell(M)$が有限であることと,~ArtinかつNoether的であることは同値.
\end{prop}
\begin{proof}
	\begin{eqv}
		\item 
		$M_1\subset M_2\subset\dots\subset M_i\subset\cdots$を$M$の増大列とする.各$M_i$について$\ell(M_i)<\ell(M_{i+1})<\ell(M)$で,~$\ell(M)$が有限なので,これは必ず停まる.減少列は明らかである.
		\item 
		$M_0=M$とし,~$M$の真部分加群の中で極大なものを$M_1$とする.同様に$M_i$を$M_{i-1}$の真部分加群で極大なものとしてとる.~$M$のDCCから,これは必ずある$n$で$M_n=0$となり停まる.ここで
		\[M=M_0\supset M_1\supset\dots\supset M_i\supset\dots\supset M_n=0\]
		は組成列.実際,~$N\subset M_i/M_{i+1}$がとれるとすると$M_{i+1}\subset N\oplus M_{i+1}\subset M_i$となり,~$N=0$または$N=M_i/M_{i+1}$である.よって,~$M_i/M_{i+1}$は単純である.
	\end{eqv}
\end{proof}
\begin{prop}\label{prop:Artin的なベクトル空間はNoether}
	体$K$上のベクトル空間$V$について, Artin的であることとNoether的であることは同値である.
\end{prop}
\begin{proof}
	まず,~$V$が有限次元であると仮定すると,~$V$の基底$\{\nitem{u}\}$に対して$V_n=\sum Ku_i$とすると,これは組成列をなすから,\ref{prop:有限な組成列の同値条件}より,~$V$はArtinかつNoether的である.よって,~Noether的(resp.Artin的)ならば$V$が有限次元であることを示せば,~$V$はArtin的(resp.Noether的)であることが従う.
	
	背理法で示す.~$V$が無限次元であるとすると, $V$の一次独立な元の無限列$\{x_i\}_{i\in\N}$が存在する.このとき,~$U_n=(x_1,x_2,\dots,x_n),V_n=(x_{n+1},x_{n+2},\dots)$によって生成される部分空間とすると,それぞれ無限な真増加列と減少列をなすので,~$V$がArtin的(Noether的)であることに矛盾.よって示された.
\end{proof}
\begin{thm}[秋月,1935]\label{thm:秋月}
	Artin環はNoether環である.
\end{thm}
\begin{proof}
	$A$をArtin環とする.まず,~$A$の極大イデアルは有限個であった(\ref{prop:Artinの素イデアル}).~$ P_1,\dots, P_r$を$A$のすべての極大イデアルとする.~$I= P_1 P_2\dots P_r=\rad(A)$とおく.~DCCより$I\supset I^2\supset\cdots$は有限で停まるので,ある$s$がとれて$I^s=I^{s+1}$となる.~$J=\ann(I^s)$とおくと	
	\[(J:I)=((0:I^s):I)=(0:I^{s+1})=J\]
	である.ここで$J=A$を示す.
	
	$J\neq A$と仮定する.すると,~$J$より真に大きいイデアルのなかで極小な$J'$がとれる.ここで$x\in J'-J$とすると$Ax\subset J'$であり,~$Ax+J$は$J$より真に大きいイデアルとなる.よって$J'$の作り方から$J'=Ax+J$である.いま,~$(J:I)=J$より$IJ=J$である.~$J'\neq 0$なので,~NAKから$IJ'=Ix+IJ=Ix+J\neq J$となり,~$J\subset IJ'\subsetneq J'$より,極小性から$Ix+J=J$である.ゆえに$Ix\subset J$であるが,これは$x\in(J:I)=J$を意味し,~$x\in J'-J$に矛盾.よって$J=A$である.
	
	ここで,イデアルの減少列;\displaystar
	\[\label{eq:秋月}
	A\supset P_1\supset P_1 P_2\supset\dots\supset P_1 P_2\dots P_{r-1}\supset I\supset I P_1\supset\dots\supset I^2\supset\dots\supset I^s=0\]
	を考える.短完全列
	\[0\longrightarrow P_1\longrightarrow A\longrightarrow A/ P_1\longrightarrow 0\]
	を考えると,~$A$がArtin的なので$A/ P_1$と$ P_1$もArtin的.同様に,~(\ref{eq:秋月})の隣り合う2項をそれぞれ$M,M P_1$とすると$M/M P_1$がArtin的で,体$A/ P_1$上のベクトル空間なので,~\ref{prop:Artin的なベクトル空間はNoether}より$M/M P_1$はNoether的.すると,特に$M=I^{s-1} P_1 P_2\dots P_{r-1}$に対し$M/M P_r=M$がNoether的なので,帰納的に$A$もNoether的であることがわかる.
\end{proof}

可換環について示したのは秋月(1935)であり,~Hopkins(1939)は非可換環に対して証明した.加群については,秋月-Hopkins-Levitzkiの定理などが知られている(前述のように,~Artin的だがNoether的でない加群は存在するが,ある環上の加群ではACCとDCCが同値であることを示している).

\begin{prop}\label{prop:faithfulでNoetherな加群があればNoether環}
	忠実かつNoether的な加群を持つ環はNoether環である.
\end{prop}
\begin{proof}
	$M$を忠実なNoether加群とする.特に$M$は有限生成で,ある$n\in\N$が存在して$M=Ax_1+\dots+Ax_n$とかける.すると
	\[\varphi:A\longrightarrow M^n;a\longmapsto(ax_1,\dots,ax_n)\]
	は準同型で,~$\ker\varphi=\ann(M)$であるので,~$\ker\varphi=0$となり,~$\varphi$は単射.よって$A$はNoether加群$M^n$の部分加群と同型なので,~$A$はNoether加群である.
\end{proof}
\begin{thm}[\citealp{formanek}]
	$A$を環とし,~$B$を有限生成かつ忠実な$A$加群とする.~$A$のイデアル$I$に対し,~$IB$の形の部分加群のなす集合が極大条件を満足すれば,~$A$はNoether環である.
\end{thm}
\begin{proof}
	背理法で示す.~$B$がNoether的でないとする.すると
	\[0\in\Sigma=\mkset{IB}{I\text{は}A\text{のイデアルで,~}B/IB\text{はNoether的でない}}\neq\emptyset\]
	より,仮定から極大元$IB$がとれる.~$\bar{B}=B/IB,\bar{A}=A/\ann(\bar{B})$とすると,対応定理から0でない$\bar{A}$の任意のイデアル$\bar{J}$に対し$\bar{B}/\bar{J}\bar{B}$は$\bar{A}$加群としてNoether的である($\bar{J}$は$\bar{A}$のイデアルなので$I\subset J$より$B/JB$がNoether).また,~$\bar{A},\bar{B}$は定理の仮定を満たす($\bar{A},\bar{B}$は$\Sigma$の極大元を$0$になるように潰したものと考えることができる).
	
	次に,~$\Gamma=\mkset{N}{N\text{は}\bar{B}\text{の部分加群,~}\bar{B}/N\text{は忠実}}$とおくと,~$\bar{B}/N$が忠実であることは,~$\bar{B}$の生成系$\{\nitem{u}\}$に対し,任意の0でない$x\in A$が$xu_i\not\in N$を満たすことであるので,~$\Gamma$は帰納的順序集合をなす.~Zornの補題から極大元$N_0$がとれる.ここで,~$\bar{B}/N_0$がNoether的であるとすると,~\ref{prop:faithfulでNoetherな加群があればNoether環}から$\bar{A}$はNoether的となる.よって$\bar{B}$は有限生成だからNoether的となり,仮定に反するので$\bar{B}/N_0$はNoether的でない.~$\bar{B}'=\bar{B}/N_0$とおくと,次の性質を満たす(これも$\Gamma$の極大元を0にすることに相当する).
	\begin{sakura}
		\item $\bar{B}'$は$\bar{A}$加群としてNoether的でない.
		\item $I$が0でない$\bar{A}$のイデアルなら,~$\bar{B}'/I\bar{B}'$はNoether的.
		\item $N$が0でない$\bar{B}'$の部分加群なら,~$\bar{B}'/N$は$A$加群として忠実でない.
	\end{sakura}
	
	ここで,~$N$を任意の$\bar{B}'$の部分加群とする.~(iii)より,ある0でない$x\in\bar{A}$が存在して$x(\bar{B}'/N)=0,~$すなわち$x\bar{B}'\subset N$である.すると,~(ii)より$\bar{B}'/x\bar{B}'$はNoether的だから,その部分加群$N/x\bar{B}'$は有限生成で,~$x\bar{B}'$も有限生成なので$N$も有限生成.よって$\bar{B}'$はNoether加群となるが,これは(i)に矛盾している.
\end{proof}
\begin{cor}[Eakin-永田,1968]\index{#Eakinながた@Eakin-永田の定理}
	$B$はNoether環,~$A$をその部分加群とする.~$B$が$A$上有限生成なら,~$A$はNoether環である.
\end{cor}
\begin{cor}[\citealp{bjork}]
	\begin{sakura}
		\item $B$を右イデアルについてACCが成り立つ非可換環,~$A$を$B$の可換な部分加群とする.~$B$が左$A$加群として有限生成なら,~$A$はNoether環である.
		\item $B$を両側イデアルについて極大条件をみたす非可換環,~$A$を$B$の中心に含まれる部分環とする.~$B$が$A$加群として有限生成ならば,~$A$はNoether環である.
	\end{sakura}
\end{cor}
 %Noether
	\newpage
\part[Integral extension and The elemnts of dimension theory]{整拡大と次元論初歩}\label{part:整拡大と次元論初歩}

代数曲線,特に古典的な代数多様体論では,曲線をより単純な曲線,特に直線に射影して性質を見る,という方法をとる.それは環の準同型としては$K[X]\to K[X,Y]/I$に対応し,これは\textbf{整拡大}というものになっている.また,多様体(Variety)と環を対応させて調べるという発送において,特に議論の対象となる環は代数閉体$K$上の有限生成代数であることが多い.そこで,一般の環についてある種の\textbf{次元}を定義するとともに,それが代数多様体の自然な次元と一致することを確認することを目標にしよう.すなわち,多項式環$K[\nitem{X}]$の自然な次元$n$を,多項式環の性質によらずに一般の環に拡張する.

\section{環の拡大}

体の拡大の基礎については第0章で少しまとめているが,Galois理論などの入門書を適宜参照のこと.体の拡大$L/K$において,代数的,超越的という概念があった.その概念を環の場合に拡張することを考えてみよう.代数$f:A\to B$を考え,これを環の拡大とみなしたい.体の場合は,体からの任意の準同型が単射なので,任意の準同型を埋め込みとして問題はなかった.そこで,$f:A\to B$が単射であるとき\textbf{環の拡大}という.しばらくは一般の$A$代数$B$について議論をしていこう.まずは代数的という性質を環の拡大の場合に拡張する.

\begin{defi}[整]\index{せい@整}
	$B$を$A$代数とする.$x\in B$に対し,$A$係数のモニックな多項式$g\in A[X]$が存在して;
	\[g(x)=x^n+a_1x^{n-1}+\dots+a_n=0\]
	が$A$加群$B$で成立するとき,$x$は$A$上\textbf{整}(integral)であるという.
\end{defi}

任意の$x\in B$が$A$上で整のとき,$B$は$A$上整であるという.$\Z$上で整である複素数のことを\textbf{代数的整数},$\Q$上で整である複素数のことを\textbf{代数的数}といっていたことを思い出そう.

\begin{ex}\label{ex:Q/Zの整閉包}
	環の拡大$\Z\subset\Q$を考えよう.ここでは$\Z$上整な元をすべて求める.$r=p/q ~(p,q\text{は互いに素})$が$Z$上整であると仮定すると,適当な$a_i\in\Z$がとれて;
	\[r^n+a_1r^{n-1}+\dots+a_n=0\]
	が成り立つ.分母を払うと;
	\[p^n+a_1qp^{n-1}+\dots+a_nq^n=0\]
	となり,$p^n$は$q$の倍数となる.$p$と$q$は互いに素なので,$q=\pm 1$が従う.ゆえに$r$は$\Z$の元である.$\Z$の元が$\Z$上整なことは明らか.
\end{ex}

整拡大を考えるときには,有限生成代数が度々登場する.そこで$A$上有限生成代数$B$について,全射$\varphi:A[\nitem{X}]\to B$を考えると,これは多項式環に何かを代入した準同型になっている.これを説明するため,単に$B$が1変数多項式環の像になっているときを考えよう.任意の$b\in B$について,ある$f\in A[X]$が存在して$b=\varphi(f)$とかける.ここで$f=a_nX^n+a_{n-1}X^{n-1}+\dots+a_0$とおくと;
\[b=\varphi(a_n)\varphi(X)^n+\dots+\varphi(a_0)\]
であるが,$\varphi(X)$は$B$の元であり,$B$を自然に$A$代数とみると,これは次のように書くのが自然である($a\cdot b=\varphi(a)b$としてスカラーを定義したことを思い出そう);
\[b=a_n\varphi(X)^n+a_{n-1}\varphi(X)^{n-1}+\dots+a_0\]
このようにして,$B$は多項式環$A[X]$の$X$を$\varphi(X)$で置き換えた環になっている.多変数の場合も同じである.そこで,全準同型$\varphi:A[\nitem{X}]\to B$が存在するとき,$B=A[\varphi(X_1),\varphi(X_2),\dots,\varphi(X_n)]$とかく.ただし,各$\varphi(X_i)$たちが変数のように独立に振る舞うとは限らない.例えば,$A[X,Y]$に$X\mapsto T^2,Y\mapsto T^4$と定めることで1変数多項式環への準同型が定まる.この準同型の像を$B$とすると先の議論から$B=A[T^2,T^4]$とかけるが,これは環としては$A[T^2]$にほかならない($T^4$は$T^2$で表現できる).

整であることは次のように特徴づけされる.
\begin{prop}\label{prop:整拡大の特徴づけ}
	$B$を$A$代数とすると,次は同値である.
	\begin{sakura}
		\item	$x\in B$は$A$上で整である.
		\item	$A[x]$は$A$加群として有限生成である.
		\item	有限生成$A$加群$C\subset B$が存在して,$A[x]\subset C$が成り立つ.
		\item	$A$上有限生成な$A[x]$加群$M$であって,任意の$f\in A[x]$について,任意の$m\in M$に対し$fm=0$ならば$f=0$を満たすものが存在する(有限生成で忠実な$A[X]$加群が存在する).
	\end{sakura}
\end{prop}
\begin{proof}
	\begin{eqv}[4]
		\item $x^n+a_1x^{n-1}x^{n-1}+\dots+a_0=0$とすると,任意の$k\geq n$に対して$x_k\in \bigoplus_{i=0}^{n-1}Ax_i$である(厳密には帰納法を用いる).ゆえに$A[x]=\bigoplus Ax_i$となり有限生成である.
		
		\item 明らか.
		\item $M=C$とすればよい.
		\item $\varphi:M\to M=m\mapsto xm$は$A$加群としての自己準同型になり,$\varphi(M)=x\cdot M\subset M$より,Cayley-Hamiltonの定理(\ref{thm:Cayley-Hamilton})を$I=A$として使うことができる.すると;
		\[\varphi^n+a_1\varphi^{n-1}+\dots+a_n=0\]
		となるから,この式の左辺は$x^n+a_1x^{n-1}+\dots+a_n$倍写像であるので,忠実性から従う.
	\end{eqv}
\end{proof}

整な元の集まり$\{x_1,\dots,x_n\}$についても,以下の補題から類似する性質を示すことができる.
\begin{lem}
	$B$を$A$代数とし,$M$を$B$加群とする.$B$が$A$加群として有限生成かつ$M$が$B$加群として有限生成であるとき,$M$は$A$加群として有限生成である.
\end{lem}
\begin{proof}
	$B,M$それぞれの生成元を$\{x_i\},\{y_j\}$として$B=\bigoplus Ax_i,M=\bigoplus By_j$とすると,$M=\bigoplus_i \left(\bigoplus_j Ax_iy_j\right)$であることがわかり,$A$加群としても有限生成である.
\end{proof}

\begin{prop}\label{prop:整な元で生成される代数はfinite}
	$B$を$A$代数とし,$\nitem{b}\in B$を$A$上整な元たちであるとする.このとき$A[b_1,\dots,b_n]$は$A$加群として有限生成である.
\end{prop}
\begin{proof}
	帰納法で示す.$n=1$のときは\ref{prop:整拡大の特徴づけ}より従う.$A_i=A[b_1,\dots,b_i]$とおこう.すると$A_{n-1}$が有限生成$A$加群であるとき,$A_n=A_{n-1}[b_n]$であるので,$b_n$は$A_{n-1}$上でも整だから$A_n$は有限生成$A_{n-1}$加群である.すると,補題より$A_n$は有限生成$A$加群になる.
\end{proof}

\begin{prop}
	$B$を$A$代数とする.$B$が$A$上有限型であり整であることと,$A$加群として有限生成であることは同値である.
\end{prop}

\begin{proof}
	$(\Longrightarrow)$は\ref{prop:整な元で生成される代数はfinite}から即座に従う.$B$が$A$加群として有限生成であるとする.まず,任意の$x\in B$をとると,\ref{prop:整拡大の特徴づけ} (iii)において$C=B$とできるので,$x$は$A$上整,すなわち$B$は$A$上整である.また,$B$の$A$加群としての生成系を$\{b_1,\dots,b_n\}$とすれば;
	\[A[X_1,\dots,X_n]\to B;f\mapsto f(b_1,\dots,b_r)\]
	が全準同型となり,$B$は有限型である.
\end{proof}

これは標語的に;
\[\text{有限型}+\text{整}=\text{有限生成}\]
と言い表すことができる,とても便利な性質である.

\begin{prop}[整従属の推移性]
	$A\subset B\subset C$を環の拡大とする.$B$が$A$上整でありかつ,$C$が$B$上整ならば,$C$は$A$上整である.
\end{prop}

\begin{proof}
	任意の$x\in C$をとる.$B$上整であるから,$b_r,\dots,b_n\in B$がとれて;
	\[x^n+b_1x^{n-1}+\dots+b_n=0\]
	が成り立つ.このとき$B'=A[b_1,\dots,b_n]$とおくと,\ref{prop:整な元で生成される代数はfinite}よりこれは有限生成$A$加群であり,$x$は$B'$上でも整なので$B'[x]$も有限生成$A$加群となる.よって\ref{prop:整拡大の特徴づけ}より,$x$は$A$上整である.
\end{proof}
\section{整閉整域と正規環}
\begin{defi}[整閉包]\index{せいへいほう@整閉包}
	$A\subset B$を環の拡大とする.$A$上整な$B$の元全体,すなわち;
	\[\mkset{b\in B}{\text{ある}a_1,\dots,a_m\in A\text{が存在して},b^m+a_1b^{m-1}+\dots+a_m=0}\]
	を$A$の$B$における\textbf{整閉包}(integral closure)という.ここでは$\bar{A}_B$と書くことにする.\footnotemark
\end{defi}\footnotetext{{Globalに通用する記号はないように思われる.}}

例えば,\ref{ex:Q/Zの整閉包}より$\Z\subset\Q$の整閉包$\bar{\Z}_{\Q}$は$\Z$である.

次が成り立つ.

\begin{prop}
	\begin{sakura}
		\item $\bar{A}_B$は$B$の部分環である.
		\item $\bar{A}_B$の$B$での整閉包は$\bar{A}_B$である.
	\end{sakura}
\end{prop}

\begin{proof}
	\begin{sakura}
		\item $x,y\in\bar{A}_B$とする.まず$x,y$は$A$上整なので$A[x,y]$は有限生成である.ゆえに$x+y,xy\in A[x,y]$だからこれらも整となり,$\bar{A}_B$は部分環をなす.
		\item $b\in B$が$\bar{A}_B$上整であるとすると,$a_1,\dots,a_m\in\bar{A}_B$がとれて;
		\[b^m+a_1b^{m-1}+\dots+a_m=0\]
		である.このとき$A'=A[a_1,\dots,a_n]$とおくと$A[b]$は$A'$加群として有限生成であり,$a_i$たちは$A$上整だから$A'$は$A$加群として有限生成.よって$A[b]$は$A$加群として有限生成になり,$b\in\bar{A}_B$が従う.
	\end{sakura}
\end{proof}

\begin{defi}[整閉]\index{せいへいせいいき@整閉整域}
	$A\subset B$を環の拡大とする.$\bar{A}_B=A$のとき,$A$は$B$において\textbf{整閉}(integrally closed)であるといい,特に整域$A$がその商体$\Frac A$で整閉のとき,\textbf{整閉整域}(integrally closed domain)という.
\end{defi}

\begin{ex}
	用意に計算できるようにUFDは整閉整域である(\ref{ex:Q/Zの整閉包}で計算した$\Z$についての類似である).それはGCD整域(\ref{defi:GCDdomain})に対して用意に一般化できる.
\end{ex}

\begin{defi}[正規環]\index{せいきかん@正規環}
	環$A$について,任意の$P\in\spec A$に対し$A_P$が整閉整域であるとき,$A$を\textbf{正規環}(normal ring)という.
\end{defi}

これは幾何的にも意味のある命題であるが,正規環は被約である.

\begin{prop}
	$A$を正規環とすると,$A$は被約である.
\end{prop}

\begin{proof}
	$\nil A=0$を言えばよい.任意の$P\in\spec A$について\ref{prop:局所化はいろんな操作と可換}より$(\nil A)_P=\nil(A_P)$であり,$A$は正規なので$A_P$は整域だから$\nil (A_P)=0$である.すると\ref{prop:局所化したら0は局所的}より$\nil A=0$となり正規環は被約である.
\end{proof}

\begin{lem}\label{lem:整閉包の局所化}
	環の拡大$A\subset B$について,$S$を$A$の積閉集合とする.このとき$S^{-1}\bar{A}_B=\bar{S^{-1}A}_{S^{-1}B}$が成り立つ.
\end{lem}

\begin{proof}
	任意の$x/s\in S^{-1}\bar{A}_B$をとる.$x$は$A$上整なので;
	\[x^n+a_1x^{n-1}+\dots+a_n=0\]
	となる$a_i\in A$が存在する.このとき;
	\[(x/s)^n+(a_1/s)(x/s)^{n-1}+\dots+a_n/s^n=0\]
	となるので$x/s\in\bar{S^{-1}A}_{S^{-1}B}$である.
	
	また,任意の$x/s\in\bar{S^{-1}A}_{S^{-1}B}$をとる.
	\[(x/s)^n+(a_1/s_1)(x/s)^{n-1}+\dots+a_n/s_n=0\]
	となる$a_i\in A,s_i\in S$が存在する.ここで$t=s_1\dots s_n$とおき,両辺に$(ts)^n$をかけると$tx\in\bar{A}_B$となる.よって$x/s=(tx)/(ts)\in S^{-1}\bar{A}_B$である.
\end{proof}

以下の命題より整閉整域は正規環である.
\begin{prop}\label{prop:整域の整閉性はlocal}
	$A$を整域とすると,$A$が整閉であることと$A$が正規環であることは同値(整閉整域は局所的な性質である).
\end{prop}

\begin{proof}
	$\bar{A}_{\Frac A}=A$であることは,自然な包含射$\iota:A\to \bar{A}_{\Frac A}$が全射であることと同値である.さて\ref{lem:整閉包の局所化}より$S^{-1}\bar{A}_{\Frac A}=\bar{S^{-1}A}_{\Frac A}$である.\ref{prop:局所的性質}より$\iota$が全射であることと$\iota$の局所化$A_P\to (\bar{S^{-1}A}_{\Frac A})_P$が全射であることは同値だから主張が従う.
\end{proof}

整域においては整閉性と正規性は同値だが,整域でない正規環が存在することに注意が必要である.$A_1,A_2$を整閉整域とし,直積$A=A_1\times A_2$の素イデアルによる局所化を考えよう.$P\times A_2\in\spec A$について,$A_{P\times A_2}$は${A_1}_{P}$と同型である.$\spec A$は$P_1\times A_2~(P_1\in\spec A_1),A_1\times P_2~(P_2\in\spec A_2)$のみからなるので,$A$は整域でない正規環である.実はすべてのNoether正規環は有限個の整閉整域の直積である.\ref{cor:Noether環の極小素イデアルは有限個}によりNoether環の極小な素イデアルは有限個であることに注意しよう.

\begin{thm}
	$A$をNoetherな正規環とすると,$A$は有限個の整閉整域の直積である.
\end{thm}

\begin{proof}
	$A$の極小な素イデアル全体を$P_1,\dots,P_r$とする.このとき$P_i+P_j=A~(i\neq j)$が成り立つ.実際$P_i+P_j\subsetneq A$とすると,これはイデアルをなすのである極大イデアル$\ideal{m}$に含まれる.すると$A_{\ideal{m}}$において$P_iA_{\ideal{m}},P_jA_{\ideal{m}}$は異なる極小素イデアルであるが(\ref{prop:Spec S^-1Aの引き戻し}),$A_{\ideal{m}}$は整域なのでただ1つの極小素イデアル$(0)$を持つ.これは矛盾.また$A$は被約なので$P_1\cap\dots\cap P_r=(0)$である.よって中国剰余定理(\ref{thm:中国剰余定理})から;
	\[A\cong \prod_{i=1}^r A/P_i\]
	である.ここで任意の$P\in\spec A/P_i$をとると,\ref{prop:直積環のspec}より$P\times\prod_{i\neq j}A/P_j\in\spec A$による$A$の局所化は$(A/P_i)_P$と同型で,$A$が正規なのでこれは整閉整域である.よって$A/P_i$は整閉整域となり,$A$は有限個の正規環の直積である.
\end{proof}

また,一般の正規環$A$は被約かつ全商環で整閉である.これの逆はNoether環について成り立つ.

\begin{prop}\label{prop:正規なら全商環で整閉}
	$A$を正規環とする.$A$の全商環を$Q(A)$とおくと,$\bar{A}_{Q(A)}=A$である.
\end{prop}

\begin{proof}
	$x/s\in Q(A)$が$A$上整であるとしよう.次のイデアル$I=\mkset{a\in A}{ax/s\in A}$が$A$であることを示せばよい.任意の$P\in\spec A$をとる.自然な単射$A\to Q(A)$にテンソルすることで,単射$A_P\to Q(A)\otimes_A A_P=Q(A)_P$が存在する.$x/s$は$A_P$上整なので,$x/s\in A_P$である.これは$a,t,h\in A,t,h\not\in P$が存在して$h(xt-as)=0$であることを意味する.ゆえに$ht\in I$であり,$I\not\subset P$である.よって$I=A$となり,$s$は$A$の単元である.
\end{proof}

\begin{thm}
	$A$がNoetherなら\ref{prop:正規なら全商環で整閉}の逆が成り立つ.すなわち$A$を全商環で整閉であるような被約Noether環とすると,$A$は正規である.
\end{thm}

\begin{proof}
	\begin{step}
		\item $A$の全商環$Q(A)$は有限個の体の直積であること(これは$A$が被約Noether環なら成り立つ).
		
		$A$がNoetherなので,有限個の極小素イデアルを持つ.それを$P_1,\dots,P_r$とする.\ref{cor:イデアルの準素分解}より$\ass A=\{P_1,\dots,P_r\}$であるので,$S=A\setminus(P_1\cup\dots\cup P_r)$とおくと$S^{-1}A=Q(A)$である.このとき$\spec Q(A)=\mkset{PQ(A)}{P\in\spec A,P\subset P_1\cup\dots\cup P_r}$であり,$P\in\spec A$が$P\subset P_1\cup\dots\cup P_r$を満たすならPrime avoidance(\ref{lem:Prime avoidance})より$P\subset P_i$となる$i$が存在するので,極小性から$\spec Q(A)=\{P_1Q(A),\dots,P_rQ(A)\}$である.よってこれらは全て極大で,また互いに素.$A$が被約だから$Q(A)$もまたそうなので,中国剰余定理から;
		\[Q(A)\cong\prod Q(A)/P_iQ(A)\]
		となり$Q(A)$は体の有限直積である.
		
		\item $A\cong\prod A/P_i$であること.
		
		$e_i=(0,\dots,0,1,0,\dots,0)\in Q(A)$について,$e_i^2-e_i=0$よりこれrは$A$上整である.$A$は$Q(A)$上整なので,これは$a_i+1\in\bigcap_{i\neq j}P_j,a_i\in P_i$となる$a_i\in A$の存在を意味する.よって$P_i$たちは互いに素となり,中国剰余定理から$A\cong\prod A/P_i$である.
		
		\item $A/P_i$が整閉整域であること.
		
		$\Frac(A/P_i)=A_{P_i}$に注意する.実際,準同型;
		\[\varphi:\Frac(A/P_i)\to A_{P_i};\frac{x+P_i}{y+P_i}\mapsto x/s\]
		を考えると,$x/y=0$ならばある$h\not\in P_i$が存在して$xh=0\in P_i$となるので$x+P_i=0$となり,$\varphi$は単射.また任意の$x/s\in A_{P_i}$について$x\not\in P_i$ならば$\varphi((x+P_i)/(y+P_i))=x/s$であり,$x/s\in P_iA_{P_i}$ならば$P_i=\ann y$となる$y\in A$について,$A$が被約なので$y\not\in P_i$だから$xy=0$となり$P_iA_{P_i}=0$,すなわち$\varphi(0)=x/s=0$である.以上より$\varphi$は全単射となり同型である.ここで,自然な準同型$Q(A)\to A_{P_i};x/s\mapsto x/s$を考えると,この核は$P_iQ(A)$であるので,$Q(A)/P_iQ(A)$は$A_{P_i}$の部分体である.一方で自然に$\Frac(A/P_i)\subset Q(A)/P_iQ(A)$であるから,$Q(A)/P_iQ(A)=A_{P_i}$であることがわかる.ゆえに$Q(A)\cong\prod A_{P_i}$である.よって$A\cong\prod A/P_i$は$Q(A)$で整閉であることから,各$A/P_i$は整閉整域である.
	\end{step}
	以上より$A$は整閉整域の直積なので正規であることが示された.
\end{proof}
\section{超越次数}
体の理論において,拡大の次数を考えることができたことを思い出そう.一般の環拡大について同様の概念を考えたい.少し条件を制限して,整域$A$について全商環を考えるとこれは体になる.これを使って整域の拡大,とくに体上の有限型の代数について考えよう.まずは次の命題に倣って,元の超越性を複数個の場合に拡張しよう.

\begin{prop}
	$L/K$を体の拡大とする.$\alpha\in L$が$K$上超越的であることと,次の準同型;
	\[\varphi:K[X]\longrightarrow L;f(X)\longmapsto f(\alpha)\]
	が単射であることは同値である.
\end{prop}

\begin{proof}
	$(\Longleftarrow$)は明らかだろう.$\alpha\in L$が$K$上超越的と仮定する.$\varphi(f)=\varphi(g)$としよう.このとき$\varphi(f-g)=(f-g)(\alpha)=0$となるが,$\alpha$は超越元なので$f-g=0$となり,$f=g$である.
\end{proof}

\begin{defi}[代数的に独立]\index{だいすうてきにどくりつ@代数的に独立}
	$L/K$を体の拡大とする.$\nitem[r]{\alpha}\in L$に対し,
	\[\varphi:K[\nitem[r]{X}]\longrightarrow L;f(\nitem[r]{X})\longmapsto f(\nitem[r]{\alpha})\]
	が単射であるとき,$\nitem[r]{\alpha}$は$K$上\textbf{代数的に独立}(algebraically independent)であるという.また,部分集合$S\subset L$に対し,その任意の有限部分集合が代数的に独立であるとき,$S$を$K$上代数的に独立な集合であるという.
\end{defi}

これを用いて超越次数を定義しよう.

\begin{defi}[超越基底]\index{ちょうえつきてい@超越基底}\index{ちょうえつじすう@超越次数}
	$L/K$を体の拡大とする.$\{\nitem[r]{\alpha}\}$が$K$上代数的に独立な集合で,$L/K(\nitem[r]{\alpha})$が代数拡大であるとき,$\nitem[r]{\alpha}$を$L/K$の\textbf{超越基底}(transcendental basis)といい,$r=\trdeg_K L$とかいてこれを\textbf{超越次数}(transcendence degree)という.
\end{defi}

$L/K$が代数拡大であるときは$\emptyset$を超越基底とし,$\trdeg_K L=0$と定める.超越次数のwell-definednessを示そう.

\begin{prop}[超越次数のwell-definedness]
	拡大$L/K$に対し,超越基底は存在し,その個数(濃度)は一致する.
\end{prop}
\begin{proof}
	超越基底が有限のときのみ示す.
	\begin{step}
		\item まず超越基底が存在することを示す.集合族:
		\[\Sigma=\mkset{S}{S:\text{代数的に独立な集合}}\]
		は空ではなく,包含関係により帰納的順序集合をなす.よってZornの補題より極大元$\mathcal{S}$が存在する.このとき,$L/K(\mathcal{S})$を考えると,ある$x\in L$が$K(\mathcal{S})$上代数的でないと仮定したとき,$\mathcal{S}\cup\{x\}$は代数的に独立で,$x\not\in\mathcal{S}$なので$\mathcal{S}$の極大性に反する.よって$L/K(\mathcal{S})$は代数拡大で,$\mathcal{S}$が超越基底となる.
			
		\item 次数のwell-definednessを示そう.$\{\nitem[r]{\alpha}\}$と$\{\nitem[s]{\beta}\}$がどちらも$L/K$の超越基底であるとする.$s\leq r$を言えばよい.$L/K(\nitem[r]{\alpha})$は代数拡大なので,$\beta_1$は$K(\nitem[r]{\alpha})$上代数的.すなわち,$a_i\in K(\nitem[r]{\alpha})$によって,関係式;
		\displaystar
		\[a_n{\beta_1}^n+a_{n-1}{\beta_1}^{n-1}+\dots+a_1\beta_1+a_0=0\quad(a_n\neq0)\]
		を満たす.$\beta_1$は$L/K$において超越的なので,少なくとも1つの$\alpha_i$が存在して,$a_0,\dots,a_n$のうち少なくとも1つには$\alpha_i$が表れる.番号を取り替えて,それを$\alpha_1$とすると,$(\ast)$は$\alpha_1$が$K(\alpha_2,\dots,\alpha_r,\beta_1)$上代数的であることを示している.すると,$L$の任意の元は$K(\nitem[r]{\alpha})$上代数的なので,$L/K(\alpha_2,\dots,\alpha_r,\beta_1)$は代数拡大である.すると,$\beta_2$は$K(\alpha_2,\dots,\alpha_r,\beta_1)$上代数的だから,$b_i\in K(\alpha_2,\dots,\alpha_r,\beta_1)$がとれて;
		\[b_m{\beta_2}^m+\dots+b_1\beta_2+b_0=0\quad(b_m\neq0)\tag{($\ast\ast$)}\]
		とできる.ここで,$\beta_1$と$\beta_2$は$K$上で代数的に独立だから,$b_i$の少なくとも1つには,$\alpha_2,\dots,\alpha_r$のうち1つ以上が表れる.番号を取り替えてそれを$\alpha_2$とすると,$(\ast\ast)$は$\alpha_2$の$K(\alpha_3,\dots,\alpha_r,\beta_1,\beta_2)$上の関係式と見ることができる.よって,同様の議論から$L/K(\alpha_3,\dots,\alpha_r,\beta_1,\beta_2)$は代数拡大.これを続けると$L/K(\alpha_{i+1},\dots,\alpha_r,\nitem[i]{\beta})$はすべて代数拡大である.
		
		ここで$r<s$を仮定する.$L/K(\alpha_r,\beta_1,\dots,\beta_{r-1})$は代数拡大で,$\beta_r$はこの上で代数的になる.いま$\{\nitem[r]{\beta}\}$は代数的に独立なので,先程の議論と同様に$\alpha_r$は$K(\nitem[r]{\beta})$上代数的である.しかし$\alpha_1,\dots,\alpha_{r-1}$は$K(\alpha_r,\nitem[r]{\beta})$上代数的なので,$L/K(\nitem[r]{\beta})$は代数拡大になり矛盾.よって$r\leq s$であることがわかった.
	\end{step}
\end{proof}

\begin{defi}
	$K$を体とする.$K$代数$A$が整域であるとき,体の拡大$\Frac A/K$の超越次数を$A$の($K$上の)超越次数と定義し,$\trdeg_K A$とかく.
\end{defi}

\begin{prop}\label{prop:weeknullstellensatzの補題}
	$K$を体とし,$A$を$k$上の有限型整域とすると,0でない$A$の素イデアル$P$に対し,
	\[\trdeg_K A/P<\trdeg_K A\]
	が成り立つ.
\end{prop}

\begin{proof}
	仮定から$K[\nitem{X}]$の0でない素イデアル$P'$が存在して$A=K[\nitem{X}]/P'$とかけている.簡単のため$K[X]=K[\nitem{X}]$と略記しよう.このとき,$P\subset Q$となる素イデアル$Q$について$\trdeg K[X]/Q<\trdeg K[X]/P$を示せば良い.$m=\trdeg A$とし,$m\leq\trdeg K[X]/Q,r=\trdeg K[X]/Q$とおき,$m\leq r$と仮定する.それぞれ$K[X]/P=K[\nitem{\alpha}],K[X]/Q=K[\nitem\beta]$とし,適切に並べ替えて$\nitem[r]{\beta}$を$K[X]/Q$の超越基とする.また$\nitem[m]{\alpha}\in K[\nitem\alpha]$を自然な準同型$K[\nitem\alpha]\longrightarrow K[\nitem\beta]$において$\nitem[m]{\beta}$に移すものとする.また,$P,Q$はそれぞれ次の代入する写像;
	\[\varphi:K[\nitem{X}]\longrightarrow K[\nitem\alpha]\]
	\[\psi:K[\nitem{X}]\longrightarrow K[\nitem\beta]\]
	の核であるので,$\alpha_i$の組の関係式が存在すれば,それは対応する$\beta_i$の組の関係式でもある.さて,任意の$0$でない$q\in Q$をとると,$\trdeg K[X]/P=m$だから,$p,\nitem[m]{\alpha}$は代数的に独立でない.よって,ある$K$上の多項式$f$が存在して$f(q,\nitem[m]{\alpha})=0$とできる.すると,$f(p,\nitem[m]{X})$は0でない関係式であり,これにより$\nitem[m]{\beta}$が消える.これは矛盾である.
\end{proof}

\begin{lem}[Artin--Tateの補題]\label{lem:Artin-Tateの補題}
	$A\subset B\subset C$を環の拡大とする.$A$がNoether環であり,$C$は有限型$A$代数でかつ$B$加群として有限生成であるとすると,$B$は有限型$A$代数である.
\end{lem}

\begin{proof}
	$C=A[c_1,\dots,c_r]$とおくと,$C$は有限$B$加群なので$c_i$たちは$B$上整である.すると;
	\[c_i^{n_i}+b_{i1}c_i^{n_i-1}+\dots+b_{n_i}=0\]
	となる$b_{ij}\in B(1\leq i\leq r,1\leq j\leq n_i)$たちがとれる.これらが$A$上生成する有限型$A$代数を$B'=A[b_{ij}]$とおく.このとき,構成から$C$は$B'$上整であり,また$C=B'[c_1,\dots,c_r]$すなわち$C$は有限型$B'$代数である.これより$C$は有限$B'$加群となる.いま$A$がNoetherなので$B'$もNoether(Hilbertの基底定理)で,$C$の部分$B'$加群$B$は有限生成である.よって$B$は有限型$B'$代数で,$B'$は有限型$A$代数だから$B$は有限型$A$代数である.
\end{proof}

\begin{prop}[Zariskiの補題]\label{prop:Zariskiの補題}
	$K$を体とし,$L$を$K$上の有限型代数とする.このとき,$L$が体ならば体の拡大$L/K$は有限次拡大である.
\end{prop}

\begin{proof}
	$L=K[x_1,\dots,x_n]$としよう.$L$が$K$上代数的でないと仮定すると,並び替えることで$x_1,\dots x_r$は$k$上代数的に独立で,$x_{r+1},\dots,x_n$は$F=K(x_1,\dots,x_r)$上代数的であるようにとれる.このとき$L$は$F$の有限次拡大体となるので,$F$加群として有限生成である.
	
	Artin--Tateの補題(\ref{lem:Artin-Tateの補題})から$F$は有限型$K$代数である.$F=K(x_1,\dots,x_r)$であったので,$f_i,g_i\in K[x_1,\dots,x_r]$が存在して;
	\[F=K[g_1/f_1,\dots,g_s/f_s]\]
	とかける.すると$f=f_1\dots f_s$とおくと,$F=K[x_1,\dots,x_r][1/f]$と表せる.さて$1/(f+1)\in K(x_1,\dots,x_r)=F$より,多項式$h\in K[x_1,\dots,x_r][X]$が存在して$1/(f+1)=h(1/f)$となる.よって$N\gg0$をとれば$f^N/(f+1)\in K[x_1,\dots,x_r]$とかける.$N$をこのような条件を満たすものの中で最小のものとしよう.このとき$g\in K[x_1,\dots,x_r]$が存在して$f^N=gf+g$とかけている.$N\geq1$と仮定すると;
	\[g/f=f^{N-1}+g\in K[x_1,\dots,x_r]\]
	となるので,$g$は$f$で割り切れる.これは$N$の最小性に矛盾.よって$N=0$である.すなわち$1/(f+1)\in K[x_1,\dots,x_r]$となり,$f\in K$でなければならない.
	
	よって$F=K[x_1,\dots,x_r]$とかけるが,$F=K(x_1,\dots,x_r)$であったことに矛盾.よって$L$は$K$上代数的である.
\end{proof}

Zariskiの補題の直接の応用の一つとして,Hilbertの零点定理(\ref{thm:Nullstellensatz})の弱い形である次の定理を示しておこう.

\begin{thm}[弱零点定理,week nullstellensatz]\index{じゃくれいてんていり@弱零点定理}\label{thm:week nullstellensatz}
	$K$が代数閉体であるとき,$K[\nitem{X}]$のすべての極大イデアル$\ideal{m}$は,ある$\nitem{a}\in K$が存在して;
	\[\ideal{m}=(X_1-a_1,X_2-a_2,\dots,X_n-a_n)\]
	と表せる.
\end{thm}
\begin{proof}
	簡単のため$K[\nitem{X}]=K[X]$とかく.$\ideal{m}$が極大であるから,$K[X]/\ideal{m}$は体である.すると,\ref{prop:Zariskiの補題}より$K[X]/\ideal{m}$は$K$代数拡大体であるので,$K$が代数閉体だから$K[X]/\ideal{m}\cong K$である.すると,適当な全準同型写像$\varphi:K[X]/\ideal{m}\twoheadrightarrow K$が存在する.ここで$\varphi(X_i)=a_i$とすれば,明らかに\displaystar
	\[(X_1-a_1,\dots,X_n-a_n)\subset\ker\varphi=\ideal{m}\]
	である.また,$(X_1-a_1,\dots,X_n-a_n)$は$a=(a_1,\dots,a_n)$とするときの点$a$の代入写像;
	\[\phi_a:K[\nitem{X}]\longrightarrow K;f\longmapsto f(a)=f(a_1,\dots,a_n)\]
	の核であるので,同型$K[X]/(X_1-a_1,\dots,X_n-a_n)\cong K$がなりたち,$(X_1-a_1,\dots,X_n-a_n)$は極大イデアルである.よって$(\ast)$において等号が成り立つことがわかる.
\end{proof}
\section{Krull次元と超越次元}

まず,環の次元の1つを定義する.環に次元を導入する方法はいくつかあるが,ある仮定のもとでそれらは一致してしまう(Krullの次元定理,\ref{thm:Krullの次元定理}).ここでは1番初等的に導入が行えるKrull次元というものを紹介しよう.
\begin{defi}[Krull次元]\index{#Krullじげん@Krull次元}\label{defi:Kurll次元}
		環$A$に対し,素イデアルの真の増大列
		\[P_*:P_0\subsetneq P_1\subsetneq\dots\subsetneq P_n\]
		に対して,$n$を列$ P_*$の\textbf{長さ}という.最長の素イデアルの列の長さを,環$A$のKrull\textbf{次元}といい,$\dim A$とかく.
\end{defi}
	
自明な例として,体の次元は0である.(体でない)PIDの次元は1である.また,\ref{prop:Artinの素イデアル}よりArtin環の次元も0である.実は,次元0のNoether環であることとArtin環であることは同値である.
\begin{thm}
	$A$がArtin環であるためには,Noether環かつ$\dim A=0$であることが必要十分である.
\end{thm}
\begin{proof}
	Artin環$A$は$\dim A=0$のNoether環であることは見た(\ref{thm:秋月}).$A$を次元0のNoether環とする.\ref{cor:Noether環の極小素イデアルは有限個}より$A$の極小素イデアルは有限個である.それを$\nitem[r]{P}$とする.また,$A$の次元は0なので$P_i$が$A$の素イデアルのすべてである.$\sqrt{0}=\bigcap_{i=1}^rP_i$であり,$\sqrt{0}$は$A$のイデアルだから有限生成なのである$k\in\Z_+$に対して$\sqrt{0}^k=0$である.よって,$P_1^kP_2^k\dots P_r^k\subset(\bigcap P_i)^k=\sqrt{0}^k=0$なので$P_1^kP_2^k\dots P_r^k=0$である.よって,\textbf{重複を許すことで}極大イデアル$P_1,\dots,P_n$を$P_1\dots P_n=0$となるようにできる(最大でこの中には$P_i$が$k$回ずつ現れる).ここで,イデアルの減少列;
	\[A\supset P_1\supset P_1P_2\supset\dots\supset P_1P_2\dots P_n=0\]
	を考えると,$A$がNoether的なので\ref{thm:秋月}の証明と同様に,隣り合う2項の剰余加群;
	\[M_i=P_1\dots P_i/P_1\dots P_{i-1}\]
	はNoether的.$M_i$は体$A/P_i$上の加群なので,\ref{prop:Artinなベクトル空間はNoether}より$M_i$はArtin的である.帰納的に$A$もArtin的である.
\end{proof}
	
ここで,次元と関連したイデアルの情報を定義しておこう.
\begin{defi}[高さ]\index{しんど@深度(イデアル)}\index{たかさ@高さ(イデアル)}
	$A$を環,$P$を素イデアルとする.
	\[\idht P=\sup\mkset{r\in\Z}{P=P_0\supsetneq P_1\supsetneq\dots\supsetneq P_r:\text{素イデアルの真減少列}}\]
	\[\coht P=\sup\mkset{r\in\Z}{P=P_0\subsetneq P_1\subsetneq\dots\subsetneq P_r:\text{素イデアルの真増大列}}\]
	を,それぞれ素イデアル$P$の\textbf{高さ(高度)},\textbf{余高度}(height,coheight)という.coheightのことを\textbf{深度}(depth)ということもある.一般のイデアル$I$については;
	\[\idht I=\min\mkset{\idht P}{P\in V(I)}\]
	で$I$の高さを定義する.
\end{defi}
	
定義から$\dim A$は$\sup\mkset{\idht P}{P\in\spec A}$である.高さと環の次元には次の関係がある.
	
\begin{prop}\label{prop:htとcohtの定義}
	以下の3つが成り立つ.
	\begin{sakura}
		\item $\dim A_P=\idht P$
		\item $\dim A/P=\coht P$
		\item $\idht P+\coht P\leq\dim A$
	\end{sakura}
\end{prop}

\begin{proof}
	\begin{sakura}
		\item 
			\ref{prop:Spec S^-1Aの引き戻し}より,$\spec A_p$と$\mkset{ P '\in\spec A}{ P '\cap (A\setminus P )=\emptyset}=\mkset{ P '\in\spec A}{ P '\subset P }$の間には包含関係を保つ全単射がある.よって,$\dim A_p=\idht P $である.
		\item 
			素イデアルの対応定理を用いれば,(i)と同様.
		\item 
			$A$の素イデアルの最長の鎖の中に$ P $が含まれるとき,前後で区切れば最長の真増大列と真減少列になるから,$\idht P +\coht P =\dim A$である.そうでないときは,$ P $の最長の真増大列と真減少列をつなげたものは素イデアルの鎖の1つとなるから,$\idht P +\coht P \leq\dim A$が成り立つ.
	\end{sakura}
\end{proof}
	
この命題の(iii)の不等式がいつ成り立つか,というのは重要な命題である.それについての一つの結果が\ref{thm:次元公式}である.

ここでは次の定理により,具体的な環の次元を計算する方法の1つを紹介しよう.

\begin{thm}
	体$K$上有限生成な整域$A$に対し,$\dim A=\trdeg_K A$である.
\end{thm}
\begin{proof}
	\begin{step}
		\item $\dim A\leq\trdeg A$であること.
		
		$\dim A=r$とし,$0\subsetneq P _1\subsetneq\dots\subsetneq P _r$とすると,\ref{prop:weeknullstellensatzの補題}より;
		\[0\leq\trdeg A/ P _r<\trdeg A/ P _{r-1}<\dots<\trdeg A/ P _1<\trdeg A\]
		であるので,$r\leq\trdeg A$である.
		
		\item $\trdeg A\leq\dim A$であること.
		
		$\trdeg A=n$とおく.基礎体$K$を動かすと$n$が動く.$n$についての帰納法で示す.まず,$n=0$のときは明らかに$0\leq\dim A$である.次に,$n-1$で正しいとする.$\trdeg A=n$となる$A$を,$A=K[\nitem[r]{\alpha}]$とおく.すべての$\alpha_i$は代数的でないから,$\alpha_1$が$K$上超越的としてよい.$S=K[\alpha_1]\setminus\{0\}$は積閉で,これで局所化すると$A_S=K(\alpha_1)[\alpha_2,\dots,\alpha_r]$となる.$K(\alpha_1)=K'$とすると,$\trdeg_K' A_S=n-1$であるので,帰納法の仮定から(任意の基礎体について成り立っているので)$n-1\leq\dim A_S$である.ゆえに,$A_S$の素イデアルの列;
		\[ P _0\subsetneq P _1\subsetneq\dots\subsetneq P _{n-1}\]
		がとれる.よって\ref{prop:Spec S^-1Aの引き戻し}より$ Q _i\cap S=\emptyset$となる$A$の素イデアルの列;
		\[ Q _0\subsetneq\dots\subsetneq Q _{n-1}\]
		がとれる.ここで,$ Q _{n-1}$が極大でなければ$ Q _{n-1}\subsetneq Q _n$となる真のイデアル$ Q _n$が存在するので,$A/ Q _{n-1}$が体でないことを示せばよい.さて,$ Q _{n-1}\cap S=( Q _{n-1}\cap K[\alpha_1])\setminus\{0\}=\emptyset$であるので,$\alpha_1\not\in Q _{n-1}$である.そこで,もし$\bar{\alpha_1}\in A/ Q _{n-1}$が$K$上代数的であると仮定すると,$f(\bar{\alpha_1})=0$となる$K$上の多項式$f$があり,これは$f(\alpha_1)\in Q _{n-1}$を意味する.ところが$f(\alpha_1)\in K[\alpha_1]$であるから,矛盾.よって$\bar{\alpha_1}$は$A/ Q _{n-1}$の超越的な元であるから,$\trdeg_K A/ Q _{n-1}>0$である.よって\ref{prop:Zariskiの補題}より$A/ Q _{n-1}$は体ではない.よって$n\leq\dim A$が示された.
	\end{step}
\end{proof}

この定理の系として,直感的な次の事実が得られる.
\begin{cor}\label{cor:多項式環の次元}
	$\dim K[\nitem{X}]=n$である.また,$n\neq m$なら$K[\nitem{X}]$と$K[\nitem[m]{X}]$は同型ではない.
\end{cor}

\section{上昇定理と下降定理}
この節では,次節でNoetherの正規化定理(\ref{thm:正規化定理})を証明する際に有力な道具となる\textbf{上昇定理}(going up theorem,\ref{thm:going up})と\textbf{下降定理}(going down theorem,\ref{thm:going down})を示そう.

\begin{defi}
	$A\subset B$を環の拡大とする.$A$のイデアル$I'$に対し,$B$のイデアル$I$が存在して$I\cap A=I'$となっているとき,$I$は$I'$の\textbf{上にある}(lying over)という.
\end{defi}

次の命題は簡単だが大切である.
\begin{prop}\label{prop:整域の整拡大と体}
	$A\subset B$を整域の拡大とする.$B$が$A$上整であるとき,$A$が体であることと$B$が体であることは同値.
\end{prop}

\begin{proof}
	\begin{eqv}
		\item $A$が体であるとき,任意の$0\neq x\in B$について;
		\[x^n+a_1x^{n-1}+\dots+a_n=0\]
		となる$a_i$に対して$B$が整域だから$a_n\neq0$である.よって$x^{-1}=-{a_n}^{-1}(x^{n-1}+\dots+a_{n-1})$であることがわかる.
		
		\item $B$が体であると仮定すると,任意の$0\neq x\in A$について$x^{-1}\in B$であるから;
		\[x^{-n}+a_1x^{-n+1}+\dots+a_n=0\]
		となる$a_i$が存在する.両辺に$x^{n-1}\in A$をかけると$x^{-1}=-(a_1+a_2x+\dots+a_nx^{n-1})\in A$であることがわかる.
	\end{eqv}
\end{proof}

\begin{lem}\label{lem:整従属は剰余環に落ちる}
	$A\subset B$を整拡大とする.このとき,$I$が$B$のイデアルならば$B/I$は$A/(I\cap A)$上整である.
\end{lem}

証明は明らかなので省略する.

\begin{prop}\label{prop:簡略版lying over theorem}
	$A\subset B$を整拡大とする.$P\in\spec B$について,$P$が極大であることと$P\cap A$が極大であることは同値.
\end{prop}

\begin{proof}
	補題より$B/P$は$A/(P\cap A)$上整であるので,\ref{prop:整域の整拡大と体}より従う.
\end{proof}

\begin{thm}\label{thm:上にある素イデアルの存在}
	$A\subset B$を整拡大とする.任意の$P'\in\spec A$について,$P'$の上にある$P\in\spec B$が存在する.
\end{thm}

\begin{proof}
	$B$の積閉集合$A\setminus P'$による局所化を$B_{P'}$と書くことにする.自然な単射$\iota:A\to B$と局所化が導く可換図式を考えよう;
	\[\begin{tikzcd}
		A\arrow[d]\nxcell[\iota] B\arrow[d]\\
		A_{P'}\nxcell[\iota_{P'}] B_{P'}
	\end{tikzcd}\]
	ここで素イデアルの準同型による逆像も素イデアルであることに注意しよう.$\ideal{m}$を$B_{P'}$の極大イデアルとすると,$\ideal{m}\cap B\in\spec B$である.これを$P$とおくと,$P\cap A=P'$となる.
	
	実際,可換図式をみると$\iota_{P'}((P\cap A)A_{P'})=\ideal m$となるが,\ref{prop:簡略版lying over theorem}より$\iota_{P'}^{-1}(\ideal{m})=\ideal m\cap A=P'A_{P'}$であり,\ref{prop:局所的性質}より$\iota_P$は単射なので$(P'\cap A)A_{P'}=P'A_{P'}$である.ゆえに$P\cap A=P'$である.
\end{proof}

\begin{thm}[上昇定理]\label{thm:going up}\index{じょうしょうていり@上昇定理}
	$A\subset B$を整拡大とする.$P_1'\subset\dots\subset P_n'$を$A$の素イデアルの昇鎖とすると,$B$の素イデアルの昇鎖$P_1\subset\dots\subset P_n$で,$P_i\cap A=P_i'$となるものが存在する.
\end{thm}

\begin{proof}
	帰納法によって$n=2$の場合についてのみ示せば十分である.$\bar{A}=A/P_1',\bar{B}=B/P_1$とする.\ref{lem:整従属は剰余環に落ちる}より$\bar{B}$は$\bar{A}$上整である.そこで$\bar{P_2'}\in\spec \bar A$について\ref{thm:上にある素イデアルの存在}より$\bar{P_2}\in\spec \bar{B}$が存在して$\bar{P_2}\cap\bar{A}=\bar{P_2'}$である.$\bar{P_2}$に対応する$P_2\in\spec B$について$P_2\cap A=P_2'$となり,示された.
\end{proof}

\begin{cor}
	$A\subset B$を整拡大とする.このとき$\dim A=\dim B$である.
\end{cor}

\begin{proof}
	上昇定理により$\dim A\leq\dim B$が従い,$B$の極大な昇鎖を引き戻せばそれは$A$のイデアルの昇鎖となるから$\dim B\leq\dim A$がわかる.
\end{proof}

つぎに下降定理(\ref{thm:going down})を示していくが,上昇定理とは異なり若干の仮定が必要である.そのためにいくつか準備をしよう.

\begin{defi}
	$A\subset B$を環の拡大,$I$を$A$のイデアルとする.$x\in\bar{A}_B$について,$x^n+a_1x^{n-1}+\dots+a_n=0$となる$a_i$をすべて$I$の元であるようにとれるとき,$x$は$I$\textbf{上整}であるという.$I$上整である$B$の元全体を$\bar{I}_B$とかき,$B$における$I$の\textbf{整閉包}という.
\end{defi}

\begin{lem}\label{lem:イデアルの整閉包と根基}
	$A\subset B$を整拡大,$I$を$A$のイデアルとする.自然な準同型$A\to \bar{A}_B$による$I$の像が$\bar{A}_B$で生成するイデアルと$I'$とおくと,$\bar{I}_B=\sqrt{I'}$である.
\end{lem}

\begin{proof}
	$x\in\bar{I}_B$をとる.このとき,$a_i\in I$がとれて;
	\[x^n+a_1x^{n-1}+\dots+a_n=0\]
	となる.すると$x^n=-(a_1x^{n-1}+\dots+a_n)$であるので,$x\in\bar{A}_B$であるから$x^n\in I'$すなわち$x\in\sqrt{I'}$である.
	
	一方,$x\in\sqrt{I'}$とすると,$x^n\in I'$となる$n$が存在する.よって,$c_i\in\bar{A}_B$と$x_i\in I$によって$x^n=\sum_{i=1}^m c_ix_i$とかける.\ref{prop:整な元で生成される代数はfinite}より,$\varphi:A[c_1,\dots,c_m]\to A[c_1,\dots,c_1];y\mapsto x^ny$に対してCayley-Hamiltonの定理(\ref{thm:Cayley-Hamilton})を適用できる.これにより$x^n$は$I$上整であり,$x\in\bar{I}_B$である.
\end{proof}

\begin{prop}\label{prop:整なら最小多項式が根基からとれる}
	$A\subset B$を整域の拡大とし,さらに$A$は整閉であるとする.$A$のイデアル$I$と$x\in\bar{I}_B$について,$x$の$K=\Frac A$上の最小多項式$F_x=T^n+c_1T^{n-1}+\dots+c_n$をとると,$c_i\in\sqrt{I}$とできる.
\end{prop}

\begin{proof}
	$I$上整なので,$K$上代数的なことは明らか.$F_x$のすべての根を$x=x_1,\dots,x_m$とし,それらを$K$に添加した体を$L$とすると,各$x_j$は$x$と同じ関係式によって$I$上整である.すると,$L$において$F_x$の係数$c_i$は$x_j$の多項式であるので,$c_i$は$I$上整である.すると,$c_i\in K$であったから$c_i\in\bar{I}_K$である.ここで,補題を$A\subset K$について考えると,$A$は整閉なので$I'=I$であるから$c_i\in\bar{I}_K=\sqrt{I}$である.
\end{proof}

\begin{thm}[下降定理]\label{thm:going down}\index{かこうていり@下降定理}
	$A\subset B$を整域の拡大とし,さらに$A$は整閉であるとする.$P_1'\supset P_2'\supset\dots\supset P_n'$を$A$の素イデアルの降鎖とすると,$B$の素イデアルの降鎖$P_1\supset P_2\supset\dots\supset P_n$で,$P_i\cap A=P_i'$となるものが存在する.
\end{thm}

\begin{proof}
	上昇定理と同様に$n=2$の場合に帰着できる.$P_2'$が$B$で生成するイデアルを$BP_2'$とかく.$B_{P_1}$について同様に考えると,\ref{prop:上にイデアルがあることの同値条件}より$B_{P_1}P_2'\cap A=P_2'$を示せば良いことがわかる.
	
	$x/s\in B_{P_1}P_2'\cap A$をとる.このとき$x\in BP_2',s\in B-P_1$である.ここで\ref{lem:イデアルの整閉包と根基}において$B$が$A$上整なので$\bar{A}_B=B$であり,$\sqrt{BP_2'}=\bar{P_2'}_B$となるので$x$は$P_2'$上整である.すると\ref{prop:整なら最小多項式が根基からとれる}より,$x$の$\Frac A$上の最小多項式は;
	\[F_x=T^n+a_1T^{n-1}+\dots+a_n\quad(a_i\in P_2')\]
	とかける.ここで$x/s\in A$より$x/s=y$とおくと,$sy=x$が成り立つので;
	\[F_s=T^n+\frac{a_1}{y}T^{n-1}+\dots+\frac{a_n}{y^n}\]
	である.
	
	$P_2'\subset B_{P_1}P_2'\cap A$は明らかなので,$y\in B_{P_1}P_2'\cap A$について$y\not\in P_2'$と仮定して矛盾を導こう.$s$は$A$上整だから,\ref{prop:整なら最小多項式が根基からとれる}を$I=A$として適用すると各$a_i/y^i\in A$である.すると$a_i/y^i\cdot y^i=a_i\in P_2'$で$y^i\not\in P_2'$だから$a_i/y^i\in P_2'$である.すると;
	\[s^n=-\left(\frac{a_1}{y}s^{n-1}+\dots+\frac{a_n}{y^n}\right)\]
	であるから$s^n\in BP_2'\subset BP_1'\subset P_1$であり,$s\in P_1$となるがこれは矛盾である.
\end{proof}
\begin{cor}\label{cor:上にあるイデアルの高さ}
	$A\subset B$が整域の拡大で,$A$が整閉であるとする.$B$のイデアル$I\neq(1)$について,$\idht I=\idht (I\cap A)$が成り立つ.	
\end{cor}

最初の仮定はただ下降定理を適用するために必要である.

\begin{proof}
	\begin{step}
		\item $I$が素イデアルのとき.
		
		$\idht I=r$とおき,$I=P_0\supset\dots\supset P_r$を$B$の素イデアルの鎖とすると,$P_i\cap A\in\spec A$によって$I'\cap A=P_0\cap A\supset\dots\supset P_r\cap A$より$\idht I\leq \idht (I\cap A)$が成り立つ.また,$\idht (I\cap A)=r'$とおくと,$I\cap A=P_0'\supset\dots\supset P_{r'}'$となる$A$の素イデアルの鎖が存在する.\ref{thm:going down}より$P_i'$のうえにある$B$の素イデアル$P_i'$たちで降鎖をなすものが存在するから,$r'=\idht(I\cap A)\leq\idht I$が成り立つ.
		
		\item $I$が一般のイデアルのとき.
		
		以下の式が成り立つことがわかる;
		\[\idht I=\min_{P\in V(I)}\idht P=\min_{P\in V(I)}\idht (P\cap A)\geq\min_{P'\in V(I\cap A)}\idht P'=\idht (I\cap A)\]
		ここで$P'\in V(I\cap A)$について,$A/(I\cap A)\hookrightarrow B/I$について\ref{thm:上にある素イデアルの存在}から$P\in V(I)$が存在して$P\cap A=P'$となることがわかるので,実は等号が成立している.
	\end{step}
\end{proof}

\section{Noetherの正規化定理}
この節ではNoetherの正規化定理(\ref{thm:正規化定理})を証明し,体上の有限生成整域について\ref{prop:htとcohtの定義} (iii)の不等式で等号が成り立つことをみよう.すなわち,この節では次の定理;
\begin{thm}\label{thm:次元公式}
	体$K$上有限生成な整域$A$に対して$\idht P +\coht P =\dim A$が成り立つ.
\end{thm}
を示すことを目標とする.
%これを定式化すると;
%
%\begin{defi}[次元公式]\index{じげんこうしき@次元公式}
%	環$A$と任意の$P\in\spec A$について;
%	\[\idht P+\coht P=\dim A\]
%	が成り立つとき,$A$で\textbf{次元公式}(dimension formula)が成り立つという.
%\end{defi}
%
%
%を証明しよう,ということである.この定理は代数幾何学で大活躍する.

\begin{lem}\label{lem:正規化の補題1}
	$K[X_1,\dots,X_n]$について,任意の$f\in K[X_1,\dots,X_n]$に対し,$f\not\in K$ならば,与えられた自然数$q$の倍数$m_2,\dots,m_n$が存在して,$y_1=f_1,y_2=X_2+X_1^{m_2},\dots,y_n=X_n+X_1^{m_n} (m_i\leq0)$とおくとき$K[X_1,\dots,X_n]$は$K[y_1,\dots,y_n]$上整である.
\end{lem}

\begin{proof}
	$f$を単項式の和として$f=\sum a_iM_i$とする.$\deg f=d$とおき,$t$を$d$より大きい$q$の倍数としよう.$i\leq2$について,$m_i=t^{i-1}$とおく.$M=X_1^{k_1}X_2^{k_2}\dots X_n^{k_n}$について,$X_i=y_i+X_1^{m_i}$を代入すると;
	\[M=X_1^{\sum k_it^{i-1}}+(X_1\text{について低次の}X_1,y_2,\dots,y_n\text{の項})\]
	である.そこで$\omega(M)=\sum k_it^{i-1}$とおく.$f$を構成する単項式$M=X_1^{k_1}\dots X_n^{k_n},M'=X_1^{l_1}\dots,X_n^{l_n}$に対して,辞書式順序において$(l_n,\dots,l_1)\leq (k_n,\dots,k_1)$ならば$\omega(M')\leq\omega(M)$となるから,$M_i$のうちで$\omega(M)$が最大のものは唯一つしかない.それを$M_1$とおく.このとき;
	\[f=a_1X_1^{\omega(M_1)}+(X_1\text{について低次の}X_1,y_2,\dots,y_n\text{の項})\]
	であるので,$y_1=f$であったことを思い出すと,$X_1$は$K[y_1,\dots,y_n]$を係数とする多項式;
	\[X^{\omega(M_1)}+(X\text{についての低次の項})+\frac{1}{a_1}y_1\]
	の根である.よって$X_1$は$K[y_1,\dots,y_n]$上整である.すると,$2\leq i$について$X_i$は$X_1,y_i$でかけているので,$K[X_1,\dots,X_n]$は$K[y_1,\dots,y_n]$上で整である.
\end{proof}

\begin{thm}[多項式環の正規化定理]
	$K[X_1,\dots,X_n]$とそのイデアル$I$を考える.$\idht I=r$のとき,$y_1,\dots,y_n\in K[X_1,\dots,X_n]$が存在して,$K[X_1,\dots,X_n]$は$K[y_1,\dots,y_n]$上整であって,$K[\nitem{y}]$のイデアルとして$I\cap K[\nitem[n]{y}]=(\nitem[r]{y})$とできる.
\end{thm}

\begin{proof}
	$r$についての帰納法で示す.
	
	\begin{step}\item $r=0$のときは$I=0$なので$y_i=X_i$とすればよい.
	
	\item	$I'\subset I$を$\idht I'=r-1$となるものとする.帰納法の仮定から$\nitem{y'}$で$I'\cap K[\nitem{y'}]=(\nitem[r]{y'})\subset I\cap K[\nitem{y'}]$となるものがとれる.\ref{cor:上にあるイデアルの高さ}より$\idht I'\cap K[\nitem{y'}]=r-1,\idht I\cap K[\nitem{y'}]=r$であるので,ある$f\in I\cap K[\nitem{y'}]$で$f\not\in I'\cap K[\nitem{y'}]$となるものがある.$I'\cap K[\nitem{y'}]=(\nitem[r]{y'})$より,$f(0,\dots,0,y_r',\dots,y_n')$も同じ条件を満たす.よって$f\in K[\nitem<r>{y'}]$としてよい.ここで\ref{lem:正規化の補題1}を用いると$y''_r=f,\dots,y''_n$で$K[\nitem<r>{y'}]$が$K[\nitem<r>{y''}]$上整であるものがとれる.ここで;
	\[y_1=y_1',\dots,y_{r-1}=y_{r-1}', y_r=y_r'',\dots,y_n=y_n''\]
	とすると$K[\nitem{X}]$は$K[\nitem{y}]$上整で,$I\cap K[\nitem{y}]\supset (\nitem[r]{y})$であって$\idht(\nitem[r]{y})\geq r$だから$\supsetneq$ではありえず,$I\cap K[\nitem{y}]=(\nitem[r]{y})$である.
	\end{step}
\end{proof}
\begin{thm}[Noetherの正規化定理]\index{#Noetherのせいきかていり@Noetherの正規化定理}\label{thm:正規化定理}
	$K$を体,$A$を有限生成$K$代数とする.このとき,$K$上代数的に独立であるような$z_1,\dots,z_s\in A$がとれて,$A$は$K[z_1,\dots,z_s]$上整である.
\end{thm}

\begin{proof}
	$A$が有限生成$K$代数であるから,全準同型$\varphi:K[\nitem{X}]\to A$が存在する.$I=\ker\varphi, \idht I=r$とおき,前定理を適用して$\nitem{y}$を得る.$y_{r+j}$の$\varphi$による像を$z_j$とすると,$K[\nitem{X}]$は$K[\nitem{y}]$上整なので,$A=\varphi(K[\nitem{X}])$は$K[\nitem[n-r]{z}]=\varphi(K[\nitem{y}])$上整である.よって$\nitem[n-r]{z}$が$K$上代数的に独立ならばよい.$z_i$についての関係式があったとすると,それの$z_i$を$y_{r+i}$に置き換えたものは$I=\ker\varphi$の元であるが,$I\cap K[\nitem{y}]=(\nitem[r]{y})$なので係数は0である.よって代数的に独立となり,主張が従う.
\end{proof}

Noetherの正規化定理によって,体上有限生成な整域$A$が$\idht P +\coht P =\dim A$を満たすことが証明できる.その本質は$A$が\textbf{鎖状環}になる,というところにあるので,それについて説明しよう.
\begin{defi}[鎖状環]\index{さじょうかん@鎖状環}
	環$A$の素イデアルの真増大列$P_0\subset P_1\subset\cdots$についてどの隣接した2項の間にも素イデアルが存在しないとき,その鎖は飽和しているという.任意の$P\subset P'$となる素イデアルについて,次の飽和したイデアルの鎖;
	\[P=P_0\subsetneq\dots\subsetneq P_n= P'\]
	の長さがすべて同一の有限値であるとき,$A$を\textbf{鎖状環}(catenary ring)という.
\end{defi}

明らかに$A$が鎖状環ならば,任意の$P\in\spec A$について$\idht P +\coht P =\dim A$が成り立つ.よって体$K$上有限生成な整域が鎖状環であることを示せばこの節の目標は果たされる.

\begin{thm}
	体$K$上有限生成な整域$A$に対して,飽和した素イデアル鎖$P_0\subset\dots\subset P_t$について;
	\[t=\dim A/P_0-\dim A/P_t\]
	である.つまり$A$は鎖状環である.
\end{thm}

\begin{proof}
		$0=P_0\subset\dots\subset P_t$が飽和していて,$P_t$が極大であるとする.環を適当に割ることで上の状態に帰着できるので,これについて$t$についての帰納法を用いて,$t=\dim A$を示せばよい.
		
		$t=0$のとき,$A$は体であるので明らか.$t-1$まで正しいとする.$\dim A=\trdeg A=d$とおくと,Noetherの正規化定理より,$\nitem[d]{z}\in A$が存在して$A$は$K[\nitem[d]{z}]$上整である.ここで,任意の$i$について$z_i\not\in P_1$と仮定する.このとき$P_1\cap K[\nitem[d]{z}]\subset K$なので,$P_1\cap K[\nitem[d]{z}]=0$である.ところが\ref{cor:上にあるイデアルの高さ}より$1=\idht P_1=\idht (P_1\cap K[\nitem[d]{z}])$であるので矛盾.よって$z_1\in P_1$としてよい.このとき$z_j\not\in P_1 (j\neq1)$に注意して,$A/P_1$と$K[\nitem[d]{z}]/(P_1\cap K[\nitem[d]{z}])=K[\nitem<2>[d]{z}]$について帰納法の仮定から$t-1=d-1$である.よって$t=d$である.
\end{proof}

これにより\ref{thm:次元公式}が示された.ではその逆,任意の素イデアル$P$について$\idht P +\coht P =\dim A$が成り立つならば$A$が鎖状環である,は成り立つだろうか?これは$A$がNoether局所整域ならば正しい(Ratliff,1972)がその証明は難しい(\cite{matsu},定理31.4.).実際のNoether環はほとんどが鎖状環であることが知られているが,その証明はCohen-Macaulay環の登場を待たなければならない.また,鎖状でないNoether環の例は\cite{Nag73}で与えられている.
\section{Hilbertの零点定理}

この節では,(古典的)代数幾何学の基礎を成す零点集合について紹介しよう.
	
\begin{defi}[Affine-$n$空間]\index{#Affineくうかん@Affine空間}
	$k$を(代数閉)体とする.$k$の元$n$個の組すべてからなる集合を$n$次元Affine\textbf{空間}(Affine space)という.
\end{defi}

体$k$は代数閉でなくても構わないが,後述するHilbertの零点定理は代数閉体でしか成り立たないため,注意が必要となる.

$A=k[\nitem{X}]$に対し,$f\in A$と$(\nitem{a})=P\in k^n$について$f(P):=f(\nitem{a})$と定めることにより,$A$の元を$k^n$から$k$への写像と解釈することができる.

\begin{defi}[零点集合]\index{れいてんしゅうごう@零点集合}
	$T\subset A$とする.$Z(T)=\mkset{P\in k^n}{\text{任意の}f\in T\text{に対し}f(P)=0}$を,$T$の零点集合という.
\end{defi}

$A$はNoether環なので,イデアル$ I $は有限個の生成元$(\nitem{f})$を持つ.よって,$Z(I)$は有限個の多項式$\nitem{f}$の共通の零点と考えられる.

\begin{defi}[代数的集合]\index{だいすうてきしゅうごう@代数的集合}
	$X\subset k^n$に対し,ある$T\subset A$が存在して,$X=Z(T)$となるとき,$X$を\textbf{代数的集合}(algebraic set)という.
\end{defi}

$ I $を$T$によって生成される$A$のイデアルとすると,$Z(T)=Z( I )$が成り立つので,代数的集合$X$に対応する$T$をイデアルとなるように取れる.

\begin{prop}
	代数的集合全体は閉集合系の公理を満たす.
\end{prop}
\begin{proof}
	\begin{step}
		\item $Z(0)=\A^n,Z(1)=\emptyset$である.
		\item 有限和について.
		
		$X=Z(T_X),Y=Z(T_Y)$を代数的集合とする.$X\cup Y=Z(T_XT_Y)$である.実際$\subset$は明らかで,$p\in Z(T_XT_Y)$とすると,$P\not\in X$ならば,ある$f\in T_X$が存在して$f(P)\neq 0$であるので,任意の$g\in T_Y$について$f(P)g(P)=0$であることから$g(P)=0$である.よって,$P\in Y$となる.ゆえに$P\in X\cup Y$である.
		
		\item 交わりについて.
		
		$X_\lambda$を代数的集合とし,$X_\lambda=Z(T_\lambda)$とすると
		\[\bigcap_{\lambda\in\Lambda}X_\lambda=Z\left(\bigcup_{\lambda\in\Lambda}T_\lambda\right)\]
		が成立する.実際$\subset$は明らかで,$P\in Z(\cup T_\lambda)$に対し,任意の$f\in\cup T_\lambda$について$f(P)=0$だから,特に各$\lambda$に対して任意の$g\in T_\lambda$について$g(P)=0$である.よって$P\in X_\lambda$となり,$P\in\bigcap X_\lambda$が従う.
	\end{step}
\end{proof}

\begin{defi}[Zariski位相]\index{#Zariskiいそうだいすうたようたい@Zariski位相(代数多様体)}
	$\A^n$に代数的集合全体を閉集合系とする位相を定める.これをZariski位相という.
\end{defi}

例として,$k$上のZariski位相(これをAffine直線という)を考える.$A=k[X]$はPIDだから,すべての代数的集合は1つの多項式の零点の集まりである.$k$は代数閉体なので,0でない多項式$f$はその字数が$n$のとき
\[f(x)=c(x-a_1)\dots(x-a_n)\quad(c\in k)\]
と分解できる.このとき$Z(f)=\{\nitem{a}\}$である.よって,$\A^1$の代数的集合は,有限な部分集合または$k$である.よって,開集合系は空集合及び有限部分集合の補集合となる(補有限位相).特にこれはHausdorffではないが,コンパクトである位相の大事な例である(ここでは「任意の開被覆を有限個で取り直せる」という性質を指して\quo{コンパクト}と呼んだ.代数幾何学,特にBourbakiの流儀では慣習的に上の性質に加えてHausdorffを課してコンパクトといい,Hausdorffでないときに\textbf{準コンパクト}(quasi-compact)と呼ぶことがあるので注意してほしい).

ここで\ref{defi:Zariski位相}を思い出そう.$k^n$のZariski位相において,点$\{(a_1,\dots,a_n)\}$は$f=c(X_1-a_1)\cdots(X_n-a_n)$の零点集合となり,$K^n$の閉点になる.これは弱零点定理(\ref{thm:week nullstellensatz})によって$K[X_1,\dots,X_n]$の極大イデアル,言い換えれば$\spec K[X_1,\dots,X_n]$の閉点が$(X_1,\dots,X_n)$しかないことと対応している.

また,Zariski位相は体$k=\Co$としたとき,Euclid位相より真に弱い位相となる.実際に多項式$f$を$\Co^n$から$\Co$への連続写像とみなせば,代数的集合はEuclid位相における閉集合$\{0\}$の$f$による引き戻しにほかならない.

$Y\subset \A^n$に対し
\[I(Y)=\mkset{f\in A}{\text{任意の}P\in Y\text{に対して}f(P)=0}\]
と定めると,これはイデアルをなす.これらについて性質をまとめよう.

\begin{prop}
	\begin{sakura}
		\item $T_1\subset T_2\subset A$とすると$Z(T_2)\subset Z(T_1)$である.
		\item $Y_1\subset Y_2\subset \A^n$とすると$I(Y_2)\subset I(Y_1)$である.
		\item $Y_1,Y_2\subset \A^n$について$I(Y_1\cup Y_2)=I(Y_1)\cap I(Y_2)$である.
	\end{sakura}
\end{prop}
\begin{proof}
	(iii)のみ示す.$f\in I(Y_1\cup Y_2)$とすると,任意の$P\in Y_1$について$f(P)=0$であるので,$f\in I(Y_1)$である.同様に$f\in I(Y_2)$であることがわかる.逆に,$f\in I(Y_1)\cap I(Y_2)$とすると任意の$P\in Y_1$と$Q\in Y_2$について$f(P),f(Q)=0$であるので,$f\in I(Y_1\cup Y_2)$である.
\end{proof}

\begin{thm}[Hilbertの零点定理]\index{#Hilbertのれいてんていり@Hilbertの零点定理}\label{thm:Nullstellensatz}
	$k$を代数的閉体,$ I $を$A=k[\nitem{X}]$のイデアルとし,$f\in A$を$Z( I )$のすべての点で消える多項式とする.このとき$f\in\sqrt{ I }$である.
\end{thm}
\begin{proof}
	$f\not\in\sqrt{I}$と仮定する.すると$P\in V(I)$であって,$f\not\in P$であるものがとれる.このとき$\bar{A}=A/P$において$\bar{A}_{\bar{f}}=\bar{A}[1/\bar{f}]$と,極大イデアル$\ideal{m}$を考えよう.体$\bar{A}_{\bar{f}}/\ideal{m}$は有限型$k$代数であるから,Zariskiの補題(\ref{prop:Zariskiの補題})より$k$の有限次拡大体,すなわち代数拡大体であり$k$が代数閉体だから,これは$k$に同型である.各$X_i$の$\bar{A}_{\bar{f}}/\ideal{m}$への像を$a_i$とおいて$a=(a_1,\dots,a_n)\in k^n$を定めると,任意の$g\in I$について$g(a)=g(\bar{X_1}+\ideal{m},\dots,\bar{X_n}+\ideal{m})=g(\bar{X_1},\dots,\bar{X_n})+\ideal{m}=0$である.一方で$\bar{f}\not\in\ideal{m}$より$f(a)\neq0$であるので,仮定に反する.よって$f\in\sqrt{I}$である.
\end{proof}

\begin{prop}
	$\A^n$の代数的集合全体を$As,A$の根基イデアル全体を$Ri$とすると,次の2つの写像
	\[\varphi:As\longmapsto Ri;Y\longmapsto I(Y), \psi:Ri\longrightarrow As; I \longmapsto Z( I )\]
	が包含関係を逆にする全単射となる.
\end{prop}
\begin{proof}
	\begin{step}
		\item Hilbertの零点定理より,$ I \in Ri$とすると,$I(Z( I ))=\sqrt{ I }= I $となるので,$\varphi\circ\psi=\id{Ri}$である.
		\item まず,任意の$Y\subset \A^n$に対し,$Z(I(Y))$は$Y$の閉包$\bar{Y}$に等しいことを示す.簡単に確かめられるように$Y\subset Z(I(Y))$であって,$Z(I(Y))$は閉なので,$\bar{Y}\subset Z(I(Y))$である.ここで,$W$を$Y\subset W$となる閉集合とする.すると,$W=Z( I )$となるイデアル$ I $がとれる.$Y\subset Z( I )$なので$I(Z( I ))\subset I(Y)$である.ここで$ I \subset \sqrt{ I }=I(Z( I ))$だから,$Z(I(Y))\subset Z( I )=W$が成立.よって$Z(I(Y))\subset \bar{Y}$となる.以上より$Z(I(Y))=\bar{Y}$であることがわかった.ここで,$\bar{Y}$が代数的集合,すなわち$Y\in As$なら$\bar{Y}=Y$であるので,$\psi\circ\varphi=\id{As}$である.
	\end{step}
\end{proof}
 %整拡大
	\part[completion,and Artin--Rees lemma]{完備化とArtin--Reesの補題}

この章では環や加群に位相を定めること,またその位相の完備化について考察する.また,その過程でArtin--Reesの補題を始めとする応用も広い諸結果に触れていこう.この章では\quo{極限}について触れる必要があるため,必要に応じて付録の極限の節を参照してもらいたい(圏の言葉を知らずともよいように配慮してある).
\section{位相群}

\begin{defi}[位相群]\index{いそうぐん@位相群}
	$G$をAbel群とする.また,集合として$G$が位相空間であり,次の写像;
	\[p:G\times G\to G;(x,y)\mapsto x+y\]
	\[m:G\to G;x\mapsto -x\]
	が連続であるとき,$G$を\textbf{位相群}(topological group)という.
\end{defi}

$g\in G$について,写像$p_g:G\to G;x\mapsto x+g$は$G$上の自己同相写像となる.よって,任意の点$g$の近傍はすべて$0$の近傍$U$を用いて$g+U$と表せる.よって$G$の位相は$0$の近傍によって決定される.

\begin{prop}
	$H$を$G$における$0$のすべての近傍の共通部分とする.$H$は$G$の部分群であり,$\{0\}$の閉包に等しい.
\end{prop}

\begin{proof}
	まず$H$が部分群であることを示そう.$0\in H$より$H\neq\emptyset$である.任意の$x,y\in H$をとる.すると$0$の任意の近傍$U$に対して$x,y\in U$である.
	
	$p$の連続性から$p^{-1}(U)$は$G\times G$の開集合である.よって$G$の開集合族$\{U_i\},\{V_i\}$が存在して$p^{-1}(U)=\bigcup (U_i\times V_i)$とかける.ここで$(0,0)\in p^{-1}(U)$だから適当な$i$について$(0,0)\in U_i\times V_i$である.このとき$U_i,V_i$は$0$の近傍となり,仮定から$x\in U_i,y\in V_i$とできる.よって$(x,y)\in p^{-1}(U)$すなわち$x+y\in U$である.よって$x+y\in H$である.同様に$m^{-1}(U)$を考えれば$x^{-1}\in H$がわかる.
	
	次に$H=\bar{\{0\}}$を示そう.$x\in\bar{\{0\}}$であることを($\ast$)とすると,これは任意の$x$の開近傍$U$が$0\in U$となることと同値である.ここで$U$は$0$の開近傍$V$を用いて$x+V$とかけることに注意すると,$(\ast)$は$0$の任意の開近傍$V$に対して$0\in x+V$となることと同値.これは更に$-x\in V$と言い換えることができ,$-x\in H$と同値である.ゆえに$x\in\bar{\{0\}}$は$x\in H$と同値である.
\end{proof}

$H$を用いて位相群$G$がHausdorffであることの判定条件を与えよう.ここで位相空間に関する次の命題を思い出しておく.

\begin{prop}\label{prop:Hausdorffと対角線}
	位相空間$M$がHausdorffであることと,$\Delta=\mkset{(x,y)\in M\times M}{x=y}$が$M\times M$が閉集合であることは同値.
\end{prop}
\begin{proof}
	$M$がHausdorffであるとき,$\Delta^c$が開であることを示す.任意の$(x,y)\in\Delta^c$に対して$x\neq y$である.$M$がHausdorffなので,開近傍$x\in U_x,y\in U_y$で$U_x\cap U_y=\emptyset$であるものがとれる.すると$U_x\times U_y$は$(x,y)$の開近傍で$U_x\times U_y\subset\Delta^c$である.よって$(x,y)$は内点なので$\Delta^c$は開である.
	
	同値であることはこの議論を逆にたどることで簡単にわかる.
\end{proof}

\begin{prop}\label{prop:位相群がHdfになる条件}
	$G$がHausdorffであることと$H=0$すなわち$\{0\}$が閉集合であることは同値である.特に$G/H$はHausdorffである.
\end{prop}

\begin{proof}
	\begin{eqv}
		\item $G$がHausdorffであるとき$T_1$であるので,1点は閉である.
		\item 連続写像$G\times G\to G;(x,y)\mapsto x-y$による$\{0\}$の引き戻しは$\Delta$である.$\{0\}$が閉なのでこれは閉であるからHausdorffとなる.
	\end{eqv}
\end{proof}

以降極限を扱うため$G$は第1可算公理を満たす,すなわち$0$は可算個の近傍を持つと仮定する.解析的な完備化はCauchy列によって与えられていたことを思い出そう.まずは位相群についてもCauchy列を定義する.

\begin{defi}[Cauchy列]\index{#Cauchyれつ@Cauchy列}
	$G$の元の族$\{x_i\}$であって,$0$の任意の近傍$U$について,ある整数$n$が存在して;
	\[\text{任意の}i,j\geq n\text{に対し} x_i-x_j\in U\]
	が成り立つとき,$\{x_i\}$をCauchy\textbf{列}(Cauchy sequence)であるという.
\end{defi}

Cauchy列を扱うには\quo{収束}の概念が不可欠であるため,距離空間でない一般の位相空間における収束を振り返っておこう.

\begin{defi}[収束]
	位相空間$M$について,系列$\{x_i\}$に対してある$x\in M$が存在して,任意の$x$の開近傍$U$に対してある$n>0$が存在して$i\geq n$ならば$x_i\in U$となるものが存在するとき,$\{x_i\}$は$x$に\textbf{収束}(converge)するという.
\end{defi}

$G$のCauchy列全体には,今までと同じように要素ごとの和をとることで和が定義され,次の関係;
\[\{x_i\}\sim\{y_i\}\Longleftrightarrow \lim_{i\to\infty} x_i-y_i\to 0\tag{$\ast$}\]
は同値関係となる.

\begin{defi}[完備化]\index{かんびか@完備化}
	同値関係$(\ast)$による$G$のCauchy列全体の同値類を$\widehat G$とかき,$G$の\textbf{完備化}(completion)という.
\end{defi}

この定義により,$\Q$を加法群としてみたときの完備化$\widehat{\Q}$は$\R$になることが自然に納得されるだろう.

各元$g\in G$に対して,定数列$\{g\}$は明らかにCauchy列となる.これによって自然な写像$\varphi:G\to\widehat{G}$を得ることができるが,一般にはこれは埋め込みにならない,すなわち単射ではない.

\begin{lem}
	$\ker\varphi=H$である.
\end{lem}

\begin{proof}
	$x\in\ker\varphi$とすると,$(0)\sim(x)$であるので,任意の$0$の近傍$U$に対して$x\in U$となる.よって$x\in H$であり,逆も成り立つ.
\end{proof}

\begin{prop}
	$\varphi$が単射であることと,$G$がHausdorffであることは同値である.
\end{prop}

\begin{proof}
	補題と\ref{prop:位相群がHdfになる条件}より従う.
\end{proof}

この節の最後に,完備化はテンソル積や局所化といったこれまでの操作と同様に関手になる,ということを注意しておく.というのも$G,H$を位相群(先程まで考えていた$H$とは異なる)としたとき,準同型$f:G\to H$により$G$のCauchy列は$H$のCauchy列を定めるから,群準同型$\widehat{f}:\widehat{G}\to\widehat{H}$が定義される.このとき$(g\circ f)\widehat{\mathstrut}=\widehat{g}\circ\widehat{f}$となるからである.
\section{線形位相と代数的な完備化}

ここまでは一般的な位相で考えてきたが,代数的な都合から$G$に\textbf{線形位相}という位相を入れて考えていくことにする.

\begin{defi}[線形位相]\index{せんけいいそう@線形位相}
	群$G$の部分群の列;
	\[G=G_0\supset G_1\supset G_2\dots\supset G_n\supset\cdots\]
	が与えられたとき,$U$が$0$の開近傍になることは$U$がある部分群$G_n$を含むことであると定義するとこれは$G$の位相になる.これを部分群の列が定める\textbf{線形位相}(liniear topology)という.
\end{defi}

この条件は位相群$G$が部分群の減少列$\{G_n\}$からなる$0$の基本近傍系を持つ,と言いかえることもできる.

位相になることを確かめておこう.そのために$g\in G$について$p_g:G\to G;x\mapsto x+g$が全単射であることを用いる.$V\subset G$が$x\in G$の開近傍であるとは,$p_g^{-1}(V)$が$0$の開近傍になることであると定めよう.このとき$x$の開近傍全体$\mathcal{V}(x)$は$x$の近傍系の公理を満たす.

\begin{lem}
	$\mathcal{V}(x)$を上のように定める.このとき;
	\begin{sakura}
		\item 任意の$V\in\mathcal{V}(x)$に対して$V\subset U$ならば$U\in\mathcal{V}(x)$である.
		\item $U_1,\dots,U_n\in\mathcal{V}(x)$ならば$\bigcap_{i=1}^n U_i\in\mathcal{V}(x)$である.
		\item 任意の$U\in\mathcal{V}(x)$に対して,$x\in U$である.
		\item 任意の$U\in\mathcal{V}(x)$に対して,ある$V\in\mathcal{V}(x)$が存在して,任意の$y\in V$に対して$U\in\mathcal{V}(y)$である.
	\end{sakura}
	が成り立ち,$\mathcal{V}(x)$は$x$の近傍系をなす.
\end{lem}

\begin{proof}
	(i)から(iii)はほぼ明らかであるから,(iv)だけ示す.
	
	$U$を$x$の近傍とする.このとき,ある$n$が存在して$G_n\subset p_x^{-1}(U)$となる.このとき$p_x(G_n)=V$とおくと,$p_x^{-1}(V)=G_n$であるから,$V$も$x$の近傍となる.任意の$y\in V$をとると,ある$g_0\in G_n$がとれて$y=g_0+x$である.このとき$p_y^{-1}(V)=(G_n+x)-(g_0+x)=G_n-g_0=G_n$であり,$V\subset U$であるから$G_n\subset p_y^{-1}(U)$が成り立つ.よって$U$は$y$の近傍となっている.
\end{proof}

以上より,任意の点の近傍系が定まったから,$G$全体に位相が定まることがわかった.この位相のもとでも$p_g$は自然な自己同相写像であることに注意しよう.また,$\{G_n\}$が定める線形位相において,各$G_n$は開かつ閉であることに注意しよう.

先の節で,位相的な完備化$\widehat{G}$と$\varphi:G\to\widehat{G}$を定義した.線形位相については,$\varphi$が埋め込みであることと$\bigcap G_n=0$が成り立つことが同値である.このとき,線形位相はHausdorffであるだけでなく非常によい位相となることが知られている.

\begin{prop}
	$G$と部分群の減少列$\{G_n\}$について,$\bigcap G_n=0$が成り立つとき;
	\[d(x,y)=\inf\mkset{\left(\frac{1}{2}\right)^n}{x-y\in G_n}\]
	と定めるとこれは$G$上の距離になり,$d$が定める位相は$\{G_n\}$による線形位相と一致する.
\end{prop}

\begin{proof}
	まずは距離になっていることを示そう.正値性,対称性,三角不等式が成り立つことは明らか.非退化であることを示す.$x=y$のとき$x-y=0$はすべての部分群に含まれるので,$d(x,y)=0$である.逆に$d(x,y)=0$とすると,任意の$\varepsilon>0$に対して,ある$n$で$x-y\in G_n$となるものが存在して$(1/2)^n<\varepsilon$である.ここで任意の$m>0$に対して$\varepsilon=(1/2)^m$とすれば,これに対してとれる$n$は$n<m$を満たし,$x-y\in G_n\subset G_m$より$x-y\in G_m$である.よって$x-y\in\bigcap G_n=0$であり$x-y=0$がわかる.
	
	以上より$d$は距離を定める.次に$0$の近傍全体が線形位相における$0$の近傍全体と一致することを見よう.$x\in G_n$のとき$d(x,0)\leq (1/2)^n$であること,$x\not\in G_n$であるとき$(1/2)^n<d(x,0)$であることに注意する.各$n$について$B_{(1/2)^{n-1}}(0)=G_n$であることを示す.$x\in G_n$ならば$d(x,0)\leq (1/2)^n<(1/2)^{n-1}$より$x\in B_{(1/2)^{n-1}}(0)$である.逆に$x\in B_{(1/2)^{n-1}}(0)$とすると,$d(x,0)<(1/2)^{n-1}$より$d(x,0)\leq(1/2)^n$でなければならないから,$x\not\in G_n$であるとすると矛盾する.
	
	さて,任意の$\varepsilon>0$をとる.ここで,$(1/2)^{n-1}<\varepsilon$となる$n$を取ることができる.このとき$B_{(1/2)^{n-1}}(0)=G_n$であるから,$G_n\subset B_\varepsilon(0)$となり$B_\varepsilon(0)$は線形位相において開であり,逆に$U$が線形位相において開ならば$G_n$は$0$の開球であり$U$に含まれるから,距離空間において開である.
	
\end{proof}

ここで,自然な全射$\theta_{n}:G/G_{n}\to G/G_{n-1}$を考えると,$\{G/G_n,\theta_n\}$は射影系をなす.射影極限;
\[\plim G/G_n=\mkset{(\xi_n)}{\text{任意の}n\leq m\text{について}\theta_m(\xi_m)=\xi_n\text{が成り立つ.}}\]

がCauchy列による完備化$\widehat{G}$と同型であることを示す.

\begin{thm}[代数的な完備化]
	$G$の部分群の列$\{G_n\}$からなる線形位相を考える.射影系$\{G/G_n\}$を考えると,次の同型が成り立つ.
	\[\widehat{G}=\plim G/G_n\]
\end{thm}

\begin{proof}
	まず,Cauchy列$(x_n)$を考えよう.各$n$について,十分大きな$m_n$をとれば,任意の$i,j\leq m_n$について$x_i-x_j\in G_n$とできる.これは$\pi_n(x_i)=\pi_n(x_j)$を意味する.このことは各$n$について$\{\pi_n(x_i)\}_i$は停まる,と言い換えることができる.それを$\xi_n$とおく.このとき$(\xi_n)$は$\plim G/G_n$の元になる.
	
	次に$(\xi_n)\in\plim G/G_n$をとる.各$x_n$を$\xi_n$の代表元とすると,$(x_n)$はCauchy列となる.実際,$\pi_n(x_{n+1})=\theta_{n+1}(\pi_{n+1}(x_{n+1}))=\theta_{n+1}(\xi_{n+1})=\xi_n=\pi_n(x_n)$であるので,$x_{n+1}-x_n\in G_n$となる.
\end{proof}

解析的な意味での完備性,すなわち実数の完備性とはある種の公理であった(同値な命題として中間値の定理やBolzano--Weierstra\ss の定理などがある)が,ここでは純粋に代数的な完備化として射影極限による構成を与えることができているので,完備であることの定義を天下り的に定義しよう.

\begin{defi}[完備]
	位相群$G$について,$\varphi:G\to\widehat{G}$が同型であるとき,\textbf{完備}(complete)であるという.
\end{defi}

$\varphi$が単射なことと$\bigcap G_n=0$が成り立つことが同値であったので,完備な位相群は距離空間である.この節の残りの部分では,完備化は完備であることを示すことを目標にする.

\begin{prop}\label{prop:位相群において完備化は完全}
	群の完全列;
	\[\ses[][p]{G'}{G}{G''}\]
	を考える.$G$に部分群の列$\{G_n\}$で定義される線形位相が入っているとき,$G',G''$にはそれぞれ$\{G'\cap G_n\},\{p(G_n)\}$による線形位相を考えることができ,次の完全列;
	\[\ses{\widehat{G}'}{\widehat{G}}{\widehat{G}''}\]
	が得られる.
\end{prop}

\begin{proof}
	一般に,定義から線形位相による射影系$\{G/G_n\}$は全射的である.よって\ref{thm:最初がsurjectiveなら射影極限は完全}を次の完全列;
	\[\ses{G'/(G'\cap G_n)}{G/G_n}{G''/p(G_n)}\]
	に適用すればよい.
\end{proof}

\begin{cor}\label{cor:完備化しても商は同型}
	$\widehat{G}_n$は$\widehat{G}$の部分群であり,$\widehat{G}/\widehat{G}_n\cong G/G_n$が成り立つ.
\end{cor}

\begin{proof}
	\ref{prop:位相群において完備化は完全}において$G'=G_n,G''=G/G_n$とすると,$G''$は離散位相を持つので,$\widehat{G}''=G''$となる.
\end{proof}

\begin{cor}
	$G$の完備化$\widehat{G}$は完備である.
\end{cor}

\begin{proof}
	先の系において射影極限を取ればよい.
\end{proof}

\section{I進位相とArtin--Reesの補題}

位相群の例で重要なものは,やはり加群についての応用である.環$A$のイデアル$I$により定義される線形位相のなかで重要なものに,$I$進位相がある.

\begin{defi}[$I$進位相]\index{#Iしんいそう@$I$進位相}
	$A$加群$M$と$A$のイデアル$I$を考える.$\{I^nM\}$は$M$の部分加群の減少列をなし,これによる線形位相を$I$\textbf{進位相}($I$-adic topology)という.
\end{defi}

これからは特筆しない限り,$A$加群$M$の位相は$I$進位相を考える.$M$の完備化$\widehat{M}$は$\widehat{A}$加群になることに注意しよう.また,$f:M\to N$を$A$加群の準同型とすると,$f(I^nM)=I^nf(M)\subset I^nN$であるので,$f$は$I$進位相について連続である.よって$\widehat{f}:\widehat{M}\to\widehat{N}$が定まる.

この節の目標は,群の場合と同様に$A$加群の完全列;
\[\ses[][p]{M'}{M}{M''}\]
に対して;
\[\ses{\widehat{M}'}{\widehat{M}}{\widehat{M}''}\]
が完全であることを示すことである(実はこれは$A$がNoetherで,$M$が有限生成$A$加群であるときにしか成り立たない).群の場合と同様に考えると,$M',M''$はそれぞれ$\{(I^nM)\cap M'\},\{I^n p(M)\}$で定義される線形位相による完備化についての完全列は得られる.よって,問題はそれぞれの線形位相が$I$進位相と一致しているだろうか?ということになる.この問題を解決するために,\textbf{フィルター}という概念を導入しよう.

\begin{defi}[$I$フィルター]\index{#Iフィルター@$I$フィルター}\index{あんていしているふぃるたー@安定しているフィルター}
	$A$加群$M$と,$A$のイデアル$I$を考える.$M_n$を$M$の部分加群として,降鎖$M=M_0\supset M_1\supset\dots\supset M_n\cdots$を考える.すべての$n$に対して,$IM_n\subset M_{n+1}$が成り立つとき,降鎖$\{M_n\}$は$I$\textbf{フィルター}($I$-filtration)であるという.特に,ある$n_0\in\N$が存在して,任意の$n\geq n_0$に対して$IM_n=M_{n+1}$が成り立つとき,そのフィルターは\textbf{安定している}(stable)という.
\end{defi}

\begin{defi}[有界な差]\index{ゆうかいなさふぃるたー@有界な差(フィルター)}
	$\{M_n\},\{M_n'\}$を$M$の$I$フィルターとする.ある$n_0\in\N$が存在して,任意の$n\in\N$について;
	\[M_{n+n_0}\subset M_n',M_{n+n_0}'\subset M_n\]
	が成り立つとき,$\{M_n\}$と$\{M_n'\}$は\textbf{有界な差}(bounded difference)を持つという.
\end{defi}

\begin{lem}\label{lem:安定しているフィルターは有界な差を持つ}
	$\{M_n\},\{M_n'\}$を$M$の安定している$I$フィルターとすると,それらは有界な差を持つ.
\end{lem}

\begin{proof}
	$M_n'=I^nM$としてよい.ある$n_0$が存在して,任意の$n\leq n_0$について$IM_n=M_{n+1}$なので,$M_{n+n_0}=I^nM_{n_0}\subset I^nM$となる.
	
	また,定義より任意の$n\in\N$について,$IM_n\subset M_{n+1}$なので,帰納的に$I^nM\subset M_n$となり,$I^{n+n_0}M\subset I^nM\subset M_n$であることがわかる.
\end{proof}

\begin{cor}
	安定しているすべての$I$フィルターの定める位相は$I$進位相に一致する.
\end{cor}

よって,問題は$\{(I^nM)\cap M'\},\{I^n p(M)\}$がそれぞれ$M',M''$の安定しているフィルターとなるかどうかに帰着することがわかった.$\{I^n p(M)\}$は定義から安定しているので,$\{(I^nM)\cap M'\}$について考えればよい.これにはArtin--Reesの補題が有効に働くため,この節の残りではこれを示すことにしよう.

Artin--Reesの補題の証明にはRees環という次数付き環が活躍するので,次数付き環についていくつか思い出しておこう.環$S$について,加法群としての部分加群の族$\{S_d\}$が存在して,$S=\middleoplus_d S_d$であり,すべての$d,e$について$S_dS_e\subset S_{d+e}$を満たすとき,$S$を\textbf{次数付き環}(graded ring)というのであった.$S_+=\middleoplus_{d>0}S_d$は$S$のイデアルとなっていて,無縁イデアルという.$S$を次数付き環とし,$S$加群$M$とその部分加群の族$\{M_n\}$について,$M=\middleoplus_n M_n$が成り立ち,すべての$d,n$について$S_dM_n\subset M_{n+d}$が成り立つとき,$M$を\textbf{次数付き}$S$\textbf{加群}(graded $S$-module)という.各$M_n$は$S_0$加群であることに注意しよう.次数付き$S$加群$M,N$について,$S$加群の準同型$f:M\to N$がすべての$n$について$f(M_n)\subset N_n$を満たすとき,$f$を次数付き加群の準同型であるという.

ここで,次数付き環に関して非常に有用な命題を示しておこう.

\begin{prop}\label{prop:次数付き環のNoether性}
	$A$を次数付き環とする.$A$がNoetherであることと,$A_0$がNoetherで,$A$が有限生成$A_0$代数であることは同値である.
\end{prop}

\begin{proof}
	\begin{eqv}
		\item $S_0=S/S_+$より$S_0$はNoetherである.$S_+$は$S$のイデアルなので有限生成である.そこで$x_i$たちを$S_+$の生成元とすると,これは斉次元の和でかけるので,適切に取り替えることで$x_i$は次数$k_i$の斉次元としてよい.また$S_+$の定義より$k_i>0$であることに注意しよう.
		
		$S_+=(x_1,\dots,x_s), S'=S_0[x_1,\dots,x_s]$とする.帰納法により,任意の$d$について$S_d\subset S'$であることを示す.$d=0$のときは明らかである.任意の$x\in S_d$をとる.$x\in S_+$より$x=\sum a_ix_i$とかける.ここで$x$は斉次なので$a_i\in S_{d-k_i}$でなければならない($m<0$のとき$S_m=0$と考えている).帰納法の仮定より$a_i\in S'$なので,$x\in S'$であることがわかる.よって$S_d\subset S$が成り立ち,$S=S'$である.
		
		\item Hilbertの基底定理(\ref{thm:Hilbertの基底定理})より従う.
	\end{eqv}
\end{proof}

\begin{defi}[Rees環]\index{#Reesかん@Rees環}
	環$A$のイデアル$I$と$A$加群$M$に対し,次数付き環;
	\[R_A(I)=\middleoplus I^n\]
	をイデアル$I$のRees\textbf{環}(Rees ring)という.特に$I$の生成元を$\{x_i\}$とするとき,$R_A(I)$は$\{x_i\}$で生成される$A$代数であることに注意しよう.
\end{defi}

Hilbertの基底定理より,$A$がNoetherなら$R_A(I)$もNoetherである.

\begin{lem}
	$A$をNoether環,$M$を有限生成$A$加群とする.$M$の$I$フィルター$\{M_n\}$に対しが安定していることと,$M^\ast\coloneq \middleoplus M_n$が有限生成$R_A(I)$加群であることは同値である.
\end{lem}

\begin{proof}
	各$n$について,$M_n^\ast\coloneq M_0\middleoplus\dots\middleoplus M_n\middleoplus IM_n\dots\middleoplus I^nM_n\middleoplus\cdots$を考える.各$M_n$は有限生成$A$加群であるから,$M_n^\ast$は有限生成$R_A(I)$加群である.ここで,$\{M_n\}$が安定しているならば,ある$n_0$が存在して,$n\geq n_0$について$M_n^\ast=M^\ast$であるので,$M^\ast$は有限生成$R_A(I)$加群である.
	
	逆に$M^\ast$が有限生成$R_A(I)$加群であるとすると,$M^\ast$はNoetherである.いま$\{M_n^\ast\}$は$M^\ast$の部分加群からなる昇鎖であるからこれは停まる.ここで$\bigcup M_n^\ast=M^\ast$であるので,ある$n_0$が存在して$M_{n_0}^\ast=M^\ast$となり,これは任意の$n\geq0$について$I^nM_{n_0}=M_{n_0+n}$が成り立つことを意味している.よって$\{M_n\}$は安定している.
\end{proof}

\begin{prop}[Artin--Reesの補題]\index{#Artin--Reesのほだい@Artin--Reesの補題}\label{prop:Artin--Reesの補題}
	$A$をNoether環とし,$A$のイデアル$I$と有限生成加群$M$を考える.$M$の安定している$I$フィルター$\{M_n\}$と部分加群$M'$について,$\{M'\cap M_n\}$は$M'$の安定している$I$フィルターである.
\end{prop}

\begin{proof}
	$\{M'\cap M_n\}$が$I$フィルターであることは明らかである.このフィルターが定める次数付き加群$\middleoplus (M'\cap M_n)$は$M^\ast$の部分加群であり,補題から$M^\ast$は有限生成$R_A(I)$加群であり,$R_A(I)$はNoetherだから$\middleoplus (M'\cap M_n)$も有限生成である.再び補題より$\{M'\cap M_n\}$は安定している.
\end{proof}

これにより以下の定理が得られた.

\begin{thm}
	$A$をNoether環とし,$M$を有限生成$A$加群とする.$A$加群の完全列;
	\[\ses{M'}{M}{M''}\]
	について,$I$進位相による完備化;
	\[\ses{\widehat{M}'}{\widehat{M}}{\widehat{M}''}\]
	は完全である.
\end{thm}

\section{Krullの交叉定理}

この節では,自然な準同型$A\to\widehat{A}$により,$\widehat{A}$を$A$代数としてみて,完備化についての考察を続けよう.まずはテンソル積$\widehat{A}\otimes_A \widehat{M}$が$\widehat{M}$と一致する条件を調べよう.$A$加群の準同型$M\to\widehat{M}$が誘導する準同型;
\[\begin{tikzcd}
	M\otimes_A 
	\widehat{A}\nxcell\widehat{M}\otimes_A\widehat{A}\nxcell\widehat{M}\otimes_{\widehat{A}}\widehat{A}=\widehat{M}
\end{tikzcd}\]
を考えよう.

\begin{prop}\label{prop:有限生成なら完備化は係数拡大}
	環$A$について,$M$が有限生成ならば$M\otimes_A \widehat{A}\to\widehat{M}$は全射である.また$A$がNoetherならばこれは同型である.
\end{prop}

\begin{proof}
	$M$は有限生成なので,ある$n$について;
	\[\ses[\iota][\varphi]{\ker\varphi}{A^n}{M}\]
	が完全である.テンソル積,完備化はそれぞれ左完全,完全であるから,先の準同型により,次の可換図式を考えることができる.
	\[\begin{tikzcd}[sep=large]
		&\ker\varphi\otimes_A \widehat{A}\arrow[d,"f"]\nxcell[\iota\otimes\id]A^n\otimes_A\widehat{A}\arrow[d,"g"]\nxcell[\varphi\otimes\id]M\otimes_A\widehat{A}\arrow[d,"h"]\nxcell0\\
		0\nxcell\widehat{\ker\varphi}\nxcell[\widehat{\iota}]\widehat{A^n}\nxcell[\widehat\varphi]\widehat{M}\nxcell0
	\end{tikzcd}\]
	ここで,加群の圏において射影極限は有限個の直和と可換であるから$g$は同型射である.すると$g$は全射なので$h$も全射である.
	
	また,$A$がNoetherであったとしよう.このとき$\ker\varphi$は有限生成なので,$f$も全射である.このとき,図式追跡によって$h$が単射であることを示すことができる($\iota\otimes\id$が単射ではないので,five lemmaなどは使えないことに注意).よって同型となることがわかる.
\end{proof}

\begin{cor}
	$A$がNoetherであるとき,$\widehat{A}$は平坦$A$代数である.
\end{cor}

\begin{proof}
	\ref{prop:平坦性は有限生成を調べればよい}からわかる.
\end{proof}

次に,これらの結果から極大イデアルによる完備化は局所環であることを示そう.

\begin{lem}\label{lem:完備化に関してのもろもろの補題}
	$A$をNoetherとし,$I$を$A$のイデアルとする.$I$進完備化$\widehat{A}$について;
	\begin{sakura}
		\item $\widehat{I}=\widehat{A}I\cong\widehat{A}\otimes_A I$
		\item $\widehat{I^n}=\widehat{I}^n$
		\item $I^n/I^{n+1}\cong\widehat{I}^n/\widehat{I}^{n+1}$
		\item $\widehat{I}$は$\widehat{A}$のJacobson根基に含まれる.
	\end{sakura}
	が成り立つ.
\end{lem}

\begin{proof}
	\begin{sakura}
		\item $A$がNoetherなので$I$は有限生成である.よって\ref{prop:有限生成なら完備化は係数拡大}より$\widehat{A}\otimes_A I\to\widehat{I}$は同型であり,その像は$\widehat{A}I$である.
		
		\item (i)より$\widehat{I^n}=\widehat{A}I^n=(\widehat{A}I)^n=\widehat{I}^n$である.
		
		\item \ref{cor:完備化しても商は同型}より$\widehat{A}/\widehat{I}^n\cong A/I^n$であり,これから(iii)が従う.
		
		\item 任意の$x\in\widehat{I}$について,すべての$A$に対し$\{\sum_{k=0}^i x^k\}$が$\widehat{A}$の$\widehat{I}$進位相においてCauchy列をなすので,$1+x+x^2+\dots=(1-x)^{-1}$は$\widehat{A}$において収束する.よって$1-x$は単元である.よって,任意の$a\in\widehat{A}$についても$1-ax$は単元となり,$\widehat{I}$はJacobson根基に含まれる.
	\end{sakura}
\end{proof}

\begin{prop}
	環$A$の極大イデアル$\ideal{m}$による完備化$\widehat{A}$は$\widehat{\ideal{m}}$を唯一の極大イデアルとする局所環である.
\end{prop}

\begin{proof}
	まず,\ref{lem:完備化に関してのもろもろの補題}(iii)より,$\widehat{A}/\widehat{\ideal{m}}\cong A/\ideal{m}$なので,$\widehat{\ideal{m}}$は極大イデアルである.また,$\widehat{\ideal{m}}'$を$\widehat{A}$の別の極大イデアルとすると,\ref{lem:完備化に関してのもろもろの補題}(iv)より$\widehat{\ideal{m}}$は$\widehat{A}$のJacobson根基に含まれ,定義より$\widehat{\ideal{m}}\subset\widehat{\ideal{m}}'$がわかる.よって$\widehat{\ideal{m}}=\widehat{\ideal{m}}'$であり,局所環である.
\end{proof}

環$A$とそのイデアル$I$による完備化への自然な写像$M\to\widehat{M}$は,一般に単射とは限らず,それは$\bigcap I^nM$により決定されるのだった.その構造について(条件付きではあるが)次のKrullの交叉定理が知られている.

\begin{thm}[Krullの交叉定理]\index{#Krullのこうさていり@Krullの交叉定理}
	$A$をNoether,$I$をそのイデアルとする.有限生成$A$加群$M$について,$L=\bigcap I^nM$とおくとき,ある$a\in I$が存在して,$(1-a)L=0$である.すなわち;
	\[x\in L\Longleftrightarrow (1-a)x=0\]
	が成り立つ.
\end{thm}

\begin{proof}
	Artin--Reesの補題(\ref{prop:Artin--Reesの補題})より,十分大きな$n$について$I^n M\cap L=I^{n-k}(I^kM\cap L)$となる$k\geq0$をとることができる.ここで,構成から$I^n M\cap L=L$であるので$L=IL$である.よって,中山の補題(\ref{thm:NAK})からある$a\in I$が存在して,$(1-a)L=0$である.
\end{proof}

この定理から多くの重要な系が得られる.

\begin{cor}
	$A$をNoether整域とし,$I\neq A$をイデアルとすると,$\bigcap I^n=0$である.
\end{cor}

\begin{cor}\label{cor:Krullの交叉定理の系}
	$A$をNoether,$I$を$A$のJacobson根基に含まれるイデアルとする.有限生成$A$加群$M$について,$I$進位相はHausdorffである.すなわち$\bigcap I^nM=0$である.
\end{cor}

\begin{cor}
	$(A,\ideal{m})$をNoether局所環とする.有限生成$A$加群$M$に対して$M$の$\ideal{m}$進位相はHausdorffである.特に,$A$の$\ideal{m}$進位相はHausdorffである.
\end{cor}
\section{随伴次数環}

この節では,Noether環の完備化がNoetherであることを示すことを目的とする.そのために\textbf{随伴次数環}という概念を導入するが,これは完備化の議論のみならず様々なところで活躍する.

\begin{defi}[随伴次数環]\index{ずいはんじすうかん@随伴次数環}\index{ずいはんかぐん@随伴加群}\label{defi:随伴次数環}
	環$A$とそのイデアル$I$,$A$加群$M$とその$I$フィルター$\{M_n\}$を考える.このとき;
	\[G(A)(=G_I(A))\coloneq\bigoplus_{n\geq0}I^n/I^{n+1},\quad G(M)\coloneq\bigoplus_{n\geq0}M_n/M_{n+1}\]
	をそれぞれ$I$に関する$A$の\textbf{随伴次数環}(associated graded ring),$M$の\textbf{随伴}$G(A)$\textbf{加群}(associatred graded $G(A)$-module)という.
\end{defi}

積は次のように定義する.$x_n\in I^n$について,$\bar{x_n}$を$I^n/I^{n+1}$における像とする.$\bar{x_n}\cdot\bar{x_m}$を,$x_nx_m\in I^{n+m}$の$I^{n+m}/I^{n+m+1}$の像と定義すると,これは代表元のとり方によらない.作用についても同様に考えると,この演算のもとでこれらは斉次成分を$I^n/I^{n+1},M_n/M_{n+1}$としてもつ次数付き環,加群になる.

\begin{prop}\label{prop:G(hat{A})はNoether}
	$A$をNoether,$I$を$A$のイデアルとする.このとき;
	\begin{sakura}
		\item $G(A)$はNoetherである.
		\item $G_I(A)$と$G_{\widehat{I}}(\widehat{A})$は次数付き環として同型である.
	\end{sakura}
	が成り立つ.特に$G(\widehat{A})$はNoether環である.
\end{prop}

\begin{proof}
	\begin{sakura}
		\item $A$はNoetherなので,$I=(x_1,\dots, x_n)$とできる.$G(A)$の$0$次斉次成分は$A/I$であり,これはNoetherである.$x_i$の$I/I^2$における像を$\bar{x_i}$とすれば$G(A)=A/I[\bar{x_1},\dots,\bar{x_n}]$であり,Hilbertの基底定理から$G(A)$はNoether.
		
		\item $I^n/I^{n+1}\cong\widehat{I}^n/\widehat{I}^{n+1}$であったことから従う.
	\end{sakura}
\end{proof}

\begin{prop}\label{prop:フィルターが安定していればG(M)は有限生成}
	$A$をNoether,$I$を$A$のイデアルとする.$M$が有限生成$A$加群であり,$\{M_n\}$が安定しているフィルターのとき,$G(M)$は$G(A)$加群として有限生成である.
\end{prop}

\begin{proof}
	$\{M_n\}$が安定しているので,ある$n_0$が存在して,任意の$n\geq0$について$M_{n_0+n}=I^nM_{n_0}$である.よって$G(M)=\middleoplus_{n\leq n_0}M_n/M_{n+1}$であり,各斉次成分はNoetherなので,$G(M)$はNoetherである.
\end{proof}

\begin{prop}
	環$A$とそのイデアル$I$,$M$加群$A$と$I$フィルター$\{M_n\}$を考える.$A$が$I$進位相について完備であり,$\bigcap M_n=0$がなりたち(すなわち$M$はHausdorff),$G(M)$が有限生成$G(A)$加群ならば$M$は有限生成$A$加群である.
\end{prop}

\begin{proof}
	次数環$G(A)$の$d$次斉次成分を$G(A)_d$と表すことにする.
	
	$G(M)$は有限生成$G(A)$加群なので,ある$y_i\in M_{d_i} (1\leq i\leq r)$たちを,$y_i$の$M_{d_i}/M_{d_i+1}$における像$\bar{y_i}$が$G(M)$の生成系になるようにとれる.ここで,背理法を用いて$M=\sum_{i=1}^r Ay_i$を示す.
	
	ある$x_0$が存在して,$x_0\not\in\sum Ay_i$であると仮定する.$x_0\neq0$より,$x_0\in M_{k_0}$となる最大の$k_0$が存在する.$\bar{x_0}$を$M_{k_0}/M_{k_0+1}$における$x_0$の像とすると,$\bar{x_0}=\sum_{i=1}^r\bar{a_{0i}}\bar{y_i}$とかけている.ここで$\bar{a_{0i}}\in G(A)_{k_0-d_i}=I^{k_0-d_i}/I^{k_0-d_i+1}$である.このとき$x_1=x_0-\sum a_{0i}y_i$とおくと,$x_1\in M_{k_0+1}$かつ$x_1\not\in\sum Ay_i$である.
	
	$x_0$のかわりに$x_1$を用いて同様の操作を行うことで,$\{x_n\}$たちを$x_n\in M_{k_0+n},x_n-x_{n+1}=\sum_{i=1}^r a_{ni}y_i$となるようにとることができる.このとき,作り方から;
	\[x-x_{n+1}=\sum_{i=1}^r\biggl(\sum_{j=0}^n a_{ji}\biggr) y_i\tag{$\ast$}\]
	である.ここで,各$i$について$\{\sum a_{ji}\}_j$は$A$内のCauchy列をなすので収束する.極限を$a_i$とおくと,$x=\sum_{i=1}^r a_iy_i$となることを示そう.$\bigcap M_n=0$より,任意の$l\geq0$に対して $x-\sum a_iy_i\in M_{K_0+l}$を示せばよい.まず,$(\ast)$より,任意の$n$に対し;
	\[x-\sum a_iy_i=x_{n+1}+\sum_{i=1}^r\biggl(\sum_{j=0}^{n} a_{ij}-a_i\biggr)y_i\]
	が成り立つ.ここで,$a_i$のつくりかたから,ある$n_i$が存在して$n\geq n_i$では$\sum_{j=0}^n a_{ji}-a_i\in I^{k_0+l-d_i}$とできる.$i$は有限個だから,$n_i$の最大値を$M$と置くことで$n\geq M$について$\sum_{i=1}^r(\sum_{j=0}^n a_{ij}-a_i)y_i\in M_{k_0+l}$である.よって,$n$を$k_0+l,M$より大きくとれば$x-\sum a_iy_i\in M_{k_0+l}$となる.
	
	以上より$x=\sum a_iy_i\in\sum Ay_i$となり,矛盾した.
\end{proof}

\begin{cor}\label{cor:G(M)がNoetherならMもNoether}
	環$A$とそのイデアル$I$,$A$加群$M$と$I$フィルター$\{M_n\}$に対して,$A$は$I$進完備で$\bigcap M_n=0$が成り立つとする.このとき$G(M)$がNoether$G(A)$加群ならば$M$はNoether$A$加群である.
\end{cor}

\begin{proof}
	$N$を$M$の部分$A$加群とすると,$\{N\cap M_n\}$が$N$の$I$フィルターとなり,$G(N)$は$G(M)$の部分$G(A)$加群なので有限生成である.明らかに$\bigcap N\cap M_n=0$であるから,先の命題より$N$は有限生成$A$加群である.
\end{proof}

\begin{thm}
	$A$をNoether環とする.$A$の$I$進位相による完備化$\widehat{A}$はNoetherである.
\end{thm}

\begin{proof}
	$\widehat{A}$の$\widehat{I}$進位相を考える.	$\widehat{A}$は完備なので$\bigcap \widehat{I}^n\widehat{A}=0$であり,また\ref{prop:G(hat{A})はNoether}より$G(\widehat{A})$はNoetherだから\ref{cor:G(M)がNoetherならMもNoether}より$\widehat{A}$はNoetherである.
\end{proof}
 %完備化
	\newpage
\part[Local ring and Dimension theory]{局所環と次元論}

\section{離散付値環}

局所環の例の1つに付値環というものがある.そのなかでも特に離散付値環は,後に見るように1次元Noether局所環の特徴付を与えている,重要なクラスの1つである.

\begin{defi}[付値環]\index{ふちかん@付値環}
	$A$を整域とし,$k=\Frac A$とする.任意の$0\neq x\in k$について,$x\in A$か$x^{-1}\in A$のどちらかが成り立つとき,$A$を$k$の\textbf{付値環}であるという.
\end{defi}

定義から付値環$A$は$k$上整閉であることがすぐにわかる.よって付値環は整閉整域である.

\begin{prop}
	$A$を$k$の付値環とする.このとき,$A$は局所環である.
\end{prop}

\begin{proof}
	$A$の可逆でない元全体を$\ideal{m}$とおく.これがイデアルをなすことを示そう.任意の$a\in A$と$0\neq x\in\ideal{m}$をとる.$ax\not\in\ideal{m}$と仮定すると$(ax)^{-1}\in A$となり,$x^{-1}=a(ax)^{-1}\in A$となるので矛盾する.よって$ax\in \ideal{m}$である.また$0$でない$x,y\in\ideal{m}$について,$xy^{-1}\in k$に対して仮定から$xy^{-1}\in A$または$x^{-1}y\in A$が成り立つ.$xy^{-1}\in A$のとき$x+y=(1+xy^{-1})y\in \ideal{m}$が成り立ち,$x^{-1}y\in A$のときも同様である.よって$\ideal{m}$はイデアルとなり,$A$は局所環である.
\end{proof}

次に付値環の定義のもととなった,付値と呼ばれる関数について説明しよう.

\begin{defi}[付値]\index{ふち@付値}
	$k$を体とし,$G$を全順序なAbel群とする.関数$v:k^\times\to G$が,全射であり,すべての$a,b\in k^\times$に対して;
	\begin{sakura}
		\item $v(ab)=v(a)+v(b)$
		\item $v(a+b)\geq\min (v(a),v(b))$
	\end{sakura}
	が成立するとき,$v$を\textbf{付値}(valuation)という.
\end{defi}

まず,自明な性質として$v(1)=0$であり,$v(x^{-1})=-v(x)$である.

\begin{defi}
	集合$A=\mkset{x\in k}{v(x)\geq0}\cup\{0\}$は$k$の部分環で,これを$v$の\textbf{付値環}(valuation ring)という.
\end{defi}

体の部分環であるから付値環は必ず整域である.また付値環$A$について$\Frac A=k$となることに注意しよう.これは$k$の付値環をなす.その極大イデアルは$\ideal{m}=\mkset{x\in k}{v(x)>0}$で与えられる.

次の離散付値環が特に大切である.
\begin{defi}[離散付値]\index{りさんふち@離散付値}\index{#DVR@DVR(離散付値環)}\index{りさんふちかん@離散付値環}
	$G=\Z$としたときの付値$v$を\textbf{離散付値}(discrete valuation)という.対応する付値環を\textbf{離散付値環}(discrete valuation ring)といい,DVRと略す.このとき,便宜上$v(0)=\infty$とする.
\end{defi}

付値環についていくつかの性質を示しておこう.

\begin{lem}
	$x,y\in A$について$v(x)\geq v(y)$ならば$x\in (y)$である.特に$v(x)=v(y)$ならば$(x)=(y)$である.
\end{lem}

\begin{proof}
	$v(xy^{-1})=v(x)-v(y)\geq0$より,$xy^{-1}\in A$である.これは$x\in (y)$を導く.
\end{proof}

\begin{prop}\label{prop:離散付値環の性質}
	$A$を離散付値環とする.$I$を$A$の任意のイデアルとすると,ある$n\in\N$がとれて$I=\ideal{m}^n$である.特に$A$はPIDである.
\end{prop}

\begin{proof}
	まず,$\ideal{m}^n=(x^n)$とかけることを示そう.付値は全射であるから,$v(x)=1$となる$x\in\ideal{m}$が存在する.このとき$\ideal{m}=(x)$となることを示そう.$(x)\subset\ideal{m}$は明らかである.$y\in\ideal{m}$とすると,$v(y)\geq1$より,$v(x^{v(y)})=v(y)$が成立する.よって補題から$(y)=(x^{v(y)})\subset(x)$である.
	
	さて,$I$を$A$のイデアルとする.$v(I)=\mkset{v(y)\in\N\cup\{\infty\}}{y\in A}$は最小元を持つ.それを$n=v(y') (y'\in A)$とおこう.すると任意の$y\in A$について$v(y)\geq n=v(x^n)$より,補題から$y\in (x^n)$である.逆に$y\in(x^n)$とすると,$v(y)\geq n=v(y')$より,$y\in(y')\subset I$が従う.よって$I=(x^n)$であることがわかる.
\end{proof}

\begin{cor}
	離散付値環$A$は1次元のNoether局所整域で,$\spec A=\{0,\ideal{m}\}$である.
\end{cor}

\ref{prop:離散付値環の性質}は離散付値環の著しい特徴付けを与えており,次が成り立つ.

\begin{thm}\label{thm:DVRの特徴づけ}
	$(A,\ideal{m})$を1次元Noether局所整域とする.次は同値である.
	\begin{sakura}
		\item $A$は離散付値環である.
		\item $A$は整閉である.
		\item $\ideal{m}$は単項イデアルである.
		\item $\dim_k (\ideal{m}/\ideal{m}^2)=1$である.
		\item すべての$A$の0でないイデアルは$\ideal{m}$の冪である.
		\item ある$x\in A$が存在して,すべての0でないイデアルは$(x^k) (k\geq0)$とかける.
	\end{sakura}
\end{thm}

これを示すためにいくつかの補題を示していこう.

\begin{prop}\label{prop:局所環がm^n=0ならArtin}
	$(A,\ideal{m})$をNoether局所環とすると,次のうちどちらか1つだけが成り立つ.
	\begin{sakura}
		\item 任意の$n\geq0$について,$\ideal{m}^n\neq\ideal{m}^{n+1}$が成り立つ.
		\item ある$n>0$が存在して$\ideal{m}^n=0$である.特に,この場合$A$は0次元すなわちArtin環である.
	\end{sakura}
\end{prop}

\begin{proof}
	$\ideal{m}^n=\ideal{m}^{n+1}$となる$n$があるとする.すると,中山の補題(\ref{thm:NAK})より$\ideal{m}^n=0$が成り立つ.任意の$P\in\spec A$について,$\ideal{m}^n\subset P$より根基をとると$\ideal{m}=P$が成り立つ.ゆえに$A$はArtinである.
\end{proof}

\begin{cor}\label{lem:DVR-A}
	$(A,\ideal{m})$を$\dim A\geq1$となるNoether局所環とすると,任意の$n$について$\ideal{m}^n\neq\ideal{m}^{n+1}$である.
\end{cor}

\begin{lem}\label{lem:Artinならnil Aは冪零}
	$A$をArtin環とすると,$\nil A$は冪零である.
\end{lem}

\begin{proof}
	DCCよりある$k>0$がとれて$(\nil A)^k=(\nil A)^{k+1}=\cdots$となる.これを$I$とおこう.$I\neq0$と仮定する.このとき$IJ\neq0$となるイデアル$J$の集合$\Sigma$は$I\in\Sigma$となり空ではない.よって$\Sigma$の極小元がとれるので,それを改めて$J$とおこう.このとき,ある$x\in J$がとれて$xI\neq0$となる.極小性より$(x)=J$であることがわかる.ここで$(xI)I=xI^2=xI\neq0$より再び極小性から$xI=(x)$となる.よって,ある$y\in I$について$xy=x$とかける.ここで$y\in I\subset\nil A$より$y^n=0$となる$n$がとれる.すると$x=xy=xy^2=\dots=xy^n=0$となり,$J=0$となるから矛盾.よって$I=0$である.
\end{proof}

\begin{prop}\label{lem:DVR-B}
	$(A,\ideal{m})$をArtin局所環とすると,$A$のすべてのイデアルが単項であることと$\dim_k(\ideal{m}/\ideal{m}^2)\leq1$であることは同値である.
\end{prop}

\begin{proof}
	$(\Longrightarrow)$は明らかなので,逆を示す.$\dim_k \ideal{m}/\ideal{m}^2=0$なら$\ideal{m}=\ideal{m}^2$となり中山の補題から$\ideal{m}=0$すなわち$A$は体となるので,示すことはない.
	
	$\dim_k \ideal{m}/\ideal{m}^2=1$と仮定すると\ref{prop:Atimac_prop_2.8}より$\ideal{m}$は単項生成である.$\ideal{m}=(x)$とする.$I$を$0$でも$A$でもない$A$のイデアルとすると,Artin環において冪零根基とJacobson根基(極大イデアルの共通部分)は等しいので,$\nil A=\ideal{m}$である.\ref{lem:Artinならnil Aは冪零}より$I\subset(\nil A)^k,I\not\subset(\nil A)^{k+1}$となる$k\in\N$がとれる.よってある$y\in I$と$a\in A$がとれて$y=ax^k,y\not\in(x^{k+1})$とできる.よって$a\not\in(x)$でなければならないので$a$は可逆である.よって$x^k\in I$となり,$I=(x^k)$となることがわかった.
\end{proof}

\begin{lem}\label{lem:DVR-C}
	$(A,\ideal{m})$を1次元Noether局所整域とする.$0$でも$A$でもない$A$のイデアル$I$について,$I$は$\ideal{m}$準素であり,特にある$n$について$\ideal{m}^n\subset I$である.
\end{lem}

\begin{proof}
	$\ideal{m}$は$A$のただ1つの0でない素イデアルなので,$\sqrt{I}=\ideal{m}$である.よって\ref{prop:sqrt{I}が極大なら準素}より$\ideal{m}$準素である.
\end{proof}

\begin{proof}[\textbf{\ref{thm:DVRの特徴づけ}の証明}]
	示すことは1次元Noether局所整域について;
	\begin{sakura}
		\item $A$は離散付値環である.
		\item $A$は整閉である.
		\item $\ideal{m}$は単項イデアルである.
		\item $\dim_k (\ideal{m}/\ideal{m}^2)=1$である.
		\item すべての$A$の0でないイデアルは$\ideal{m}$の冪である.
		\item ある$x\in A$が存在して,すべての0でないイデアルは$(x^k) (k\geq0)$とかける.
	\end{sakura}
	の同値性である.
	\begin{eqv}[6]
		\item 明らか.
		\item $0\neq a\in\ideal{m}$をとる.\ref{lem:DVR-C}より$\ideal{m}^n\subset(a),\ideal{m}^{n-1}\not\subset(a)$となる$n>0$がとれる.$b\in\ideal{m}^{n-1}$かつ$b\not\in(a)$となる$b$をとる.$x=a/b\in k$とおくと,$b\not\in(a)$より$x^{-1}\not\in A$である.$A$は整閉なので$x^{-1}$は整ではなく,\ref{prop:整拡大の特徴づけ}より$x^{-1}\ideal{m}\not\subset\ideal{m}$である.いま$x$の構成から$x^{-1}\ideal{m}\subset A$なので,$x^{-1}\ideal{m}=A$が成り立ち,$\ideal{m}=(x)$である.
		
		\item \ref{lem:DVR-A}よりわかる.
		
		\item $I$を$A$の$0$でも$A$でもないイデアルとする.\ref{lem:DVR-C}より$\ideal{m}^n\subset I$となるものがとれる.$A/\ideal{m}^n$に\ref{lem:DVR-B}を使うと,その証明から$I$は$\ideal{m}$の冪になる.
		
		\item \ref{lem:DVR-A}より$\ideal{m}\neq\ideal{m}^2$であるので,$x\not\in\ideal{m}^2$となる$x\in\ideal{m}$がとれる.仮定より$(x)=\ideal{m}^r$となる$r$がとれるが,$x$のとりかたから$r=1$でなければならない.よって$\ideal{m}=(x)$とできるので,すべてのイデアルは$(x^k)$の形に書ける.
		
		\item $(x)=\ideal{m}$なので,\ref{lem:DVR-A}より$(x^k)\neq(x^{k+1})$である.よって0でない任意の$a\in A$について,唯1つ$(a)=(x^k)$となる$k$が定まる.$v(a)=k$とし,$v(ab^{-1})=v(a)-v(b)$として$v$を$k$全体に定義することで離散付値環となる.
	\end{eqv}
\end{proof}

\section{Dedekind整域}

\begin{defi}[Dedekind整域]\index{#Dedekindせいいき@Dedekind整域}
	$A$を整域とする.すべての$A$の0でも$A$でもないイデアル$I$が有限個の素イデアルの積に(一意的に)かけるとき,$A$を\textbf{Dedekind整域}という.
\end{defi}

この条件は$\Z$の素因数分解の拡張を与えるために素イデアルが考案されたという歴史的経緯を考えると,素因数分解ができる環,というように捉えることができ,$\Z$のよい一般化になっている.この節ではDedekind環の公理的特徴づけを与えよう.

まず,判定方法として1次元のNoether整閉整域ならばDedekind整域であることを示そう.実際にはこれが同値条件を与えていることを後に示す(\ref{thm:Dedekind同値条件}).

\begin{lem}\label{lem:Dedekind-1}
	$A$を環,$P,P_0\in\spec A$とし,$
	q_0$を$P_0$準素イデアルとする.$P\neq P_0$であるとき,$q_0A_P=A_P$が成り立つ.
\end{lem}

\begin{proof}
	$x\not\in P$となる$x\in P_0$をとる.このとき$P_0=\sqrt{q_0}$なので,$x^n\in q_0$となる$n\geq1$がとれる.また$x^n\not\in P$なので,$x^n/1\in q_0A_P$は可逆である.よって$q_0A_P=A_P$が成り立つ.
\end{proof}

\begin{lem}\label{lem:Dedekind-2}
	環$A$のイデアル$I,J$について,$I$と$J$が互いに素であることは$\sqrt{I}$と$\sqrt{J}$が互いに素であることと同値.
\end{lem}

\begin{proof}
	逆は明らかなので根基が互いに素なら$I,J$も互いに素であることを見れば十分である.$x+y=1$となる$x\in\sqrt{I},y\in\sqrt{J}$をとる.適当な$n,m$をとって$x^n\in I,y^m\in J$としたとき,$1=(x+y)^{n+m}$であって,これは$k+l=m+n$となる$k,l$についての$x^ky^l$の線形和である.いま$k<n$なら$m<l$が成り立ち,常に$x^ky^l\in I\cup J$が成り立つ.よって各項は$I$か$J$に含まれるから,$I$と$J$は互いに素である.
\end{proof}

\begin{thm}\label{thm:1dimNoether整閉整域ならDedekind}
	$A$が1次元Noether整閉整域ならば$A$はDedekind整域である.
\end{thm}

\begin{proof}
	$I$を$A$の0でも$A$でもないイデアルとする.$A$はNoetherなので,$I=\cap q_i$と無駄のない準素分解ができる.$P_i=\sqrt{q_i}$とおく.このとき$A$は1次元の整域だから$P_i$は極大イデアルであることに注意する.特に$IA_{P_i}$は$A_{P_i}$の0でないイデアルになる.ここで;
	\[IA_{P_i}=\cap (q_iA_{P_i})\]
	であり,$i\neq j$のとき$P_i\neq P_j$なので\ref{lem:Dedekind-1}より$q_jA_{P_i}=A_{P_i}$である.よって$IA_{P_i}=q_iA_{P_i}$が成り立つ.ここで$A_{P_i}$は1次元のNoether局所整閉整域なので,\ref{thm:DVRの特徴づけ}より $q_iA_{P_i}=IA_{P_i}=P_i^{n_i}A_{P_i}$が成り立つ.ここで$\sqrt{P_i^{n_i}}=P_i$であり,$P_i$は極大なので\ref{prop:sqrt{I}が極大なら準素}より$P_i^{n_i}$も$P_i$準素イデアルである.すると;
	\[q_i=q_iA_{P_i}\cap A=P_i^{n_i}A_{P_i}\cap A=P_i^{n_i}\]
	である.ここで$P_i$たちは極大なので互いに素である.よって\ref{lem:Dedekind-2}より$q_i$たちも互いに素なので,中国剰余定理から$I=\prod P_i^{n_i}$とかける.
\end{proof}

逆向きの証明を行うために,分数イデアルという概念を導入しよう.

\begin{defi}[分数イデアル]\index{ぶんすういである@分数イデアル}
	$A$を整域とし,$k$をその商体とする.$k$の0でない$A$部分加群$M$で,ある$0\neq x\in A$が存在して$xM\subset A$となっているとき,$M$を$A$の\textbf{分数イデアル}(fractional ideal)という.
\end{defi}

通常の$A$のイデアルは分数イデアルであることに注意せよ.ここでは$A$の通常のイデアルを区別する目的で\textbf{整イデアル}と呼ぶことがある.

$k$の有限生成$A$部分加群$M$は分数イデアルである.なぜならば,生成元たちを\quo{通分}して,その分母をかければよいからである.

\begin{defi}[可逆イデアル]\index{かぎゃくいである@可逆イデアル}
	$M,N$を$k$の$A$部分加群とする.$MN=A$となっているとき,$M,N$を\textbf{可逆イデアル}(invertible ideal)という.
\end{defi}
	実際には,$M$が可逆イデアルであるとき,$MN=A$となる$N$は$(A:M)=\mkset{u\in k}{uM\subset A}$に一致する.実際$N\subset (A:M)=(A:M)MN\subset AN=N$が成り立つ.特に$(A:M)=M^{-1}$と略記する.
	
	一般の分数イデアル$M$について同様に$M^{-1}$を考えると$MM^{-1}$は$A$の整イデアルになる.次の補題から可逆イデアルは分数イデアルであるから,分数イデアル$M$が可逆であることは$M(A:M)=A$となること,と定式化できる.

\begin{lem}
	$M$が可逆なら有限生成であり,分数イデアルとなる.
\end{lem}

\begin{proof}
	 $MM^{-1}=A$であるので,$x_i\in M$と$y_i\in M^{-1}$がとれて$\sum x_iy_i=1$が成立する.ここで,任意の$x\in M$に対して$y_ix\in A$であるから,$x=\sum(y_ix)x_i$により$M$は$x_i$たちによって生成される.
\end{proof}

可逆性は局所的な性質であることを示そう.
	
\begin{prop}\label{prop:可逆性は局所的}
	分数イデアル$M$について,次は同値である.
	\begin{sakura}
		\item $M$は可逆である.
		\item $M$は有限生成で,任意の$P\in\spec A$について$M_P$は可逆.
		\item $M$は有限生成で,任意の$\ideal{m}\in\spm A$について$M_{\ideal{m}}$は可逆.
	\end{sakura}
\end{prop}

\begin{proof}
	\begin{eqv}[3]
		\item $M$は可逆なので有限生成である.ここで$M(A:M)=A$であるから,\ref{prop:イデアル商は局所化と可換}より$A_P=M_P(A_P:M_P)$が成り立つ.
		\item 明らか.
		\item $I=M(A:M)$とおくと,これは$A$の整イデアルとなる.$I\neq A$とすると,$I\subset \ideal{m}$となる極大イデアル$\ideal{m}$について$M_{\ideal{m}}$は可逆だから$I_{\ideal{m}}=M_{\ideal{m}}(A_{\ideal{m}}:M_{\ideal{m}})=A_{\ideal{m}}$が成り立つ.よって$I\subset\ideal{m}$ではありえず,$I=A$である.
	\end{eqv}
\end{proof}

\begin{lem}\label{lem:局所の場合の可逆性}
	$(A,\ideal{m})$を局所整域とする.$A$の0でないすべてのイデアルが可逆ならば,$A$はDVR,すなわち1次元局所Noether整閉整域である.
\end{lem}

\begin{proof}
	可逆な分数イデアルは有限生成なので,$A$はNoether環である.$A$のすべてのイデアルが$\ideal{m}$の冪になっていればよい.$\Sigma$を$\ideal{m}$の冪でない$A$のイデアル全体の集合とし,これが空でないと仮定する.$I$を$\Sigma$の極大元とする.このとき$I\subsetneq\ideal{m}$でなければならない.よって$\ideal{m}^{-1}I\subsetneq \ideal{m}^{-1}\ideal{m}=A$もイデアルで,$I\subset \ideal{m}^{-1}I$である.ここで,もし$\ideal{m}^{-1}I=I$ならば中山の補題(\ref{thm:NAK})より$I=0$となってしまうから,$I\subsetneq m^{-1}I$である.よって極大性から$\ideal{m}^{-1}I$は$\ideal{m}$の冪になるが,これは$I$が$\ideal{m}$の冪であることを即座に導き矛盾.
\end{proof}

\begin{prop}\label{prop:すべてのイデアルが可逆ならば1-dimNoether整閉整域}
	$A$を整域とする.$A$の0でないすべてのイデアルが可逆ならば,$A$は1次元Noether整閉整域である.
\end{prop}

\begin{proof}
	補題と同様に$A$はNoether環である.$A_P$は局所整域となる.$A_P$のイデアルがすべて可逆であることを示そう.$I$を$A_P$のイデアルとすると,$I\cap A$は$A$のイデアルなので可逆である.よって,$I$はこれを局所化したものだから\ref{prop:可逆性は局所的}より可逆.\ref{lem:局所の場合の可逆性}より$A_P$は1次元局所Noether整閉整域である.よって$\idht P=\dim A_P=1$より$\dim A=1$が従い,\ref{prop:整域の整閉性はlocal}より$A$は整閉であることがわかる.
\end{proof}

\begin{thm}\label{thm:Dedekind同値条件}
	$A$を整域とする.$A$がDedekind整域であることと,$A$が1次元Noether整閉整域であることは同値である.
\end{thm}

\begin{proof}
	\ref{thm:1dimNoether整閉整域ならDedekind}と\ref{prop:すべてのイデアルが可逆ならば1-dimNoether整閉整域}より,$A$がDedekind整域ならば$A$の0でないすべてのイデアルが可逆であることを示せばよい.以下の証明は\cite{matsu}に拠っている.
	
	\begin{step}
		\item $M,N$を0でない分数イデアルとする. $M,N$が可逆であることと$MN$が可逆であることは同値である.
		
		$MN=B$とおく.$M,N$が可逆なら$B$が可逆なことは明らかなので,逆を示そう.$B$が可逆であると仮定する.簡単な計算で$M^{-1}N^{-1}\subset B^{-1}$であることがわかる.また$B^{-1}M\subset N^{-1},B^{-1}N\subset M^{-1}$であるので,$B^{-1}=B^{-1}B^{-1}B=(B^{-1}M)(B^{-1}N)\subset M^{-1}N^{-1}$が成立する.よって$B^{-1}=M^{-1}N^{-1}$であるから;
		\[A=BB^{-1}=(MM^{-1})(NN^{-1})\]
		が従う.ここで$MM^{-1}\subset A$であり,これは$A$のイデアルを成すので$MM^{-1}=NN^{-1}=A$でなければならない.
	
		\item $0\neq P\in\spec A$に対して,$P\subset I$となるイデアル$I$について$IP=P$である.
		
		$a\not\in P$をとる.$I=P+(a)$の形のときに示せば十分である.$I^2$と$P+(a^2)$を素イデアル分解して$I^2=P_1\dots P_r,P+(a^2)=Q_1\dots Q_s$とする.$P\subset I\subset P_i,Q_j$なので,$\bar{A}=A/P$における像$\bar{P_i},\bar{Q_j}$はすべて0でない.このとき;
		\[\bar{P_1}\dots\bar{P_r}=(\bar{a}^2)=\bar{Q_1}\dots\bar{Q_s}\tag{$\ast$}\]
		となる.$\bar{P_i}$が$\bar{P_i}$のなかで極小であるとしてよい.ここで$\bar{Q_j}$のすべてが$\bar{P_i}$に含まれないと仮定すると,$x_j\in \bar{Q_j}-\bar{P_1}$がとれる.一方で$x_1\dots x_s\in \bar{Q_1}\dots\bar{Q_s}=\bar{P_1}\dots\bar{P_r}\subset\bar{P_1}$より,$\bar{P_1}$が素であることに矛盾する.よって$\bar{Q_1}\subset\bar{P_1}$としてよい.ここで$\bar{P_1}\dots\bar{P_r}\subset\bar{Q_1}$だから,同様にして$\bar{P_i}\subset\bar{Q_1}$となる$i$がある.このとき$\bar{P_1}$の極小性から$\bar{P_1}=\bar{Q_1}$である.
		
		また,$(\ast)$において$(\bar{a}^2)$は可逆なイデアルなので,Step1 から$\bar{P_i},\bar{Q_j}$はすべて可逆である.よって両辺に$\bar{P_1}{}^{-1}$をかけることで$\bar{P_2}\dots\bar{P_r}=\bar{Q_2}\dots\bar{Q_s}$がわかる.上と同様にして$r=s$であり,$\bar{P_i}=\bar{Q_i}$となるように並び替えることができることがわかる.
		
		よって$P_i=Q_i$が従うから,$P+(a^2)=P^2+aP+(a^2)$である.よって,任意の$x\in P$は;
		\[x=y+az+a^2t\quad y\in P^2,z\in P,t\in A\]
		とかけるが,このとき$a^2t\in P$で$a\not\in P$から$t\in P$である.よって$P\subset P^2+aP=IP\subset P$であるから,主張が従う.
		
		\item $0\neq x\in A$に対して,$(x)=P_1\dots P_r$と素イデアル分解したとき,各$P_i$は極大イデアルである.
		
		実際,イデアル$I$が$P_i\subset I$を満たすならStep2より$IP_i=P_i$であるが,$P_i$はStep1より可逆だから$I=A$である.
		
		\item すべての$A$のイデアルは可逆である.
		
		任意の0でも$A$でもないイデアル$I$について,$I=P_1\dots P_r$と素イデアル分解する.各$P_i$が可逆なら$I$も可逆になるので,すべての$0\neq P\in\spec A$が可逆ならよい.
		
		$0\neq P\in\spec A$と$0\neq x\in P$をとる.Step3より$(x)=Q_1\dots Q_s$と分解したとき各$Q_i$は極大イデアルである.ここでどれかの$Q_i$は$P$に含まれるから,$Q_i=P$が成り立つ.よって$P$は可逆である.
	\end{step}
\end{proof}

\section{Krullの次元定理}
\ref{defi:Krull次元}で環のKrull次元を定義した.素イデアルの長さをもってして環の大きさを計ったわけであるが,この節ではPoincar\'e\textbf{級数}を用いた別の\quo{計りかた}を,次数付きNoether環に,そして\ref{defi:随伴次数環}で定義した随伴次数環を用いてNoether局所環に対して考えよう.
まず,一般に\textbf{加法的関数}というものを定義する.
\begin{defi}[加法的関数]\index{かほうてきかんすう@加法的関数}
	ある加群の族$\{M_i\}$上定義された$\Z$への関数$\lambda$で,任意の短完全列;
	\[\ses{M_1}{M_2}{M_3}\]
	について,$\lambda(M_1)-\lambda(M_2)+\lambda(M_3)=0$であるとき,$\lambda$を\textbf{加法的}(additive)関数という.
\end{defi}

さて,$A=\bigoplus_{n\geq0}A_n$を次数付きNoether環とする.このとき,$A=A_0[x_1,\dots,x_s]$とできる(\ref{prop:次数付き環のNoether性}).ここで$x_i$を斉次元で取り替え,それらの次数を$k_i$としよう.また,$M=\bigoplus_{n\geq0}M_n$を有限生成$A$加群とし,斉次な生成元を$m_1,\dots,m_t$,それぞれの次数は$r_1,\dots,r_t$とする.このとき$M_n$のすべての元は$f_i\in A_{n-r_i}$によって$\sum f_im_i$の形で書けるので,$M_n$は有限生成$A_0$加群である.以後,しばらくはこの記号で話をすすめる.

\begin{defi}[Poincar\'e級数]\index{#Poincareきゅうすう@Poincar\'e級数}
	次数付き環$A$,次数付き有限生成$A$加群$M$について,$\lambda$をすべての有限生成$A_0$加群からなる集合族上の加法的関数とする.このとき;
	\[P(M,t)=\sum_{n=0}^\infty \lambda(M_n)t^n\]
	を$M$の\textbf{Poincar\'e級数}(Poincar\'e series)という.
\end{defi}

Poincar\'e級数の考察には,次の定理により与えられる表示が強力である.

\begin{thm}[Hilbert, Serre]
	$P(M,t)$は有理関数である.特に$\Z$係数の多項式$f(t)\in\Z[t]$が存在して;
	\[P(M,t)=\frac{f(t)}{\prod_{i=1}^s(1-t^{k_i})}\]
	が成り立つ.
\end{thm}

\begin{proof}
	$A=A_0[x_1,\dots,x_s]$とおく.$s$についての帰納法で示す.
	\begin{step}
		\item $s=0$のとき.
		
		$A=A_0$なので,$M$は有限生成$A_0$加群となり,十分大きな$M_n$について$M_n=0$である.よって$P(M,t)$は多項式となる.
		
		\item $s-1$まで正しいとする.
		
		$\begin{tikzcd}
			M_n\nxcell[\times x_s]M_{n+k_s}
		\end{tikzcd}$の核,余核を$K_n,L_{n+k_s}$とおくと;
		\[\begin{tikzcd}
			0\nxcell K_n\nxcell M_n\nxcell[\times x_s]M_{n+k_s}\nxcell L_{n+k_s}\nxcell0
		\end{tikzcd}\]
		が完全である.$K=\bigoplus K_n,L=\bigoplus L_n$とおくと,$M$は有限生成$A$加群なので,その部分加群,剰余加群である$K,L$も有限生成である.どちらも$x_s$で零化されるので$A_0[x_1,\dots,x_{s-1}]$加群である.ここで$\lambda$が加法的なので,$(\ast)$において;
		\[\lambda(K_n)-\lambda(M_n)+\lambda(M_{n+k_s})-\lambda(L_{n+k_s})=0\]
		である.$t^{n+k_s}$をかけて,$n$について加えると,補正項を$g(t)$として;
		\[t^{k_s}P(K,t)-t^{k_s}P(M,t)+P(M,t)-P(L,t)-g(t)=0\]
		となり;
		\[(1-t^{k_s})P(M,t)=-t^{k_s}P(K,t)+P(L,t)+g(t)\]
		となるので,帰納法の仮定から条件を満たす$f(t)$が見つかる.
	\end{step}
\end{proof}

もっとも簡単な,かつ多項式のように重要な$k_1=\dots=k_s=1$の場合を考えてみよう.このとき$P(M,t)=f(t)(1-t)^{-s}$とかけるが,$f(1)=0$ならば約分して$f$をとりかえることで;
\[P(M,t)=\frac{f(t)}{(1-t)^d},\quad f(1)\neq0\]
とできる.ここで$d$は$P(M,t)$の$1$における極の位数であることに注意する.$(1-t)^{-1}=1+t+t^2+\cdots$の両辺を$t$で微分して(あるいは$(1+t+^t2+\cdots)^{d}$を展開して);
\[(1-t)^{-d}=\sum_{n=0}^\infty\binom{d+n-1}{d-1} t^n\]
を得る.よって,$f(t)=\sum_{k=0}^N a_kt^k$とすると;
\[\lambda(M_n)=a_0\binom{d+n-1}{d-1}+a_1\binom{d+n-2}{d-1}+\dots+a_k\binom{d+n-k-1}{d-1}\tag{$\ast$}\]
とかけている($m<d-1$なら$\binom{m}{d-1}=0$とする($\ast$)の先頭項(最高次の係数と次数)は$f(1)/(d-1)! n^{d-1}$である.これをまとめると次のようになる.

\begin{defi}[Hilbert多項式]\index{#Hilbertたこうしき@Hilbert多項式}\label{defi:Hilbert多項式}
	$k_1=\dots=k_s=1$のとき,有理係数で$d-1$次の($n$に関する)多項式$\varphi_M(n)$が存在して,$N
	\leq n$ならば$\lambda(M_n)=\varphi_M(n)$が成り立つ.$\varphi_M$を,$M$の$\lambda$に関する\textbf{Hilbert多項式}(function,polymonial)という.
\end{defi}

$A_0$がArtin(特に体)のとき,$M$は有限生成だからArtinかつNoether的なので\ref{prop:有限な組成列の同値条件}より組成列の長さ$l(M)$は有限である.そして$l(M)$は加法的である(確かめよ).$x_i$を$A_0$上の不定元として$A=A_0[x_1,\dots,x_s]$とすると,$A_n$は$x_1^{m_1}\dots x_s^{m_s} (\sum m_i=n)$により生成される.$n$次の単項式は$\binom{n+s}{s}$個あるので,$l(A_n)=l(A_0)\binom{n+s}{s}$となり;
\[\varphi_A(n)=\frac{l(A_0)}{s!}(n+s)(m+s-1)\dots(n+1)\]
である.

次に$(A,\ideal{m})$をNoether局所環,$M$を有限生成$A$加群とし,随伴次数環について考えよう.$I$を$\ideal{m}$準素イデアルとし,$G(A)=(G_I(A)=)\bigoplus I^n/I^{n+1},G(M)=\bigoplus M_n/M_{n+1}$を考える.このとき$G_0(A)=A/I$は$I$が$\ideal{m}$準素なのでArtinである.ここでフィルターについて思い出してみよう.

\begin{prop}\label{prop:フィルターとHilbert多項式}
	上の設定のもとで,$\{M_n\}$を$M$の安定している$I$フィルターとする.$x_1,\dots,x_s$を$I$の極小の生成系とすると;
	\begin{sakura}
		\item $l(M/M_n)$は有限である.
		\item すべての十分大きな$n$について,次数が$s$以下の多項式$g(n)$が存在して$l(M/M_n)=g(n)$となる.
		\item $g(n)$の先頭項は$M$と$I$のみに依存する(フィルター$\{M_n\}$には依存しない).
	\end{sakura}
	が成り立つ.また(i), (ii)は$\{M_n\}$の安定性を仮定せずに成り立つ.
\end{prop}

\begin{proof}
	\begin{sakura}
		\item \ref{prop:G(hat{A})はNoether},\ref{prop:フィルターが安定していればG(M)は有限生成}より$G(A)$はNoetherで,$G(M)$は有限生成$G(A)$加群である.各$G_n(M)=M_n/M_{n+1}$は$I$で零化されるのでNoether $A/I$加群であるから,$A/I$はArtinなので$l(M_n/M_{n+1})$は有限である.ここで$l(M/M_n)=\sum_{i=1}^n l(M_{i-1}/M_i)$であるから,$l(M/M_n)$も有限である.
		\item $I=(x_1,\dots,x_s)$のとき,$G(A)=A/I[\bar{x_1},\dots,\bar{x_s}]$であった.よって \ref{defi:Hilbert多項式}の条件を満たし,次数が$s-1$次以下の$\varphi_{G(M)}(n)$が存在して,$l(M_n/M_{n+1})=\varphi_{G(M)}(n)$となる.(i)より$l(M/M_{n+1})-l(M/M_n)=\varphi_{G(M)}(n)$なので,題意が従う.
		\item $\{M_n'\}$を$M$の安定している$I$フィルターとすると,$g'(n)=l(M/M_n')$とおく.\ref{lem:安定しているフィルターは有界な差を持つ}より$\{M_n\},\{M_n'\}$は有界な差を持つ.よって,ある$n_0\geq0$が存在して,すべての$n$について;
		\[g'(n)\leq g(n+n_0),g(n)\leq g'(n+n_0)\]
		が成り立つ.はさみうちの原理から$\lim_{n\to\infty} g(n)/g'(n)=1$となり,先頭項は一致する.
	\end{sakura}
\end{proof}

フィルター$\{I^n M\}$に対応する$g(n)$は$\chi_I^M(n)$で表される.

\begin{defi}[特性多項式]\index{とくせいたこうしき@特性多項式}
	$M=A$のとき,$\chi_I(n)(=\chi_I^A(n))$を$\ideal{m}$準素イデアル$I$の\textbf{特性多項式}(characteristic polynomial)という.
\end{defi}

$\chi_I(n)$の次数は$I$の極小の生成系の個数以下であることに注意する.

\begin{prop}
	$I,I'$を$\ideal{m}$準素イデアルとすると,$\chi_I(n)$と$\chi_I'(n)$の次数は等しい.
\end{prop}

\begin{proof}
	$\deg\chi_I(n)=\deg\chi_{\ideal{m}}(n)$を示せばよい.$A$がNoetherで$I$が$\ideal{m}$準素なので$\sqrt{I}=\ideal{m}$だから,ある$r\geq0$が存在して$\ideal{m}^r\subset I\subset\ideal{m}$である.よって$\ideal{m}^{nr}\subset I^n\subset\ideal{m}^n$なので;
	\[\chi_{\ideal{m}}(n)\leq\chi_I(n)\leq\chi_{\ideal{m}^r}(nr)\]
	である.右辺と左辺の次数は等しいので題意が成り立つ.
\end{proof}

$\deg\chi_I(n)=d(A)$とかく.また,先程まで$s$とかいていた$I$の極小の生成系の個数を$\delta(A)$とかくことにする.$d(A)$を$A$のHilbert--Samuel次元,$\delta(A)$を$A$の座標次元と呼ぶ.この節の残りの目標はNoether局所環$(A,\ideal{m})$について,次のKrullの次元定理;
\begin{thm}[Krullの次元定理]\index{#Krullのじげんていり@Krullの次元定理}\label{thm:Krullの次元定理}
	$(A,\ideal{m})$をNoether局所環とする.$\chi_{\ideal{m}}(n)=l(A/\ideal{m}^n)$の次数を$d(A),A$の$\ideal{m}$準素イデアルの極小の生成系の個数を$\delta(A)$とおくと;
	\[\dim A=d(A)=\delta(A)\]
	が成り立つ.
\end{thm}

を示すことである.先程も注意したように,$\delta(A)\geq d(A)$が成り立っている.次に$d(A)\geq\dim A$を示していこう.

\begin{lem}\label{lem:商の特性多項式を上から抑える}
	Noether局所環$(A,\ideal{m})$と$\ideal{m}$準素イデアル$I$について,$M$を有限生成$A$加群,$x\in A$を$M$の零因子でない元として$M'=M/xM$とおく.このとき$\deg\chi_I^{M'}\leq\deg\chi_I^M-1$である.特に,$M=A$としたとき$d(A/(x))\leq d(A)-1$が成り立つ.
\end{lem}

\begin{proof}
	$N=xM$とおき,$N_n=N\cap I^nM$とする.このとき;
	\[\ses{N/N_n}{M/I^nM}{M'/I^nM'}\]
	が完全である.Artin--Reesの補題(\ref{prop:Artin--Reesの補題})より,$\{N_n\}$は$N$の(安定している) $I$フィルターだから,\ref{prop:フィルターとHilbert多項式}より,十分大きな$n$について$l(M/M_n)=g(n)$となる$g(n)$がとれる.同様に,十分大きな$n$をとれば$g(n)-\chi_I^M(n)+\chi_I^{M'}(n)=0$が成り立つ.また,仮定より$N=xM\cong M$であるから,\ref{prop:フィルターとHilbert多項式}(iii)より$\deg g(n)=\deg\chi_I^M$であるので,主張が従う.
\end{proof}

\begin{prop}\label{prop:d(A)geq dim A}
	$d(A)\geq\dim A$である.
\end{prop}

\begin{proof}
	$d(A)=d$についての帰納法で示す.
	\begin{step}
		\item $d=0$のとき.
		
		十分大きな$n$に対して$l(A/\ideal{m}^n)$は定数である.これは$\ideal{m}^n=\ideal{m}^{n+1}$であるので,中山の補題(\ref{thm:NAK})より$\ideal{m}^n=0$である.よって\ref{prop:局所環がm^n=0ならArtin}から$A$はArtinであり,$\dim A=0$である.
		
		\item $d-1$まで正しいとする.
		
		$\dim A=r$とおき,$P_0\subset P_1\subset\dots\subset P_r$を$A$の素イデアルの列とする.$x\in P_1\setminus P_0$をとる.$A'=A/P_0$とおき,$x'$を$x$の$A'$への像とする.$A'$は整域で,$x'\neq0$であるので,\ref{lem:商の特性多項式を上から抑える}より$d(A'/(x')))\leq d(A')-1$が成り立つ.ここで$\ideal{m}'$を$\ideal{m}$の$A'$への像とすると,$(A',\ideal{m}')$はNoether局所環である.ここで$l(A'/\ideal{m}')\leq l(A/\ideal{m})$であるので$d(A')\leq d(A)$である.よって$d(A'/(x'))\leq d(A)-1$であるから,帰納法の仮定から$\dim A'/(x')\leq d-1$である.$P_1,\dots,P_r$の$A'/(x')$への像は長さ$r-1$の素イデアルの列をなし,$r\leq d$が成り立つ.よって示された.
	\end{step}
\end{proof}

\begin{cor}
	Noether環$A$の素イデアル$P$について$\idht P<\infty$である.
\end{cor}

\begin{proof}
	Noether局所環$(B,\ideal{m})$について,$\delta(B)<\infty$が明らかに成り立つ.よって$\dim B<\infty$である.よって,Noether環$A$とその素イデアル$P$について,$A_P$は局所環となるので,$\idht P=\dim A_P<\infty$である.
\end{proof}

しかしながら,無限次元のNoether環(整域)が存在することに注意しなければならない(\ref{ex:無限次元Noether環}).

\begin{lem}
	$A$を$1\leq\dim A$なる有限次元Noether環とする.このとき,単元でない$x\in A$が存在して,$\dim A/(x)<\dim A$が成り立つ.
\end{lem}

\begin{proof}
	$P\in\spec A$を$\idht P>0$となるものとする.ここで,$A$がNoetherなので$A$の極小な素イデアルは有限個しか存在しない(\ref{cor:Noether環の極小素イデアルは有限個}).よって,それらを$P_1,\dots, P_n$とすると,すべての$i$について$P\not\subset P_i$であるから,Prime avoidance(\ref{lem:Prime avoidance})より$P\not\subset\bigcup_{i=1}^nP_i$が成り立つ.よって$x\in P\setminus\bigcup P_i$をとると,$\dim A/(x)<\dim A$である.

\end{proof}
\begin{prop}\label{prop:dim A geq delta(A)}
	$(A,\ideal{m})$をNoether局所環とすると,$\dim A\geq s$個の元からなる$\ideal{m}$準素イデアル$I$が存在する.特に$\delta(A)\leq\dim A$である.
\end{prop}

\begin{proof}
	$\dim A=d$についての帰納法で示す.
	
	\begin{step}
		\item $d=0$のとき.
		
		$A$はArtinなので,\ref{prop:局所環がm^n=0ならArtin}より,ある$n$が存在して$\ideal{m}^n=0$である.よって$\ideal{m}=\nil A$が成り立ち,$0$は$\ideal{m}$準素イデアルである.ゆえに$\delta(A)=0$がわかる.
		
		\item $d-1$まで正しいとする.
		
		補題より,単元でない$x\in A$で,$\dim A/(x)\leq d-1$となるものがとれる.$A'=A/x$とおくと,これはNoether局所環である.$d'=\dim A'$とすると,帰納法の仮定より$d'\geq s$個の元で生成される$\ideal{m}'$準素イデアル$I'=(x_1',\dots,x_s')$が存在する.ここで$I=(x_1,\dots,x_s,x)$が$\ideal{m}$準素イデアルであること,すなわち$\ideal{m}\subset\sqrt{I}$であることを示そう.
		
		任意の$y\in\ideal{m}$をとる.もし$y\in (x)$のときは$y\in I$であるので,$y\not\in(x)$すなわち$y'$を$A'$への像とすると$y'\neq0$としてよい.$I'$は$\ideal{m}'$準素なので,ある$n$が存在して$y'^n\in I'$である.よって$y'^n=a_1'x_1'+\dots+a_s'x_s'$とかける.よって$y^n-a_1x_1+\dots+a_sx_s\in (x)$であるから,$y^n\in I$である.
		
		よって$\delta(A)=s+1\leq d'+1\leq (d-1)+1=d$であることがわかった.
	\end{step}
\end{proof}

以上より,\ref{prop:フィルターとHilbert多項式},\ref{prop:d(A)geq dim A},\ref{prop:dim A geq delta(A)}によってKrullの次元定理(\ref{thm:Krullの次元定理})が示された.

\section{Krullの次元定理の系たち}

準素イデアルの節で注意しておいたことだが,極小素イデアルについてもう一度注意しておこう.Noether環$A$とそのイデアル$I$について$V(I)$の極小元,つまり$I$を含む素イデアルで極小なものは$\ass(A/I)$の元であるから有限個である.また$I=\bigcap q_i$と準素分解したとき,$P=\sqrt{q_i}$となる$i$が存在する.
\begin{thm}[Krullの標高定理]\index{#Krullのひょうこうていり@Krullの標高定理}\label{thm:Krullの標高定理}
	$A$をNoether環とし,$\nitem{f}\in A$とする.イデアル$(\nitem{f})$の極小素イデアル$P$について$\idht P\leq n$が成り立つ.
\end{thm}

\begin{proof}
	$P$のとりかたから,$(\nitem{f})=\bigcap_{i=1}^m q_i$と準素分解したとき,$P=\sqrt{q_j}$となる$j$がある.ここで$A_P$において$(\nitem{f})A_P$が$PA_P$準素イデアルであることを示そう.そのために$\sqrt{(\nitem{f})A_P}=PA_P$を示せば十分である.任意の$x/s\in PA_P$をとる.$x\in P=\sqrt{q_j}$より,ある$n$がとれて$x^n\in q_j$である.ここで$x^n\not\in q_i$となる$i$たちをまとめて$i_1,\dots,i_r$とする.$\sqrt{q_i}=P_i$とおくと,$P$は極小なので$P_i\not\subset P$が成り立つ.よって$\nitem[r]{i}$について,ある$y_k\in P_{i_k}$が存在して$y_k\not\in P$である.$y_k\in\sqrt{q_{i_k}}$よりある$n_k$が存在して$y_k^{n_k}\in q_{i_k}$である.$n,\nitem[r]{n}$の最大値を$n$ととりなおし,$xy_1\dots y_r$を$x$,$sy_1\dots y_r$を$s$と置き直すと$x^n=\bigcap_{i=1}^m q_i=(\nitem{f})$であり,$s\not\in P$であるので,$x/s\in\sqrt{(\nitem{f})A_P}$である.よってKrullの次元定理から$\idht P=\dim A_P\leq n$となる.
\end{proof}

\begin{cor}[Krullの単項イデアル定理]\index{#Krullのたんこういであるていり@Krullの単項イデアル定理}
	$A$をNoether環とし,$x$を$A$の零因子でも単元でもない$A$の元とする.このとき,$(x)$のすべての極小素イデアル$P$について$\idht P=1$である.
\end{cor}

\begin{proof}
	標高定理より$\idht P\leq1$である.$\idht P=0$であるとすると,$\spec A_P=\{P\}$であるので,$\nil A_P=PA_P$である.$x\in P$であるので,$x$は零因子でないことに矛盾する.よって$\idht P=1$である.
\end{proof}

\begin{thm}[Krullの標高定理の逆]
	$A$をNoether環とし,$P\in\spec A$が$\idht P=n$であったとすると,ある$a_1,\dots,a_n\in P$が存在して$P$は$(a_1,\dots,a_n)$の極小素イデアルとなる.
\end{thm}

\begin{proof}
	$A_P$は$n$次元局所環なので,$n$個の元で生成される$PA_P$準素イデアル$(x_1/s_1,\dots,x_n/s_n)$が存在する.このとき$PA_P$は$(x_1/s_1,\dots,x_n/s_n)$の極小素イデアルになる.ここで$a_i=s_1\dots s_n(x_i/s_i)\in A$とおくと,$s_1\dots s_n$は$A_P$の単元なので,イデアルとして$(a_1,\dots,a_n)A_P=(x_1/s_1,\dots,x_n/s_n)$である.よって$P$は$(a_1,\dots,a_n)$の極小素イデアルとなる.
\end{proof}

\begin{prop}\label{prop:標高定理の系}
	$A$をNoether環とし,$P\in\spec A$が$(a_1,\dots,a_n)$の極小素イデアルなら$\idht (P/(a_1,\dots,a_i))=n-i (1\leq i\leq n)$である.
\end{prop}

\begin{proof}
	$P$の$\bar{A}=A/(a_1,\dots,a_n)$における像を$\bar{P}$とおく.$\idht \bar{P}=r$としよう.$\bar{A}$において$\bar{P}$は$(\bar{a_{i+1}},\dots,\bar{a_n})$の極小素イデアルである.よってKrullの標高定理から$r\leq n-i$である.
	
	また,Krullの標高定理の逆より$\bar{x_1},\dots,\bar{x_r}\in\bar{A}$が存在して$\bar{P}$は$(\bar{x_1},\dots,\bar{x_r})$の極小素イデアルである.よって$P$は$(a_1,\dots,a_i,x_1,\dots,x_r)$の極小素イデアルとなり,$n\leq r+i$である.よって$r=n-i$が示された.
\end{proof}

\begin{prop}\label{prop:dim A/xA=dim A-1}
	Noether局所環$(A,\ideal{m})$と$A$の零因子でない$x\in\ideal{m}$について,$\dim A/(x)=\dim A-1$が成り立つ.
\end{prop}

\begin{proof}
	$d=\dim A/(x)$とおく.\ref{lem:商の特性多項式を上から抑える}より$d\leq \dim A-1$である.一方,$\ideal{m}$の$A/(x)$への像を$\bar{\ideal{m}}$とすると,局所環$(A/(x),\bar{\ideal{m}})$において$d$個の元で生成される$\bar{\ideal{m}}$準素イデアルがある.$x_1,\dots,x_d\in A$をそれらの$A/(x)$への像$\bar{x_i}$が$\bar{\ideal{m}}$準素イデアルを生成するような元としよう.このとき$(x,x_1,\dots,x_d)$は$\ideal{m}$準素イデアルとなる.よって$\dim A\leq d+1$となり,示された.
\end{proof}

\begin{thm}
	Noether局所環$(A,\ideal{m})$において,$\ideal{m}$進完備化を$\widehat{A}$とすると$\dim\widehat{A}=\dim A$である.
\end{thm}

\begin{proof}
	\ref{lem:完備化に関してのもろもろの補題}より$A/\ideal{m}^n=\widehat{A}/{\widehat{\ideal{m}}}^n$であるので,$\chi_{\ideal{m}}(n)=\chi_{\widehat{\ideal{m}}}(n)$が成り立つ
\end{proof}

\begin{prop}\label{prop:イデアルの高さと生成系の個数}
	Noether環$A$のイデアル$I$について,$\idht I=r$ならばある$\nitem[r]{f}\in I$が存在して,任意の$1\leq i\leq r$について$\idht (\nitem[i]{f})=i$が成り立つ.
\end{prop}

\begin{proof}
	まず,$f_1$を$A$の零因子で単元でもない元とすると,Krullの単項イデアル定理より$\idht (f_1)=1$である.そこで,ある$i<r$について$\nitem[i]{f}$がすべての$1\leq j\leq i$について$\idht (\nitem[j]{f})=j$となるように選ばれているとする.素イデアル$P\in V(\nitem[i]{f})$について$\idht P=i$であるものは$(\nitem[i]{f})$の極小素因子なので\ref{cor:Noether環の極小素イデアルは有限個}より有限個しかない.また,$I\subsetneq P$であるので,Prime avoidance(\ref{lem:Prime avoidance})より$f_{i+1}\in I$を$f_{i+1}\not\in\bigcup_{P\in V(\nitem[i]{j}),\idht P=j} P$となるようにとることができる.すると$\idht (\nitem[i+1]{f})=i+1$が成り立つ.よって帰納的に主張が従う.
\end{proof}

\begin{thm}
	環$A$について,$\dim A+1\leq\dim A[X]\leq 2\dim A+1$が成り立つ.
\end{thm}

\begin{proof}
	$\dim A+1\leq\dim A[X]$は明らかである.$\dim A[X]=d$とおこう.$A[X]$の素イデアル鎖$P_0\subsetneq P_1\subsetneq\dots\subsetneq P_d$について,$P_i'=P_i\cap A$とおいて$A$の素イデアル鎖$P_0'\subset P_1'\subset\dots\subset P_d'$を考える.この鎖の長さの取りうる値の最小値を求めよう.
	
	$P\in\spec A[X]$と$P'=P\cap A\in\spec A$を考える.積閉集合$A-P'$による$A,A[X]$の局所化はそれぞれ$A_P',A_P'[X]$である.特に$\spec A_P'[X]$は$\mkset{Q\in\spec A[X]}{(Q\cap A)\subset P'}$と対応する.また,$A_P'[X]$において$V(P'A_P'[X])$は$\spec (A_P'[X]/P'A_P'[X])=\spec (k(P)[X])$と対応するので,これらから$P$について$\mkset{Q\in\spec A[X]}{Q\cap A=P'}$からなる$A[X]$の素イデアル鎖の長さは高々1である.
	
	さて,$A$の素イデアル鎖$P_0'\subset\dots\subset P_d'$にもどると,もし$P_{i-1}'=P_i'$ならば先の議論より$P_{i-2}',P_{i+1}'$は$P_i'$と異なる.よってこの鎖は最短でも長さが$d$が偶数なら$d/2$,奇数なら$(d-1)/2$である.よって$(d-1)/2\leq \dim A$すなわち$d\leq 2\dim A+1$が成り立つ.
\end{proof}

次の結果は\ref{cor:多項式環の次元}の拡張である.

\begin{cor}\label{cor:Noether多項式の次元}
	$A$がNoether環ならば,$\dim A[X]=\dim A+1$が成り立つ.帰納的に$\dim A[X_1,\dots,X_n]=\dim A+n$である.
\end{cor}

\begin{proof}
	$d=\dim A[X]$としたとき,$A[X]$の素イデアル鎖$P_0\subsetneq P_1\subsetneq\dots\subsetneq P_d$について,先の定理と同様に$A$の素イデアル鎖$P_0'\subset\dots\subset P_d'$を考える.このとき$d-1\leq\idht P_d'$を示したい.
	
	$P'\in\spec A$について,$P'A[X]$を係数がすべて$P'$の元である多項式全体とすると,これは$A[X]$の素イデアルになる.$\idht P'=r$とし,$A$の素イデアル鎖$Q_0'\subsetneq Q_1'\subsetneq\dots\subsetneq Q_r'=P'$を考えよう.このとき$Q_0'A[X]\subsetneq Q_1'A[X]\subsetneq\dots\subsetneq Q_r'A[X]$は$A[X]$の素イデアル鎖になるので$r\leq\idht P'A[X]$が成り立つ.また$\idht P'=r$より,$P$はある$f_1,\dots,f_r$について$I=(\nitem[r]{f})$の極小素イデアルである.このとき$P'A[X]$は$I[X]$の極小素イデアルになるので,Krullの標高定理より$\idht P'A[X]\leq r$である.
	
	すると,$A$の素イデアル鎖$P_0'\subset\dots\subset P_d'$において$\idht P_d'=\idht P_d'A[X]$であり,$P_d'A[X]\cap A=P_d\cap A$であるので,$P_d'A[X]\subset P_d$でありかつ前定理の議論からその間に素イデアルはない.よって$d-1\leq\idht P_d'A[X]=\idht P_d'\leq\dim A$である.よって$d\leq\dim A+1$が従う.
\end{proof}

\section{正則環}

以後2つの節,正則環とCohen--Macaulay環については,深い結果を証明するにはホモロジー代数の考え方が不可分なので,定義とそこから予見される性質を紹介するに留める.

Noether局所環$(A,\ideal{m},k)$について,$\dim A=d$とすると$A$の$\ideal{m}$準素イデアルは少なくとも$d$個の元で生成される.ちょうど$d$個の元で$\ideal{m}$準素イデアルが生成されているときを考えよう.
\begin{defi}[巴系]\index{ぱけい@巴系}
	$(A,\ideal{m})$を$d$次元Noether局所環とする.$x_1,\dots,x_d\in\ideal{m}$が$\ideal{m}$準素イデアルを生成するとき$x_1,\dots,x_d$を$A$の\textbf{巴系}(system of parameters)という.
\end{defi}

また\ref{prop:Atimac_prop_2.8}より,$\ideal{m}$を生成するのに必要な元の個数は$\dim_k\ideal{m}/\ideal{m}^2$である.$\dim_k\ideal{m}/\ideal{m}^2$を$A$の\textbf{埋め込み次元}(embedding dimension)といい$\emdim A$と表す.次元定理より$\dim A\leq\emdim A$が成り立つ.等号が成り立つとき,すなわち$\ideal{m}$が$\dim A$個の元で生成されているとき$A$を\textbf{正則局所環}という.

\begin{defi}[正則局所環]\index{せいそくきょくしょかん@正則局所環}
	$(A,\ideal{m})$をNoether局所環とする.$d=\dim A$個の元$x_1,\dots,x_d\in A$が存在して$\ideal{m}=(\nitem[d]{x})$となっているとき$A$を\textbf{正則局所環}(regular local ring)といい,$\ideal{m}$を生成する巴系を\textbf{正則巴系}という.
\end{defi}

環$A$のすべての素イデアル$P$による局所化$A_P$が正則局所環であるような$A$を,\textbf{正則環}(regular ring)という.\index{せいそくかん@正則環}
\begin{prop}\label{prop:正則局所環は落ちる}
$d$次元Noether局所環$(A,\ideal{m})$について,$x_1,\dots,x_i$が$A$の正則巴系の1部分であることと$A/(x_1,\dots,x_i)$が$d-i$次の正則局所環であることは同値.
\end{prop}

\begin{proof}
	\begin{eqv}
		\item $A/(x_1,\dots,x_i)$の極大イデアルは$x_{i+1},\dots,x_n$の像で生成されており,\ref{prop:標高定理の系}より$\dim A/(x_1,\dots,x_i)=d-i$である.よって正則局所環となる.
		
		\item $A/(x_1,\dots,x_i)$の極大イデアル$\ideal{m}'=\ideal{m}/(x_1,\dots,x_i)$が$y_1,\dots,y_{n-i}$の像で生成されているとすると$\ideal{m}=(x_1,\dots,x_i,y_1,\dots,y_{n-i})$である.
	\end{eqv}
\end{proof}

\begin{thm}\label{thm:正則局所環は整域}
	正則局所環$(A,\ideal{m})$は整域である.
\end{thm}

\begin{proof}
	$\dim A$についての帰納法で示す.$\dim A=0$のときは$A$が体であるので明らか.$\dim A\geq 1$とし,$\dim A-1$次までの正則局所環は整域であると仮定する.$A$の高さ$0$の素イデアルを$P_1,\dots,P_r$とする.\ref{prop:局所環がm^n=0ならArtin}より$\ideal{m}\not\subset\ideal{m}^2$であり$\ideal{m}\not\subset P_i$であるので,Prime avoidanceよりある$x\in\ideal{m}$が存在して$x\not\in\ideal{m},P_1,\dots,P_r$が成り立つ.すると$x$の$\ideal{m}/\ideal{m}^2$における像$\bar{x}$は$0$でないので延長して$\{\bar{x},\bar{y_1},\dots,\bar{y_{n-1}}\}$を$\ideal{m}/\ideal{m}^2$の基底にできる.このとき$x,y_1,\dots,y_{n-1}$は$A$の正則巴系となるので,前命題より$A/xA$は$n-1$次元の正則局所環である.仮定より整域なので$(x)$は素イデアル.すると$P_i$たちは極小なので$P_i\subset(x)$となる$i$がある.すると任意の$y\in P_i$について$y=ax\in P_i$となる$a\in A$が存在し,$x\not\in P_i$なので$a\in P_i$である.よって$P_i=xP_i$となり,中山の補題から$P_i=0$である.よって$A$は整域である.
\end{proof}

\ref{thm:DVRの特徴づけ}より次の結果が従う.

\begin{cor}
	$(A,\ideal{m})$が$1$次元の正則局所環であることとDVRであることは同値.
\end{cor}

DVRは整閉なので,1次元の正則局所環は整閉である.またより強くDVRはUFDでもあるが,次の結果が知られている.

\begin{thm}[Auslander--Buchsbaumの定理]\index{#Auslander--Buchsbaumのていり@Auslander--Buchsbaumの定理}\label{prethm:Auslander-Buchsbaum}
	正則局所環はUFDである.
\end{thm}

特にすべての正則局所環は整閉整域である.この結果の証明にはホモロジー代数的手法を本質的に必要とするため,ホモロジー代数を導入した後に証明を与える(\ref{thm:Auslander--Buchsbaumの定理}).また,一見するとイデアル論の範疇で(ひょっとすると簡単に?)示せそうに見えるが,ホモロジー代数を要する結果として;

\begin{thm}[Serreの定理]\index{#Serreのていり@Serreの定理}\label{prethm:Serre}
	$A$を正則局所環とすると,任意の$P\in\spec A$について$A_P$も正則局所環である.
\end{thm}
がある(証明は\ref{thm:Serreの定理}).この定理の証明はSerreによる方法以外には証明法が知られていない(\cite{goto}).このように1960年代は,Serre, Auslander, Bachsbaum, Bass, Grothendieck らによってホモロジー的な手法が本格的に導入され,可換環論は大規模に発展することとなった.

次の節ではホモロジー代数を導入する前に,現在の可換環論の中心をなすCohen--Macaulay性について紹介することにする.

\section{Cohen--Macaulay 加群}

重要な局所環のクラスの1つに\textbf{Cohen--Macaulay環}(略してCM\textbf{環}と書くことも多い)がある.Cohen--Macaulay環は現在の可換環論で中心的な存在の1つであり,この節と続くもう1つの節(Cohen--Macaulay環について述べる)ではホモロジー代数を(ほとんど)要しない範囲で性質をまとめておこう.まずそのために\textbf{正則列}と\textbf{深さ}を導入する.

\begin{defi}[正則元]
	$A$加群$M$について,$a\in A$が$M$に非零因子として作用する,すなわち任意の$0\neq x\in M$について$ax\neq0$であるとき,$a$は$M$\textbf{正則}(regular)であるという.
\end{defi}

\begin{defi}[正則列]\index{せいそくれつ@正則列}
	$A$をNoether環とし,$M$を有限生成$A$加群とする.$a_1,\dots,a_r\in A$が任意の$1\leq i\leq r$について$a_i$が$M/(a_1,\dots,a_{i-1})M$正則であるとき,$a_1,\dots,a_r$を$M$\textbf{正則列}(regular sequence)という.
\end{defi}

例えば体上の多項式$A=k[X_1,X_2,X_3]$において$f_1=X_1(X_2-1),f_2=X_2,f_3=X_3(X_1-1)$とすると$f_1,f_2,f_3$は$A$正則列をなす.ところが$X_1$は$A/(f_1)$において$0$ではないが,$f_3X_3$は$0$であるので$f_1,f_3,f_2$は$A$正則列ではない.このように正則列は順序を考慮する必要があるが,局所環においては極大イデアルの元について順序によらないことが知られている.ここではより強く$\rad A$の元においてそのことを示そう.

\begin{lem}\label{lem:正則元と素因子}
	Noether環$A$上の加群$M$において$a\in A$が$M$正則でありかつ$\bigcap a^nM=0$であるものとする.このとき$M$の素因子$P\in\ass M$について$P+aA\subset Q$となる$Q\in\ass (M/aM)$が存在する.
\end{lem}

この補題の仮定はKrullの交叉定理(\ref{cor:Krullの交叉定理の系})より$a\in\rad A$のとき満たされる.

\begin{proof}
	$x\in M$によって$P=\ann (x)$とかける.ここで$\bigcap a^nM=0$より,ある$k\geq0$が存在して$x\in a^kM$かつ$x\not\in a^{k+1}M$である.このとき$y\in M$によって$x=a^ky$と書けているとすると$y\not\in aM$である.ここで$\bar{y}$を$y$の$M/aM$における像とすると,$a$が$M$正則なので$(P+aA)\bar{y}=0$が成り立つ.これは$P+aA \subset\ann (\bar{y})$を導き,\ref{prop:素因子の存在}よりある$Q\in\ass (M/aM)$で$P+aA\subset Q$となるものが存在する.
\end{proof}

\begin{prop}
	Noether環$A$上の有限生成加群$M$において$a_1,\dots,a_r\in\rad A$とする.このとき$a_1,\dots,a_r$が$M$正則列ならその並べ替えも$M$正則である.
\end{prop}

\begin{proof}
	まず2つの場合に帰着できることをみよう.$a_1,\dots,a_i,a_{i+1},\dots,a_r$が$M$正則列であるとする.$N=M/(a_1,\dots,a_{i-1})M$とおくと$a_i$は$N$正則であり,$a_{i+1}$は$N/a_iN=M/(a_1,\dots,a_i)M$正則である.このとき$a_1,\dots,a_{i+1},a_i,\dots,a_r$が$M$正則列であることをみるには$a_{i+1}$が$N$の,$a_i$が$N/a_{i+1}N=M/(a_1,\dots,a_{i-1},a_{i+1})M$正則であることを示せばよい.
	
	よって$a_1$が$M$の,$a_2$が$M/a_1M$正則なら$a_2$が$M$の,$a_1$が$M/a_2M$正則であることを示せば十分である.まず$a_2$が$M$の零因子とすると,\ref{prop:素因子の存在}よりある$P\in\ass M$に対して$a_2\in P$である.ここで$a_1\in\rad A$より,Krullの交叉定理から$\bigcap a_1^nM=0$となるので,補題から$P+a_1A \subset Q$となる$Q\in\ass (M/a_1M)$がある.すると$a_2\in P\subset Q$だから$a_2$が$M/a_1M$正則であることに矛盾する.よって$a_2$は$M$正則.また$a_1$が$M/a_2M$の零因子とすると,ある$x\not\in a_2M,y\in M$について$a_1x=a_2y$だが,このとき$a_2y\in a_1M$より$a_2$が$M/a_1M$正則だから$y\in a_1M$である.$y=a_1z$とおくと$a_1(x-a_2z)=0$となり,$a_1$は$M$正則なので$x=a_2z$となって矛盾する.よって$a_1$は$M/a_2M$正則である.
\end{proof}

\begin{cor}
	Noether局所環$(A,\ideal{m})$上の有限生成加群$M$について,$a_1,\dots,a_r\in\ideal{m}$が$M$正則列であるとき,その並べ替えも$M$正則列である.
\end{cor}

次元と関連して,加群の\textbf{深さ}というものを導入しよう.

\begin{defi}[深さ]\index{ふかさ@深さ}
	$A$をNoether環とし,$M$を有限生成$A$加群とする.$A$のイデアル$I$について$IM\neq M$であるとき,$I$の元からなる$M$正則列の長さの最大値を$M$の$I$における\textbf{深さ}(depth)といい,$\mdepth_I M$とかく.$IM=M$であるときは$\mdepth_I M=\infty$と定義する.特に$A$が極大イデアルが$\ideal{m}$である局所環のとき,$\ideal{m}$における$M$の深さを$\mdepth M$とかいて単に$M$の\textbf{深さ}という.
\end{defi}

この定義における$M\neq IM$という条件は,$M\neq0$のとき$(A,\ideal{m})$が局所環ということと中山の補題から$I\subset\ideal{m}$と同値である.

さて,Noether局所環$(A,\ideal{m})$について$\dim A\leq \emdim A=\dim_k\ideal{m}/\ideal{m}^2$であり,これの等号が成り立つものを正則局所環というのであった.深さは次元を下から抑えるものとなっている.そのことを示そう.

\begin{prop}
	Noether局所環$(A,\ideal{m})$と,有限生成$A$加群$M\neq0$について,任意の$P\in\ass M$に対し;
	\[\mdepth M\leq\dim A/P\]
	が成り立つ.
\end{prop}

\begin{proof}
	$\mdepth M$についての帰納法で示す.まず$\mdepth M=0$のときは明らかに成り立っている.$\mdepth M=r$とおき,$\mdepth M'\leq r-1$となる$M'$について成り立っていると仮定しよう.$M$正則列$a_1,\dots,a_r$を考える.このとき$\mdepth (M/a_1M)=r-1$である.また任意の$P\in\ass M$について,\ref{lem:正則元と素因子}から$P+(a_1)\subset Q$となる$Q\in\ass (M/a_1M)$が存在し,$a_1\not\in P,a_1\in Q$であるから$P\subsetneq Q$である.よって$\dim A/Q<\dim A/P$が成り立つ.帰納法の仮定から$r-1=\mdepth (M/a_1M)\leq\dim A/Q<\dim A/P$であるので,$r\leq\dim A/P$であることがわかる. 
\end{proof}

\begin{cor}
	Noether局所環$(A,\ideal{m})$と,有限生成$A$加群$M\neq0$について,$\mdepth M\leq\dim M$である.
\end{cor}

これにより次の定義を導入する.

\begin{defi}[Cohen--Macaulay加群]\index{#Cohen--Macaulayかぐん@Cohen--Macaulay加群}
	Noether局所環$(A,\ideal{m})$上の有限生成加群$M$について,$\dim M=\mdepth M$であるとき$M$を\textbf{Cohen--Macaulay加群}という.以後単にCM加群とかく.
\end{defi}

まず,わかりやすいご利益として定義から次が従う.

\begin{prop}\label{prop:CM加群は非孤立素因子を持たない}
	$M$をNoether局所環$(A,\ideal{m})$上のCM加群とすると,$M$は非孤立素因子を持たない.
\end{prop}

\begin{proof}
	一般に$P\in\ass M$なら$\mdepth M\leq\dim A/P\leq\dim M$だが,$M$がCMなので$\dim A/P$は$P$によらず$\mdepth M=\dim M$に等しい.
\end{proof}
これからCM性についての考察を進めていきたいところではあるが,加群の深さについては Ext と呼ばれる関手を使った言い換えがあり,それにより扱いが簡明になるところがある.ホモロジー代数を仮定しないここでは,次の補題;

\begin{lem}\label{lem:depth M/aM=depth M-1}
	$(A,\ideal{m})$をNoether局所環とし,$M$を$A$上の有限生成加群とする.$a\in I$が$M$正則ならば,イデアル$I\subset\ideal{m}$について;
	\[\mdepth_I (M/aM)=\mdepth_I M-1\]
	が成り立つ.
\end{lem}

を仮定して(証明は\ref{lem:depth M/aMについての証明}で与える),CM性のご利益について見ていくことにしよう.まず補題を用意する.

\begin{lem}\label{lem:dim M/aM=dim M-1}
	$(A,\ideal{m})$をNoether局所環とし、$M\neq0$を有限生成$A$加群とする.$a\in\ideal{m}$が$M$正則元であるとき,$\dim M/aM=\dim M-1$が成り立つ.
\end{lem}

\begin{proof}
	$a$は$A/\ann M$の非零因子なので,\ref{prop:dim A/xA=dim A-1}より$\dim M-1=\dim A/\ann M-1=\dim A/(\ann M+(a))$が成り立つ.また$\dim M/aM=\dim A/\ann(M/aM)$であるので,$V(\ann M+(a))=V(\ann M/aM)$を示せばよい.
	
	容易に$\ann(M/aM)\subset P$なら$\ann M\subset P$かつ$a\in P$であることがわかる.一方で$\ann M\subset P$かつ$a\in P$ならば$M_P\neq0$であり,また$a\in P$だから$(M/aM)_P=M_P/aM_P$であり,$a/1\in\rad (A_P)=PA_P$だから中山の補題より$M_P/aM_P\neq0$である.よって$P\in\supp M/aM$すなわち$\ann(M/aM)\subset P$となり,求める等式が示された.
\end{proof}

この2つの補題から帰納法により容易に次が従う.
\begin{cor}\label{cor:CMの正則列による商もCM}
	$M$をNoether局所環$(A,\ideal{m})$上のCM加群とする.$a_1,\dots,a_r\in\ideal{m}$が$M$正則列なら($r$は$\mdepth M$と一致しているとは限らない),$M/(a_1,\dots,a_r)M$も次元が$\dim M-r$のCM加群である.
\end{cor}

最後に技術的な命題たちを述べておこう.

\begin{lem}\label{lem:depth 0とass}
	$A$をNoether環,$M$を$A$のイデアルとし$I$を$IM\neq M$となる$A$のイデアルとする.$\mdepth_I M=0$,すなわち任意の$x\in I$が$M$の零因子であることと,ある$P\in\ass M$が存在して$I\subset P$であることは同値.
\end{lem}

\begin{proof}
	$(\Longleftarrow)$は明らか.$(\Longrightarrow)$は対偶を示す.任意の$P\in\ass M$について$I\not\subset P$であると仮定する.$\ass M$は有限だからPrime avoidance より$I\not\subset\bigcup_{P\in\ass M}P$であり,\ref{cor:加群の零因子は素因子と等しい}より$x\in I$で$M$正則元となるものが存在する.
\end{proof}

\begin{prop}\label{prop:CMの局所化もCM}
	$M$をNoether局所環$(A,\ideal{m})$上のCM加群とすると,任意の$P\in\supp M$について;
	\[\mdepth_P M=\mdepth_{PA_P} M_P=\dim M_P\]
	が成り立つ.特に$M_P$もCM $A_P$加群である.
\end{prop}

\begin{proof}
	一般に$\mdepth_P M\leq\mdepth_{PA_P}M_P\leq\dim M_P$であることを容易に確かめることができる.よって$\dim M_P\leq\mdepth_P M$であることを示せばよい.
	
	$\mdepth_P M$についての帰納法で示そう.まず$\mdepth_P M=0$ならば,
	補題よりある$\in\ass M$が存在して$P\subset Q$である.ここで\ref{lem:素因子と局所化}より$V(\ann (M_P))=\ass M_P=\mkset{Q\in\ass M}{Q\subset P}\neq\emptyset$であり,$M$がCM加群だから非孤立素因子を持たないので$V(\ann(M_P))=PA_P$となり,$\dim M_P=\dim A_P/PA_P=0$である.さて$\mdepth_P M=r>0$として,$r-1$まで正しいとする.定義より$a\in P$で$M$正則なものがある.ここで\ref{lem:depth M/aM=depth M-1}より$\mdepth_P M/aM=\mdepth_P M-1$であり,帰納法の仮定と\ref{lem:dim M/aM=dim M-1}から$\mdepth_P M/aM=\dim (M/aM)_P=\dim M_P/aM_P=\dim M_P-1$である.よって$\mdepth_P M=\dim M_P$であることがわかり,証明が完了する.	
\end{proof}

この命題によりCM局所環$(A,\ideal{m})$の局所化$A_P$もまた$\mdepth A_P=\idht P$となるCM局所環である.
\section{Cohen--Macaulay環と鎖状環}

正則環と同様に,局所環でない環は局所化により局所環とできることから次の定義をする.
\begin{defi}[Cohen--Macaulay環]\index{#Cohen-Macaulayかん@Cohen--Macaulay環}
	Noether環$A$について,任意の$P\in\spec A$による局所化$A_P$が$A_P$加群としてCohen--Macaulay加群となるとき$A$を\textbf{Cohen--Macaulay環}という.
\end{defi}

明らかに体はCM環である.また簡単な計算により1次元Noether整域はCM環であることがわかる.特にPID,Dedekind整域などはCM環である.この節では,CM環の簡単な性質を見るとともに,正則局所環やCM環上の多項式環がCM環であること,またすべてのCM環が強鎖状環であることを示そう.

\begin{prop}
	環$A$について,鎖状性は局所的な性質である.すなわち$A$が鎖状環であることと,任意の$P\in\spec A$について$A_P$が鎖状環であることは同値.
\end{prop}

証明は定義から明らかであろう.この性質によりCM局所環に帰着することができる.

\begin{thm}
	$(A,\ideal{m})$をCM局所環とする.$A$において弱次元公式(\ref{defi:弱次元公式})が成り立ち,また$A$は鎖状環である.
\end{thm}

\begin{proof}
	まず弱次元公式が成り立つことを示そう.任意の$P\in\spec A$をとり,$\idht P=n$とする.\ref{prop:CMの局所化もCM}より$A_P$もCM局所環で$\idht P=\mdepth_{PA_P} A_P=\mdepth_P A$であるので,$A$正則列$a_1,\dots,a_n\in P$が存在する.$I=(a_1,\dots,a_n)$とおくと,\ref{cor:CMの正則列による商もCM}より$A/I$は$\dim A/I=\dim A-n$となるCM局所環である.また$a_1,\dots,a_n$は$A$正則なので$0<\idht(a_1)<\idht(a_1,a_2)<\cdots$であるから,$n\leq\idht I$が成り立つ.また$\idht I\leq\idht P=n$より$\idht I=\idht P$となり,$P$は$I$の極小素イデアルである.よって$P$は$A/I$の素因子なので,$\dim A/P=\dim A/I=\dim A-n$が成り立つ.よって$\dim A/P=\coht P$だから$\idht P+\coht P=\dim A$である.
	
	次に鎖状であることを示す.任意の素イデアル鎖$P\subsetneq Q$をとる.局所化$A_Q$もCM局所環なので,弱次元公式が成り立つから$\idht Q-\idht P=\dim A_Q/PA_Q$
	が成り立つ.そこで$P\subsetneq P_1\subsetneq \dots\subsetneq P_s=Q$を飽和した素イデアル鎖とすると,$P_i\subsetneq P_{i+1}$について同様に$\idht P_{i+1}=\idht P_i+\dim A_{P_i}/P_{i-1}A_{P_i}=\idht P_i+1$が成り立つ.よって$\idht Q=\idht P+s$であり,$s=\dim A_Q/PA_Q$がわかるので$A$は鎖状である.
\end{proof}

\begin{cor}
	CM環は鎖状環である.
\end{cor}

次にCM環においては巴系と正則列の間に相互によい関係があることを見よう.まず一般に正則局所環はCM環であることを確認しておく.

\begin{prop}[Cohen]
	正則局所環はCM環である.
\end{prop}

\begin{proof}
	$(A,\ideal{m})$を正則局所環とし,$\dim A=n$とおく.$x_1,\dots,x_d\in\ideal{m}$を正則巴系としよう.このとき\ref{prop:正則局所環は落ちる},\ref{thm:正則局所環は整域}によりこれらが$A$正則列をなすことがわかる.
\end{proof}

CM環という条件を仮定すると,正則列と巴系の概念が一致する.

\begin{prop}\label{prop:CM局所環での正則列と巴系}
	$(A,\ideal{m})$をCM局所環とすると,$a_1,\dots,a_r\in\ideal{m}$について,以下の命題;
	\begin{sakura}
		\item $a_1,\dots,a_r$は$A$正則列をなす.
		\item 任意の$1\leq i\leq r$について$\idht (a_1,\dots,a_i)=i$である.
		\item $\idht (a_1,\dots,a_r)=r$である.
		\item $a_1,\dots,a_r$は$A$の巴系の一部分をなす.
	\end{sakura}
	は同値である.	
\end{prop}

\begin{proof}	
	CM性が必要になるのは(iv)$\Longrightarrow$(i)のみであることを注意しておく.
	\begin{eqv}[4]
		\item 
		$a_1,\dots,a_r\in\ideal{m}$を$A$正則列とすると,$0<\idht(a_1)<\idht(a_1,a_2)<\cdots$よりKrullの標高定理(\ref{thm:Krullの標高定理})から$1\leq i\leq r$について$\idht (a_1,\dots,a_i)=i$である.
		\item 自明.
		\item もし$\dim A=r$なら,$(a_1,\dots,a_r)$の極小素イデアルは$\ideal{m}$のなので$\sqrt{(a_1,\dots,a_r)}=\ideal{m}$となり$(a_1,\dots,a_r)$は$\ideal{m}$準素だから$\nitem[r]{a}$は巴系をなす.また$r<\dim A$なら$\ideal{m}$は$(a_1,\dots,a_r)$の極小素イデアルではないので,Prime avoidanceから$a_{r+1}\in\ideal{m}$を$(a_1,\dots,a_r)$のすべての極小素イデアルに含まれないようにとれる.このとき$\idht(a_1,\dots,a_r,a_{r+1})>r$なので,高さが$\dim A$と一致するまで続けることで$a_1,\dots,a_r$は$A$の巴系の一部分であることがわかる.
		\item $\dim A=n$として,$a_1,\dots,a_n$が巴系なら$A$正則であることを示せばよい.$n$についての帰納法を用いる.まず$n=1$のとき,$a_1$が巴系であるとする.ここで$a_1$が$A$正則でない,すなわち$a_1x=0$となる$x\neq0\in A$が存在するとすると,\ref{prop:素因子の存在}から$a_1\in P$となる$P\in\ass A$が存在するが,\ref{prop:CM加群は非孤立素因子を持たない}の証明からわかるように$\dim A/P=1$であるので,$P\subsetneq\ideal{m}$を意味する.これは$(a_1)$が$\ideal{m}$準素であることに矛盾する.よって$a_1$は$A$正則である.
		
		次に$n-1$まで正しいとする.$A$の巴系$a_1,\dots,a_n$について,$a_1$が$A$正則元でないと仮定すると,$n=1$の場合と同様に$a_1\in P$となる$P\in\ass A$が存在して$\dim A/P=n$である.一方で$\sqrt{(a_1,\dots,a_n)}=\ideal{m}$であるので,$A/P$において$\sqrt{(\bar{a_2},\dots,\bar{a_n})}=\ideal{m}$であり,次元定理から$\dim A/P\leq n-1$が言えるがこれは矛盾である.ゆえに$a_1$は$A$正則で,$A/a_1A$は$a_2,\dots,a_n$の像を巴系とする$n-1$次のCM局所環である.よって帰納法の仮定から$a_2,\dots,a_n$の像は$A/a_1A$の正則列であり,$a_1,\dots,a_n$は$A$の正則列をなす.
	\end{eqv}
\end{proof}

次にCM環上の多項式環もCM環になることを示そう.そのために,CM環についてMacaulay,Cohenらが考察していた当時の定義を紹介しよう.

\begin{defi}[純性定理]\index{じゅんせいていり@純性定理}\index{じゅんいである@純(イデアル)}
	Noether環$A$のイデアル$I$について,任意の$P\in\ass A/I$の高さが$\idht I$に等しいとき,$I$は\textbf{純}(unmixied)であるという.特にNoether環$A$について$r$個の元で生成されるイデアル$I=(a_1,\dots,a_r)$について$\idht I=r$ならば$I$は必ず純であるとき,$A$において\textbf{純性定理}(unmixedness theorem)が成り立つという.
\end{defi}

Krullの標高定理(\ref{thm:Krullの標高定理})より,$r$個の元で生成され高さが$r$のイデアル$I$の極小素イデアルの高さはすべて$r$であるので,$I$が純であるということは非孤立素因子を持たないということを意味する.Macaulay (1916)は体上の多項式環について純性定理が成り立つことを,Cohen (1946)は正則局所環について純性定理が成り立つことを証明した.Noether環$A$について純性定理が成り立つことと,(本書の意味で)CM環であることは同値であり,これがCM環の由来である.正則局所環がCM環であることは(Cohenは純性定理が成り立つことを証明したためだいぶ異なる道筋で)上に示した通り.CM環の定義と純性定理が成り立つことを同値であることを確かめておこう.

\begin{thm}
	Noether環$A$について,純性定理が成り立つこととCM環であることは同値である.
\end{thm}

\begin{proof}
	\begin{eqv}
		\item 任意の$P\in\spec A$をとり,$\idht P=r$とする.\ref{prop:イデアルの高さと生成系の個数}より$a_1,\dots,a_r\in P$を$\idht(a_1,\dots,a_i)=i$がすべての$1\leq i\leq r$で成り立つようにとれる.このとき$a_1,\dots,a_r$が$A_P$正則列になる.各$i$について純性定理により$(a_1,\dots,a_i)$の素因子の高さはすべて$i$であるので,特に$a_{i+1}$を含まない.すると\ref{cor:加群の零因子は素因子と等しい}により$a_{i+1}$は$A/(a_1,\dots,a_i)$で正則である.よって$\mdepth A_P=r=\dim A_P$であり$A_P$はCM局所環である.すなわち$A$はCM環.
		\item $I=(a_1,\dots,a_r)$が$\idht I=r$であるとする.$P\in\ass A/I$をとり,$P$が極小でないと仮定する.$P$を含む極大イデアル$\ideal{m}$で局所化して考える.このとき$IA_{\ideal{m}}$はすくなくとも素因子$P'A_{\ideal{m}}\subsetneq PA_{\ideal{m}}$を持つ.ところが\ref{prop:CM局所環での正則列と巴系}より$a_1,\dots,a_r$は$A_{\ideal{m}}$の正則列であり,\ref{cor:CMの正則列による商もCM}より$A_{\ideal{m}}/IA_{\ideal{m}}$もCM局所環である.よって$IA_{\ideal{m}}$の素因子はすべて極小でなければならないので矛盾である.よって純性定理が成り立つ.
	\end{eqv}
\end{proof}

この証明の$(\Longleftarrow)$に注目すると,$A$の極大イデアルによる局所化がCM局所環でありさえすれば純性定理が成り立つ.よって次の系が従う.

\begin{cor}
	Noether環$A$がCM環かどうかを確かめるには任意の極大イデアルによる局所化がCM局所環であるかどうかを確かめれば十分である.
\end{cor}

CM環上の多項式環がCM環であること,そしてCM環は強鎖状環であることを証明してこの節を締めくくろう.

\begin{thm}
	$A$をCM環とすると,$A$上の多項式$A[X_1,\dots,X_n]$もCM環である.
\end{thm}

\begin{proof}
	先の系により$A[X]$の極大イデアル$\ideal{m}'$による局所化がCM局所環であることを示せば十分である.まずは$A$が局所環の場合に帰着できることを見る.$A\cap\ideal{m}'=\ideal{m}\in\spec A$とおくと,自然な対応で;
	\[A[X]_{\ideal{m}'}=(A_{\ideal{m}}[X])_{\ideal{m}'A_{\ideal{m}}[X]}, A_{\ideal{m}}[X]/(\ideal{m}'A_{\ideal{m}}[X])=A[X]/\ideal{m}'\]であり,$A$を局所環と仮定してよいことがわかる.また$\ideal{m}'A_{\ideal{m}}[X]\cap A_{\ideal{m}}=\ideal{m}A_{\ideal{m}}$であるので,$\ideal{m}$も極大と仮定してよい.
	
	以上より,$A$が局所環であって,$A[X]$の極大イデアル$\ideal{m}'$が$\ideal{m}'\cap A=\ideal{m}\in\spm A$を満たすときに$A[X]_{\ideal{m}'}$がCM局所環であることを示せばよい.体$A/\ideal{m}$を$k$とおくと,準同型;
	\[\varphi:A[X]\to k[X];a_nX^n+\dots+a_0\mapsto \bar{a_n}X^n+\dots+\bar{a_0}\quad\text{( $\bar{a_i}$は$a_i$の$A/\ideal{m}$における像.)}\]
	により環同型$A[X]/\ideal{m}A[X]\cong k[X]$がある.いま$\ideal{m}A[X]\subset\ideal{m}'$であり,$A[X]/\ideal{m}A[X]$は1次元の整域であることに注意すると$\ideal{m}'$の像は極大イデアルである.よってモニックで既約な$f'\in k[X]$によって生成されている.$f\in\ideal{m}'$を$\varphi(f)=f'$となるようにとると,$\ideal{m}'=\ideal{m}A[X]+(f)$である.よって$\ideal{m}=(a_1,\dots,a_r)$とすれば$A[X]$上$\ideal{m}'$は$a_1,\dots,a_r,f$で生成されている.またこれらは$A[X]$正則列をなす.$a_i$たちについては\ref{prop:CM局所環での正則列と巴系}から$A$正則列をなすことから従い,$f$は$A[X]/(a_1,\dots,a_r)A[X]=k[X]$でモニックなので正則である.よって$r+1\leq\mdepth_{\ideal{m}'}A[X]$であることがわかる.また$\ideal{m}'$は$r+1$個の元で生成されているからKrullの標高定理により$\dim A[X]_{\ideal{m}'}\leq r+1$である.これらに\ref{prop:CMの局所化もCM}をあわせて;
	\[\dim A[X]_{\ideal{m}'}\leq r+1\leq\mdepth_{\ideal{m}'}A[X]=\mdepth A[X]_{\ideal{m}'}\]
	がわかり,$A[X]_{\ideal{m}'}$はCM局所環である.	
\end{proof}

\begin{cor}[Macaulay]
	体上の多項式環$k[X_1,\dots,X_n]$はCM環である.
\end{cor}

\begin{cor}\label{cor:CM環は強鎖状}
	CM環は強鎖状環である.
\end{cor}

それでは,いよいよ次の章からは圏論とホモロジー代数を導入し,より現代的な手法を用いた可換環の考察についてみていこう.
 %次元論
	\part[Homological algebra]{ホモロジー代数}
ホモロジー代数とは圏の手法を用いてホモロジーの考察を行うものだが,その手法がSerreらの手によって可換環論に応用され革命をもたらした.現代では代数を研究する際の非常に有用な道具として使われている.初等的には完全列とその乱れを調べる手法のことだと思って構わない.この章以降\textbf{圏の簡単な知識を仮定する.}その内容については付録,特に圏,関手,Abel圏の節を見よ.
\section{基本命題}
この節では完全列について考えるときの基本的な道具となる,\textbf{5項補題}(five lemma),\textbf{蛇の補題}(snake's lemma),\textbf{分裂補題}(splitting lemma)を紹介しよう.また常に$A$加群の圏$\operatorname{Mod}(A)$を考える.

\begin{lem}\label{lem:核余核の可換性}
	図式Figure.\ref{fig:核余核の可換性1}が可換ならば,核,余核に誘導される可換図式Figure.\ref{fig:核余核の可換性2}がある.
	
	\begin{minipage}{.29\linewidth}
		\begin{figure}[H]
			\centering
			\begin{tikzcd}[row sep=scriptsize, column sep=scriptsize]
			M_1\darrow[h_1]\nxcell[f]M_2\darrow[h_2]\\
			N_1\nxcell[g]N_2
			\end{tikzcd}
			\caption{}\label{fig:核余核の可換性1}
		\end{figure}
	\end{minipage}
	\hfill
	\begin{minipage}{.70 \linewidth}
		\begin{figure}[H]
			\centering
			\begin{tikzcd}[row sep=scriptsize, column sep=scriptsize]
			0\nxcell\ker f\darrow[\varphi]\nxcell M_1\darrow[h_1]\nxcell[f]M_2\darrow[h_2]\nxcell\coker f\darrow[\psi]\nxcell0\\
			0\nxcell\ker g\nxcell N_1\nxcell[g]N_2\nxcell\coker g\nxcell0
			\end{tikzcd}
			\caption{}\label{fig:核余核の可換性2}
		\end{figure}
	\end{minipage}

\end{lem}

\begin{proof}
	任意の$x\in\ker f$について$h_1(x)\in\ker g$であるので,$\varphi:\ker f\to\ker g;x\mapsto h_1(x)$が定まる.
	
	また,$\psi:\coker f\to\coker g;y+\im f\mapsto h_2(y)+\im g$が求める準同型を与える.実際$y+\im f=y'+\im f$ならば$y-y'\in\im f$なので,ある$x\in M_1$がとれて$h_2(y)-h_2(y')=g(h_1(x))\in\im g$である.よってwell-defined.
\end{proof}
\begin{lem}[5項補題]\index{ごこうほだい@5項補題}
	2つの行が完全であるような次の可換図式;
	\[\begin{tikzcd}
		M_1\arrow[d,"h_1"]\nxcell[f_1]M_2\arrow[d,"h_2"]\nxcell[f_2]M_3\arrow[d,"h_3"]\nxcell[f_3]M_4\arrow[d,"h_4"]\nxcell[f_4]M_5\arrow[d,"h_5"]\\
		N_1\nxcell[g_1]N_2\nxcell[g_2]N_3\nxcell[g_3]N_4\nxcell[g_4]N_5
	\end{tikzcd}\]
	について次が成り立つ.
	\begin{sakura}
		\item $h_1$が全射で$h_2,h_4$が単射ならば$h_3$は単射.
		\item $h_5$が単射で$h_2,h_4$が全射ならば$h_3$は全射.
	\end{sakura}

	特に$h_1,h_2,h_4,h_5$が同型ならば$h_3$も同型である.
\end{lem}

\begin{proof}
	\begin{sakura}
		\item $h_3(x_3)=h_3(x_3')$とする.すると$f_3(x_3-x_3')\in\ker h_4=0$である.よって$x_3-x_3'\in\ker f_3=\im f_2$であるので, $x_3-x_3'=f_2(x_2)$とかける.ここで$g_2(h_2(x_2))=h_3(x_3-x_3')=0$より$h_2(x_2)\in\ker g_2=\im g_1$である.よって$h_2(x_2)=g_1(y_1)$とかけている.すると$h_1$が全射なので$h_2(x_2)=g_1(h_1(x_1))=h_2(f_1(x_1))$とかけている.よって$h_2$が単射だから$x_2-f_1(x_1)=0$である.すると$f_2(x_2-f_1(x_1))=x_3-x_3'=0$である.よって$h_3$は単射.
		
		\item
		 任意の$y_3\in N_3$に対して, $h_4$が全射なので$g_3(y_3)=h_4(x_4)$となる$x_4$がとれる.すると$h_5(f_4(x_4))=g_4(h_4(x_4))=0$より$h_5$が単射であるから, $f_4(x_4)=0$である.よって$x_4\in\ker f_4=\im f_3$である.ゆえに$f_3(x_3)=x_4$となる$x_3$がとれる.このとき$g_3(y_3)=h_4(x_4)=h_4(f_3(x_3))=g_3(h_3(x_3))$であるので, $y_3-h_3(x_3)\in\ker g_3=\im g_2$である.よって$h_2$の全射性から$y_3-h_3(x_3)=g_2(y_2)=g_2(h_2(x_2))=h_3(f_2(x_2))$とかける$x_2,y_2$が存在する.ゆえに$y_3=h_3(x_3+f_2(x_2))$となり全射である.
	\end{sakura}
\end{proof}

\begin{lem}[蛇の補題]\index{へびのほだい@蛇の補題}
	2つの行が完全であるような次の可換図式;
	\[\begin{tikzcd}
		& M_1\arrow[d,"f_1"]\nxcell[\varphi]M_2\arrow[d,"f_2"]\nxcell[\psi]M_3\arrow[d,"f_3"]\nxcell0\\
		0\nxcell N_1\nxcell[\varphi']N_2\nxcell[\psi']N_3
	\end{tikzcd}\]
	を考えると,自然に誘導される射たちが存在して;
	\[\begin{tikzcd}	
	\ker f_1\nxcell[\bar{\varphi}]\ker f_2\nxcell[\bar{\psi}]\ker f_3\\[-1em]
	&{}\nxcell[d]\coker f_1\nxcell[\bar{\varphi'}]\coker f_2\nxcell[\bar{\psi'}]\coker f_3
	\end{tikzcd}\]
	が完全になる.
\end{lem}

また$\varphi$が単射であることと$\bar{\varphi}$が単射であること,$\psi'$が全射であることと$\bar{\psi'}$が全射であることは同値.
\begin{figure}[H]
	\centering
	\begin{tikzcd}
		(0\nxcell)\ker f_1\arrow[d]\nxcell[\bar{\varphi}]\ker f_2\arrow[d]\arrow[ddd,phantom,""{coordinate,name=Z}]\nxcell[\bar{\psi}]
		\ker f_3\arrow[d]\arrow[r,"d",no head]
		\arrow[llddd,near start,
			rounded corners,
			to path={ -- ([xshift=4.5em]\tikztostart.east)
			|- (Z) [near end]\tikztonodes
			-| ([xshift=-4.5em]\tikztotarget.west)
			-- (\tikztotarget)}
		]&{}\\
		(0\nxcell) M_1\arrow[d,"f_1",near start]\nxcell[\varphi]M_2\arrow[d,"f_2",near start]\nxcell[\psi]M_3\arrow[d,"f_3",near start]\nxcell0\\[2em]
		0\nxcell N_1\arrow[d]\nxcell[\varphi']N_2\arrow[d]\nxcell[\psi']N_3\arrow[d](\nxcell0)\\
		&\coker f_1\nxcell[\bar{\varphi'}]\coker f_2\nxcell[\bar{\psi'}]\coker f_3(\nxcell0)
	\end{tikzcd}
	\caption{蛇の補題}
\end{figure}

\begin{proof}
	\begin{step}
		\item $\bar{\varphi},\bar{\psi}$の定義.
		
			$\varphi,\psi$を制限することで定義しよう.実際,$x\in\ker f_1$について$f_2(\varphi(x))=\varphi'(f_1(x))=0$であるので, $\varphi(x)\in\ker f_2$である.$\psi$についても同様.
			
		\item $\bar{\varphi'},\bar{\psi'}$の定義.
		
			$\varphi',\psi'$を$\coker $に誘導することで定義しよう. well-definednessを確認しておく. $\bar{y}=\bar{y'}$と仮定すると,$y-y'\in\im f_1$よりある$x\in M_1$で$y-y'=f_1(x)$となるものが存在する.すると$\varphi'(y-y')=f_2(\varphi(x))\in\im f_2$より$\bar{\varphi'(y)}=\bar{\varphi'(y')}$である.
			
		\item $d$の定義.
		
			$d:\ker f_3\to\coker f_1$を次のように定めよう.$x\in\ker f_3$に対し$\psi$が全射なので$\psi(x_2)=x$となる$x_2\in M_2$がとれる.このとき$f_2(x_2)\in\ker\psi'=\im\varphi'$となり,ただ1つの$y_1\in N_1$が存在して$f_2(x_2)=\varphi'(y_1)$である. $d(x)=\bar{y_1}\in\coker f_1$と定義する.
			
			この定義においてwell-deinfednessを確かめるには,$x_2$のとり方によらないことを見ればよい.$\psi(x_2)=\psi(x_2')=x$となっているとしよう.このとき$\varphi'(y_1')=f_2(x_2')$となる$y_1'$をとると, $\varphi'(y_1'-y_1)=f_2(x_2'-x_2)$であり, $x_2'-x_2\in\ker\psi=\im\varphi$より$\varphi(x_1)=x_2'-x_2$となる$x_i\in M_1$がとれる.すると$\varphi'(f_1(x_1))=f_2(x_2'-x_2)=\varpi'(y_1'-y_1)$となり, $\varphi'$は単射だから$y_1'-y_1=f_1(x_1)\in\im f_1$である.
			
		\item $d$の完全性のみ確認しておこう.
		
		\begin{sakura}
			\item $\im\bar\psi=\ker d$であること.
			
			任意の$x\in\im\bar\psi$について$d(x)=0$を示せばよい. 定義から$x_2\in\ker f_2$が存在して$x_3=\psi(x_2)$とかけている.すると$\varphi'(y_1)=f_2(x_2)=0$となる$y_1$をとれば$d(x)=\bar{y_1}$だが, $y_1\in\ker\varphi'=0$である.よって$d(x)=0$である.
			
			$x\in\ker d$をとると,$x=\psi(x_2)$となる$x_2\in M_2$がとれる. $f_2(x_2)=\varphi'(y_1)$となる$y_1$について, $\bar{y_1}=d(x)=0$より$y_1\in\im f_1$である.よって$y_1=f_1(x_1)$となる$x_1\in M_1$をとる.すると$f_2(x_2)=\varphi'(y_1)=\varphi'(f_1(x_1))=f_2(\varphi(x_1))$であるので,$x_2-\varphi(x_1)\in\ker f_2$となる.このとき$\psi(x_2-\varphi(x_1))=\psi(x_2)=x_3$であるので,$x_3\in\im\bar\psi$である.
		
			\item $\im d=\ker\bar{\varphi'}$であること.
			
			$\bar{y_1}\in\im d$をとる.$d(x)=\bar{y_1}$とすると, $\psi(x_2)=x_3$となる$x_2$をとったとき, $f_2(x_2)=\varphi'(y_1')$となる$y_1'$について$\bar{y_1}=\bar{y_1'}$である.いま$\varphi(y_1')\in\im f_2$なので, $\bar{\varphi'}(\bar{y_1})=0$である.
			
			$\bar{y_1}\in\ker\bar{\varphi'}$をとる.すると$\varphi'(y_1)\in\im f_2$である.よって$\varphi'(y_1)=f_2(x_2)$となる$x_2\in M_2$がとれる.このとき$d(\psi(x_2))=\bar{y_1}$となる.
		\end{sakura}
	\end{step}
\end{proof}

\begin{defi}[分裂完全列]\index{ぶんれつかんぜんれつ@分裂完全列}
	完全列;
	\[\ses[\iota][\pi]{M_1}{M}{M_2}\tag{$\ast$}\]
	について,次の条件;
	\begin{sakura}
		\item $i:M_2\to M$が存在して$\pi\circ i=\id_{M_2}$が成り立つ.
		\item $p:M\to M_1$が存在して$p\circ\iota=\id_{M_1}$が成り立つ.
	\end{sakura}
	のどちらかが成り立つとき,完全列$(\ast)$は\textbf{分裂}(split)するという. 特に(i)を\textbf{左分裂}(left split),(ii)を\textbf{右分裂}(right split)という.
\end{defi}

\begin{lem}[分裂補題]\index{ぶんれつほだい@分裂補題}
	完全列$\ses[\iota][\pi]{M_1}{M}{M_2}$について,以下の3つ;
	\begin{sakura}
		\item $M$は$M_1\oplus M_2$と同型で,$\iota,\pi$はそれぞれ自然な移入と射影に一致する.
			
		\item $i:M_2\to M$が存在して$\pi\circ i=\id_{M_2}$が成り立つ.
		
		\item $p:M\to M_1$が存在して$p\circ\iota=\id_{M_1}$が成り立つ.
	\end{sakura}
	は同値.すなわち完全列が分裂していることと,中央の項が左右の項の直和であることが同値である.
\end{lem}

\begin{proof}
	\begin{eqv}[3]
		\item 自然な$i:M_2\to M_1\oplus M_2$によって得られる.
		
		\item 任意の$x\in M$について$x-i(\pi(x))\in\ker\pi=\im\iota$より,$y\in M_1$で$\iota(y)=x-i(\pi(x))$となるものが一意的に定まる.これによって$p:M\to M_1$を$p(x)=y$と定めると題意を満たす.
		
		\item $\varphi:M\to M_1\oplus M_2;x\mapsto(p(x),\pi(x))$が同型射となる.
		\begin{step}
			\item 単射であること.
			
			$\varphi(x)=0$とすると$p(x)=\pi(x)=0$であるので, $x\in\ker\pi=\im\iota$よりある$y\in M_1$が存在して$\iota(y)=x$とかける.すると$p(x)=y=0$なので$x=0$である.
			
			\item 全射であること.
			
			任意の$(x_1,x_2)\in M_1\oplus M_2$について,$\pi$が全射なのである$x\in M$が存在して$\pi(x)=x_2$である.このとき$\varphi(x+\iota(x_1))=(x_1,x_2)$となる.
		\end{step}
		
	\end{eqv}
\end{proof}

\begin{thm}\label{thm:分裂しているなら完全性は保存される}
	完全列$\ses{M_1}{M}{M_2}$が分裂しているとする.加群の圏への半完全関手$F$に対して;
	\[\ses{F(M_1)}{F(M)}{F(M_2)}\]
	は完全である.
\end{thm}

\begin{proof}
	$F(p\circ\iota)=F(p)\circ F(\iota)=F(\id_{M_1})=\id_{F(M_1)}$より$F(\iota)$が単射であることがわかり,同様にして$F(\pi)$は全射である.
\end{proof}
\section{複体とホモロジー,コホモロジー}

この節ではAbel圏$\mathscr{A}$で考えていく(付録Bもみよ)が,埋め込み定理(\ref{thm:埋め込み定理})により加群だとおもって話を進めていく.つまり核,余核は今までどおりの見知った対象であると考え,元についての操作を行う.圏論的な考え方で議論を押し切って行くことを「アブストラクト・ナンセンス(abstract nonsense)」とよく言うが,これはごちゃごちゃした計算に頼ることなく,いわば考えている対象を\quo{上に}あげて,コホモロジーやスペクトル系列などの道具で計算してから地上に戻してみると証明したかったことがわかっている,といった状況のことをいう.\cite{kato}によれば「使われているすべてのコホモロジーは,みな導来関手」である.まずは導来関手を考えるためにホモロジー代数の基礎知識を集め,定義していこう.
	
\begin{defi}[複体]\index{ふくたい@複体}
	Abel圏$\mathscr{A}$の対象の族$\{A_i\}_{i\in\N}$と射$d_i:A_{i}\to A_{i-1}$の族$\{d_i\}$で,$d_{i}\circ d_{i+1}=0$を満たすものを\textbf{(鎖)複体}(chain complex)という.これらをまとめて$A_\bullet$とかく.また$\mathscr{A}$の対象の族$\{A^i\}_{i\in\N}$と射$d^i:A^i\to A^{i+1}$の族$\{d^i\}$で,$d^{i+1}\circ d^{i}=0$を満たすものを\textbf{余鎖複体}(cochain complex)といい,これらを$A^\bullet$とかく.
\end{defi}

射$d_i$を$i$次\textbf{境界作用素}(boundary operator)ともいう.\index{きょうかいさようそ@境界作用素} この定義の意味を考えてみよう.列(鎖);
\[\begin{tikzcd}
\cdots\nxcell[d_{i+2}]A_{i+1}\nxcell[d_{i+1}]A_i\nxcell[d_i]A_{i-1}\nxcell[d_{i-1}]\cdots\nxcell[d_1] A_0\nxcell[]0
\end{tikzcd}\]
\[\begin{tikzcd}
0\nxcell[]A_0\nxcell[d^0]\cdots\nxcell A^{i-1}\nxcell[d^{i-1}]A^i\nxcell[d^i]A^{i+1}\nxcell[d^{i+1}]\cdots
\end{tikzcd}\]
があったときここから情報としてホモロジー,コホモロジーを取りたいわけだが,その定義は$\ker$を$\im$で割ったものであるので,この定義が意味を持つのは$\im d_{i+1}\subset\ker d_i$であるとき,つまり$d_i\circ d_{i+1}=0$であるときである.これが複体の意味である.以後,単に複体といえば鎖複体を表すものとする.
\begin{defi}[ホモロジー群,コホモロジー群]\index{ほもろじーぐん@ホモロジー群}\index{こほもろじーぐん@コホモロジー群}
	$A_\bullet$を複体とする.各$i$に対して$\ker d_i/\im d_{i+1}$を$A_\bullet$の$i$次\textbf{ホモロジー群}(homology group)といい,$H_i(A_\bullet)$で表す.余鎖複体$A^\bullet$については $H^i(A^\bullet)=\ker d^i/\im d^{i-1}$を$i$次\textbf{コホモロジー群}(cohomology group)という.
\end{defi}

ホモロジー群とは$A_i$の部分対象である.自然な全射をそれぞれ$\pi_i:\ker d_i\to H_i(A_\bullet),\pi^i:\ker d^i\to H^i(A^\bullet)$という.それぞれの対象がどうなっているかについていろいろと名前があるので,それを挙げておく.
\begin{defi}[非輪状]\index{ひりんじょう@非輪状}
	すべての$i>0$について$H_i(A_\bullet)=0$となるとき,つまり$\ker d_i=\im d_{i+1}$となるとき$A_\bullet$を完全(exact),ひいては\textbf{非輪状}(acyclic)という.$A^\bullet$についても同様.
\end{defi}

完全列そのものではどこのホモロジー,コホモロジーをとっても消えてしまう.しかし,見方を返せば完全列に関手を施して複体を作ったときそこでホモロジーが消えない,ということは完全性が乱れてしまったということにほかならない.以前取り上げた左(右)完全関手はその一例である.このようにホモロジーは鎖がどれだけ完全列から離れているかという\quo{乱れ}を計測する手段であるといえる.

\begin{defi}[コサイクル,コバウンダリー]\index{コサイクル}\index{コバウンダリー}
	$\ker d_i$の元を\textbf{サイクル,輪体}(cycle),$\im d^{i+1}$の元を\textbf{バウンダリー,境界輪体}(boundary)という.$\ker d^i,\im d_{i-1}$については\textbf{コサイクル,余輪体}(cocycle), \textbf{コバウンダリー,余境界輪体}(coboundary)という.
\end{defi}

次に複体の圏について考えたい.そのために複体の間の射$f_\bullet:A_\bullet\to B_\bullet$を考えねばならない.射$f_i:A_i\to B_i$の族$\{f_i\}$ないし$\{f^i\}$で,次の図式;

%\begin{minipage}{.45\linewidth}
	\begin{figure}[H]
		\centering
		\begin{tikzcd}
		\cdots\arrow[r]&A_{i+1}\arrow[r,"d_{i+1}"]\arrow[d,"f_{i+1}"]&A_i\arrow[r,"d_i"]\arrow[d,"f_i"]&A_{i-1}\arrow[r]\arrow[d,"f_{i-1}"]&\cdots\\
		\cdots\arrow[r]&B_{i+1}\arrow[r,"{d'}_{i+1}"]&B_i\arrow[r,"{d'}_{i}"]&B_{i-1}\arrow[r]&\cdots
		\end{tikzcd}
		\caption{複体の射}
	\end{figure}
%\end{minipage}
%\hfill
%\begin{minipage}{.45\linewidth}
%\begin{figure}[H]
%	\centering
%	\begin{tikzcd}
%	\cdots\arrow[r]&A^{i-1}\arrow[r,"d^{i-1}"]\arrow[d,"f^{i-1}"]&A^i\arrow[r,"d^i"]\arrow[d,"f^i"]&A^{i+1}\arrow[r]\arrow[d,"f^{i+1}"]&\cdots\\
%	\cdots\arrow[r]&B^{i-1}\arrow[r,"{d'}^{i-1}"]&B^i\arrow[r,"{d'}^{i}"]&B^{i+1}\arrow[r]&\cdots
%	\end{tikzcd}
%	\caption{余鎖複体の射}\label{fig:余鎖複体の射}
%\end{figure}
%\end{minipage}

を可換にするものを\textbf{複体の射}といい$f_\bullet$とかく.余複体についても双対的に考え$f^\bullet$とかく.まとめよう.
\begin{defi}[複体の圏]
	$\mathscr{A}$をAbel圏とする.対象を$\mathscr{A}$の複体,射を$f_\bullet$として定める圏を$\ch{A}$とかき, \textbf{複体の圏}という.また余鎖複体と$f^\bullet$のなす圏を$\coch{A}$とかく.
\end{defi}

コホモロジーとは導来関手である,と述べたが,まずホモロジーをとることは関手になることをみよう.
\begin{prop}
	各$i$に対し,ホモロジー$H_i$は$\ch{A}$から$\mathscr{A}$への関手になる.
\end{prop}
\begin{proof}
	複体の射$f_\bullet$について$H_i(f_\bullet)$を次で定めよう;
	\[H_i(f_\bullet)=\widetilde{f}_i:H_i(A_\bullet)\to H_i(B_\bullet);x+\im d_{i+1}\mapsto f_i(x)+\im d_{i+1}'\]
	$x\in\ker d_i$ならば$f_i(x)\in\ker d'_i$であるので,この定義は意味を持つ.well-definednessについても計算すれば明らかである.
\end{proof}

コホモロジーについても同様に$\coch{A}$から$\mathscr{A}$への関手になる.ではいつ$H_i(f_\bullet)=H_i(g_\bullet)$となるかを考えよう.まず$H_i(f_\bullet)=H_i(g_\bullet)$であることと,任意の$x\in\ker d_i$に対して$f_i(x)-g_i(x)\in\im d'_{i+1}$であることは同値である.ここで射$s_i:A_i\to B_{i+1}$が存在して$f_i-g_i=(d'_{i+1}\circ s_i)+(s_{i-1}\circ d_i)$となるとき,$x\in\ker d_i$なら$f_i(x)-g_i(x)\in\im d'_{i+1}$である.このような$s^i$がすべての$i$でとれるとき,任意の次数のホモロジーが一致する.このとき$f_\bullet$と$g_\bullet$は\textbf{ホモトピー同値}であるという.
\begin{defi}[ホモトピー同値]\index{ほもとぴーどうち@ホモトピー同値}
	複体の射$f_\bullet,g_\bullet$に対し,射$s_i:A_i\to B_{i+1}$の族$\{s_i\}$で,各$i$に対し$f_i-g_i=(d'_{i+1}\circ s_i)+(s_{i-1}\circ d_i)$となるものが存在するとき$f_\bullet$と$g_\bullet$は\textbf{ホモトピー同値,ホモトピック}(homotopic)であるという.
\end{defi}

\begin{figure}[H]
	\centering
	\begin{tikzcd}[row sep=huge, column sep=huge]
	\cdots\arrow[r]&A_{i+1}\arrow[r,"d_{i+1}"]\arrow[d,"f_{i+1}",shift left=.5ex]\arrow[d,"g_{i+1}",swap,shift right =.5ex]\arrow[dl,"s_{i+1}",swap]&A_i\arrow[r,"d_i"]\arrow[d,"f_i",shift left=.5ex]\arrow[d,"g_i",swap,shift right=.5ex]\arrow[dl,"s_i",swap]&A_{i-1}\arrow[r]\arrow[d,"f_{i-1}",shift left=.5ex]\arrow[d,"g_{i-1}",swap,shift right=.5ex]\arrow[dl,"s_{i-1}",swap]&\cdots\arrow[dl]\\
	\cdots\arrow[r]&B_{i+1}\arrow[r,"{d'}_{i+1}"]&B_i\arrow[r,"{d'}_{i}"]&B_{i-1}\arrow[r]&\cdots
	\end{tikzcd}
	\caption{複体の射のホモトピー同値}
\end{figure}

余鎖複体についても同様に少し調整し,$f^i-g^i=(d'^{i-1}\circ s^i)+(s^{i+1}\circ d^i)$とすればよい.
\begin{figure}[H]
	\centering
	\begin{tikzcd}[row sep=huge, column sep=huge]
	\cdots\arrow[r]&A^{i-1}\arrow[r,"d^{i-1}"]\arrow[d,"f^{i-1}",shift left=.5ex]\arrow[d,"g^{i-1}",swap,shift right =.5ex]\arrow[dl,"s^{i-1}",swap]&A^i\arrow[r,"d^i"]\arrow[d,"f^i",shift left=.5ex]\arrow[d,"g^i",swap,shift right=.5ex]\arrow[dl,"s^i",swap]&A^{i+1}\arrow[r]\arrow[d,"f^{i+1}",shift left=.5ex]\arrow[d,"g^{i+1}",swap,shift right=.5ex]\arrow[dl,"s^{i+1}",swap]&\cdots\arrow[dl]\\
	\cdots\arrow[r]&B^{i-1}\arrow[r,"{d'}^{i-1}"]&B^i\arrow[r,"{d'}^{i}"]&B^{i+1}\arrow[r]&\cdots
	\end{tikzcd}
	\caption{余鎖複体の射のホモトピー同値}
\end{figure}
\begin{exer}
	ホモトピー同値は同値関係である.
\end{exer}
複体に加法的関手を施すことを考えよう.というのも,加法的関手$F$に対して$\mathscr{A}$の対象$A$における導来関手$R^iF(A)$を対応させたいからである.前述の通り$F$が完全関手でなければ非輪状な(完全な)複体は,移した先ではもはや完全ではない.そこでのホモロジーを見ることで,どの程度関係が乱れたかを測りたい.そのためには移した先でも複体になっていることが必要である.
\begin{prop}
	$\mathscr{A},\mathscr{B}$をAbel圏とし,$F:\mathscr{A}\to\mathscr{B}$を加法的関手とする.複体$A_\bullet\in\ch{A}$に対し;
	\[\begin{tikzcd}
	\cdots\nxcell F A_{i+1}\nxcell[F d_{i+1}]F A_i\nxcell[F d_i]F A_{i-1}\nxcell[F d_{i-1}]\cdots
	\end{tikzcd}\]
	は$\mathscr{B}$の複体となる.これを$F A_\bullet$とかく.
\end{prop}
\begin{proof}
	$A_\bullet$が複体なので$d_i\circ d_{i+1}=0$である.$F$が加法的関手なので$F(d_i)\circ F(d_{i+1})=F(d_i\circ d_{i+1})=F(0)=0$である.ゆえに複体となる.
\end{proof}

よってホモロジーをとることができる.

\begin{exer}
	ホモトピー同値は加法的関手で保たれることを示せ.
\end{exer}

しかし,そもそもは対象$A$の情報を取りたかったのである.では$A$から定まる自然な複体についてコホモロジーを考えてみるのはどうだろうか.
\begin{prop}
	Abel圏$\mathscr{A}$は$\ch{A}$の部分圏になる.
\end{prop}
\begin{proof}
	次の自然な鎖;
	\[\begin{tikzcd}
	\cdots\nxcell0\nxcell A\nxcell 0\nxcell\cdots
	\end{tikzcd}\tag{$\ast$}\]
	は複体になる.
\end{proof}

とはいえこの構成はあまりに自然すぎて,ここに加法的関手を施してもたかだか$A$の前後でしか完全性は乱れない(加法的関手は0を0に移すから).そこで$A$を分解してみよう.とはいえその分解はあくまで$A$の代わりであるので,(移す前の)ホモロジーは($\ast$)と一致することを要求する.それを実現してくれるのがここから話す\textbf{射影分解}である(コホモロジーでは\textbf{入射的分解}を用いる).

\section{射影分解と入射的分解}
前節の最後に話したことを図式で書いてみよう;
\begin{figure}[H]
	\centering
	\begin{tikzcd}
	\cdots\nxcell P_2\arrow[d]\nxcell[d_2]P_1\arrow[d]\nxcell[d_1]P_0\arrow[d,"\varepsilon"]\nxcell[d_0]0\arrow[d]\nxcell\cdots\\
	\cdots\nxcell0\nxcell0\nxcell A\nxcell 0\nxcell\cdots\\
	\end{tikzcd}
	\caption{}\label{fig:射影分解したい}
\end{figure}
となる複体$P_\bullet$で,ホモロジーを取ると$H_0(P_\bullet)=\ker d_0/\im d_1=A,H_i(P_\bullet)=\ker d_i/\im d_{i+1}=0~(i\geq1)$となるものをうまく取りたいということであった.高次のホモロジーが消えているような$P_\bullet$を定義するために,まずはうまい$P_i$を取る必要がある.
\begin{defi}[射影対象]\index{しゃえいたいしょう@射影対象}
	Abel圏の対象$P$で,任意の完全列;
	\[\begin{tikzcd}
	A\nxcell[\varepsilon]A''\nxcell0
	\end{tikzcd}\]
	と$f:P\to A''$が与えられたとき,$\varepsilon\circ\widetilde{f}=f$となる$f:P\to A$が必ず存在するような$P$を\textbf{射影対象}(projective object)という.
\end{defi}

これは次の図式が可換になる$\widetilde{f}$の存在といえる.
\begin{figure}[H]
	\centering
	\begin{tikzcd}[row sep=huge, column sep=huge]
		A\nxcell[\varphi]A''\nxcell0\\
		P\arrow[u,"\widetilde{f}",dashed]\arrow[ur,"f"]
	\end{tikzcd}
	\caption{}
\end{figure}
射影加群(injective module)の定義を思い出そう.一般のAbel圏でも関手$\hom(N,-)$は左完全関手になる.射影加群の定義はこの関手を完全にする$M$のことであったが,射影対象も$\hom_{\mathscr{A}}(P,-)$が($0\to A'\to A\to A''\to0$に対して)完全になる関手のことである,といってよい.これを用いて$A$の都合のよい分解を与える.

\begin{defi}[射影分解]\index{しゃえいてきぶんかい@射影分解}
	Abel圏$\mathscr{A}$の対象$A$について,入射的対象$P^i~(i\geq0)$がとれて;
	\[\begin{tikzcd}
	\cdots\nxcell[d_2]P_1\nxcell[d_1]P_0\nxcell[\varepsilon]A\nxcell0
	\end{tikzcd}\]
	が完全列であるとき,$d_0$を自然な零射とする複体;
	\[\begin{tikzcd}
	\cdots\nxcell P_2\nxcell[d_2]P_1\nxcell[d_1]P_0\nxcell[d_0]0
	\end{tikzcd}\]
	を$P_\bullet$と書いて$A$の\textbf{射影分解}(projective resolution)という.
\end{defi}

このとき$H_0(P_\bullet)=\ker d_0/\im d_1=P_0/\ker\varepsilon=A$となっていることに注意しよう.

どんな$A$でも射影分解が行えるとは限らないが,分解の存在を保証してくれる条件がある.
\begin{defi}
	Abel圏$\mathscr{A}$の任意の対象$A$について,$\mathscr{A}$の射影対象$P$が存在して;
	\[\begin{tikzcd}
	P\nxcell[\varepsilon] A\nxcell0
	\end{tikzcd}\]
	が完全になるような$\varepsilon$が存在するとき,$\mathscr{A}$は\textbf{射影対象を十分に持つ}(has enough projectives)という.
\end{defi}

射影対象を十分に持つなら$A$の射影分解が構成できることは演習問題としよう.
\begin{prop}
	$A$加群の圏$\operatorname{Mod}(A)$は射影対象を十分に持つ.
\end{prop}

\begin{proof}
	$A$加群$M$について自由加群$F$からの全射$\varepsilon:F\to M$が存在し,自由加群は射影加群(\ref{prop:自由加群は射影加群})なので題意は満たされる.
\end{proof}

次に入射的分解について考えよう.Figure.\ref{fig:射影分解したい}の余鎖複体バージョンを考えると,次のようになる;

\begin{figure}[H]
	\centering
	\begin{tikzcd}
	\cdots\nxcell0\arrow[d]\nxcell A\arrow[d,"\varepsilon"]\nxcell 0\arrow[d]\nxcell0\arrow[d]\nxcell\cdots\\
	\cdots\nxcell0\nxcell I^0\nxcell[d^0] I^1\nxcell[d^1]I^2\nxcell[d^2]\cdots
	\end{tikzcd}
	\caption{}
\end{figure}

となる複体$I^\bullet$で,コホモロジーを取ると$H^0(I^\bullet)=\ker d^0=A,H^i(I^\bullet)=\ker d^i/\im d^{i-1}=0~(i\geq1)$となるものをうまく取りたい.

\begin{defi}[入射的対象]\index{にゅうしゃてきたいしょう@入射的対象}
	Abel圏の対象$I$で,任意の完全列;
	\[\begin{tikzcd}
	0\nxcell A'\nxcell[\varepsilon]A
	\end{tikzcd}\]
	と$f':A'\to I$が与えられたとき,$f\circ\varepsilon=f'$となる$f:A\to I$が必ず存在するような$I$を\textbf{入射的対象}(injective object)という.
\end{defi}

これは次の図式が可換になる$f$の存在といえる.
\begin{figure}[H]
	\centering
	\begin{tikzcd}[row sep=huge, column sep=huge]
	&&I\\
	0\arrow[r]&A'\arrow[ur,"f'"]\arrow[r,"\varepsilon"]&A\arrow[u,dashed,"f"]
	\end{tikzcd}
	\caption{}
\end{figure}

\begin{defi}[入射的分解]\index{にゅうしゃてきぶんかい@入射的分解}
	Abel圏$\mathscr{A}$の対象$A$について,入射的対象$I^i~(i\geq0)$がとれて;
	\[\begin{tikzcd}
	0\nxcell A\nxcell[\varepsilon]I^0\nxcell[d^0] I^1\nxcell[d^1]\cdots
	\end{tikzcd}\]
	が完全列であるとき,複体;
	\[\begin{tikzcd}
		0\nxcell I^0\nxcell[d^0] I^1\nxcell[d^1]I^2\nxcell[d^2]\cdots
	\end{tikzcd}\]
	を$I^\bullet$と書いて,$A$の\textbf{入射的分解}(injective resolution)という.
\end{defi}

同様に$H^0(I^\bullet)=\ker d^0=\im\varepsilon=A$であることに注意しよう.また射影分解と同様に\textbf{入射的対象を十分に持つ}なら入射的分解が必ずできる.

\begin{defi}
	Abel圏$\mathscr{A}$の任意の対象$A$について $\mathscr{A}$の入射的対象$I$が存在して;
	\[\begin{tikzcd}
		0\nxcell A\nxcell[\varepsilon]I
	\end{tikzcd}\]
	が完全になるような$\varepsilon$が存在するとき,$\mathscr{A}$は\textbf{入射的対象を十分に持つ}(has enough injectives)という.
\end{defi}

$A$加群の圏$\operatorname{Mod}(A)$が入射的対象を十分に持つことを示すには,射影加群の場合と違って多少手間がかかる.

\begin{thm}[Baerの基準]\index{#Bearのきじゅん@Bearの基準}\label{thm:Bear's Criterion}
	$A$加群$I$が入射的であることと,すべての$A$のイデアル$\ideal{a}$に対して$\ideal{a}\to I$が$A\to I$に拡張できることは同値である.
\end{thm}

\begin{proof}
	片方の矢印は明らか.$M\subset N$を$A$加群とし,$\varphi:M\to I$とする.ここで,Zornの補題から$N'$を$M\subset N'\subset N$なる加群のうちで$\varphi$を拡張できる極大のものとしてとれる.ここで$N'\neq N$を仮定する.任意の$x\in N-N'$をとる.ここで$\ideal{a}=\mkset{a\in A}{ax\in N'}$は$A$のイデアルとなる.このとき$\ideal{a}\to N';a\mapsto ax$と$\varphi':N'\to A$の合成は,仮定より$\psi:A\to I$に持ち上がる.ここで;
	\[\varphi'':N'+Ax\to I:n'+an\mapsto \varphi'(n')+\psi(a)\]
	はその構成から$N'$に制限すると$\varphi$に一致する.これは$N'$の極大性に矛盾.よって$I$は入射的となる. 
\end{proof}

この定理から直ちに整域$A$の商体$K$は入射的$A$加群である.実際$f:\ideal{a}\to K$を$A$のイデアルからの$A$準同型とすると,任意の$x,y\in \ideal{a}$について$yf(x)=xf(y)$が成り立つ.すなわち固定された1つの元$x_0\in \ideal{a}$について$f(x)=x\cdot f(x_0)/x_0$が常に成り立つ.よって$f':A\to K;x\mapsto xf(x_0)/x_0$が$f$の拡張になる.

$A$がPIDという特殊な場合には入射加群はよりわかりやすい表示を持つ.
\begin{defi}[可除加群]
	$A$を環,$M$を$A$加群とする.任意の$x\in M$と$a\neq0\in A$について$ay=x$となる$y\in M$が存在するような$M$を\textbf{可除加群}(divisible module)という.\index{かじょかぐん@可除加群}
\end{defi}

\begin{prop}
	$A$をPIDとする.$A$加群$I$が可除加群であることと,入射加群であることは同値.
\end{prop}

\begin{proof}
	\begin{eqv}
		\item $A$のイデアルはすべて$(a)$の形をしている.このとき$A$準同型$f':(a)\to I$は$f(a)$で決まる.$I$が可除なのである$y\in I$が存在して$ay=f(a)$とかけるから,$f:A\to I;x\mapsto xy$が$f'$の延長になる.よって$I$は入射的.
		
		\item 任意の$x\in I,a\neq0\in A$をとる.$f':(a)\to I;a\mapsto x$は$f:A\to I$に延びる.すると$1\in A$より$x=f(a)=af(1)$となるので$I$は可除加群である.
	\end{eqv}
\end{proof}

この命題と商体の場合を組み合わせて$\Z$加群$\Q/\Z$が入射的であることがわかる.この加群は以下の議論で黒子的な活躍をする.$A$加群$M$に対し$T$を$\Z$加群とすると,$M$の$T$双対$M^\ast=\hom_{\Z}(M,T)$は$a\cdot f:x\mapsto f(ax)$と定めることで$A$加群になる.
\begin{lem}
	$I$を入射的$\Z$加群とすると,射影加群$P$に対しその$I$双対$P^\ast$は入射的である.
\end{lem}
\begin{proof}
	$M\subset N$と$\varphi:M\to P^\ast$をとると;
	\[M\times P\to I;(m,y)\mapsto \varphi(m)(y)\]
	が$A$双線形になり,テンソル積の普遍性から$\psi:M\otimes P\to I$がある.ここで$P$は射影加群なので,平坦だから自然な$M\hookrightarrow N$と$\id_P$のテンソル$M\otimes P\to N\otimes P$は単射.このとき$I$が入射的なので$\hat\psi:N\otimes P\to I$が存在し,これにより;
	\[\hat\varphi:N\to P^\ast; x\mapsto (y\mapsto \hat\psi(x\otimes y))\]
	が定まるが,これは構成から$x\in M$ならば$\hat\psi(x\otimes y)=\psi(x\otimes y)=\varphi(x)(y)$となり$\hat\varphi(x)=\varphi(x)$となる.よって$P^\ast$は入射的である.
\end{proof}
\begin{lem}
	$A$加群$M$に対して,$T=\Q/\Z$双対$M^\ast$の双対$M^{\ast\ast}$を考えると単射$M\to M^{\ast\ast}$が存在する.
\end{lem}
\begin{proof}
	次の準同型;
	\[i:M\to M^{\ast\ast};x\mapsto (f\mapsto f(x))\]
	が単射になる.$i(x)=0$と仮定しよう.このとき$x=0$を示したいので,$x\neq0$ならばある$f\in M^\ast$が存在して$f(x)\neq0$を示せばよい.
	
	$x$の$M$での位数が$n$ならば$f(x)=1/n+\Z$,位数が無限なら$f(x)=1/2+\Z$と定義すると$\Z$準同型$m\Z\to T$が定まる.$T$の単射性からこれは$M\to T$に持ち上がり$f(x)\neq0$である.
\end{proof}

\begin{thm}\label{thm:加群の圏はhas enough injectives}
	$A$加群の圏は入射的対象を充分に持つ.
\end{thm}
\begin{proof}
	$M$を$A$加群とする.$M$の$T$双対$M^\ast$は生成系をとると,自由加群$P$からの全射$s:P\to M^\ast$が存在する.$T$双対をとって$s^\ast:M^{\ast\ast}\to P^\ast;\varphi\mapsto\varphi\circ s$を考える.これは$s$が全射なので単射である.上の補題たちより$P^\ast$は入射的で,単射$M\hookrightarrow M^{\ast\ast}$が存在するので,$M\hookrightarrow P^\ast=I$が存在する.
\end{proof}

\section{導来関手}

いよいよ導来関手の定義である.
\begin{defi}[左導来関手]\index{ひだりどうらいかんしゅ@導来関手}
	$\mathscr{A,B}$をAbel圏とし,対象$A\in\mathscr{A}$と加法的右完全関手$F:\mathscr{A}\to\mathscr{B}$を考える.$A$の射影分解$P_\bullet$について,$F P_\bullet$も複体になる.これに対する$\mathscr{B}$の中でのホモロジー$H_i(F P_\bullet)$を$L_i F (A)$とかいて,$L_iF$を$F$の$A$における$i$次\textbf{左導来関手}(left derived functor)という.
\end{defi}

\textbf{左}導来関手は\textbf{誤植ではない.} 右完全関手によって,複体(左に伸びる鎖)が左に伸びるホモロジーの列を作る(\ref{thm:左導来関手の特徴付け}をみよ)から左導来関手と呼ぶ.

この定義では$P_\bullet$を無視して$L_i F(A)$と書いているのだから,射影分解のとり方によらないことを証明する必要がある(\ref{prop:左導来関手のwell-definedness}).つまり$J_\bullet$を$A$の別の射影分解とすると,$F$で送ったときにホモロジーが一致せねばならない.このことを\textbf{擬同型}であるという.

\begin{defi}[擬同型]\index{ぎどうけいふくたい@擬同型(複体)}
	複体$A_\bullet,B_\bullet$に対し,各$i$について$H_i(A_\bullet)=H_i(B_\bullet)$であるとき$A_\bullet$と$B_\bullet$は\textbf{擬同型}(quasi-isomorphic)であるという.
\end{defi}

この用語は余鎖複体$A^\bullet,B^\bullet$についてコホモロジーが一致するときにも用いられる.

\begin{prop}\label{prop:左導来関手のwell-definedness}
	$P_\bullet,Q_\bullet$を$A$の射影分解とすると,加法的右完全関手$F$について$F P_\bullet$と$F Q_\bullet$は擬同型である.
\end{prop}
\begin{proof}
	
まず,複体の射$f_\bullet:P_\bullet\to Q_\bullet,g_\bullet:Q_\bullet\to P_\bullet$を構成しよう.
	
全射$\varepsilon:P_0\to A,\varepsilon':Q_0\to A$を考える.$P_0$は射影的なので,$\varepsilon'$に対して$f_0:P_0\to Q_0$が存在する.次に$d_1'\circ f_1=f_0\circ d_1$となるような$f_1:P_1\to Q_1$を作りたい.
	
\begin{figure}[H]
	\centering
	\begin{tikzcd}	
		\cdots\nxcell P_2\arrow[dd,"f_2",dashed]\nxcell[d_2]P_1\arrow[dd,dashed,"f_1"]\nxcell[d_1]P_0\arrow[dd,"f_0"]\arrow[dr,"\varepsilon"]\\[-1.5em]
		&&&&A\nxcell0\\[-1.5em]
		\cdots\nxcell Q_2\nxcell[d_2']Q_1\nxcell[d_1']Q_0\arrow[ur,"\varepsilon'"]
	\end{tikzcd}
	\caption{}
\end{figure}
	
ここで,$\im(f_0\circ d_1)\subset\im d_1'$であるので次の図式を考えることができ,$P_1$の射影性から$f_1$がとれる.可換性は構成から明らか.

\begin{figure}[H]
	\centering
	\begin{tikzcd}
		P_1\arrow[d,dashed,"f_1"]\arrow[dr,"f_0\circ d_1"]\\
		Q_1\arrow[r,"d_1'"]&\im d_1'\nxcell0
	\end{tikzcd}
	\caption{}
\end{figure}	
	
これを続けることで$f_\bullet$が構成され,$g_\bullet$も同様に作ることができる.

関手$F$を施してホモロジーをとることで,$H_i(Fg_\bullet),H_i(Ff_\bullet)$が同型射であることを示せばよい.ここで;
\[H_i(Fg_\bullet)\circ H_i(Ff_\bullet)=H_i(F(g_\bullet\circ f_\bullet))\]
であるので,$F(g_\bullet\circ f_\bullet)$と$F(\id_{P_\bullet}),F(f_\bullet\circ g_\bullet)$と$F(\id_{Q_\bullet})$がホモトピックであることを示せばよい. $g_\bullet\circ f_\bullet$と$\id_{P_\bullet}$がホモトピックであることを示そう.

$h_n=\id_{P_n}-(f_n\circ g_n)$とおき,$d_{n+1}\circ s_n=h_n-(s_{n-1}\circ d_n)$となる$\{s_n\}$を帰納的に作ろう.

\begin{step}
	\item $n=0$のとき.
	
	$\im h_0\subset\im d_1=\ker\varepsilon$なので,$P_1$の射影性から次の図式のように$s_0:P_0\to P_1$が$d_1\circ s_0=h_0$となるように作れる.
	
	\begin{figure}[H]
		\centering
		\begin{tikzcd}
			&P_0\arrow[d,"h_0"]\arrow[ld,"s_0",dashed,swap]\\
			P_1\nxcell[d_1]\im d_1\nxcell0
		\end{tikzcd}
		\caption{}
	\end{figure}

	\item $n=1$のとき.
	
	$\im (h_1-s_0\circ d_1)\subset\im d_2=\ker d_1$より,$P_2$の射影性から次の図式;
	
	\begin{figure}[H]
		\centering
		\begin{tikzcd}
			&P_1\arrow[d,"h_1-(s_0\circ d_1)"]\arrow[ld,"s_1",swap,dashed]\\
			P_2\nxcell[d_2] \im d_2\nxcell 0
		\end{tikzcd}
		\caption{}
	\end{figure}
	
	のように$s_1:P_1\to P_2$を$d_2\circ s_1=h_1-(s_0\circ d_1)$となるようにできる.

	\item $n-1$まで正しいとき.
	
	同様に$\im (h_n-(s_{n-1}\circ d_n))\subset \im d_{n+1}=\ker d_n$なので,$P_{n+1}$の射影性から$s_n$が定まる.
\end{step}


よって$g_\bullet\circ f_\bullet$と$\id_{P_\bullet}$はホモトピックで, $f_\bullet\circ g_\bullet$と$\id_{Q_\bullet}$についても同様.以上から$H_i(FI_\bullet)=H_i(FJ_\bullet)$であることがわかった.
\end{proof}

これにより,左導来関手は射影分解のとり方によらない.次に右導来関手を考えよう.

\begin{defi}[右導来関手]\index{みぎどうらいかんしゅ@右導来関手}
	$\mathscr{A,B}$をAbel圏とし,対象$A\in\mathscr{A}$と加法的左完全関手$F:\mathscr{A}\to\mathscr{B}$を考える.$A$の入射的分解$I^\bullet$について,$F I^\bullet$も(余鎖)複体になる.これに対する$\mathscr{B}$の中でのコホモロジー$H^i(F I^\bullet)$を$R^i F (A)$とかいて,$F$の$A$における$i$次\textbf{右導来関手}(right derived functor)という.
\end{defi}

右導来関手も入射的分解のとり方によらない.

\begin{prop}
	$I^\bullet,J^\bullet$を$A$の入射的分解とすると,$F I^\bullet$と$F J^\bullet$は擬同型である.
\end{prop}
\begin{proof}
	単射$\varepsilon:A\to I^0,\varepsilon':A\to J^0$を考える.このとき$J^0$が入射的対象なので$f^0:I^0\to J^0$が存在する.
	\begin{figure}[H]
		\centering
		\begin{tikzcd}
		&&I^0\arrow[r,"d^0"]\arrow[dd,dashed,"f^0"]&I^1\nxcell[d^1]\cdots\\[-1.5em]
		0\nxcell A\arrow[ru,"\varepsilon"]\arrow[dr,"\varepsilon'",swap]\\[-1.5em]
		&&J^0\nxcell[d'^0]J^1\nxcell[d'^1]\cdots
		\end{tikzcd}
		\caption{}
	\end{figure}

	$\ker d^0\subset\ker(d'^0\circ f^0)$(たしかめよ)であるので,$\varphi:\im d^0\to \im (d'\circ f^0)\subset J^1$が$\varphi\circ d^0=d'^0\circ f^0$となるように定まる.これを$J^1$の入射性から持ち上げて$f^1$を得る.
	
	\begin{figure}[H]
		\centering
		\begin{tikzcd}
			I^0\arrow[drr,"d'^0\circ f^0",bend right]\nxcell[d^0]\im d^0\arrow[rd,"\varphi"]\nxcell I^1\arrow[d,"f^1",dashed]\\
			&&\ J^1
		\end{tikzcd}
		\caption{}
	\end{figure}
	
	これを繰り返して複体の射$f^\bullet$を作ることができる.$g^\bullet$も同様.
	
	$H^i(Fg^\bullet),H^i(Ff^\bullet)$が同型射であることを示せばよい.左導来関手のときと同様に$g^\bullet\circ f^\bullet$と$\id_{I^\bullet}$がホモトピックであることを示そう.
	
	$h_n=\id_{I^n}-(g^n\circ f^n)$とおき,$s^n\circ d^{n-1}=h_{n-1}-(d^{n-1}\circ s^{n-1})$となるような$\{s^n\}$を帰納的に作ろう.左導来関手と違い$n=1$からの議論なことに注意する.
	
	まず$n=1$のとき,$\ker d^0\subset\ker h^0$なので$\varphi:\im d^0\to\im h^0\subset I$が$\varphi\circ d^0=h^0$となるように定まる.これを$I^1$の入射性から持ち上げて$s^1$を得る.
	
	\begin{figure}[H]
		\centering
		\begin{tikzcd}
			I^0\arrow[d,"h^0"]\nxcell[d^0]\im d^0\arrow[dl,"\varphi"]\nxcell I^1\arrow[lld,bend left,dashed,"s^1"]\\
			I^0
		\end{tikzcd}
		\caption{}
	\end{figure}
	
	次に$s^{n-1}$まで存在するとする.このとき$s^{n-1}\circ d^{n-2}=h_{n-2}-(d^{n-3}\circ s^{n-2})$であることに注意すると,$\ker d^{n-2}\subset\ker (h_{n-1}-(d^{n-2}\circ s^{n-1}))$であることが確かめられる.よって次の図式から$s^n$の存在がわかる.
	
	\begin{figure}[H]
		\centering
		\begin{tikzcd}
			I^{n-1}\arrow[d,"h_{n-1}-(d^{n-2}\circ s^{n-1})",swap]\nxcell[d^{n-1}]\im d^{n-1}\arrow[ld]\nxcell I^n\arrow[lld,dashed,"s^n",bend left]\\
			I^{n-1}
		\end{tikzcd}
		\caption{}
	\end{figure}
	
	 これを繰り返して$\id_{I^\bullet}$と$g^\bullet\circ f^\bullet$がホモトピックであることを得る.よって$H^i(FI^\bullet)=H^i(FJ^\bullet)$であることがわかった.
\end{proof}

ここでは共変関手のみ考えていたが,加法的反変右完全関手は入射分解から左導来関手を導き,加法的反変左完全関手は射影分解から右導来関手を導く.以後証明は共変関手のことしか考えない.また,これらの証明における$f_\bullet,f^\bullet$の構成を真似することで次の補題を得る.

\begin{lem}\label{lem:分解への持ち上げ}
	任意の射$\varphi:A\to B$と$A,B$の射影分解$P_\bullet,Q_\bullet$について,複体の射$\varphi_\bullet:P_\bullet\to Q_\bullet$を次の図式が可換になるようにとれる(入射分解についても同様);
	\begin{figure}[H]
		\centering
		\begin{tikzcd}
			\cdots\nxcell P_2\arrow[dd,"\varphi_2",dashed]\nxcell[d_2]P_1\arrow[dd,"\varphi_1",dashed]\nxcell[d_1]P_0\arrow[dd,"\varphi_0",dashed]\nxcell[\varepsilon]A\arrow[dd,"\varphi"]\arrow[rd]\\[-1.5em]
			&&&&&0\\[-1.5em]
			\cdots\nxcell Q_2\nxcell[d_2']Q_1\nxcell[d_1']Q_0\nxcell[\varepsilon']B\arrow[ur]
		\end{tikzcd}
	\end{figure}
	
\end{lem}

次に,複体の完全列について$H_n$を施すとどうなるのかを観察しよう.

\begin{prop}[ホモロジー長完全列と連結射の存在]\label{lem:ホモロジー長完全列と連結射の存在}
	複体の完全列$\ses[\varphi_\bullet][\psi_\bullet]{A_\bullet}{B_\bullet}{C_\bullet}$について,任意の$n$について$\partial_n:H_n(C_\bullet)\to H_{n-1}(A_\bullet)$が存在して;
	\[\begin{tikzcd}
	\cdots\nxcell[\partial_{n+1}]H_n(A_\bullet)\nxcell[\varphi_n]H_n(B_\bullet)\nxcell[\psi_n]H_n(C_\bullet)\nxcell[\partial_{n}]\cdots\\
	\nxcell[\partial_{1}]H_0(A_\bullet)\nxcell[\varphi_0] H_0(B_\bullet)\nxcell[\psi_0]H_0(C_\bullet)\nxcell 0
	\end{tikzcd}\]
	が完全.
\end{prop}

この長完全列を$\ses{A_\bullet}{B_\bullet}{C_\bullet}$に伴う\textbf{ホモロジー長完全列}(long exact sequence of homologies)といい, $\partial_n$を\textbf{連結射}(connecting morphism)という.\index{れんけつしゃ@連結射}

\begin{proof}
	\begin{step}
		\item 
		
		$A_\bullet$において,$d_n:A_n\to A_{n-1}$に対して;
		\[\widetilde{d_n}:\coker d_{n+1}\to\ker d_{n-1};x+\im d_{n+1}\mapsto d_n(x)\]
		と定めるとこれはwell-definedである.このとき;
		\[\begin{tikzcd}
		0\nxcell\ker\widetilde{d_n}\nxcell\coker d_{n+1}\nxcell[\widetilde{d_n}]\ker d_{n-1}\nxcell\coker\widetilde{d_n}\nxcell0
		\end{tikzcd}\]
		は完全で,構成から$\ker\widetilde{d_n}=\ker d_n/\im d_{n+1}=H_n(A_\bullet),\coker\widetilde{d_n}=\ker/d_{n-1}/\im\widetilde{d_n}=H_{n-1}(A_\bullet)$なので,完全列;
		\[\begin{tikzcd}
		0\nxcell H_n(A_\bullet)\nxcell\coker d_{n+1}\nxcell[\widetilde{d_n}]\ker d_{n-1}\nxcell H_{n-1}(A_\bullet)\nxcell0
		\end{tikzcd}\]
		が得られた.
		
		\item 
		
		それぞれの複体の境界作用素を$d_{n,1},d_{n,2},d_{n,3}$とおくと,蛇の補題から次の可換図式がある.
		\begin{figure}[H]
			\centering
			\begin{tikzcd}
			0\nxcell\ker d_{n,1}\arrow[d]\nxcell[]\ker d_{n,2}\arrow[d]\arrow[ddd,phantom,""{coordinate,name=Z}]\nxcell[]
			\ker d_{n,3}\arrow[d]\arrow[r,"",no head]
			\arrow[llddd,near start,
			rounded corners,
			to path={ -- ([xshift=4.5em]\tikztostart.east)
				|- (Z) [near end]\tikztonodes
				-| ([xshift=-4.5em]\tikztotarget.west)
				-- (\tikztotarget)}
			]&{}\\
			0\nxcell A_n\arrow[d,near start]\nxcell[\varphi_n]B_n\arrow[d,near start]\nxcell[\psi_n]C_n\arrow[d,near start]\nxcell0\\[2em]
			0\nxcell A_{n-1}\arrow[d]\nxcell[\varphi_{n-1}]B_{n-1}\arrow[d]\nxcell[\psi_{n-1}]C_{n-1}\arrow[d]\nxcell0\\
			&\coker d_{n-1,1}\nxcell[]\coker d_{n-2,2}\nxcell[]\coker d_{n-1,3}\nxcell0
			\end{tikzcd}
		\end{figure}
		特に;
		\[\begin{tikzcd}
		\coker d_{n+1,1}\nxcell\coker d_{n+1,2}\nxcell\coker d_{n+1,3}\nxcell0
		\end{tikzcd}\]
		\[\begin{tikzcd}
		0\nxcell\ker d_{n-1,1}\nxcell\ker d_{n-1,2}\nxcell\ker d_{n-1,3}
		\end{tikzcd}\]
		が完全である.すると再び蛇の補題とStep.1から;
		\begin{figure}[H]
			\centering
			\begin{tikzcd}
			&H_n(A_\bullet)\arrow[d]\nxcell[\varphi_n]H_n(B_\bullet)\arrow[ddd,phantom,""{coordinate,name=Z}]\arrow[d]\nxcell[\psi_n]H_n(C_\bullet)\arrow[d]\arrow[r,"\partial_n",no head]
			\arrow[llddd,near start,
			rounded corners,
			to path={ -- ([xshift=4.5em]\tikztostart.east)
				|- (Z) [near end]\tikztonodes
				-| ([xshift=-4.5em]\tikztotarget.west)
				-- (\tikztotarget)}
			]&{}\\
			&\coker d_{n+1,1}\arrow[d,"\widetilde{d_{n,1}}",near start]\nxcell\coker d_{n+1,2}\arrow[d,"\widetilde{d_{n,2}}",near start]\nxcell\coker d_{n+1,3}\arrow[d,"\widetilde{d_{n,3}}",near start]\nxcell0\\[2em]
			0\nxcell\ker d_{n-1,1}\arrow[d]\nxcell\ker d_{n-1,2}\arrow[d]\nxcell\ker d_{n-1,3}\arrow[d]\\
			&H_{n-1}(A_\bullet)\nxcell[\varphi_{n-1}]H_{n-1}(B_\bullet)\nxcell[\psi_{n-1}]H_{n-1}(C_\bullet)
			\end{tikzcd}
		\end{figure}
		が得られる.
	\end{step}
\end{proof}
		連結射$\partial_n$は記号的には$\varphi_{n-1}^{-1}\circ d_{n,2}\circ \psi_n^{-1}$と書くことができることに注意しよう.

\begin{prop}[連結射の可換性]\label{prop:連結射の可換性}
	各行が完全であるような複体の可換図式;
	\begin{figure}[H]
		\centering
		\begin{tikzcd}
		0\nxcell A_\bullet\arrow[d,"f_\bullet"]\nxcell[\varphi_\bullet]B_\bullet\arrow[d,"g_\bullet"]\nxcell[\psi_\bullet]C_\bullet\arrow[d,"h_\bullet"]\nxcell0\\
		0\nxcell A_\bullet'\nxcell[\varphi_\bullet']B_\bullet'\nxcell[\psi_\bullet']C_\bullet'\nxcell0
		\end{tikzcd}
	\end{figure}
	について,各$n$と連結射$\partial_n:H_n(C_\bullet)\to H_{n-1}(A_\bullet),\delta_n:H_n(C_\bullet')\to H_{n-1}(A_\bullet')$に対して;
	\begin{figure}[H]
		\centering
		\begin{tikzcd}
		H_n(C_\bullet)\arrow[d,"H_n(h_n)"]\nxcell[\partial_n]H_{n-1}(A_\bullet)\arrow[d,"H_n(f_n)"]\\
		H_n(C_\bullet')\nxcell[\delta_n]H_{n-1}(A_\bullet')
		\end{tikzcd}
	\end{figure}
	が可換である.
\end{prop}

\begin{proof}
	$\partial_n=\varphi_{n-1}^{-1}\circ d_{n,2}\circ\psi_n^{-1},\delta_n={\varphi_{n-1}'}^{-1}\circ{d_{n,2}}'\circ{\psi_n'}^{-1}$に注意すると,$\ker{d_{n,3}}/\im d_{n+1,3}$上で${\varphi_{n-1}'}^{-1}\circ{d_{n,2}'}\circ{\psi_n'}^{-1}\circ h_n=f_{n-1}\circ\varphi_{n-1}^{-1}\circ d_{n,2}\circ\psi_n^{-1}$に帰着するが,これは次の図式を追うことでわかる.
	\begin{figure}[H]
		\centering
		\begin{tikzcd}[row sep=scriptsize, column sep=scriptsize]
		& A_n \arrow[dl] \arrow[rr] \arrow[dd] & & B_n \arrow[dl,"d_{n,2}",swap] \arrow[dd] \arrow[rr,"\psi_n"]&&C_n\arrow[dd,"h_n"]\arrow[ld]\\
		A_{n-1} \arrow[rr, crossing over,"\varphi_{n-1}",near end] \arrow[dd,"f_{n-1}"] & & B_{n-1} \arrow[rr,crossing over]&& C_{n-1}\\
		& A_n' \arrow[dl] \arrow[rr] & & B_n' \arrow[dl,"d_{n,2}'"] \arrow[rr,near start,"\psi_n'"]&&C_n'\arrow[ld]\\
		A_{n-1}' \arrow[rr,"\varphi_{n-1}'"] & & B_{n-1}' \arrow[from=uu, crossing over]\arrow[rr]&&C_{n-1}'\arrow[from=uu,crossing over]\\
		\end{tikzcd}
		\caption{}
	\end{figure}
\end{proof}

以上2つの結果はコホモロジーについても同様に成り立ち,\textbf{コホモロジー長完全系列}と連結射$\partial^n$の存在と可換性がいえる.

\begin{thm}[左導来関手の特徴付け]\label{thm:左導来関手の特徴付け}
	$F$を$\mathscr{A}\to\mathscr{B}$の加法的右完全関手とする.このとき$F$の左導来関手$L_iF$に対し次が成り立つ.
	\begin{defiterm}{LDF}
		\item $L_0 F\cong F$である.
		\item $\mathscr{A}$の完全列$\ses[\varphi][\psi]{A_1}{A_2}{A_3}$
		に対し,各$i\geq0$について連結射$\partial_{i+1}:L_{i+1}F(A_3)\to L_{i}F(A_1)$が存在して;
		\[\begin{tikzcd}
			\cdots\nxcell[\partial_{n+1}]L_nF(A_1)\nxcell[L_nF(\varphi)]L_nF(A_2)\nxcell[L_nF(\psi)]L_nF(A_3)\\
			\nxcell[\partial_{n}]\cdots\\
			\nxcell[\partial_{2}]L_1F(A_1)\nxcell[L_1F(\varphi)]L_1F(A_2)\nxcell[L_1F(\psi)]L_1F(A_3)\\
			\nxcell[\partial_{1}]F(A_1)\nxcell[F(\varphi)]F(A_2)\nxcell[F(\psi)]F(A_3)\nxcell 0
		\end{tikzcd}\]
		が$\mathscr{B}$の完全列になる.
		
		\item $\mathscr{A}$の可換図式;
		\[\begin{tikzcd}
		0\nxcell A_1\arrow[d,"f"]\nxcell[\varphi]A_2\arrow[d,"g"]\nxcell[\psi]A_3\arrow[d,"h"]\nxcell0\\
		0\nxcell B_1\nxcell[\lambda]B_2\nxcell[\mu]B_3\nxcell0
		\end{tikzcd}\]
		に対して,下の列についての連結射を$\delta_i:L_{i+1}F(B_3)\to L_{i}F(B_1)$とすると,図式;
		\[\begin{tikzcd}
		L_{i+1}F(A_3)\arrow[d,"L_{i+1}F(h)"]\nxcell[\partial_i]L_iF(A_1)\arrow[d,"L_iF(f)"]\\
		L_{i+1}F(B_3)\nxcell[\delta_i]L_iF(B_1)
		\end{tikzcd}\]
		が可換である.
		
		\item $P$を射影的対象とすると,$i>0$について$L_iF(P)=0$である.
	\end{defiterm}
\end{thm}

逆に, $F:\mathscr{A}\to\mathscr{B}$を加法的右完全な関手とする.このとき,$T_0,T_1,\cdots$という加法的な関手の列が;
\begin{sakura}
	\item 同型な自然変換$T^0\cong F$がある.
	
	\item $\mathscr{A}$の完全列$\ses[\varphi][\psi]{A_1}{A_2}{A_3}$について$\mathscr{B}$の射$\partial_n:T_{n+1}(A_3)\to T_n(A_1)$が存在して;
	\[\begin{tikzcd}
		\cdots\nxcell T_{n+1}(A_3)\nxcell[\partial_n] T_n(A_1)\nxcell T_n(A_2)\nxcell T_n(A_1)\nxcell[\partial_{n-1}]\cdots\\
		&\cdots\nxcell[\partial_1] F(A_1)\nxcell F(A_2)\nxcell F(A_3)\nxcell 0 
	\end{tikzcd}\]
	が完全.
	
	\item $\mathscr{A}$での可換図式;
	\begin{figure}[H]
		\centering
		\begin{tikzcd}
			0\nxcell A_1\arrow[d]\nxcell A_2\arrow[d]\nxcell A_3\arrow[d]\nxcell0\\
			0\nxcell B_1\nxcell B_2\nxcell B_3\nxcell0
		\end{tikzcd}
	\end{figure}
	について,$\mathscr{B}$内で;
	
	\begin{figure}[H]
		\centering
		\begin{tikzcd}
			T_{i+1}(A_3)\arrow[d]\nxcell[\partial_i]T_i(A_1)\arrow[d]\\
			T_{i+1}(B_3)\nxcell[\delta_i] T_i(B_1)
		\end{tikzcd}
	\end{figure}
	が可換.

	\item $\mathscr{A}$の射影的対象$P$について$i\geq1$について$T_i(P)=0$.
\end{sakura}

を満たすとき,$T_i$を$F$の左導来関手と定義することができる.

\begin{proof}[\textbf{\ref{thm:左導来関手の特徴付け}の証明}]
	\begin{defiterm}{LDF}
		\item $A$の射影分解$P_\bullet$について, $L_0F(A)=H_0(FP_\bullet)=\ker F(d_0)/\im F(d_1)=F(P_0)/\im F(d_1)$であるが,$F$は右完全なので$\begin{tikzcd}
			F(P_1)\nxcell[F(d_1)]F(P_0)\nxcell[F(\varepsilon)]F(A)\nxcell0
		\end{tikzcd}$は完全.よって$\im F(d_1)=\ker F(d_1)$であるので,$L_0F(A)=F(P_0)/\ker F(d_1)=F(A)$である.
		
		\item $A_1,A_2,A_3$の射影分解からなる複体の完全列で,分裂しているものを作りたい(これ自身はHorseshoe\textbf{の補題}と呼ばれる\index{#Horseshoeのほだい@Horseshoeの補題}). $A_1,A_3$の射影分解の初項を$P_{0,1},P_{0,3}$とする.ここで $P_{0,2}=P_{0,1}\midoplus P_{0,3}$とおくと,これは射影的である.また自然な単射,全射があって次の図式が考えられる;
		\begin{figure}[H]
			\centering
			\begin{tikzcd}
				&P_{0,1}\arrow[d,"\varepsilon_1"]\arrow[r,shift left=.5ex,"\iota_0"]&P_{0,2}\arrow[d,dashed,"\varepsilon_2"]\arrow[l,shift left=.5ex,"p_0"]\arrow[r,shift left=.5ex,"\pi_0"]&P_{0,3}\arrow[d,"\varepsilon_3"]\arrow[l,shift left=.5ex,"i_0"]\\
				0\nxcell A_1\arrow[d]\nxcell[\varphi]A_2\arrow[d]\nxcell[\psi]A_3\arrow[d]\nxcell 0\\
				&0&0&0
			\end{tikzcd}
			\caption{}\label{fig:LDF-1}
		\end{figure}
		これが可換になるような全射$\varepsilon_2:P_{0,2}\to A_2$を作りたい.まず$P_{0,2}$の射影性から,$\varepsilon_3\circ\pi_0:P_{0,2}\to A_3$の拡張$\varepsilon_2':P_{0,2}\to A_2$が定まる.ここで$\varepsilon_2=\varphi\circ\varepsilon_1\circ p_0+\varepsilon_2'$とおくと,これは図式を可換にする全射となる($\varepsilon_2'$だけみていると$P_{0,3}$の情報はでてくるが$P_{0,1}$の情報はでてこないので,そこを補おうという気持ち).すると蛇の補題から;
		\[\ses{\ker\varepsilon_1}{\ker\varepsilon_2}{\ker\varepsilon_3}\]
		は完全.ここで$A_1,A_3$の射影分解について境界作用素をそれぞれ$d_{i,1},d_{i,3}$としたとき$\ker\varepsilon_1=\im d_{1,1},\ker\varepsilon_3=\im d_{1,3}$であるので,$P_{1,2}=P_{1,1}\oplus P_{1,3}$とおくことで次の可換図式がある.
		\begin{figure}[H]
			\centering
			\begin{tikzcd}
				&P_{1,1}\arrow[d,"d_{1,1}"]\arrow[r,shift left=.5ex,"\iota_1"]&P_{1,2}\arrow[d,dashed,"d_{1,2}"]\arrow[l,shift left=.5ex,"p_1"]\arrow[r,shift left=.5ex,"\pi_1"]&P_{1,3}\arrow[d,"d_{1,3}"]\arrow[l,shift left=.5ex,"i_1"]\\
				0\nxcell \im d_{1,1}\arrow[d]\nxcell[]\ker\varepsilon_2\arrow[d]\nxcell[]\im d_{1,3}\arrow[d]\nxcell 0\\
				&0&0&0
			\end{tikzcd}
			\caption{}
		\end{figure}
		これはFigure.\ref{fig:LDF-1}と全く同様にして全射$d_{1,2}$の存在を導く.ゆえに,$A_1,A_2,A_3$の射影分解$P_{\bullet,1},P_{\bullet,2},P_{\bullet,3}$からなる複体の完全列で,分裂しているものができる.すると\ref{thm:分裂しているなら完全性は保存される}により$F$を施しても完全なので,\ref{lem:ホモロジー長完全列と連結射の存在}を適用することができる.
		
		\item 
		
		\ref{lem:分解への持ち上げ}と(LDF1)の証明から,それぞれの射影分解からなる複体の完全列で,行は分裂しているものができる.これに$F$と\ref{prop:連結射の可換性}を施して求める結果を得る.
		
		\item $\begin{tikzcd}\cdots\nxcell0\nxcell0\nxcell P\nxcell0
		\end{tikzcd}$
		自体が射影分解となることから明らか.
	\end{defiterm}
\end{proof}

右導来関手についても同様に得られる.証明は省略するが,結果だけ述べておこう.

\begin{thm}[右導来関手の特徴付け]
	$F$を$\mathscr{A}\to\mathscr{B}$の加法的左完全関手とする.このとき$F$の導来関手$R^iF$に対し;
	\begin{defiterm}{RDF}
		\item $R^0 F\cong F$である.
		\item $\mathscr{A}$の完全列$\ses[\varphi][\psi]{A_1}{A_2}{A_3}$に対し,各$i\geq0$について連結射$\partial^i:R^iF(A_3)\to R^{i+1}F(A_1)$が存在して;
		\[\begin{tikzcd}[row sep=tiny, column sep=scriptsize]
		0\nxcell F(A_1)\nxcell[F(\varphi)]F(A_2)\nxcell[F(\psi)]F(A_3)\nxcell[\partial^0]{}\cdots\\%	R^1F(A_1)\nxcell[R^1F(\varphi)]R^1F(A_2)\nxcell[R^1F(\psi)]R^1F(A_3)\nxcell[\partial^1]R^2F(A_1)\nxcell[R^2F(\varphi)]\cdots\\
		{}\nxcell[\partial^{i-1}]R^iF(A_1)\nxcell[R^iF(\varphi)]R^iF(A_2)\nxcell[R^iF(\psi)]R^iF(A_3)\nxcell[\partial^i]\cdots
		\end{tikzcd}\]
		が$\mathscr{B}$の完全列になる.
		
		\item $\mathscr{A}$の可換図式;
		\[\begin{tikzcd}
		0\nxcell A_1\arrow[d,"f"]\nxcell[\varphi]A_2\arrow[d,"g"]\nxcell[\psi]A_3\arrow[d,"h"]\nxcell0\\
		0\nxcell B_1\nxcell[\lambda]B_2\nxcell[\mu]B_3\nxcell0
		\end{tikzcd}\]
		に対して,下の列についての連結射を$\delta^i:R^iF(B_3)\to R^{i+1}F(B_1)$とすると,図式;
		\[\begin{tikzcd}
		R^iF(A_3)\arrow[d,"R^iF(h)"]\nxcell[\partial^i]R^{i+1}F(A_1)\arrow[d,"R^{i+1}F(f)"]\\
		R^iF(B_3)\nxcell[\delta^i]R^{i+1}F(B_1)
		\end{tikzcd}\]
		が可換である.
		
		\item $I$を入射的対象とすると,$i>0$について$R^iF(I)=0$である.
	\end{defiterm}
\end{thm}
\section{二重複体}
導来関手の例として$\Tor,\Ext$を定義したいのだが,計算の必要性から\textbf{二重複体}についての知識が必要となる.

\begin{defi}[二重複体]\index{にじゅうふくたい@二重複体}
	Abel圏$\mathscr{A}$の対象の族$\{X_{p,q}\}_{p,q\in\N}$と,射$d_{p,q}':X_{p,q}\to X_{p-1,q}, d_{p,q}'':X_{p,q}\to X_{p,q-1}$の族$\{d_{p,q}'\},\{d_{p,q}''\}$について;
	\[d_{p-1,q}'\circ d_{p,q}'=0,\quad d_{p,q-1}''\circ d_{p,q}''=0,\quad d_{p-1,q}''\circ d_{p,q}'+d_{p,q-1}'\circ d_{p,q}''=0\]
	が成り立つとき,これらをまとめて$X_{\bullet,\ast}$とかいて\textbf{二重複体}(double chain complex)という.
\end{defi}

\begin{figure}[H]
	\centering
		\begin{tikzcd}
			&\vdots\arrow[d]&\vdots\arrow[d]&\vdots\arrow[d]\\
			\cdots\nxcell X_{p+1,q+1}\arrow[d,"d_{p+1,q+1}''"]\nxcell[d_{p+1,q+1}']X_{p,q+1}\arrow[d,"d_{p,q+1}''"]\nxcell[d_{p,q+1}']X_{p-1,q+1}\arrow[d,"d_{p-1,q+1}''"]\nxcell\cdots\\
			\cdots\nxcell X_{p+1,q}\arrow[d,"d_{p+1,q}''"]\nxcell[d_{p+1,q}']X_{p,q}\arrow[d,"d_{p,q}''"]\nxcell[d_{p,q}']X_{p-1,q}\arrow[d,"d_{p-1,q}''"]\nxcell\cdots\\
			\cdots\nxcell X_{p+1,q-1}\arrow[d]\nxcell[d_{p+1,q-1}']X_{p,q-1}\arrow[d]\nxcell[d_{p,q-1}']X_{p-1,q-1}\arrow[d]\nxcell\cdots\\
			&\vdots&\vdots&\vdots
	\end{tikzcd}
	\caption{二重複体}\label{fig:二重複体}
\end{figure}

双対的に\textbf{二重余鎖複体}についても同様の定義ができる.
\begin{figure}[H]
	\centering
	\begin{tikzcd}
		&\vdots\arrow[d]&\vdots\arrow[d]&\vdots\arrow[d]\\
		\cdots\nxcell X^{p-1,q-1}\arrow[d,"{d''}^{p-1,q-1}"]\nxcell[{d'}^{ p-1,q-1}]X^{p,q-1}\arrow[d,"{d''}^{p,q-1}"]\nxcell[{d'}^{p,q-1}]X^{p+1,q-1}\arrow[d,"{d''}^{p+1,q-1}"]\nxcell\cdots\\
		\cdots\nxcell X^{p-1,q}\arrow[d,"{d''}^{p-1,q}"]\nxcell[{d'}^{p-1,q}]X^{p,q}\arrow[d,"{d''}^{p,q}"]\nxcell[{d'}^{p,q}]X^{p+1,q}\arrow[d,"{d''}^{p+1,q}"]\nxcell\cdots\\
		\cdots\nxcell X^{p-1,q+1}\arrow[d]\nxcell[{d'}^{p-1,q+1}]X^{p,q-1}\arrow[d]\nxcell[{d'}^{p,q-1}]X^{p+1,q-1}\arrow[d]\nxcell\cdots\\
		&\vdots&\vdots&\vdots
	\end{tikzcd}
	\caption{二重余鎖複体}
\end{figure}

本によって射が満たすべき性質が異なることに注意しておく.ここでは\cite{kawada},\cite{siho}に合わせた.\cite{kato}ではFigure.\ref{fig:二重複体}が可換であることを要請している.\cite{kato}のように;
\[d_{p-1,q}'\circ d_{p,q}'=0,\quad d_{p,q-1}''\circ d_{p,q}''=0,\quad d_{p-1,q}''\circ d_{p,q}'=d_{p,q-1}'\circ d_{p,q}''\]
を仮定すると,これは複体の複体となるので$\ch(\ch(\mathscr{A}))$の対象となる.一般にこれは二重複体にはならないが,$d_{p,q}''$を$-d_{p,q}''$に変えることにより二重複体が得られる.この対応は$\ch(\ch(\mathscr{A}))$と二重複体の圏の間の圏同値を与える.

複体を二重複体とみなす自然な方法は(2つ)あることがすぐにわかるが,二重複体から複体を得ることもできる.以下ではとりあえず3通り紹介しよう.

\begin{defi}[全複体]\index{ぜんふくたい@全複体}
	二重複体$X_{\bullet,\ast}$について;
	\[T_n=\bigoplus_{p+q=n} X_{p,q},\quad d_n=\sum_{p+q=n}(d_{p,q}'+d_{p,q}''):T_n\to T_{n-1}\]
	と定めると$T_\bullet$は複体となる.これを$X_{\bullet,\ast}$の\textbf{全複体}(total chain complex)という.
\end{defi}

また,各行,列からも複体が作られる.
\begin{defi}
	二重複体$X_{\bullet,\ast}$について;
	\[A_q=\coker d_{1,q}',\quad B_p=\coker d_{p,1}''\]
	とおくと,\ref{lem:核余核の可換性}より定まる$d_q^A:A_q\to A_{q-1},d_p^B:P_p\to B_{p-1}$によって$\{A_q,d_q^A\},\{B_p,d_p^B\}$は複体となる.これを$X_{\bullet,\ast}$の\textbf{辺複体}(bordered chain complex)という.
\end{defi}

\begin{figure}[H]
	\centering
	\begin{tikzcd}[row sep=scriptsize]
		&&&&\vdots\arrow[d]&\vdots\arrow[d]&\vdots\arrow[d]\\
		&&&\cdots\nxcell X_{1,q}\arrow[d,"d_{1,q}''"]\nxcell[d_{1,q}'] X_{0,q}\darrow[d_{0,q}'']\nxcell[\varepsilon_q']A_q\darrow[d_q^A]\nxcell0\\
		&&&\cdots\nxcell X_{1,q-1}\darrow\nxcell[d_{1,q-1}'] X_{0,q-1}\darrow\nxcell[\varepsilon_{q-1}']A_{q-1}\darrow\nxcell0\\
		&\vdots\darrow&\vdots\darrow&{}&\vdots\darrow&\vdots\darrow&\vdots\darrow\\
		\cdots\nxcell X_{p,1}\darrow[d_{p,1}'']\nxcell[d_{p,1}']X_{p-1,1}\darrow[d_{p-1,1}'']\nxcell\cdots\nxcell X_{1,1}\darrow[d_{1,1}'']\nxcell[d_{1,1}']X_{0,1}\darrow[d_{0,1}'']\nxcell[\varepsilon_1']A_1\darrow[d_1^A]\nxcell0\\
		\cdots\nxcell X_{p,0}\darrow[{\varepsilon_q}'']\nxcell[d_{p,0}']X_{p-1,0}\darrow[{\varepsilon_{p-1}}'']\nxcell\cdots\nxcell X_{1,0}\darrow[\varepsilon_1'']\nxcell[d_{1,0}']X_{0,0}\darrow[\varepsilon_0'']\nxcell[\varepsilon_0']A_0\nxcell0\\
		\cdots\nxcell B_p\darrow\nxcell[d_p^B] B_{p-1}\darrow\nxcell\cdots\nxcell B_1\darrow\nxcell[d_1^B]B_0\darrow\\
		&0&0&&0&0
	\end{tikzcd}
	\caption{}
\end{figure}

\begin{thm}\label{thm:辺複体のhomologyは同型}
	二重複体$X_{\bullet,\ast}$の全複体$T_\bullet$,辺複体$A_\bullet,B_\bullet$について,各$p,q$に対して次の列;
	\[\begin{tikzcd}
		\cdots\nxcell[d_{2,q}']X_{1,q}\nxcell[d_{1,q}']X_{0,q}\nxcell[\varepsilon_q']A_q\nxcell0
	\end{tikzcd},\quad\begin{tikzcd}
		\cdots\nxcell[d_{p,2}'']X_{p,1}\nxcell[d_{p,1}'']X_{p,0}\nxcell[\varepsilon_q'']B_q\nxcell0
	\end{tikzcd}\]
	が完全であるならば,ホモロジーについて$H_n(T_\bullet)=H_n(A_\bullet)=H_n(B_\bullet)$が成り立つ.
\end{thm}
\begin{proof}
	全複体$T_\bullet$,辺複体$A_\bullet,B_\bullet$の間に複体の射;
	\[\varphi_n:T_n\to A_n;(x_{p,q})_{p+q=n}\mapsto\varepsilon_n'(x_{0,n}),\quad\psi_n:T_n\to B_n;(x_{p,q})_{p+q=n}\mapsto\varepsilon_n''(x_{n,0})\]
	が定義できる.例えば,$\varphi$について$\varphi_{n+1}\circ d_{n+1}=d_{n+1}^A\circ\varphi_n$を確かめることは簡単である.$\psi$も同様.

	$H_n(\varphi_n):H_n(T_\bullet)\to H_n(A_\bullet)$が全単射であることを示そう.$H_n(\psi_n)$についても同様に示すことができる.
	
	\begin{step}
		\item 単射であること.
		
		$(x_{p,q})_{p+q=n}\in\ker d_n$に対して,$\varepsilon_n'(x_{0,n})\in\im d_{n+1}^A$ならば$(x_{p,q})\in\im d_{n+1}$を示せばよい.まず$\varepsilon_{n+1}'$は全射なので,ある$x_{0,n+1}\in X_{0,n+1}$が存在して$x_{0,n}-d_{0,n+1}''(x_{0,n+1})\in\ker\varepsilon_n'=\im d_{1,n}$である.ゆえに,ある$x_{1,n}\in X_{1,n}$が存在して$x_{0,n}=d_{1,n}'(x_{1,n})+d_{0,n+1}''(x_{0,n+1})$である.
		
		次に,$x_{1,n-1}$について,仮定から$d_{1,n-1}'(x_{1,n-1})+d_{0,n}''(x_{0,n})=0$である.すると$x_{0,n}=d_{1,n}'(x_{1,n})+d_{0,n+1}''(x_{0,n+1})$であったので;
		\[d_{1,n-1}'(x_{1,n-1})+d_{0,n}''(x_{0,n})=d_{1,n-1}'(x_{1,n-1})+d_{1,n-1}'(d_{1,n}''(x_{1,n}))=0\]
		となる.すなわち$x_{1,n-1}-d_{1,n}''(x_{1,n})\in\ker d_{1,n-1}'=\im d_{2,n-1}'$であるから,$x_{1,n-1}=d_{1,n}''(x_{1,n})+d_{2,n-1}'(x_{2,n-1})$となる$x_{2,n-1}\in X_{2,n-1}$がみつかる.
		
		以後帰納的に$x_{n,0}$まで続けることで$(x_{p,q})\in\im d_{n+1}$を示すことができる.
		
		\item 全射であること.
		
		任意の$x_n+\im d_{n+1}^A\in\ker d_n^A$について,$(x_{p+q})_{p+q=n}$を$(x_{p+q})\in\ker d_n,\varepsilon_n'(x_{0,n})=x_n$となるようにとりたい.
		
		まず$\varepsilon_n'$は全射なので,ある$x_{0,n}$で$\varepsilon_n'(x_{0,n})=x_n$となるものが存在する.次に$d_{1,n-1}'(x_{1,n-1})+d_{0,n}''(x_{0,n})=0$となる$x_{1,n-1}$の存在を言いたいので,$-d_{0,n}''(x_{0,n})\in\im d_{1,n-1}'=\ker\varepsilon_{n-1}'$を示せばよいが,$d_n^A$の定義から$\varepsilon_{n-1}'\circ d_{0,n}''=d_n^A\circ\varepsilon_n'$であるので, $\varepsilon_n'(x_{0,n})=x_n\in\ker d_A$より成り立っていることがわかる.以後帰納的に続けることで条件を満たす$(x_{p,q})_{p+q=n}$を構成することができる.
		
	\end{step}
\end{proof}

全複体,辺複体の定義と\ref{thm:辺複体のhomologyは同型}は余鎖複体についても双対的に行うことができる.

満を持して$\Tor$と$\Ext$の登場である.
\begin{defi}[$\Tor$関手]\index{#Tor@$\Tor$}
	加群$M$について,関手$M\otimes -$は右完全である.これによる左導来関手を$\Tor_n(M,-)$とかく.これをTor\textbf{関手}(Tor functor, torsion functor)という.
\end{defi}

この定義からは$\Tor(M,N)$を計算するには$N$の射影分解$P_\bullet$を計算する必要があるように思われるが,\ref{thm:辺複体のhomologyは同型}より次の定理を言うことができる.

\begin{thm}[$\Tor$の可換性]
	$A$加群$M,N$について,$M$の射影分解に$N$をテンソルした複体;
	\[\begin{tikzcd}
		\cdots Q_2\otimes N\nxcell Q_1\otimes N\nxcell Q_0\otimes N\nxcell0
	\end{tikzcd}\]
	の$n$次のホモロジーは$\Tor_n(M,N)$と同型である.特に$\Tor(M,N)\cong\Tor(N,M)$である.
\end{thm}

\begin{proof}
	$M,N$の射影分解からなる複体を$Q_\bullet,P_\bullet$とする.ここで射影加群は平坦であることに注意すると,各$i,j$について次の複体たちは完全である;
	\[\begin{tikzcd}
		\cdots\nxcell P_1\otimes Q_j\nxcell[d_1\otimes\id_{Q_j}] P_0\otimes Q_j\nxcell[\varepsilon\otimes\id_{Q_j}]N\otimes Q_j\nxcell0
	\end{tikzcd}\]
	\[\begin{tikzcd}
		\cdots\nxcell P_i\otimes Q_1\nxcell[\id_{P_i}\otimes d_1'] P_i\otimes Q_0\nxcell[\id_{P_i}\otimes\varepsilon']P_i\otimes M\nxcell0
	\end{tikzcd}\]
	よって,辺複体が$M\otimes P_\bullet,N\otimes Q_\bullet$である二重複体$P_\bullet\otimes Q_\ast$で,各行,列が完全なものができる.これに\ref{thm:辺複体のhomologyは同型}を適用して$\Tor(M,N)\cong\Tor(N,M)$を得る.
\end{proof}

\begin{defi}[$\Ext$関手]\index{#Ext@$\Ext$}
	加群$M$について,関手$\hom(M,-)$は左完全である.これによる右導来関手を$\Ext^n(M,-)$とかき,$\Ext$\textbf{関手}(Ext functor, extension functor)という.
\end{defi}

$\Tor$と同様に,$\Ext(M,N)$の計算は$N$の単射分解と$M$の射影分解のどちらを計算してもよい($\hom(-,N)$は反変左完全であるから).

ここで普遍性と\ref{prop:テンソル積は直和と可換}より次が成り立っている.

\begin{prop}
	$A$加群の圏において,以下が成り立つ.
	\begin{sakura}
		\item $\hom(\bigoplus_{\lambda\in\Lambda} M_\lambda,N)\cong\prod_{\lambda\in\Lambda}\hom(M_\lambda,N)$
		\item $\hom(M,\prod_{\lambda\in\Lambda}N_\lambda)\cong\prod_{\lambda\in\Lambda}\hom(M,N_\lambda)$
		\item $M\otimes\bigoplus_{\lambda\in\Lambda}N_\lambda\cong\bigoplus_{\lambda\in\Lambda}(M\otimes N_\lambda)$
	\end{sakura}
	$M$が有限生成なら次も正しい.
	\begin{sakura}\setcounter{enumi}{3}
		\item $\hom(M,\bigoplus_{\lambda\in\Lambda}N_\lambda)\cong\bigoplus_{\lambda\in\Lambda}\hom(M,N_\lambda)$
	\end{sakura}
\end{prop}

\begin{proof}
	(i)$\sim$(iii)についてはまさに直積,直和の普遍性と\ref{prop:テンソル積は直和と可換}による.(iv)は$M$が有限生成なら$f\in\hom(M,\bigoplus N_\lambda)$について有限部分集合$I\subset\Lambda$が存在して, $f(M)\subset\bigoplus_{i\in I} N_i$が成り立つ.
\end{proof}

ここから導来関手たちにも次が言えることがわかる.

\begin{prop}
	$A$加群の圏において,以下が成り立つ.
	\begin{sakura}
		\item $\Ext(\bigoplus_{\lambda\in\Lambda} M_\lambda,N)\cong\prod_{\lambda\in\Lambda}\Ext(M_\lambda,N)$
		\item $\Ext(M,\prod_{\lambda\in\Lambda}N_\lambda)\cong\prod_{\lambda\in\Lambda}\Ext(M,N_\lambda)$
		\item $\Tor(M,\bigoplus_{\lambda\in\Lambda}N_\lambda)\cong\bigoplus_{\lambda\in\Lambda}\Tor(M,N_\lambda)$
	\end{sakura}
	$M$が有限生成なら次も正しい.
	\begin{sakura}\setcounter{enumi}{3}
		\item $\Ext(M,\bigoplus_{\lambda\in\Lambda}N_\lambda)\cong\bigoplus_{\lambda\in\Lambda}\Ext(M,N_\lambda)$
	\end{sakura}
\end{prop}

\begin{proof}
	(i)だけ示す.$M_\lambda$の射影分解を$P_{\bullet,\lambda}$とすると,$\bigoplus_{\lambda\in\Lambda} P_{\bullet,\lambda}$は$\bigoplus M_\lambda$の射影分解になる.すると$\hom(-,N)$によって次の余鎖複体;
	\[\begin{tikzcd}
		\nxcell\bigoplus \hom(P_{n,\lambda},N)\nxcell\bigoplus\hom(P_{n+1,\lambda},N)\nxcell
	\end{tikzcd}\]
	を得るが,これの各成分は$\prod\hom(P_{n,\lambda},N)$と同型である.よって主張が従う.
\end{proof}

%%---archive これいらなかった気がする・・・。
%ここで$\Tor$の計算に使うため,平坦性の同値条件を示しておく.
%\begin{prop}
%	次の2つは同値である.
%	\begin{sakura}
%		\item 	$A$加群$L$は平坦である.
%		\item   $L$を第3項とする任意の完全列$\ses{M_1}{M_2}{L}$
%	について,任意の$A$加群$N$に対し;
%	\[\ses{M_1\otimes N}{M_2\otimes N}{L\otimes N}\]
%	が完全となる.
%	\end{sakura}
%\end{prop}
%
%\begin{proof}
%	\begin{eqv}
%		\item $N$について,射影加群$P$と全射$\varphi:P\to N$がとれるから,完全列;
%		\[\ses[\iota][\varphi]{\ker\varphi}{P}{N}\]
%		がある.ここで射影加群は平坦なので,次の可換図式があり,各行,列は完全である.
%		\begin{figure}[H]
%			\centering
%			\begin{tikzcd}
%				&&&0\arrow[d]\\
%				&M_1\otimes\ker\varphi\arrow[d,"\id_{M_1}\otimes\iota"]\nxcell M_2\otimes\ker\varphi\arrow[d,"\id_{M_2}\otimes\iota"]\nxcell L\otimes\ker\varphi\arrow[d,"\id_L\otimes\iota"]\nxcell0\\
%				0\nxcell M_1\otimes P\arrow[d]\nxcell M_2\otimes P\arrow[d]\nxcell L\otimes P\arrow[d]\nxcell0\\
%				&M_1\otimes N\arrow[d]\nxcell M_2\otimes N\arrow[d]\nxcell L\otimes N\arrow[d]\nxcell0\\
%				&0&0&0
%			\end{tikzcd}
%			\caption{}
%		\end{figure}
%		ここで$\coker\id_{M_1}\otimes\iota=M_1\otimes N,\coker\id_{M_2}\otimes\iota=M_2\otimes N,\coker\id_{L}\otimes\iota=L\otimes N$であるから,上2行について蛇の補題を使うと;
%		\[\ses{M_1\otimes N}{M_2\otimes N}{L\otimes N}\]
%		が完全である.
%		\item 完全列;
%		\[\ses{M_1}{M_2}{M_3}\]
%		を考える.射影加群$P$で, $L$への全射$\varphi:P\to L$が存在するものを取ると, (ii)から次の可換図式で完全なものがとれる.
%		\begin{figure}[H]
%			\centering
%			\begin{tikzcd}
%				&0\arrow[d]&0\arrow[d]&0\arrow[d]\\
%				&M_1\otimes\ker\varphi\arrow[d]\nxcell M_2\otimes\ker\varphi\arrow[d]\nxcell M_3\otimes\ker\varphi\arrow[d]\nxcell0\\
%				0\nxcell M_1\otimes P\arrow[d]\nxcell M_2\otimes P\arrow[d]\nxcell M_3\otimes P\arrow[d]\nxcell0\\
%				&M_1\otimes L\arrow[d]\nxcell M_2\otimes L\arrow[d]\nxcell M_3\otimes L\arrow[d]\nxcell0\\
%				&0&0&0
%			\end{tikzcd}
%			\caption{}
%		\end{figure}
%		先ほどと同様に,蛇の補題から;
%		\[\ses{M_1\otimes L}{M_2\otimes L}{M_3\otimes L}\]
%		が完全である.
%	\end{eqv}
%\end{proof}
%


 %Homology algebra
	\part[Homological method to ring theory]{可換環論のホモロジー代数的手法}
前章で定義した射影分解,$\Tor,\Ext$などの道具を使って可換環の理論にホモロジー代数的手法を持ち込もう.以後,$\Tor(M,N)=\Tor(N,M)$であるが$\Ext(M,N)\neq\Ext(N,M)$であることを注意してほしい.

\section{Ext と加群の深さ}
この節では,Cohen--Macaulay性を議論するために必要不可欠な深さの概念について,Extを使った言い換えを与え,ホモロジー代数の道具を用いて考察していく.まずは\ref{lem:depth M/aM=depth M-1}の証明を与えるために,深さとExtの関係,最初は簡単な場合として正則元とHomのつながりについて調べてみる.以後の命題たちの証明に\ref{lem:depth M/aM=depth M-1}は必要ないことに注意しよう.

\begin{lem}
	$A$をNoether環とし,$M$を有限生成$A$加群,$I$を$IM\neq M$となる$A$のイデアルとする.このとき,次の条件;
	\begin{sakura}
		\item $M$正則元である$a\in I$が存在する.
		\item 任意の有限生成$A$加群$N$について,$\supp N\subset V(I)$ならば$\hom(N,M)=0$である.
		\item ある有限生成$A$加群$N$が存在して,$\supp N=V(I)$かつ$\hom(N,M)=0$となる.
	\end{sakura}
	は同値である.
\end{lem}

\begin{proof}
	\begin{eqv}[3]
		\item $a$倍写像$a\cdot:M\to M$は単射である.すると,これを合成する準同型$a\cdot_ \ast:\hom(N,M)\to\hom(N,M)$も単射である.任意の$\varphi\in\hom(N,M)$をとると,$a\cdot_\ast(\varphi)$は$x\mapsto \varphi(ax)$という準同型であることに注意する.ここで$\supp N=V(\ann N)\subset V(I)$なので,\ref{prop:V(I)の包含}より$I\subset \sqrt{\ann N}$である.よってある$n>0$が存在して$a^nN=0$である.すると$\varphi$に$a\cdot_\ast$を$n$回施すとそれは$0$になり,これは単射なので$\varphi$は$0$でなければならない.
		\item $N=A/I$とすればよい.
		\item $I$は$M$正則元を持たないとする.すると\ref{lem:depth 0とass}より$P\in\ass M\cap V(I)$となる$P\in\spec A$が存在する.このとき単射$A/P\to M$が存在する($P=\ann x$とするとき,$A/P\to M;a+P\mapsto ax$とすればよい).これを$P$で局所化して$k(P)\to M_P$が存在する.また,$P\in V(I)=\supp N$であるので$N_P\neq 0$であり,中山の補題より$N_P/PN_P=N\otimes_A k(P)$は$0$でない$k(P)$ベクトル空間である.よって$0$でない$N_P/PN_P\to k(P)$がある.以上のことを組み合わせて$\hom_{A_P}(N_P,M_P)=\hom(N,M)_P\neq0$である.すると\ref{prop:局所化したら0は局所的}より$\hom (N,M)\neq0$であるので矛盾である.
	\end{eqv}
\end{proof}

これを正則列について一般化しよう.

\begin{prop}\label{prop:正則列とExt}
	$A$をNoether環とし,$M$を有限生成$A$加群,$I$を$IM\neq M$となる$A$のイデアルとする.任意の$n>0$について,次の条件;
	\begin{sakura}
		\item $n\leq\mdepth_I M$である.
		\item 任意の有限生成$A$加群について,$\supp N\subset V(I)$ならば任意の$0\leq i<n$について$\Ext^i(N,M)=0$である.
		\item ある有限生成$A$加群$N$が存在して,$\supp N=V(I)$かつ任意の$0\leq i<n$について$\Ext^i(N,M)=0$である.
		\item 任意の$0<i<n$について,$a_1,\dots,a_i\in I$が$M$正則列であるならば,ある$a_{i+1},\dots,a_n\in I$が存在して$a_1,\dots,a_n$が$M$正則列をなす.
	\end{sakura}
	は同値である.
\end{prop}

\begin{proof}
	\begin{eqv}[4]
		\item $n$についての帰納法で示す.$n=1$のときは先の補題でみたので,$n>1$とする.$a_1,\dots,a_n\in I$を$M$正則列とする.完全列;
		\[\ses[a_1\cdot]{M}{M}{M/a_1M}\]
		についてExtが導く完全列を考えて;
		\[\begin{tikzcd}{\Ext^{i-1}(N,M/a_1M)}\nxcell[\partial^{i-1}]{\Ext^i(N,M)}\nxcell[a_1\cdot]{\Ext^i(N,M)}\end{tikzcd}\]
		が完全である($\Ext^i(N,M)$の間の準同型は$a_1\cdot$が誘導するコホモロジーの間の準同型だが,これは$a_1\in A$なので$a_1\cdot$のまま変わらない).ここで$a_2,\dots,a_n$が$M/a_1M$正則列なので$n-1\leq\mdepth_I M/a_1M$だから,帰納法の仮定より$a_1\cdot:\Ext^i(N,M)\to\Ext^i(N,M)$は単射である.またExtの定義から$\ann N\subset\ann(\Ext^i(N,M))$であるので,補題と同様の議論で$\Ext^i(N,M)=0$である.
		\item $N=A/I$とすればよい.
		\item これも$n$についての帰納法で示す.$n=1$のときは見た.$n>1$とし,$i<n$について$a_1,\dots,a_i\in I$が$M$正則列であるとする.(iii)の条件を満たす有限生成$A$加群$N$と$a_1$倍写像についての完全列が導くExtの完全列;
		\[\begin{tikzcd}
		\Ext^{j}(N,M)\nxcell\Ext^{j}(N,M/a_1M)\nxcell[\partial^i]\Ext^{j+1}(N,M)
		\end{tikzcd}\]
		を考える.$j+1<n$のとき,仮定から完全列の両端は$0$となり,任意の$j<n-1$について$\Ext^j(N,M/a_1M)=0$である.$\supp N=V(I)$であるから,帰納法の仮定から$M/a_1M$正則列$a_2,\dots,a_i\in I$を長さ$n-1$に延長できる.よって,番号のズレに気をつけて$a_1,\dots,a_n\in I$を$M$正則列となるようにできることがわかった.
	\end{eqv}
\end{proof}

この命題から即座に次の定理が従う.
\begin{thm}[Rees]\label{thm:Extとdepth}
	$A$をNoether環とし,$M$を有限生成$A$加群,$I$を$IM\neq M$となる$A$のイデアルとする.$I$の元からなる極大な$M$正則列の長さは一定であり,また;
	\[\mdepth_I M=\inf\mkset{i\in\N}{\Ext^i(A/I,M)\neq0}\]
	である.
\end{thm}

\ref{prop:正則列とExt}によって,最初の目的が達成できる.
\begin{lem}[\ref{lem:depth M/aM=depth M-1}の証明]\label{lem:depth M/aMについての証明}
	$A$をNoether環とし,$M$を有限生成$A$加群,$I$を$IM\neq M$となる$A$のイデアルとする.$a\in I$が$M$正則ならば;
	\[\mdepth_I (M/aM)=\mdepth_I M-1\]
	が成り立つ.
\end{lem}

\begin{proof}
	$a_1,\dots,a_n\in I$が$M/aM$正則列ならば$a,a_1,\dots,a_n$は$M$正則であり,$\mdepth_I(M/aM)\leq\mdepth_I M-1$であることはすぐにわかる.また$\mdepth_I M=r$とすると,\ref{prop:正則列とExt}より$a,a_2,\dots,a_r\in I$を$M$正則列であるようにできる.このとき$a_2,\dots,a_r$は$M/aM$正則列をなすので$\mdepth_I M-1\leq\mdepth_I(M/aM)$であることがわかった.
\end{proof}

CM局所環とは$\dim A=\mdepth A$となっている環のことであったことを思い出すと,この定義はKrull次元がホモロジカルな量で与えられている局所環のことである,と言い換えることができる.Krull次元をホモロジカルな量に翻訳することで可換環論に新たな視点が持ち込まれ.\textbf{ホモロジカル予想}と呼ばれる一連の予想群が生まれることとなった.これらの予想については本書のところどころで目にすることになるだろう.

\section{射影被覆と入射包絡}

異なるホモロジカルな量として\textbf{ホモロジー次元}とも呼ばれる\textbf{射影次元}と,その双対概念であるところの\textbf{入射次元}を定義しよう.

\begin{defi}[射影次元]\index{しゃえいじげん@射影次元}
	$A$を環とし,$M$を$A$加群とする.$M$の射影分解の長さの最小値を$M$の\textbf{射影次元}(projective dimension)といい,$\prjdim_A M$とかく.$M=0$のときは$\prjdim M=-1$とする.
\end{defi}

$\prjdim M=0$であることと$M$が射影的であることは同値である.例を見てみよう.

\begin{ex}
	$x\in A$を単元でも零因子でもないとする.$M=A/Ax$とおくと;
	\[\ses[x]{A}{A}{M}\]
	が射影分解となり$\prjdim M=1$である.
\end{ex}

この例で$x$が冪零,例えば$A=\R[X]/(X^2),x=X+(X^2)$なら$x$倍写像$A\to A$の核が$Ax\cong M$であるので;
\[\begin{tikzcd}
	\dots\nxcell A\nxcell[x]A\nxcell[x] A\nxcell M\nxcell0
\end{tikzcd}\]
が無限に続く射影分解となる.だがここから$\projdim M=\infty$を言えるかというと,そうではない.そのためには長さが極小になるような(本質的な)射影分解を考える必要がある.

\begin{defi}[射影被覆]\index{しゃえいひふく@射影被覆}\index{よじょうかぐん@余剰加群}
	$A$加群$M$の部分加群$N$が;
	\[\text{任意の$M$の部分加群$L$について$N+L=M$なら$L=M$.}\]
	を満たすとき,$N$を$M$の\textbf{余剰部分加群}(superfluous submodule)という.$A$加群$M$について射影加群$P$と全射$\varepsilon:P\to M$が存在して$\ker\varepsilon$が$P$の余剰加群のとき,$P,\varepsilon$は$M$の\textbf{射影被覆}(projective cover)であるという.
\end{defi}

\begin{lem}\label{lem:余剰加群の補題}
	加群の準同型の列(完全性は仮定しない)$\begin{tikzcd}
	M_1\nxcell[\varphi]M_2\nxcell[\psi]M_3
	\end{tikzcd}$について, $\psi\circ\varphi$が全射であるとする. $\ker\psi $が$M_2$の余剰加群なら$\varphi$も全射である.
\end{lem}

\begin{proof}
	任意の$x\in M_2$についてある$x_1\in M_1$が存在して$\psi(\varphi(x_1))=\psi(x)$である.このとき$\varphi(x_1)-x\in\ker\psi$なので$M_2=\varphi(M_1)+\ker\psi$である.よって仮定から$M_2=\varphi(M_1)$となる.
\end{proof}

\begin{prop}\label{prop:射影被覆が同型の補題}
	$M$を$A$加群とし,$M$の射影被覆$\varepsilon:P\to M$が存在したとする.射影加群$P'$への全射$\varepsilon':P'\to M$に対して分裂全射$f:P'\to P$が存在して,次の図式;
	\[\begin{tikzcd}
		P'\arrow[dd,"f",swap]\arrow[dr,"\varepsilon'"]\\[-1.5em]
		&M\nxcell0\\[-1.5em]
		P\arrow[ur,"\varepsilon"]
	\end{tikzcd}\]
	が可換($P'$は$P$と同型な直和因子を持つ).
\end{prop}

\begin{proof}
	可換になる$f$の存在は$P'$が射影的であることから従う.いま$\varepsilon\circ f$が全射で$\ker\varepsilon$が余剰加群なので,補題より$f$も全射である.また$P$も射影的であるから,次の図式;
	\[\begin{tikzcd}
		P\arrow[d,"g"]\arrow[rd,"\id"]\\
		P'\nxcell[f]P\nxcell0
	\end{tikzcd}\]
	が可換になる$g:P\to P'$が存在し$f$は分裂全射である.
\end{proof}

この命題より射影被覆は\textbf{存在すれば}同型を除いて一意である.また,同様の議論で$M\cong M'$かつ$P,P'$がそれぞれの射影被覆なら$P\cong P'$である.

\begin{defi}[極小射影分解]\index{きょくしょうしゃえいぶんかい@極小射影分解}
	$A$加群$M$の射影分解;
	\[\begin{tikzcd}
		\dots\nxcell P_2\nxcell[d_2]P_1\nxcell[d_1]P_0\nxcell[d_0]0
	\end{tikzcd}\]
	について,各$d_i:P_i\to\ker d_{i-1}$が射影被覆であるとき\textbf{極小射影分解}(minimal projective resolution)であるという.
\end{defi}

極小射影分解は存在すれば同型を除いて一意である.よって冒頭の2つめの例に戻ると,この分解が極小射影分解を与えていることを見ればよい.それには$Ax$が$A$の余剰部分加群であることを示せば十分である.$A$のイデアル$I$について$Ax+I=A$であるとすると,$x$は冪零なので$Ax\subset\nil (A)\subset\rad(A)$であるので,中山の補題から$I=A$である.よって$\prjdim M=\infty$が示された.

しかし一般には極小射影分解(射影被覆)が存在するとは限らないことに注意しなければならない.例えば$\Z/m\Z$について;
\[\ses[m]{\Z}{\Z}{\Z/m\Z}\]
は$\Z/m\Z$の射影分解を与えるが,これは$m\Z$が$\Z$の余剰部分加群でないので極小射影分解ではない.$\Z/m\Z$は射影$\Z$加群でないので,$\prjdim \Z/m\Z=1$である.するともし極小射影分解が存在すれば;
\[\ses[d_1][\varepsilon]{P_1}{P_0}{\Z/m\Z}\]
という形をしているが,このとき$\varepsilon$は分裂全射なので$P_0$は$P_1\oplus\Z/m\Z$と同型である.すると$P_1$は$\ker\varepsilon$と同型で,これは余剰加群であるから$\Z/m\Z=P_0$となり$\Z/m\Z$が射影的となって矛盾する.

双対的に入射加群について考えたものが\textbf{入射包絡}であり,こちらは必ず存在する.
\begin{defi}[入射次元]\index{にゅうしゃじげん@入射次元}
	$M$を$A$加群とする. $M$の入射分解の長さの最小値を$M$の\textbf{入射次元}(injective dimension)といい,$\injdim_A M$とかく.$M=0$のときは$\injdim M=-1$とする.
\end{defi}

\begin{defi}[入射包絡]\index{ほんしつてきかぐん@本質的加群}\index{にゅうしゃほうらく@入射包絡}
	$A$加群$M$の部分加群$N$が;
	\[\text{任意の$M$の部分加群$L$について$N\cap L=0$なら$L=0$.}\]
	を満たすとき,$N$を$M$の\textbf{本質部分加群}(essential submodule)という.$A$加群$M$について入射加群$I$と単射$\varepsilon:M\to I$が存在して$\im\varepsilon$が$I$の本質加群のとき,$I,\varepsilon$は$M$の\textbf{入射包絡}(injective hull)であるという.
\end{defi}

本質部分加群については次の判定条件が強力である.
\begin{prop}\label{prop:本質的加群の判定条件}
	$A$加群$M$の部分加群$N$が本質的であることと,任意の$x\neq0\in M$について$Ax\cap N\neq0$であることは同値.
\end{prop}

\begin{proof}
	$(\Longleftarrow)$のみ示す.$N\cap L=0$かつ$L\neq0$とすると,$x\neq0\in L$がとれ,このとき$Ax\subset L$より$Ax\cap N\subset L\cap N=0$であるので$Ax\cap N=0$だがこれは矛盾.よって$L=0$である.
\end{proof}

\ref{lem:余剰加群の補題}の双対版を示しておこう.

\begin{lem}
	加群の準同型の列$\begin{tikzcd}
	M_1\nxcell[\varphi]M_2\nxcell[\psi]M_3
	\end{tikzcd}$について,$\psi\circ\varphi$が単射であるとする.$\im\varphi$が$M_2$の本質加群なら$\psi$も単射である.
\end{lem}

\begin{proof}
	任意の$x\neq0\in M_2$について$\psi(x)\neq0$を言えばよい.このとき$Ax\cap\im\varphi\neq0$であるので,ある$a\in A$が存在して$0\neq ax\in\im\varphi$である.すると,ある$x_1\in M_1$が存在して$ax=\varphi(x_1)$とできる.$\psi(x)=0$と仮定すると,$0=a\psi(x)=\psi(ax)=\psi(\varphi(x_1))$であり$\psi\circ\varphi$が単射なので$x_1=0$となる.よって$ax=\varphi(0)=0$となりこれは矛盾.よって$\psi(x)\neq0$である.
\end{proof}

これをつかって,射影被覆と双対的に次が示される.

\begin{prop}\label{prop:入射包絡の一意性}
	$M$を$A$加群とし,入射包絡$\varepsilon:M\to I$が存在したとする.入射加群$I'$と単射$\varepsilon':M\to I'$について分裂単射$f:I\to I'$が存在して,次の図式;
	\[\begin{tikzcd}
		&&I\arrow[dd,"f"]\\[-1.5em]
		0\nxcell M\arrow[ur,"\varepsilon"]\arrow[dr,"\varepsilon'"]\\[-1.5em]
		&&I'
	\end{tikzcd}\]
	が可換($I'$は$I$と同系な直和因子を持つ).
\end{prop}

\begin{defi}[極小入射分解]\index{きょくしょうにゅうしゃぶんかい@極小入射分解}
	$A$加群$M$の入射分解;
	\[\begin{tikzcd}
		0\nxcell I^0\nxcell[d^0]I^1\nxcell[d^1] I^2\nxcell\cdots 
	\end{tikzcd}\]
	について,各移入$\im d^i\hookrightarrow I^{i+1}$が入射包絡のとき\textbf{極小入射分解}(minimal inductive resolution)であるという.
\end{defi}

極小射影分解と同様に同型を除いて一意に定まる.射影被覆と異なるのは入射包絡が必ず存在することである.

\begin{thm}[入射包絡の存在]
	$A$加群$M$について入射包絡が必ず存在する.
\end{thm}

\begin{proof}
	$A$加群の圏は入射的対象を十分に持つ(\ref{thm:加群の圏はhas enough injectives})ので,入射加群$I$と単射$\varepsilon:M\to I$が存在する.次の集合;
	\[\mathscr{E}=\mkset{E:I\text{の部分加群}}{M\subset E, M\text{は}E\text{の本質部分加群}}\]
	は$M\in\mathscr{E}$なので空ではなく,帰納的順序集合をなす.よってZornの補題から極大元がとれ,それを$E$としよう.次に;
	\[\mathscr{L}=\mkset{L:I\text{の部分加群}}{L\cap E=0}\]
	は$0\in\mathscr{L}$より空でなく,帰納的順序集合をなすなので極大元を$L$とおく.埋め込み$\iota:E\to I$と自然な全射$\pi:I\to I/L$を考える.合成$\pi\circ\iota$は単射であり,$\pi(E)$は$I/L$で本質的.実際$L\subset N\subset I$を$I$の部分加群とすると,$\pi(E)\cap N/L=0$なら$E\cap N\subset L$だが$E\cap L=0$より$E\cap N=0$となり,$L$の極大性より$N=L$である.
	
	次に$E$は$\varphi(I/L)$の本質部分加群であることを示そう.$I$が入射的なので,次の図式;
	\[\begin{tikzcd}
	&&I\\
	0\nxcell E\arrow[ur,"\iota",hookrightarrow]\nxcell[\pi\circ\iota]I/L\arrow[u,dashed,"\varphi"]
	\end{tikzcd}\]
	が可換になる$\varphi:I/L\to I$が存在する.$\varphi(I/L)$の部分加群$N$について$E\cap N=0$であるとする.このとき$\pi(E)\cap\varphi^{-1}(N)=0$である.実際$x\in\pi(E)\cap\varphi^{-1}(N)$とすると,ある$y\in E$が存在して$\pi(y)=x$である.このとき$\varphi(\pi(y))=y\in N\cap E$より$y=0$であり,ゆえに$x=0$となる.すると$\pi(E)$は$I/L$の本質部分加群なので$\varphi^{-1}(N)=0$となり$N\subset\im\varphi$だから$N=0$となる.すると$E$は$M$の本質拡大で,$\varphi(I/L)$は$E$の本質拡大なので$\varphi(I/L)$は$M$の本質拡大だから(\ref{prop:本質的加群の判定条件}を用いて確かめよ)$E$の極大性より$E=\varphi(I/L)$である.よって$\iota:E\to I$は分裂単射となる.ゆえに$E$は入射加群であり,これが$M$の入射包絡にほかならない.
\end{proof}

この定理と\ref{prop:入射包絡の一意性}より$A$加群$M$の入射包絡は同型を除いて必ず一意に存在するので$\E(M)$とかくことにしよう.

\section{ホモロジー次元}
定義からホモロジー次元について次が従う.

\begin{lem}
	$A$を環,$M$を$A$加群とする.このとき;
	\begin{sakura}
		\item $\prjdim M\leq n$であるとき,任意の$i>n$と$A$加群$N$について$\Ext^{i}(M,N)=0$である.
		\item $\injdim M\leq n$であるとき,任意の$i>n$と$A$加群$N$について$\Ext^{i}(N,M)=0$である.
	\end{sakura}
	が成り立つ.
\end{lem}

この逆が成り立つだけでなく,\ref{prop:ホモロジー次元を抑える}により$n+1$についてのみ確かめればよいことがわかる.まず$\Ext$の長完全列を考えることにより次の補題が従うことを注意しておこう.

\begin{lem}
	$A$加群$P$が射影的であることと,任意の$A$加群$N$について$\Ext^1(P,N)=0$であることは同値.また$I$が入射的であることは任意の$N$について$\Ext^1(N,I)=0$であることと同値.
\end{lem}

\begin{proof}
	$P$についてのみ示す.$A$加群の完全列;
	\[\ses{M_1}{M_2}{M_3}\]
	から得られる$\Ext$の長完全列;
	\[\begin{tikzcd}
	0\nxcell\hom(P,M_1)\nxcell\hom(P,M_2)\nxcell\hom(P,M_3)\nxcell\Ext^1(P,M_1)\nxcell\cdots
	\end{tikzcd}\]
	を考えれば$\hom(P,-)$が完全関手であることと$\Ext^1(P,M_1)=0$が同値であることがわかる.
\end{proof}

これと全く同様にして,$A$加群$M$が平坦であることと,任意の$A$加群$N$について$\Tor_1(M,N)=0$であることが同値だとわかることを注意しておく.

\begin{prop}\label{prop:ホモロジー次元を抑える}
	$A$を環,$M$を$A$加群とする.このとき;
	\begin{sakura}
		\item $\prjdim M\leq n$であることと,任意の$A$加群$N$について$\Ext^{n+1}(M,N)=0$であることは同値.
		\item $\injdim M\leq n$であることと,任意の$A$加群$N$について$\Ext^{n+1}(N,M)=0$であることは同値.
	\end{sakura}
	が成り立つ.
\end{prop}

\begin{proof}
	(i)のみ示す.$(\Longrightarrow)$は明らかなので,逆を見ればよい.$M$の射影分解$P_\bullet$を考える.$P=\im d_n~(d_n:P_n\to P_{n-1})$とおくと,次の2つの完全列;
	\[\begin{tikzcd}
	0\nxcell P\nxcell P_{n-1}\nxcell[d_{n-1}]P_{n-1}\nxcell\cdots\nxcell P_0\nxcell M\nxcell0\end{tikzcd};\]
	\[\begin{tikzcd}
	\cdots\nxcell P_{n+1}\nxcell[d_{n+1}]P_n\nxcell[d_n]P\nxcell0
	\end{tikzcd}\]
	がある.1つめの完全列より,$P$が射影的ならば$\prjdim M\leq n$が従うので,これを示そう.2つめの完全列を$P$の射影分解とみなすと,$\Ext^1(P,N)=\Ext^{n+1}(M,N)=0$となっており,$P$は射影的である.
\end{proof}

同様にExtの長完全列を考えることで次の2つの命題がわかる.
\begin{cor}\label{cor:prjdimの有限性}
	$A$加群の完全列;
	\[\ses{M_1}{M_2}{M_3}\]
	について,2つの加群の射影元が有限なら残りの1つの射影次元も有限.
\end{cor}

\begin{cor}\label{cor:prjdim lemma}
	$A$加群の完全列;
	\[\ses{M_1}{M_2}{M_3}\]
	について,$\prjdim M_2<\infty$であるとき,$\prjdim M_1=\prjdim M_2$ならば$\prjdim M_3\leq\prjdim M_1+1$であり,$\prjdim M_1>\prjdim M_2$ならば$\prjdim M_3=\prjdim M_1+1$である.
\end{cor}

以後簡単のために局所環という条件を課すことにしよう.これにより様々な恩恵が得られることは局所環の章でみてきた通りである.ホモロジカルなご利益としては,例えばNoether局所環$(A,\ideal{m})$のもとでは射影被覆が存在し,かつ扱いやすいものとなる.

\begin{defi}[極小自由分解]\index{きょくしょうじゆうぶんかい@極小自由分解}
	Noether局所環$(A,\ideal{m})$上の有限生成加群$M\neq0$の射影分解$P_\bullet$であって,次の条件;
	\begin{sakura}
		\item $P_\bullet$は極小射影分解である.
		\item 各$P_i$は有限自由$A$加群である.
		\item $P_0\otimes A/\ideal{m}=M\otimes A/\ideal m=M/\ideal{m}M$である.
	\end{sakura}
	をみたすものが存在する.これを$M$の\textbf{極小自由分解}(minimal free resolution)という.
\end{defi}

\begin{proof}[\textbf{存在証明}]
	$M/\ideal{m}M$の(有限次元)$A/\ideal{m}$ベクトル空間としての生成系$\{e_i\}$をとり,各$e_i$の代表元$f_i\in M$を固定する.$\{e_i\}$が生成する自由$A$加群$\oplus Ae_i$を$P_0$とおく.このとき,次の$A$準同型;
	\[\varepsilon:P_0\to M;e_i\mapsto f_i\]
	を考えると,これは全射である.実際$M=\im\varepsilon+\ideal{m}M$であることが容易に確かめられ,中山の補題より$M=\im\varepsilon$であることがわかる.またこれは射影被覆になる.これも$\ker\varepsilon\subset\ideal{m}P_0$であることに注意して中山の補題から従う.同様に$\ker\varepsilon$について射影被覆$d_1:P_1\to\ker\varepsilon$がとれる.これを繰り返して完全列;
	\[\begin{tikzcd}
	\cdots\nxcell P_2\nxcell[d_2]P_1\nxcell[d_1]P_0\nxcell[\varepsilon]M\nxcell0
	\end{tikzcd}\]
	を得,各$\ker d_i$は$P_i$の余剰部分加群である.よって$P_\bullet$を各$P_i$が自由でありかつ極小射影分解となっているようにとることができた.
\end{proof}

ここから射影次元を次のように言い換えることができる.

\begin{thm}\label{thm:射影次元の言い換え}
	$(A,\ideal{m})$をNoether局所環とし,$M\neq0$をその上の有限生成加群とする.このとき;
	\[\prjdim M=\sup\mkset{i\in N}{\Tor_i(M,A/\ideal{m})\neq0}\]
	である.
\end{thm}

ここから,ある$n\geq0$を固定したとき,ある$i\geq n$について$\Tor_i(M,A/\ideal m)\neq0$ならば$n\leq i\leq\prjdim M$であることがわかり,対偶をとって$\prjdim M<n$ならばすべての$i\geq n$について$\Tor_i(M,A/\ideal m)=0$である.これを一般化したものがホモロジカル予想の1つ,\textbf{剛性}(rigidty)\textbf{予想}である.

\begin{conj}[剛性予想]
	$(A,\ideal m)$を局所環とし,$M,N$を有限生成$A$加群とする.$\prjdim M<\infty$かつ$\Tor_n(M,N)=0$となる$n$が存在するならば,任意の$i\geq n$について$\Tor_i(M,N)=0$であろう.
\end{conj}

$A$が正則局所環のときには\cite{lichtenbaum1966}によって証明されている.その他いくつかの条件のもとでは成り立つことがわかっているが,一般には反例が存在する(\cite{heitmann1993counterexample}).一方$\prjdim N<\infty$も仮定すると未解決である.

また,次のように驚くべき性質を示すことができる.


\begin{prop}
	$(A,\ideal{m})$をNoether局所環とし,$M$をその上の有限生成加群とする.このとき,次の条件;
	\begin{sakura}
		\item $M$は自由である.
		\item $M$は射影的である.
		\item $M$は平坦である.
		\item $\Tor_1(A/\ideal m,M)=0$である.
	\end{sakura}
	は同値.
\end{prop}

\begin{proof}
	(iv)が(i)を導くことのみ見れば良い.極小自由分解の構成と同じように$\varepsilon:P_0\to M$を構成する.$\varepsilon$は全射であったので,$\Tor_1(A/\ideal m,M)=0$ならば$\ker\varepsilon=0$であることを示す.完全列;
	\[\ses{\ker\varepsilon}{P_0}{M}\]
	に$-\otimes A/\ideal m$を作用させて,Torの長完全列;
	\[\begin{tikzcd}
	\cdots\nxcell\Tor_1(A/\ideal m,M)\nxcell\ker\varepsilon\otimes A/\ideal{m}\nxcell P_0\otimes A/\ideal{m}\nxcell M\otimes A/\ideal{m}\nxcell0
	\end{tikzcd}\]
	を得る.$P_0$の構成から$P_0\otimes A/\ideal{m}=M\otimes A/\ideal{m}$であるので,$\Tor_1(A/\ideal m,M)=0$だから$\ker\varepsilon\otimes A/\ideal{m}=\ker\varepsilon/\ideal{m}\ker\varepsilon=0$である.よって中山の補題から$\ker\varepsilon=0$である.
\end{proof}

このようにNoether局所環上の有限生成加群では射影的であることと自由であることが同値になるが,Noether性という条件は外すことができる(それだけでなく有限生成である必要すらないが,長くなるのでここでは省く.\cite{matsu}をみよ).

\begin{thm}
	$(A,\ideal{m})$を局所環,$M$を有限生成$A$加群とする.このとき,$M$が射影加群であることと自由加群であることは同値である.
\end{thm}

\begin{proof}
	$M$を射影加群とする.極小自由分解の構成から,有限生成自由加群$P_0$からの全射$\varepsilon:P_0\to M$が存在する.前命題と同様の議論から$\ker\varepsilon\otimes A/\ideal m=0$である(ここまでNoether性は要らず,$\ker\varepsilon$が有限生成であることにNoether性を使った).ここで射影的であることと自由加群の直和因子であることは同値であったことと,その証明(\ref{thm:射影的と自由の直和因子は同値})を思い出すと,$P_0\cong M\oplus\ker\varepsilon$であるので,とくに$\ker\varepsilon\cong P_0/M$であり,$P_0$は有限生成自由加群なので$\ker\varepsilon$も有限生成である.よって中山の補題が使えて$\ker\varepsilon=0$である.
\end{proof}

\begin{thm}[Auslander--Buchsbaumの公式]\index{#Auslander--Buchsbaumのこうしき@Auslander--Buchsbaumの公式}\label{thm:Auslander--Buchsbaumの公式}
	$(A,\ideal m)$をNoether局所環とし,$M\neq 0$をその上の射影次元有限な有限生成加群とする.このとき;
	\[\prjdim M+\mdepth M=\mdepth A\]
	が成り立つ.
\end{thm}

\begin{proof}
	$d=\mdepth A$とおく.$r=\prjdim M$についての帰納法で示す.$r=0$のとき,$M$は前定理より自由加群だから,$\mdepth M=\mdepth A$である.次に$1\leq r$とし,$r-1$まで成立しているとする.適当な$n$について全射$A^n\to M$が存在し,その核を$K$とおけば;
	\[\ses{K}{A^n}{M}\]
	が完全である.\ref{cor:prjdim lemma}より$\prjdim K=r-1$である.帰納法の仮定から$\mdepth K=d-r+1$である.よって$\mdepth M=d-r$を示せばよい.\ref{thm:Extとdepth}より$i<d-r+1$なら$\Ext^i(A/\ideal m,K)=0,\Ext^{d-r+1}(A/\ideal m,K)\neq0$である.上の完全列が導く$\Ext$の長完全列;
	\[\begin{tikzcd}
	\cdots\nxcell\Ext^i(A/\ideal m,A^n)\nxcell\Ext^i(A/\ideal m,M)\nxcell\Ext^{i+1}(A/\ideal m,K)\nxcell\Ext^{i+1}(A/\ideal m,A^n)\nxcell\cdots
	\end{tikzcd}\]
	を考えると,$i<d-r$ならば$i+1<d-r+1\leq d$なので;
	\[\begin{tikzcd}
	0\nxcell\Ext^i(A/\ideal m,M)\nxcell\Ext^{i+1}(A/\ideal m,K)\nxcell0
	\end{tikzcd}\]
	となり$\Ext^i(A/\ideal m,M)=\Ext^{i+1}(A/\ideal m,K)=0$である.また$i=d-r$のとき;
	\[\begin{tikzcd}
	0\nxcell\Ext^{d-r}(A/\ideal m,M)\nxcell\Ext^{d-r+1}(A/\ideal m,K)\nxcell\Ext^{d-r+1}(A/\ideal m,A^n)\nxcell\cdots
	\end{tikzcd}\]
	であり,$r>1$ならば$d-r+1<d$となり$\Ext^{d-r}(A/\ideal m,M)=\Ext^{d-r+1}(A/\ideal m,K)\neq0$である.
	
	$r=1$のときは別に示す必要がある.$M$の極小自由分解を考えて;
	\[\ses{P_1}{P_0}{M}\]
	とする.$\Ext^{d-1}(A/\ideal m,M)\neq0$を示す.
	\[\begin{tikzcd}
	0\nxcell\Ext^{d-1}(A/\ideal m,M)\nxcell\Ext^d(A/\ideal m,P_1)\nxcell\Ext^d(A/\ideal m,P_0)\nxcell\cdots
	\end{tikzcd}\]
	において$\Ext^d(A/\ideal m,P_1)\to\Ext^d(A/\ideal m,P_0)$を考える.$d_1:P_1\to P_0$において極小自由分解の構成から$\im d_1=\ker \varepsilon\subset\ideal{m}P_0$であったので,$P_1\to P_0$の誘導する$\hom(A/\ideal m,P_1)\to\hom(A/\ideal m,P_0)$は$0$である.ゆえに$\Ext^d(A/\ideal m,P_1)\to\Ext^d(A/\ideal m,P_0)$は$0$であり,上の完全列から$\Ext^{d-1}(A/\ideal m,M)=\Ext^d(A/\ideal m,P_1)\neq0$である.
	
	以上を組み合わせて証明が完了した.
\end{proof}

\section{大域次元とSerreの定理}

この節では,環の\textbf{大域次元}(global dimension)を導入して,Serreによる正則局所環の特徴づけとSerreの定理(\ref{prethm:Serre})の証明を行おう.まず,\ref{prop:ホモロジー次元を抑える}を入射次元についてより簡単に書き換えるところから始める.

\begin{prop}
	$A$を環,$M$を$A$加群とする.このとき,与えられた$n$について$\injdim M\leq n$であることと,任意の$A$のイデアル$I$について$\Ext^{n+1}_A(A/I,M)=0$であることは同値.
\end{prop}

\begin{proof}
	$A$加群$M$が入射的であることと,任意の$A$のイデアル$I$について$\Ext^1(A/I,M)=0$であることが同値であることを示せば十分である.$M$が入射的なら$\Ext^1$が消えることは明らかなので,逆を見ればよい.Bearの基準(\ref{thm:Bear's Criterion})より,$\hom(-,M)$が自然な$0\to I\to A$の完全性を保てばよい.それは完全列;
	\[\ses{I}{A}{A/I}\]
	に対して$\hom(-,M)$が導く長完全列;
	\[\begin{tikzcd}
	0\nxcell \hom(A/I,M)\nxcell\hom(A,M)\nxcell\hom(I,M)\nxcell\Ext^1(A/I,M)=0\nxcell\cdots
	\end{tikzcd}\]
	より成り立つことがわかり,$M$は入射的である.
\end{proof}

\begin{defi}[大域次元]\index{たいいきじげん@大域次元}
	環$A$について;
	\[\gldim A=\sup\mkset{\prjdim M}{M:A\text{加群}}\]
	と定義する.これを$A$の\textbf{大域次元}(global dimension)という.
\end{defi}

大抵の環は$\gldim A=\infty$なので,大域次元が有限となる環に興味がある.先の命題を用いて大域次元は$A$のイデアルによる商環のみを確かめればよいことを示そう.

\begin{prop}[Auslander]
	$\gldim A=\sup\mkset{\prjdim A/I}{I:A\text{のイデアル}}$が成り立つ.
\end{prop}

\begin{proof}
	右辺を$n$とおく.明らかに$n\leq\gldim A$だから逆をみればよい.$n<\infty$としてよい.任意の$A$加群$N$について,$\Ext^{n+1}(A/I,N)=0$であるので,先の命題から$\injdim N\leq n$である.よって任意の$A$加群$M$について$\Ext^{n+1}(M,N)=0$であり,$\prjdim M\leq n$である.よって示された.
\end{proof}

Noether性を課すことでさらに話は簡単になる.

\begin{thm}\label{thm:Noether局所環の大域次元}
	$(A,\ideal{m})$をNoether局所環とする.このとき$\gldim A=\prjdim A/\ideal{m}$である.
\end{thm}

\begin{proof}
$\prjdim A/\ideal{m}=\infty$なら明らかであるので,有限であると仮定する.$n=\prjdim A/\ideal m$とおき,任意の$A$加群$M$について$\prjdim M\leq n$であることを,$\dim M$についての帰納法で示す.

$\dim M=0$とする.このとき$M$を$A/\ann M$加群とみると,$M$はNoetherかつArtin加群なので組成列;
\[0=M_0\subsetneq M_1\subsetneq\dots\subsetneq M_r=M\]
を持つ(\ref{prop:有限な組成列の同値条件}).$A/\ann M$は唯一の極大イデアルを$\ideal{m}$とする局所環であるから,\ref{prop:単純加群の構造}により各$1\leq i\leq r$について$M_i/M_{i-1}\cong A/\ideal m$である.よって;
\[\ses{M_{i-1}}{M_i}{A/\ideal m}\]
が完全である.任意の$A$加群$N$について$\Ext^{n+1}(A/\ideal m,N)=0$であるので,$i=1$から順に考えて$\Ext^{n+1}(M,N)=0$である.よって$\prjdim M\leq n$である.

次に$d=\dim M>0$とし,$d-1$まで正しいと仮定する.まずは$\mdepth M>0$のときを考えると,$M$正則元$a\in\ideal m$が存在する.このとき$\dim M/aM=\dim M-1$である.よって帰納法の仮定から$\prjdim M/aM\leq n$である.次の完全列;
\[\ses[a\cdot]{M}{M}{M/aM}\]
が誘導するExtの長完全列を考えると,任意の$A$加群$N$について$\Ext^{n+1}(M/aM,N)=0$であり;
\[\begin{tikzcd}
	\Ext^n(M/aM,N)\nxcell\Ext^n(M,N)\nxcell[a\cdot]\Ext^n(M,N)\nxcell\Ext^{n+1}(M/aM,N)=0
\end{tikzcd}\]
から$a\Ext^n(M,N)=\Ext^n(M,N)$であるので,中山の補題から$\Ext^n(M,N)=0$である.よって$\prjdim M\leq n-1$である.

次に$\mdepth M=0$と仮定する.$L=\mkset{x\in M}{\text{ある}i\in\N\text{について}\ideal{m}^ix=0}$とおく.短完全列;
\[\ses{L}{M}{M/L}\]
を考えよう.まず$\dim L=0$であることを示す.$L$の生成系$\{u_1,\dots,u_r\}$をとる.各$i$について$\ideal{m}^{n_i}u_i=0$となる$n_i$がある.その最大値を$n$とすれば$\ideal{m}^n\subset\ann L$である.よって\ref{prop:局所環がm^n=0ならArtin}より$A/\ann L$はArtin,すなわち$\dim L=0$である.よって$\prjdim L\leq n$となる.また$M/L$について$\mdepth M/L>0$であることを示す.\ref{lem:depth 0とass}を局所環の場合に考えると$\mdepth M=0$であることと$\ideal m\in\ass M$であることは同値である.よって$\ideal m\not\in\ass M/L$であることを示そう.ある$x\in M$について$\ideal m=\ann (x+L)$であるとする.$\ideal m$の生成系$\{a_i\}$をとれば,任意の$i$について$a_i\ideal{m}^{n_i}x=0$となる$x_i$がとれるので,最大値を$n$として$\ideal{m}\cdot \ideal{m}^nx=\ideal{m}^{n+1}x=0$である.よって$x\in L$となる.ゆえに$\ideal m\not\in\ass M/L$となり,$\mdepth M/L>0$である.よって$\prjdim M/L\leq n-1$である.よって,$\Ext^{n+1}(M/L,N)=\Ext^{n+1}(L,N)=0$であるから$\Ext^{n+1}(M,N)=0$となり$\prjdim M\leq n$である.
\end{proof}

また,Auslander--Buchsbaumの公式(\ref{thm:Auslander--Buchsbaumの公式})より容易に次の補題を得る.

\begin{lem}\label{lem:prjdim M/aM}
	$(A,\ideal{m})$をNoether局所環とし,$M$をそのうえの有限生成$A$加群とする.$M$正則元$a\in\ideal{m}$について;
	\[\prjdim M/aM=\prjdim M+1\]
	が成り立つ.
\end{lem}

\begin{proof}
	まず$\prjdim M=\infty$のときを考える.もし$\prjdim M/aM=n<\infty$ならば,短完全列;
	\[\ses[a\cdot]{M}{M}{M/aM}\]
	を考えて$\Ext^{n+1}(M/aM,N)=0$であるから,先の定理と同様に$\Ext^n(M,N)=0$となり仮定に反する.よって$\prjdim M/aM=\infty$となり仮定は成り立つ.
	
	次に有限個の場合を考えよう.$\prjdim M=n<\infty$とおく.\ref{cor:prjdimの有限性}より$\prjdim M/aM$も有限である.よってAuslander--Buchsbaumの公式から;
	\[\prjdim M/aM=\mdepth A-\mdepth M/aM=\mdepth A-(\mdepth A-1)=\prjdim M+1\]
	であることがわかる.
\end{proof}

また,$a\in\ideal m$が$A$正則という条件を足すことで$A/aA$加群としての射影次元を上から抑えることができる.

\begin{lem}\label{lem:prjdim_A/aA M/aM}
	$(A,\ideal{m})$をNoether局所環とし,$M$をそのうえの有限生成$A$加群とする.$M$正則元$a\in\ideal{m}$について$A$正則でもあるならば;
	\[\prjdim_{A/aA} M/aM\leq \prjdim_A M\]
	が成り立つ.
\end{lem}

\begin{proof}
	$d=\prjdim_A M$とおく.$d<\infty$と仮定してよい.$d$についての帰納法で示す.$d=0$のとき,$M$は自由なので$M\cong A^n$とすると$M/aM\cong(A/aA)^n$より$A/aA$加群として自由だから$\prjdim_{A/aA}M/aM=0$となり正しい.$d>1$とし,$d-1$まで正しいと仮定する.自由加群からの全射$\varphi:F\to M$を考えよう.次の図式;
	\[\begin{tikzcd}
		0\nxcell\ker\varphi\darrow[a\cdot]\nxcell F\darrow[a\cdot]\nxcell[\varphi]M\darrow[a\cdot]\nxcell0\\
		0\nxcell \ker\varphi\nxcell F\nxcell[\varphi]M\nxcell0
	\end{tikzcd}\]
	を考える.$a$は$M$正則だから$a\cdot:M\to M$は単射で,蛇の補題から;
	\[\ses{\ker\varphi/a\ker\varphi}{F/aF}{M/aM}\]
	が完全.\ref{cor:prjdim lemma}より$\prjdim\ker\varphi=d-1$だから,帰納法の仮定から$\prjdim_{A/aA}\ker\varphi/a\ker\varphi\leq d-1$である.再び\ref{cor:prjdim lemma}より$\prjdim_{A/aA}\ker\varphi/a\ker\varphi=\prjdim_{A/aA}M/aM-1$なので$\prjdim_{A/aA}M/aM\leq d-1+1=d$である.
\end{proof}
\ref{lem:prjdim M/aM}により次の定理が従う.

\begin{thm}[Auslander--Buchsbaum]
	$(A,\ideal m)$を$d$次元の正則局所環とすると,$\gldim A=d$である.
\end{thm}

\begin{proof}
	$x_1,\dots,x_d\in\ideal{m}$を正則巴系とすると,これは$A$正則列をなし$\prjdim A/\ideal m=\prjdim A+d=d$である.よって\ref{thm:Noether局所環の大域次元}より$\gldim A=\prjdim A/\ideal m=d$である.
\end{proof}

驚くべきはこれの逆が成り立つ.すなわち,局所環の大域次元が有限であることと正則であることは同値である.

\begin{thm}[Serre]
	$(A,\ideal m,k)$を$\gldim A=d<\infty$となるNoether局所環とする.このとき$A$は$d$次元の正則局所環である.
\end{thm}

\begin{proof}
	$d=\gldim A$とおく.また$r=\emdim A=\dim_k\ideal m/\ideal{m}^2$とおこう.まず$r=0$とする.$\ideal m=\ideal{m}^2$であるので,中山の補題から$\ideal m=0$である.よって$A$は体となり,$\gldim A=\prjdim A=0$なので題意が成り立つ.逆に$\gldim A=0$とすると,$\prjdim k=0$となり$k$は自由だから$k\cong A$となり,$A$は体となる.
	
	よって$r,d>0$と仮定してよい.$r$についての帰納法で示す.$r-1$まで正しいとしよう.$\ideal m\neq0$であり,中山の補題から$\ideal m\neq\ideal{m}^2$である.まず$a\in\ideal m\setminus\ideal{m}^2$を$A$正則にとれることを示す.任意の$a$が$A$正則でないと仮定する.すると$\ideal{m}\setminus\ideal{m}^2\subset\bigcup_{P\in\ass A}P$である.よって$\ideal m\subset\bigcup_{P\in\ass A}P\subset\ideal{m}^2$であるので,Prime avoidanceからある$P\in\ass A$について$\ideal{m}\subset P$である($\ideal m\not\subset\ideal{m}^2$であるから).$\ideal{m}$は極大なので$\ideal m\in\ass A$となり,単射$k\hookrightarrow A$が存在する($\ideal m=\ann x$としたとき$a+\ideal m\mapsto ax$とすればよい).よって短完全列;
	\[\ses{k}{A}{A/k}\]
	について,$\Tor_d(A,k)=\Tor_{d+1}(A,k)=0,\Tor_d(k,k)\neq0$だからTorの長完全列を考えて;
	\[\begin{tikzcd}
		0\nxcell\Tor_{d+1}(A/k,k)\nxcell\Tor_d(k,k)\nxcell0
	\end{tikzcd}\]
	が完全であり$\Tor_{d+1}(A/k,k)\neq0$となるので\ref{thm:射影次元の言い換え}より$d+1\leq\prjdim A/k$であるが,これは$d=\gldim A$であるので矛盾.
	
	このとき,$A/aA$が$\emdim A/aA=r-1$となる大域次元有限な環であることを示そう.$r=\dim_k \ideal m/\ideal{m}^2$であるので,$a_1,\dots,a_r\in\ideal m$を$a_i+\ideal{m}^2$が$\ideal m/\ideal{m}^2$の基底になるようにとれる.ここで$a_1=a$としてよい.$I=(a_2,\dots,a_r)$とおくと,中山の補題より$aA+I=\ideal m$であり,$\emdim A/aA=\dim_k (\ideal{m}/aA)/(\ideal{m}/aA)^2=r-1$である.また$\ideal m/aA=(aA+I)/aA\cong I/(I\cap aA)$である.ここで任意の$ab\in aA\cap I$をとると,ある$c_i\in A$により$ab=c_2a_2+\dots+c_ra_r$とかけている.よって$ba_1-c_2a_2-\dots-c_ra_r=0$である.これを$k$加群$\ideal m/\ideal{m}^2$におとすと,$a_i+\ideal{m}^2$は一次独立であるからすべての係数は$k$でゼロ,つまり$b,c_i\in\ideal{m}$でなければならない.よって$I\cap aA\subset a\ideal{m}$である.よって;
	\[\ideal m/a\ideal m\to\ideal m/aA=I/(I\cap aA)\to\ideal m/a\ideal m\]
	が恒等だから$\ideal m/a\ideal m\to \ideal m/aA$は分裂する.よって$\ideal m/aA$は$\ideal m/\ideal{m}^2$の直和因子である.ゆえに$\prjdim_{A/aA}\ideal m/aA\leq\prjdim_{A/aA}\ideal m/\ideal{m}^2$であり,\ref{lem:prjdim_A/aA M/aM}より;
	\[\prjdim_{A/aA} k-1=\prjdim_{A/aA}\ideal m/aA\leq\prjdim_A \ideal m=\prjdim_A k-1\]
	よって$\gldim A/aA<\infty$であり,帰納法の仮定から$A/aA$は正則である.ゆえに$\dim A/aA=r-1$となり,$\dim A=\dim A/aA+1$だから$A$も$r$次元の正則局所環である.
\end{proof}

これが可換環論にホモロジー代数を導入したことによる大結果の最初の1つである.この定理により次の定理がほぼ自明と化した.
\begin{thm}[Serreの定理]\index{#Serreのていり@Serreの定理}\label{thm:Serreの定理}
	$A$を正則局所環とすると,任意の$P\in\spec A$について$A_P$も正則局所環である.
\end{thm}

\begin{proof}
	前定理から$\prjdim_A A/P<\infty$であり,有限な極小自由分解を得る.これを$P$で局所化して$A_P/PA_P$の極小自由分解を得,$\prjdim_{A_P} A_P/PA_P<\infty$である.
\end{proof}

\section{UFDと有限自由分解}

この節では有限自由分解と呼ばれる射影分解の一種をもつ加群について触れるとともに,Auslander--Buchsbaumの定理(\ref{prethm:Auslander-Buchsbaum})の証明を与える.まずUFDについていくつかの性質を見ていこう.まず整域$A$がUFDであることと,単項イデアルについての昇鎖律を満たし,既約元が素元であることは同値である.前半の部分は元を無限に割り続けることはできないということ,後半の部分は分解の一意性を保証する.この節では単項イデアルについての昇鎖律を(ACCP)と略記する.まずよく知られている次の判定法を紹介しよう.

\begin{prop}\label{prop:UFD判定}
	Noether整域$A$がUFDであることはすべての高さ$1$の素イデアルが単項であることと同値.
\end{prop}

\begin{proof}
	\begin{eqv}
		\item $P\in\spec A$が高さ$1$であるとする.任意の$0$でない$a\in P$について既約分解$a=f_1\cdots f_r$する.このときある$f_i$について$f_i\in (P)$であるから$(f_i)\subset P$であり,UFDにおいて既約元は素元だから$(f_i)\in\spec A$である.すると$\idht P=1$だから$(f_i)=P$である.
		\item 既約元は素元であることを示せばよい.$f$を既約元とする.$(f)$の極小素イデアル$P$について単項イデアル定理より$\idht P=1$である.よって$P$は単項だから$P=(a)$とかける.よって$f\in (a)$なのである$r$について$f=ar$であり,$f$は既約だから$r$は単元.よって$(f)=P$となり$f$は素元である.
	\end{eqv}
\end{proof}

また,Noetherと限らない一般の整域について次が成り立つ.

\begin{lem}
	$A$を整域とする.$S\subset A$を素元によって生成された積閉集合とする.$f\in A$を既約であるとする.このとき$f/1\in S^{-1}A$も既約であり,また$f$が素元であることと$f/1$が素元または単元であることは同値.
\end{lem}

\begin{proof}
	まず$f/1$が既約であることを示そう.$f/1=a/s\cdot b/t$とかけるとする.すると$stf=ab$とかける.$st\in S$なので,素元$p_1,\dots,p_n$をとって$st=p_1\cdots p_n$とかける.よって各$i$について$ab\in (p_i)$であるから,$a\in (p_i)$または$b\in (p_i)$である.よってある$r,r'\in A$と$s',s''\in S$がとれて$f=rr',a=s'r,b=s''r'$とかける.$f$は既約なので$r$または$r'$は単元である.よって$a\in S$または$b\in S$となり$a/s$または$b/t$は単元である.よって$f/1$は既約である.
	
	次に$f\in A$が素元であるとする.すると$S^{-1}A/(fS^{-1}A)=S^{-1}(A/fA)$において$A/fA$は整域であるから$S^{-1}(A/fA)$はゼロまたは整域である.よって$f/1$は素元または単元である.逆を示そう.まず$f/1$が単元であるとすると,$f/1\cdot a/s=1$となる$a/s\in S^{-1}A$が存在する.このとき$fa=s\in S$であるので,適当な$p_1,\dots,p_n$をとって$fa=p_1\cdots p_n$とかける.よって各$i$について$f\in (p_i)$または$a\in (p_i)$である.任意の$i$について$f\not\in (p_i)$ならば$f$は単元になり既約であることに矛盾.よって$f\in (p_i)$となる$i$があるが,$f$が既約なので$(f)=(p_i)$となり$f$は素元である.また$f/1$が素元であるとすると,$ab\in (f)$とすると$(f/1)$が素イデアルだからある$s\in S$と$r\in A$が存在して$sa=rf$または$sb=rf$である.$sa=rf$のときに示す.$s=p_1\cdots p_n$と表そう.このとき$rf\in (p_i)$より,ある$i$について$f\in (p_i)$なら$f$は素元であり問題ない.また任意の$i$について$f\not\in (p_i)$なら$a\in (f)$となり$(f)$が素イデアルであることが従う.
\end{proof}

\begin{thm}[永田の判定条件]\label{thm:UFDnagata}
	整域$A$についてUFDであることと,(ACCP)を満たし,かつある素元からなる$A$の積閉集合$S$について$A_S$がUFDであることは同値.
\end{thm}

\begin{proof}
	\begin{eqv}
		\item 任意の$f/s\in S^{-1}A$をとる.$f\in A$を既約分解して$f=f_1\dots f_n$とする.$A$はUFDだから$f_i$は素元なので,ある$m\leq n$について$f_1,\dots,f_m\in S,f_{m+1},\dots,f_n\not\in S$となるように並び替えることができる.すると;
		\[f/s=\frac{f_1\dots f_m}{s}\frac{f_{m+1}}{1}\dots\frac{f_n}{s}\]
		とかけ,補題から$f/s$は単元と素元の積でかけているので,$S^{-1}A$はUFDである.
		\item $f\in A$を既約元とする.補題より$f/1$は既約であり,$S^{-1}A$がUFDだから$f/1$は素元.すると再び補題より$f$は素元である.
	\end{eqv}
\end{proof}

\ref{prop:UFD判定}より素イデアルが単項かどうかに主要な興味があることがわかる.ここでは\cite{narita}による次の補題を紹介しよう.

\begin{lem}
	$A$が整域であるとする.イデアル$I$で$I\oplus A^n\cong A^{n+1}$となるものは単項イデアルである.
\end{lem}

\begin{proof}
	$A^{n+1}$の基底を$\{e_0,\dots,e_n\}$とする.また$I\oplus A^n\hookrightarrow A\oplus A^{n}$としたとき,$f_0$を$A$の,$\{f_1,\dots,f_n\}$を$A^n$の基底とすれば,同型;
	\[\varphi:A^{n+1}\to I\oplus A^n\]
	の基底による表現は$\varphi(e_i)=\sum_{j=0}^n a_{ij}f_j$とかける.行列$(a_{ij})$の$(i,0)$余因子を$d_i$とおき,$\det (a_{ij})=d$とすると,$\sum a_{i0}d_i=d,\sum a_{ij}d_i=0~(j\neq i)$である.よって$e_0'=\sum d_ie_i$とおくと$\varphi(e_0')=df_0$であり,$\varphi$は全射だから$\varphi(e_i')=f_i$となる$e_i'\in A^{n+1}$が存在する.ここで$e_i'=\sum_{j=0}^n c_{ij}e_j$とおくと,行列として;
	\[(c_{ij})(a_{ij})=
		\begin{pmatrix}
			d&0&\cdots&0\\
			0&1&&0\\
			\vdots&&\ddots\\
			0&0&&1
	\end{pmatrix}\]
	が成り立つ($e_i'=\sum_{j=0}^n c_{ij}e_j$の両辺に$\varphi$を施してみよ).よって$\det (c_{ij})=1$となるので,$e_0',\dots,e_n'$は$A^{n+1}$の基底である.ゆえに$If_0=\varphi(Ae_0')=Ad f_0$となり,$I=Ad$である.
\end{proof}
この補題の条件は一般に\textbf{安定して自由}であるといわれる.

\begin{defi}[安定して自由]
	$A$を環,$M$を$A$加群とする.有限自由加群$F,F'$が存在して$M\oplus F\cong F'$となるとき,$M$は\textbf{安定して自由}(stably free)であるという.
\end{defi}

加群が安定して自由であることを判定する条件として,\textbf{有限自由分解}が有効である.
\begin{defi}[有限自由分解]\index{ゆうげんじゆうぶんかい@有限自由分解}
	$A$を環,$M$を$A$加群とする.長さ有限の完全列;
	\[\begin{tikzcd}
		0\nxcell F_n\nxcell F_{n-1}\nxcell\dots\nxcell F_0\nxcell M\nxcell 0
	\end{tikzcd}\]
	であって,各$F_i$が有限階数の自由加群であるものが存在するとき,$M$は\textbf{有限自由分解}(finite free resolution)を持つという.
\end{defi}

まず,\ref{prop:射影被覆が同型の補題}の一般化であるSchanuelの補題を紹介しよう.
\begin{lem}[Schanuelの補題]\index{#Schanuelのほだい@Schanuelの補題}
	$A$を環,$M$を$A$加群とする.完全列;
	\[\begin{tikzcd}
		0\nxcell K\nxcell P_n\nxcell\dots\nxcell P_0\nxcell[\varepsilon] M\nxcell0\\
		0\nxcell K'\nxcell Q_n\nxcell\dots\nxcell Q_0\nxcell[\varepsilon'] M\nxcell0
	\end{tikzcd}\]
	について,各$0\leq i\leq n$について$P_i,Q_i$が射影加群ならば;
	\[P_0\oplus Q_1\oplus P_2\oplus\dots\cong Q_0\oplus P_1\oplus Q_2\oplus\dots\]
	が成り立つ.
\end{lem}

\begin{proof}
	$n$についての帰納法で示す.まず$n=0$とする.
	\[\begin{tikzcd}
		0\nxcell K\nxcell P\arrow[d,shift right=.5em,"g",swap]\nxcell[\varepsilon]M\nxcell0\\
		0\nxcell K'\nxcell Q\arrow[u,shift right=.5em, "f",swap]\nxcell[\varepsilon']M\nxcell0
	\end{tikzcd}\]
	$P,Q$は射影的なので,上のように$f,g$が存在して$\varepsilon'\circ f=\varepsilon,\varepsilon\circ g=\varepsilon'$が成り立つ.そこで同型写像$\varphi,\psi:P\oplus Q\to P\oplus Q$を構成しよう;
	\[\begin{tikzcd}
	0\nxcell K\oplus Q\nxcell P\oplus Q\arrow[d,shift right=.5em,"\varphi",swap]\nxcell[\widetilde\varepsilon]M\nxcell0\\
	0\nxcell K'\oplus P\nxcell P\oplus Q\arrow[u,shift right=.5em, "\psi",swap]\nxcell[\widetilde\varepsilon']M\nxcell0
	\end{tikzcd}\]
	$\varepsilon$を$P\oplus Q$に誘導した$(x,y)\mapsto \varepsilon(x)$を$\widetilde\varepsilon$と表す.$\varepsilon'$についても同様である.このとき;
	\[\varphi:P\oplus Q\to P\oplus Q;(x,y)\mapsto (x-g(y),y+f(x-g(y)))\]
	と定義する.同様に;
	\[\psi:P\oplus Q\to P\oplus Q;(x,y)\mapsto(x+g(f(x)-y),y-f(x))\]
	と定義すれば$\psi=\varphi^{-1}$となる.また$\widetilde\varepsilon'\circ\varphi=\widetilde\varepsilon$であり,誘導される$K\oplus Q\to K'\oplus P$について蛇の補題から同型であることがわかる.
	
	$n-1$まで正しいとしよう.次の完全列;
	\[\begin{tikzcd}
	0\nxcell K\nxcell P_n\nxcell\dots\nxcell P_1\nxcell\ker\varepsilon \nxcell0\\
	0\nxcell K'\nxcell Q_n\nxcell\dots\nxcell Q_1\nxcell\ker\varepsilon'\nxcell0
	\end{tikzcd}\]
	について,$n=0$の場合から$\ker\varepsilon\oplus Q_0\cong\ker\varepsilon'\oplus P_0$であり,次の完全列;
	\[\begin{tikzcd}
	0\nxcell K\nxcell P_n\nxcell\dots\nxcell P_1\oplus Q_0\nxcell\ker\varepsilon\oplus Q_0\arrow[d,"\cong"] \nxcell0\\
	0\nxcell K'\nxcell Q_n\nxcell\dots\nxcell Q_1\oplus P_0\nxcell\ker\varepsilon'\oplus P_0\nxcell0
	\end{tikzcd}\]
	に$n-1$の場合を適用して結論を得る.
\end{proof}

\begin{prop}
	有限生成な射影加群が有限自由分解をもてば安定的に自由である.
\end{prop}

\begin{proof}
	$P$を有限生成な射影加群とする.
	\[\begin{tikzcd}
		0\nxcell F_n\nxcell\dots\nxcell F_0\nxcell P\nxcell0\\
		0\nxcell 0\nxcell\dots\nxcell P\nxcell[\id] P\nxcell0
	\end{tikzcd}\]	
	についてSchanuelの補題を適用すればよい.
\end{proof}

これらの準備を踏まえていよいよAuslander--Buchsbaumの定理を証明しよう.
\begin{thm}[Auslander--Buchsbaumの定理]\index{#Auslander--Buchsbaumのていり@Auslander--Buchsbaumの定理}\label{thm:Auslander--Buchsbaumの定理}
	正則局所環はUFDである.
\end{thm}

\begin{proof}
	
\end{proof}
\section{Serreの条件,因子類群}

前節でUFD,特にAuslander--Buchsbaumの定理(\ref{prethm:Auslander-Buchsbaum})について紹介した.この節ではUFDであることのご利益の1つとして因子類群について説明しようと思う.

\begin{defi}[Serreの条件]\index{#Serreのじょうけん@Serreの条件}
	$A$をNoether環とする.
	\begin{sakura}
		\item 任意の$\idht P\leq n$となる$P\in\spec A$について$A_P$が正則局所環であるとき,$A$で$(R_n)$が成り立つという.
		\item 任意の$P\in\spec A$について$\min\{n,\idht P\}\leq\mdepth A_P$となるとき,$A$で$(S_n)$が成り立つという.
	\end{sakura}
\end{defi}

$(R_n)$についてはわかりやすい条件であるから,$(S_n)$について少し見てみよう.明らかに$A$で$(S_{n+1})$が成り立つならば$(S_n)$が成り立つ.また,次を確認することも容易いだろう.

\begin{prop}
	Noether環$A$がCM環であることと,任意の$n$について$(S_n)$が成り立つことは同値.
\end{prop}

また,次の言い換えが成り立つ.
\begin{lem}
	Noether環$A$について$(S_1)$が成り立つことと,任意の$P\in\ass A$が極小であることは同値である.
\end{lem}

\begin{proof}
	任意の$P\in\ass A$をとる.このとき\ref{lem:素因子と局所化}より$PA_P\in\ass A_P$である.すると\ref{lem:depth 0とass}により$\mdepth A_P=0$である.よって$\min\{1,\idht P\}\leq0$だから$\idht P=0$となる.逆はこれを反対方向にたどればよい.
\end{proof}

まず,これを用いた被約性の判定法を紹介しよう.

\begin{thm}[Serreの$(R_0)+(S_1)$判定法]
	Noether環$A$について$(R_0),(S_1)$が成り立つことと$A$が被約であることは同値である.
\end{thm}

\begin{proof}
	\begin{eqv}
		\item $\nil A=\bigcap_{\idht P=0}P=0$であることを示す.$\ass A=\{P_1,\dots,P_r\}$とおき,$0$を準素分解して$0=Q_1\cap\dots\cap Q_r$とする.ここで$Q_i$は$P_i$準素イデアルとする.$(S_1)$より$\ass A=\mkset{P\in\spec A}{\idht P=0}$であることに注意すると,局所化$f_i:A\to A_{P_i}$について\ref{thm:準素成分の一意性}より$Q_i=f_i^{-1}(0)$である.ここで$(R_0)$から$A_{P_i}$は$0$次元の正則局所環,すなわち体である.よって$Q_i=P_i$となる.ゆえに$\bigcap_{\idht P=0}P=Q_1\cap\dots\cap Q_r=0$であるので,$A$は被約. 
		\item $0=\bigcap_{\idht P=0}P$が$0$の準素分解となり,$\ass A=\mkset{P\in\spec A}{\idht P=0}$となるので$(S_1)$が成り立つ.また,$P'\in\ass A$が$P\neq P'$ならば$P'A_P=A_P$であるから;
		\[0=\bigcap_{\idht P'=0}P'A_P=PA_P\]
		となる.よって$A_P$は体となり,$(R_0)$が成り立つ.
	\end{eqv}
\end{proof}

次に$(S_2)$についてみてみよう.以下\textbf{仮定に整域を課す}ことに注意せよ.

\begin{lem}
	$A$をNoether整域とする.$(S_2)$が成り立つことと,任意の$a\in A$で$0$でも単元でもないものについて,任意の$P\in\ass A/aA$は高さ$1$であることは同値である.
\end{lem}

\begin{proof}
	\begin{eqv}
		\item 任意の$P\in\ass A/aA$をとる.$a/1\in PA_P$は$A_P$正則である.$P=\ann (x+aA)$とおくと,任意の$b/1\in PA_P$について$bx/1=0\in A_P/aA_P$であるので$A_P/aA_P$正則元は存在しない.よって$\mdepth A_P=1$である.ゆえに$(S_2)$から$\min\{2,\idht P\}\leq1$であるから$\idht P\leq1$である.一方Krullの単項イデアル定理から$1\leq\idht P$なので,$\idht P=1$である.
		\item $\idht P=0,1$のときは明らかに$\idht P\leq\mdepth A_P$が成り立つ.$2\leq\idht P$のときを考えればよい.$0\neq a_1\in P$を1つ固定する.任意の$a_2\in P$が$A/aA$の非正則元であるなら,$P\subset\bigcup_{Q\in\ass A/a_1A}$であるのでPrime avoidanceにより$P\subset Q$となる$Q\in\ass A/a_1A$が存在する.ここで仮定より$Q$は高さ1なので$\idht P\leq1$となって矛盾する.よって$A/a_1A$正則元$a_2\in P$が存在し,$2\leq\mdepth A_P$である.
	\end{eqv}
\end{proof}

\begin{thm}[Serreの$(R_1)+(S_2)$判定法]
	Noether整域$A$が整閉であることと,$A$で$(R_1),(S_2)$が成り立つことは同値である.特に次の条件;
	\begin{sakura}
		\item $A$は整閉である.
		\item $A$について$(R_1),(S_2)$が成り立つ.
		\item $A$について$(R_1)$が成り立ち,$A=\bigcap_{\idht P=1}A_P$である.
	\end{sakura}
	は同値である.
\end{thm}

\begin{proof}
	\begin{eqv}[3]
		\item $(R_1)$が成り立つことは明らかである(整閉整域は正規であって,またDVRとは$1$次元の正則局所環のことであるから).また,$0$でも単元でもない$a\in A$について$P\in\ass_A (A/aA)$を考える.$P=\ann (b+aA)$となる$b\not\in aA$をとろう.定義より$bP\subset aA$である.$P$で局所化して考えよう.$bPA_P\subset aA_P$であるから,$b/aPAP\subset A_P$である.ここで$b/aPA_P\subset PA_P$なら,Cayley--Hamiltonの定理より$b/a$は$A_P$上整である.よって$b/a\in A_P$となるが,これはある$s\not\in P$が$bs\in Aa$となることを導き矛盾.ゆえに$b/aPA_P=A_P$である.よって$bPA_P=aA_P$となり,ある$x\in P$と$s\not\in P$によって$bx/s=a$である.このとき任意の$c/t\in PA_P$について,ある$r/u\in A_P$が存在して$bc/t=ar/u=bxr/su$となる.$A$は整域だから$c/t=xr/su$が成立する.よって$PA_P=xA_P$となり,極大イデアルが単項だから$A_P$は$1$次元である.よって$\idht P=1$となり,$(S_2)$が成り立つことがわかった.
		\item $A\subset\bigcap_{\idht P=1}A_P$は明らかであるから,$x/s\in\Frac A$について$x/s\in\bigcap_{\idht P=1} A_P$かつ$x/s\not\in A$であると仮定する.$x\not\in sA$かつ$s$は$A$の非単元であることに注意する.すると$A/sA$を$A$加群とみたとき$\ann (x+sA)$は$A$の真のイデアルであるから,$s\in\ann(x+sA)\subset P$となる$P\in\ass (A/sA)$が存在する.すると,先の補題より$(S_2)$から$\idht P=1$が従う.よって$x/s\in A_P$なのである$a\in A$と$t\not\in P$が存在して$x/s=a/t$となるが,定義よりある$h\not\in P$が存在して$h(xt-as)=0$となる.$A$は整域なので$xt=as\in sA$となり,$t\in\ann(x+sA)\subset P$となって矛盾する.よって$x/s\in A$である.
		\item $a/b\in\Frac A$について$A$上整であると仮定する.よって,$A$の元たちを適当にとれば;
		\[(a/b)^n+c_1(a/b)^{n-1}+\dots+c_n=0\quad(c_i\in A)\]
		となる$n$が存在する.このとき,仮定から$c_i$たちは整閉整域$A_P$の元であるから,$a/b$は$A_P$上整となり$a/b\in A_P$である.これが任意の$\idht P=1$となる$P$で言えるから$a/b\in\bigcap_{\idht P=1}A_P=A$である.
	\end{eqv}
\end{proof}

この事実により,Auslander--Buchsbaumの定理を経由することなく次が示される.
\begin{cor}
	正則局所環は整閉整域である.
\end{cor}

\begin{proof}
	正則ならば$(R_1)$は明らかになりたち,またCM環であるので$(S_2)$も成り立つ.
\end{proof}

\begin{thm}
	正則局所環はUFDである.
\end{thm} %Homology代数的手法
	
	\def\prepartname{付録}
	\def\postpartname{}
	\thepartchange
	\part[Appendix \thepart, Various Example]{様々な例}

Counter example of Commutative Algebra を目指したい.

\section{加群の同型と相等}

加群について,全単射準同型の存在,すなわち同型$M\cong N$と,集合としての相等$M=N$は区別しなければならない.例えば中山の補題で大変なことになる.次の例を見てみよう.

\begin{ex}
	$A$を環,$x$を零因子でない元とする.$A$加群として$A\cong xA$であるので,中山の補題(\ref{cor:NAK})からある$a\in A$で$a-1\in(x),aA=0$となるものが存在する.$aA=0$より$a=0$であり,$-1\in (x)$となるので$x$は可逆である.
\end{ex}

この結果は明らかに正しくない.矛盾を引き起こした理由は\textbf{同型}であって\textbf{相等}ではないときに中山の補題を適用してしまったからである.このように加群の同型をイコールと思って扱うと大変なことになってしまうので注意を必要とする.
\section{ED,PID,UFD}
\begin{surex}
	体$K$について,$K[X,X^{-1}]$はPIDである.
\end{surex}

\begin{proof}
	$I$を$K[X,X^{-1}]$のイデアルとする.これはNoetherなので$I=(f_1,\dots,f_r)$とできる.$f_1,\dots,f_r$に現れる負ベキの項は高々有限個だから,ある$n\geq0$が存在して$X^nf_i\in K[X]$とできる.$(X^nf_1,\dots,X^nf_r)$を$K[X]$のイデアルとみると,これはPIDなので$(X^nf_1,\dots,X^nf_r)=(g)$となる$g\in K[X]$が存在し,$K[X,X^{-1}]$のイデアルとしても$(f_1,\dots,f_r)=(g)$となる.
\end{proof}
\begin{surex}[UFDだがPIDでない環]
	体$K$について,$K[X,Y]$はUFDだがPIDではない.
\end{surex}

\begin{proof}
	\ref{cor:多変数もUFD}.
\end{proof}
\begin{surex}[PIDだがEDでない環]
	$\Z[(1+\sqrt{-19})/2]$はPIDだがEDではない.
\end{surex}

PIDだがEDでない環はこのように2次体の整数環についてよく知られているが,次の定理が知られている(証明は\cite{goel2018nullstellenstze}を見よ).

\begin{thm}
	$b,c\in\R$を$b>0,c>0$とする.$\R[X,Y]/(X^2+bY^2+c)$はPIDであるがEDではない.
\end{thm}

\begin{proof}
	$\R[X,Y]/(X^2+bY^2+c)$は$b>0,c>0$のときEDではない(\cite{goel2018nullstellenstze},系2.17).また,$\R[X,Y]/(X^2+bY^2+c)$がPIDであることは$c>0$と同値である(\cite{goel2018nullstellenstze},系2.18).
\end{proof}

\section{次元論}
\begin{surex}[無限次元のNoether整域の例(\citealp{nagatakan})]\label{ex:無限次元Noether環}
	$k$を体とし,$A=k[x_1,\dots,x_n,\dots]$を可算無限個の変数をもつ多項式環とする.自然数の増加列$\{n_i\}$を,$n_i-n_{i-1}<n_{i+1}-n_i$が成り立つように取る.$P_i=(x_{n_i},\dots,x_{n_{i+1}-1})$とおく.これらは素イデアルであるから,$\bigcup P_i$の$A$における補集合を$S$とすると,これは積閉である.$S^{-1}A$は無限次元のNoether整域となる.
\end{surex}

\begin{proof}
実際,$S^{-1}A$の素イデアルは$\bigcup P_i$に含まれる$A$の素イデアルであることを考えると$\dim S^{-1}A=\infty$であることは明らか.Noetherであることは,次の補題から従う.
		
\begin{lem}
	環$A$について,以下の2つ;
	\begin{sakura}
		\item $\ideal{m}$が$A$の極大イデアルならば,$A_{\ideal{m}}$はNoetherである.
		\item 任意の$0\neq x\in A$について,$x$を含む$A$の極大イデアルは有限個しかない.
	\end{sakura}
		を満たすならば,$A$はNoetherである.
\end{lem}
		
\begin{proof}[\textbf{補題の証明}]
	$I$を$A$のイデアルとする.(ii)より,$I$を含む極大イデアルは有限個しかない.それらを$\ideal{m}_1,\dots,\ideal{m}_s$としよう.任意の$x_0\in I$をとる.$x$を含む極大イデアルは有限個であるから,それを$\ideal{m}_1,\dots,\ideal{m}_s,\ideal{m}_{s+1},\dots,\ideal{m}_{s+r}$とする.各$1\leq j\leq r$について,$x_j\in I$で$x_j\not\in\ideal{m}_j$であるものがとれる.また,(i)より各$1\leq i\leq s$について$A_{\ideal{m}_i}$はNoetherなので,$IA_{\ideal{m}_i}$を生成する$I$の元は有限個である.それらを$x_{s+1},\dots,x_{s+t}$としよう.$I'=(x_0,\dots,x_t)$とおく.明らかに$I'\subset I$であるので,$A$自然な$A$加群の準同型$\iota:I'\hookrightarrow I$が存在する.このとき\ref{prop:局所的性質}を用いて$\iota$が全射であることを示す.$A$の極大イデアル$\ideal{m}$について,$\ideal{m}$が$\ideal{m}_1,\dots,\ideal{m}_s$のどれとも異なるとき,$I,I'\not\subset\ideal{m}$であるから,$IA_{\ideal{m}}=I'A_{\ideal{m}}=A_{\ideal{m}}$である.また$\ideal{m}_i (1\leq i\leq s)$については,$I'$は$IA_{\ideal{m}_i}$の生成元をすべて含むので,$I'A_{\ideal{m}_i}=IA_{\ideal{m}_i}$である.よって,任意の極大イデアルに$\iota$を誘導したものは全単射であるから,\ref{prop:局所的性質}によって$\iota$は全射,すなわち$I=I'$である.よって$I$は有限生成.
\end{proof}
		
この補題が適用できることを見るために,$S^{-1}A$の極大イデアルは$S^{-1}P_i$のみであることを示そう.$P\in\spec A$が$P\subset\bigcup P_i$であるとすると,$P\subset P_i$となる$i$があることをいえば十分.任意の$a\in P$に対して,$a$を含む$P_j$たちは有限個なので,それをすべてとってきて$(a)\subset\bigcup_{j=1}^n P_j$とする.任意の$x\in P$に対し,$x$を含む$P_j$たちも有限個だから,ある$n'$に対して$(a,x)\subset\bigcup_{j=1}^{n'}P_j$とできる.Prime avoidanceより$(a,x)\subset P_{j}$となる$j\leq n'$がとれるが,$a\in P_j$より$j\leq n$でなければならず,すると$(a,x)\subset\bigcup_{j=1}^n P_j$である.よって,$x$は任意で,$n$のとりかたは$x$によらないので$P\subset\bigcup_{j=1}^n P_i$である.再びPrime avoidanceを使って$P\subset P_i$となる$i$がとれる.

また,${S^{-1}A}_{S^{-1}P_i}=A_{P_i}$であるので,$S^{-1}A$は補題を満たす.

\end{proof} %例
	\part[Appendix \thepart, Category theory]{圏}
\section{圏}

\begin{defi}[圏]\index{けん@圏}
	\begin{itemize}
		\item $\bullet$ \textbf{対象}(object)のクラス$\operatorname{ob}(\mathscr{A})$.
		\item $A,B\in\operatorname{ob}(\mathscr{A})$について,クラス$\hom (A,B)$と,その元$f\in\hom(A,B).$~($f$を$A$から$B$への\textbf{射}(morphism,map,arrow)という.)
	\end{itemize}
	について,以下の公理;
	\begin{itemize}
		\item $A\in\operatorname{ob}(\mathscr{A})$について,\textbf{恒等射}(identity)~$\id_A\in\hom(A,A)$が存在する.
		\item $A,B,C\in\operatorname{ob}(\mathscr{A})$とする.すべての$f\in\hom(A,B)$と$g\in\hom(B,C)$について$g\circ f\in\hom(A,C)$が定義され,~$f,g$の\textbf{合成}(composition)という.これは結合的で,恒等射を単位元とする.
	\end{itemize}
	をみたすとき,~$\operatorname{ob}(\mathscr{A})$と$\hom$のデータを合わせて\textbf{圏}(category)$\mathscr{A}$という.
\end{defi}

$\hom(A,B)$は$\mathscr{A}(A,B)$ともかく.また,簡単のために$A\in\operatorname{ob}(\mathscr{A})$を$A\in\mathscr{A},f\in\hom(A,B)$を$f:A\to B$とかく.
{
	\makeatletter
	\renewcommand{\symlist}[2]{%
		\mbox{$#1$}%
		\setlength{\m@syu@length}{2.5cm}%
		\settowidth{\m@syu@length@}{\mbox{$#1$}}%
		\addtolength{\m@syu@length}{-\m@syu@length@}%
		\leaders\hbox{\normalfont$\m@th \mkern%
			\@dotsep mu\hbox{.}\mkern \@dotsep mu$}\hskip\m@syu@length
		#2\par
	}%
\begin{ex}
	圏の例として,以下がある.
	
	\symlist{\mathbf{Set}}{集合全体,写像.}
	\symlist{\mathbf{Ab}}{Abel群全体,準同型写像.}
	\symlist{\mathbf{Ring}}{(1を持つ)可換環全体,準同型写像.}
	\symlist{\operatorname{Mod}(A)}{$A$加群全体,~$A$準同型写像.}
\end{ex}
}
\begin{defi}[局所小]\index{きょくしょしょう@局所小}
	圏$\mathscr{A}$について,任意の$A,B\in\mathscr{A}$に対して$\hom(A,B)$が集合であるとき,~$\mathscr{A}$を\textbf{局所小}(locally small)な圏であるという.
\end{defi}

上にある例はすべて局所小である.以降,すべての圏は明記しない限り局所小であることを仮定する.

どこの圏の話であるかを捉えるために,~$A,B\in\mathscr{A}$について$\hom(A,B)$を$\hom_{\mathscr{A}}(A,B)$とかいたりする.たとえばAbel群$A$は$A\in\mathbf{Ab}$かつ$A\in\mathbf{Set}$なので,射を区別する必要がある.

\section{関手}
\begin{defi}[関手]\index{かんしゅ@関手}\index{きょうへんかんしゅ@共変関手}
	$\mathscr{A},\mathscr{B}$を圏とする.
	\[F:\operatorname{ob}(\mathscr{A})\to\operatorname{ob}(\mathscr{B});A\mapsto F(A)\]
	\[F:\hom(A,B)\to\hom(F(A),F(B));f\mapsto F(f)\]
	で,次の公理;
	\begin{sakura}
		\item 任意の$A\in\mathscr{A}$について$F(\id_A)=\id_{F(A)}$
		\item $f\in\hom(A,B),g\in\hom(B,C)$に対して$F(g\circ f)=F(g)\circ F(g)$
	\end{sakura}
	を満たすとき,~$F$を$\mathscr{A}$から$\mathscr{B}$への\textbf{関手}(functor),特に\textbf{共変関手}(covariant functor)という.
\end{defi}
\begin{defi}[双対圏]\index{そうついけん@双対圏}
	圏$\mathscr{A}$について,圏$\mathscr{A}^{op}$を;
	\[\operatorname{ob}\mathscr{A}^{op}=\operatorname{ob}\mathscr{A}\]
	\[\hom_{\mathscr{A}^{op}}(A,B)=\hom_{\mathscr{A}}(B,A)\]
	と定める.これを$\mathscr{A}$の\textbf{双対圏}(dual category)という.
\end{defi}
\begin{defi}[反変関手]\index{はんへんかんしゅ@反変関手}
	圏$\mathscr{A},\mathscr{B}$について,関手$\mathscr{A}^{op}\to\mathscr{B}$を$\mathscr{A}$から$\mathscr{B}$への\textbf{反変関手}(contravariant functor)という.
\end{defi}
\begin{defi}[忠実,充満]\index{ちゅうじつかんしゅ@忠実(関手)}\index{じゅうまんかんしゅ@充満(関手)}
	$F:\mathscr{A}\to\mathscr{B}$を関手とする.各$A,B\in\mathscr{A}$について;
	\[F:\hom(A,B)\to\hom(F(A),F(B));f\mapsto F(f)\]
	が単射であるとき\textbf{忠実}(faithful),全射であるとき\textbf{充満}(full)という.
\end{defi}
\begin{figure}[H]
	\centering
	\begin{tikzcd}
		A\arrow[dd,dashed]\arrow[dd,bend right=60,phantom,"\mathscr{A}"]&&&F(A)\arrow[dd,"g"]\arrow[dd,bend left=60,phantom,"\mathscr{B}"]\\
		&{}\arrow[r,"F",Rightarrow]&{}\\
		B&&&F(B)
	\end{tikzcd}
	\caption{}
\end{figure}

上の図式において,関手$F$で$g$に移る$\mathscr{A}$の射(図式では破線)が高々1つならば忠実で,少なくとも1つあれば充満である.
\begin{defi}[自然変換]\index{しぜんへんかん@自然変換}
	関手$F,G$があり,~$F,G:\mathscr{A}\to\mathscr{B}$であって,すべての$\mathscr{A}$の射$f:A\to B$について以下の図式(Figure.\ref{fig:自然変換})を可換にするような射$\varphi(A):F(A)\to G(A)$がとれるとき,~$\varphi:F\to G$を関手$F$から$G$への\textbf{自然変換}(natural transformation)という.
\end{defi}
\begin{figure}[H]
	\centering
	\begin{tikzcd}[row sep=huge, column sep=huge]
		F(A)\arrow[r,"F(f)"]\arrow[d,"\varphi(A)"]&F(B)\arrow[d,"\varphi(B)"]\\
		G(A)\arrow[r,"G(f)"]&G(B)
	\end{tikzcd}
	\caption{}\label{fig:自然変換}
\end{figure}

すべての$A\in\mathscr{A}$に対して$\varphi(A)$が同型のとき,~$\varphi$を同型という.
\begin{defi}[圏同値]\index{けんどうち@圏同値}\index{そうついどうち@双対同値}
	関手$F:\mathscr{A}\to\mathscr{B}$に対して,~$G\circ F=\id_{\mathscr{A}},F\circ G=\id_{\mathscr{B}}$となる関手$G:\mathscr{B}\to\mathscr{A}$が存在するとき,~$\mathscr{A}$と$\mathscr{B}$は\textbf{圏同値}(equivalence of categories)という.
\end{defi}

また$\mathscr{A}^{op}$と$\mathscr{B}$が圏同値なとき$\mathscr{A}$と$\mathscr{B}$は\textbf{反変同値},双対同値(dual equivalance)という.

\section{Abel圏}

Abel圏とはホモロジー代数を展開するために必要なエッセンスを抽出した圏のことである.まず,射$f:X\to Y$についての図式追跡(いままでの例では完全列など)を考えるために核,像の議論が不可欠であった.これはホモロジー,コホモロジーの定義にも不可欠である.よって核,像,余核を持つことが要求される.そのために圏$\mathscr{A}$について$\hom_{\mathscr{A}}(X,Y)$が自然にAbel群になることを要求する.このとき,自然な準同型$h_Z:\hom(Z,X)\to\hom (Z,Y);\varphi\mapsto f\circ\varphi$が定まる.これの核を表現する対象を$\ker f$とかく.~$h_Z^{op}:\hom(Y,Z)\to\hom(X,Z)$として余核も同様に定義される.定義の形でまとめよう.
\begin{defi}[射の核,余核]
	圏$\mathscr{A}$を,任意の$X,Y\in\mathscr{A}$に対し$\hom(X,Y)$が自然にAbel群である圏とする.射$f:X\to Y$を固定する.各$Z\in\mathscr{A}$に対して;
	\[\hom(Z,\ker f)\cong \ker h_Z=\ker(\hom(Z,X)\to\hom(Z,Y))\]
	となる$\ker f\in\mathscr{A}$が存在するとき,これを射$f$の\textbf{核}という.
	
	同様に$\hom(\coker f,Z)\cong\ker h_Z^{op}=\ker(\hom(Y,Z)\to\hom(X,Z))$となる対象が存在するとき,それを$\coker f$とかいて$f$の\textbf{余核}という.
\end{defi}

この定義は$\hom(Z,\ker f)$は$\hom(Z,X)$の一部と同型だと言っているので,~$\hom(Z,\ker f)\to\hom(Z,X)$が定まる.これにより関手の間の射$\hom(-,\ker f)\to\hom(-, X)$があるから,~$\iota:\ker f\to X$がある.ここで$\coim f=\coker \iota$と定義する.また,~$\hom(\coker f,Z)\to\hom(Y,Z)$が存在するが,~$Z$として$\coker f$をとると$\hom(\coker f,\coker f)\to\ker h_{\coker f}^{op}\subset\hom(Y,\coker f)$がある.これによる$\id$の像を$\pi:Y\to\coker f$とする.ここで$\im f$を$\ker\pi$で定義する.いま$h_{\coker f}^{op}(\pi)=f\circ\pi=0$であるので,~$g:X\to\ker\pi=\im f$が存在する.また,~$\ker h_Z$を形式的に書き下せば;
\[\ker h_Z=\mkset{\varphi\in\hom(Z,X)}{f\circ\varphi=0}\]
である.~$\hom(Z,\ker f)\cong \ker h_Z$は同型,特に全単射であるから$\hom(Z,\ker f)\to\hom(Z,X)$は集合として単射である.これは$\iota$が単射であることを導く.\footnote{このあたりのことは核は\textbf{差核}(difference kernel)の特別なものであることに由来しているが,本書ではそこまで述べる余裕はない.~\cite{siho}などを参照すると良いかもしれない.}ここでは$f\circ\iota=0$であるので,~$\ker\pi$に対して$\varphi:\ker\pi\to Y$を考えると$f\circ\iota=\varphi\circ(g\circ\iota)=0$であるので,~$\varphi$は単射であるから$g\circ\iota=0$を導く.ここから$\coker\iota$の普遍性より$h:\coker\iota\to\im f$が存在する.この$h$が同型であることを要請したい.これは以下のように図式にすると見やすい.これらをまとめてAbel圏を定義しよう.
\begin{figure}[H]
	\centering
	\begin{tikzcd}[row sep=huge, column sep=huge]
	\ker f\arrow[r,"\iota"]&X\arrow[r,"f"]\arrow[rd,dashed,"g"]\arrow[d]&Y\arrow[r,"\pi"]&\coker f\\
	&\coim f\arrow[r,dashed,"h"]\arrow[d,equal]&\im f\arrow[d,equal]\arrow[u,"\varphi"]\\[-2.5em]
	&\coker\iota&\ker\pi
	\end{tikzcd}
	\caption{}
\end{figure}

\begin{defi}[Abel圏]\index{#abelけん@Abel圏}
	圏$\mathscr{A}$が次の定義;
	\begin{defiterm}{AC}
		\item 任意の$X,Y\in\mathscr{A}$に対し,~$\hom_{\mathscr{A}}(X,Y)$はAbel群としての自然な構造を持つ.
		\item $\mathscr{A}$は自然な零対象0を持つ.
		\item 任意の$X,Y\in\mathscr{A}$に対し,直和$X\oplus Y$を持つ.ここで$X\oplus Y$は,任意の$Z\in\mathscr{A}$に対し$\hom(X\oplus Y,Z)\cong\hom(X,Z)\times\hom(Y,Z)$が成り立つ対象のことである.
		\item $\mathscr{A}$の中の$f:X\to Y$について,核$\ker f$と余核$\coker f$が存在する.
		\item $\coim f$と$\im f$は同型である.
	\end{defiterm}
	を満たすとき, Abel\textbf{圏}(Abelian Category)であるという.
\end{defi}

先に出た核からの射の単射性について補足しておきたいところだが,ここは流れに乗って議論を進めよう.~Abel圏の例としては,自明なAbel群の圏, $A$加群の圏$\operatorname{Mod}(A)$などの他にスキーム$(X,\oka_X)$上の$\oka_X$加群の層の圏などがある.以後$\mathscr{A}$をAbel圏とする,として議論していきたいのだが,それではあまりにも$\mathscr{A}$が抽象的にすぎる.ここで次の定理が大切である(証明は\cite{siho}定理2.160をみよ).

\begin{thm}[Freyd-Mitchellの埋め込み定理]\index{#Freyd-Mitchellのうめこみていり@ Freyd-Mitchellの埋め込み定理}\label{thm:埋め込み定理}
	対象の全体が集合となるようなAbel圏(対象の全体が集合となる圏を\textbf{小さい圏}という)からAbel群の圏への(加法的)完全忠実充満関手が存在する.
\end{thm}

これによりAbel圏を考えるときには,加群の圏などで図式追跡により示せる事実は,一般のAbel圏では図式追跡はできないにもかかわらず,正しいということに注意する必要がある.むしろそれは恩恵であって,元を考えたくなったらすべて加群だと思ってよい,ということをこの定理は主張している.実際にあの\cite{harts}でさえも,多くの文献では図式追跡の証明のみによる,とくに加群の圏でしか示していないことに言及をし,その後この定理により\quo{正当化}している.本書でも,以後大手を振ってAbel圏についての事実の証明に図式追跡を使おう.そのための注意として,まずAbel圏$\mathscr{A}$のなかでの完全列を埋め込むと,加群としても完全であることは大切である.

ここで埋め込み定理に述べた「加法的」という言葉について,なかば明らかで推測はついていると思うが説明しておく.
\begin{defi}[加法的関手]\index{かほうてきかんしゅ@加法的関手}
	$\mathscr{A,B}$をAbel圏とし,関手$F:\mathscr{A}\to\mathscr{B}$を考える.~$\hom(X,Y)$がAbel群なので,~$f+g\in\hom(X,Y)$である.ここで;
	\[F(f+g)=F(f)+F(g)\]
	を満たす$F$を\textbf{加法的}(additive)関手という.
\end{defi}

次加法的関手なら$F(0)=0$であることに注意しよう.次に何度も取り上げてきた\textbf{完全関手}について述べる.

\begin{defi}[完全関手]\index{かんぜんかんしゅ@完全関手}
	$\mathscr{A,B}$をAbel圏とする.加法的関手$F:\mathscr{A}\to\mathscr{B}$と, $\mathscr{A}$の対象からなる短完全列;
	\[\begin{tikzcd}
		0\nxcell A\nxcell B\nxcell C\nxcell 0
	\end{tikzcd}\]
	を考える.ここで, $F(A)\to F(B)\to F(C)$が完全であるとき$F$を\textbf{半完全}(half-exact)であるといい, $0\to F(A)\to F(B)\to F(C)\to 0$が完全であるとき\textbf{完全}(exact)であるという. $G$が加法的反変関手であるときは$0\to G(C)\to G(B)\to G(A)\to0$が完全であるとき$G$は\textbf{(反変)完全}であるという. 
\end{defi}

\[\begin{tikzcd}
	0\nxcell M_0\nxcell M_1\nxcell M_2 
\end{tikzcd}\tag{$(\ast)$}\]

について,それぞれ;

\[\begin{tikzcd}
	0\nxcell F(M_0)\nxcell F(M_1)\nxcell F(M_2)
\end{tikzcd}\]
\[\begin{tikzcd}
	G(M_2)\nxcell G(M_1)\nxcell G(M_0)\nxcell 0
\end{tikzcd}\]

が完全であるとき, $F$を\textbf{左完全}(left-exact), $G$を\textbf{反変右完全}(contravariant right-exact)であるという.また;

\[\begin{tikzcd}
	M_0\nxcell M_1\nxcell M_2\nxcell0
\end{tikzcd}\tag{$(\ast\ast)$}\]

に対して;

\[\begin{tikzcd}
	F(M_0)\nxcell F(M_1)\nxcell F(M_2)\nxcell0
\end{tikzcd}\]
\[\begin{tikzcd}
	0\nxcell G(M_2)\nxcell G(M_1)\nxcell G(M_0)
\end{tikzcd}\]

が完全であるとき, $F$を\textbf{右完全}(right-exact), $G$を\textbf{反変左完全}(contravariant left-exact)であるという.

\section{射影極限,帰納極限}

極限については代数幾何の分冊にも収録してあるが,この分冊でも射影極限などを扱うため,ある程度の内容について述べておくことにする.代数幾何では帰納系が有効に働くためそちらの分冊では帰納系を中心に書いてあるから,こちらでは射影系を中心に記述することにする.

\begin{defi}[前順序]\index{まえじゅんじょ@前順序}
	集合$A$とその上の関係$\leq$に対し,反射律と推移律を満たすとき$\leq$を\textbf{前順序}(preorder)といい,~$A$を\textbf{前順序集合}(preordered set)という.
\end{defi}
\begin{defi}[有向集合]\index{ゆうこうしゅうごう@有向集合}
	$A$を前順序集合とする.任意の有限部分集合$X\subset A$が上界を持つとき,~$A$を\textbf{有向集合}(directed set)またはフィルター付き集合(filterd set)という.
\end{defi}

\begin{defi}[帰納系]\index{きのうけい@帰納系}\index{じゅんけい@順系}
	$I$を有向集合とする.各$i\in I$について集合$A_i$が存在し,また$i\leq j$となる$i,j\in I$に対して写像$\varphi_{ij}:A_i\to A_j$が与えられ,次を満たすとき$(A_i,\varphi_{ij})_{i,j\in I}$を集合の\textbf{帰納系}(inductive system)または\textbf{順系}(direct system)という.しばしば$(A_i)$と略す.
	\begin{defiterm}{IS}
		\item 任意の$i\in I$に対し$\varphi_{ii}=\id_{A_i}$である.
		\item $i,j,k\in I$が$i\leq j\leq k$を満たすなら,~$\varphi_{ik}=\varphi_{jk}\circ\varphi_{ij}$である.~(Figure.\ref{fig:順系}をみよ.)
	\end{defiterm}
\end{defi}

\begin{defi}[射影系]\index{しゃえいけい@射影系}\index{ぎゃくけい@逆系}
	順系の写像の向きを逆にしたものを\textbf{射影系}(projective system)または\textbf{逆系}(inverse system)という.すなわち各$i\in I$について集合$A_i$が存在し,また$i\leq j$となる$i,j\in I$に対して写像$\varphi_{ji}:A_j\to A_i$が与えられ,次を満たすとき$(A_i,\varphi_{ji})_{i,j\in I}$を集合の射影系といい,~$(A_i)$と略す.
	\begin{defiterm}{PS}
		\item 任意の$i\in I$に対し$\varphi_{ii}=\id_{A_i}$である.
		\item $i,j,k\in I$が$i\leq j\leq k$を満たすなら,~$\varphi_{ki}=\varphi_{ji}\circ\varphi_{kj}$である.~(Figure.\ref{fig:逆系}をみよ.)
	\end{defiterm}
\end{defi}

\begin{minipage}{.4\hsize}
	\begin{figure}[H]
		\centering
		\begin{tikzpicture}[baseline= (a).base]
		\node[scale=.8] (a) at (0,0){
			\begin{tikzcd}
			i\arrow[rr]\arrow[rdd]&&k&A_i\arrow[rr,"\varphi_{ik}"]\arrow[rdd,"\varphi_{ij}"]&&A_k\\
			&&{}\arrow[r,leftrightarrow]&{}\\
			&j\arrow[uur]&&&A_j\arrow[uur,"\varphi_{jk}"]
			\end{tikzcd}
		};
		\end{tikzpicture}
		\caption{帰納系}\label{fig:順系}
	\end{figure}
\end{minipage}
\hspace{\fill}
\begin{minipage}{.4\hsize}
	\begin{figure}[H]
		\centering
		\begin{tikzpicture}[baseline= (a).base]
		\node[scale=.8] (a) at (0,0){
			\begin{tikzcd}
			i\arrow[rr]\arrow[rdd]&&k&A_i\arrow[rr,"\varphi_{ki}",leftarrow]\arrow[rdd,"\varphi_{ji}",leftarrow]&&A_k\\
			&&{}\arrow[r,leftrightarrow]&{}\\
			&j\arrow[uur]&&&A_j\arrow[uur,"\varphi_{kj}",leftarrow]
			\end{tikzcd}
		};
		\end{tikzpicture}
		\caption{射影系}\label{fig:逆系}
	\end{figure}
\end{minipage}

射影系(resp.帰納系)の定義において,例えば集合を加群,写像を準同型写像と置き換えると,加群の射影型(resp.帰納系)が得られる.圏の言葉で言えば,各$i$について圏$\mathscr{C}$の対象$A_i$を与え,射$\varphi_{ij}(\text{resp.}\varphi_{ji})$を条件を満たすように定めれば圏$\mathscr{C}$の射影系(resp.帰納系)となる.

そのため,以下集合の写像についても射という言葉を用いる.
\begin{defi}[射影系の射]\index{しゃしゃえいけい@射(射影系,帰納系)}
	$(A_i),(B_i)$を射影系とする.このとき$(A_i)$から$(B_i)$への\textbf{射}(morphism)とは,~$f_i:A_i\to B_i$となる射の族$(f_i)$で,任意の$i\leq j$についてFigure.\ref{fig:射影系の射}が可換であるものをいう.射影系の射については,~Figure.\ref{fig:帰納系の射}が可換であるものをいう.
\end{defi}

\begin{minipage}{.45\hsize}
	\begin{figure}[H]
		\centering
		\begin{tikzcd}[row sep=huge, column sep=huge]
		A_i\arrow[r,"f_i"]&B_i\\
		A_j\arrow[r,"f_j"]\arrow[u,"\varphi_{ji}"]&B_j\arrow[u,"\psi_{ji}"]
		\end{tikzcd}
		\caption{}\label{fig:射影系の射}
	\end{figure}
\end{minipage}
\hspace{\fill}
\begin{minipage}{.45\hsize}
	\begin{figure}[H]
		\centering
		\begin{tikzcd}[row sep=huge, column sep=huge]
		A_i\arrow[r,"f_i"]\arrow[d,"\varphi_{ij}"]&B_i\arrow[d,"\psi_{ij}"]\\
		A_j\arrow[r,"f_j"]&B_j
		\end{tikzcd}
		\caption{}\label{fig:帰納系の射}
	\end{figure}
\end{minipage}

\begin{defi}[射影極限]\index{しゃえいきょくげん@射影極限}
	$(A_i)$を射影系とする.集合$A$及び射の族$\varphi_i:A\to A_i$の組$(A,\varphi_i)$で,次の条件;
	\begin{defiterm}{PL}
		\item $i\leq j$に対し$\varphi_{ji}\circ\varphi_j=\varphi_i$.
		\item 任意の集合$B$と,任意の射の族$f_i:B\to A_i$で$i\leq j$に対し$\varphi_{ji}\circ f_i=f_j$となるものに対して,射$f:B\to A$で$f_i=\varphi_i\circ f$となるものが一意的に存在する.
	\end{defiterm}
	を満たすものを\textbf{射影極限}(projective limit)または\textbf{逆極限}(inverse limit)といい,~$A:=\plim[i\in I] A_i$とかく.よく有効集合を省略し$\plim A_i$とかく.
\end{defi}
\begin{figure}[H]
	\centering
	\begin{tikzcd}
	{}&j\arrow[l]&&i\arrow[ll]&\arrow[l]\\
	\arrow[r]&A_j\arrow[rr,"\varphi_{ji}"]&&A_i\arrow[r]&{}\\
	\\
	&&B\arrow[luu,"f_j"]\arrow[ruu,"f_i",near start,swap]&&\plim A_i\arrow[llluu,crossing over,"\varphi_j",swap,near start]\arrow[luu,"\varphi_i",swap]\arrow[ll,dashed,"f",leftarrow]
	\end{tikzcd}
	\caption{射影極限}
\end{figure}

同様に射の向きを逆にしたものを\textbf{帰納極限}(inductive limit),順極限(direct limit)という.\index{きのうきょくげん@帰納極限}\index{じゅんきょくげん@順極限}
\begin{figure}[H]
	\centering
	\begin{tikzcd}
	{}\arrow[r]&i\arrow[rr]&&j\arrow[r]&{}\\
	{}\arrow[r]&A_i\arrow[rr,"\varphi_{ij}"]\arrow[rdd,"\varphi_{i}"]\arrow[rrrdd,"f_i",near start]&&A_j\arrow[r]\arrow[ldd,"\varphi_{j}",crossing over,near start,swap]\arrow[rdd,"f_j"]&{}\\
	\\
	&&\ilim A_i\arrow[rr,dashed,"f"]&&B
	\end{tikzcd}
	\caption{帰納極限}
\end{figure}

普遍性からの標準的な結果によって,射影極限や帰納極限は同型を除いて一意的である.

\begin{thm}
	加群の射影系$(A_i)$については,必ず射影極限が存在する.
\end{thm}
\begin{proof}
	直積加群$\prod A_i$の部分加群;
	\[A=\mkset{(x_i)\in\prod A_i}{\text{任意の}i\leq j\text{について, } x_j=\varphi_{ji}(x_i)}\]
	を考える. $\varphi_i$を自然な射影$A\to A_i$と定めることで, $(A,\varphi_i)$は射影極限の普遍性を満たす.
\end{proof}


\begin{defi}
	射影(帰納)系$(A_i),(B_i),(C_i)$に対し,その間の射$(f_i),(g_i)$が各$i$について系列;
	\[\begin{tikzcd}
	0\arrow[r]&A_i\arrow[r,"f_i"]&B_i\arrow[r,"g_i"]&C_i\arrow[r]&0
	\end{tikzcd}\]
	を完全にするとき;
	\[\begin{tikzcd}
	0\arrow[r]&(A_i)\arrow[r,"(f_i)"]&(B_i)\arrow[r,"(g_i)"]&(C_i)\arrow[r]&0
	\end{tikzcd}\]
	を射影系の(短)完全列であるという.
\end{defi}

加群の射影系の系列に対して,射影極限をとることで自然な系列が誘導される.すなわち,射影系の射$(f_i):(A_i)\to (B_i)$について$f:\plim A_i\to\plim B_i;(x_i)\mapsto(f_i(x_i))$と定めると,これは準同型になる.

帰納極限は完全関手となる,すなわち完全列の帰納極限は完全列となるが,射影極限についてはそうとは限らない.代数幾何の分冊では帰納極限が完全関手であることを示しているので,ここでは射影極限が完全になる条件を考えよう.

\begin{prop}
	加群の射影極限は左完全関手である.すなわち,射影系$(A_i,\varphi_{ij}),(B_i,\psi_{ij}),(C_i,\omega_{ij})$の完全列;
	\[\ses[(f_i)][(g_i)]{(A_i)}{(B_i)}{(C_i)}\]
	に対して;
	\[\begin{tikzcd}
	0\nxcell \plim A_i\nxcell[f]\plim B_i\nxcell[g]\plim C_i
	\end{tikzcd}\]
	は完全である.
\end{prop}

\begin{proof}
	構成から左完全であることは明らか(本質的に図式追跡)である.
\end{proof}

\begin{defi}[全射的系]
	射影系$(A_i,\varphi_{ij})$について,各$i$について$\varphi_{i+1,i}$が全射であるとき, $(A_i)$を\textbf{全射的系}(surjective system)であるという.
\end{defi}

\begin{thm}\label{thm:最初がsurjectiveなら射影極限は完全}
	加群の射影系の完全列;
	\[\ses[(f_i)][(g_i)]{(A_i)}{(B_i)}{(C_i)}\]
	において, $(A_i)$が全射的な系ならば;
	\[\ses[f][g]{\plim A_i}{\plim B_i}{\plim C_i}\]
	は完全である.
\end{thm}

\begin{proof}
	$g$の全射性のみ示せばよい.任意の$(z_i)\in\plim C_i$をとる.帰納的に$(y_i)$を構成しよう.すなわち$y_i\in B_i$で$\psi_i(y_i)=z_i$となるものがあるとする. $y_{i+1}\in B_{i+1}$で$\psi_{i+1}(y_{i+1})=y_i,g_{i+1}(y_{i+1})=z_{i+1}$となるものを作ればよい.
	
	$z_{i+1}$に対して$g_{i+1}(y_{i+1}')=z_{i+1}$となるものをとる.すると$g_i(y_i)=z_i=\omega_{i+1}(z_{i+1})=g_i(\psi_{i+1}(y_{i+1}'))$より$y_i-\psi_{i+1}(y_{i+1}')\in\im f_i$である.よって$y_i-\psi_{i+1}(y_{i+1}')=f_i(x_i)$となる$x_i\in A_i$がとれる.いま$\varphi_{i+1}:A_{i+1}\to A_i$は全射なので, $\varphi_{i+1}(x_{i+1})=x_i$となるものをとると, $y_i-\psi_{i+1}(y_{i+1}')=f_i(\varphi_{i+1}(x_{i+1}))=\psi_{i+1}(f_{i+1}(x_{i+1}))$となり, $y_i=\psi_{i+1}(y_{i+1}'+f_{i+1}(x_{i+1}))$である.ここで$y_{i+1}=y_{i+1}'+f_{i+1}(x_{i+1})$とおけば条件を満たす.
\end{proof} %圏
%	\input{ringApe3} 
	


	
	\addcontentsline{toc}{part}{索引}%
	\printindex
	
	\newpage
	\bibliographystyle{askw-jecon-no-sort} %%hyperref,natbib必須
	\addcontentsline{toc}{part}{参考文献} %bookのときは切る
	\bibliography{ring}
	
\end{document}
ref