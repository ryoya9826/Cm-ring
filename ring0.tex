\setcounter{part}{-1}
\part[Definition of Ring and more...]{環の基礎(かけあし)}
\section{環の定義}

環について,非常に圧縮してではあるが基礎事項を説明しておこう.本書では\textbf{環に可換性を仮定する}. 可換でない環は\textbf{非可換環}と呼ばれ,これは\textbf{作用素環}(operator algebra)などが有名で,こちらも活発な研究が行われている.
\begin{defi}[環]\index{かん@環}
	集合$R$に対し2つの演算$x+y,x*y$が備わっていて,次の性質を満たすとき,代数構造$(R,+,*)$を\textbf{環}(ring)という.
	\begin{defiterm}{R}
		\item $R$は$+$についてAbel群である.
		\item 任意の$x,y,z\in R$に対し,~$x*(y*z)=(x*y)*z$が成立する.
		\item ある$1\in R$が存在して,任意の$x\in R$に対し,~$x*1=1*x=x$が成立する.
		\item 任意の$x,y,z\in R$に対し,~$x*(y+z)=x*y+x*z,(x+y)*z=x*z+y*\z$が成立する.
	\end{defiterm}
\end{defi}

このとき,演算$+$を加法,~$*$を乗法という.ここでは乗法の可換性を仮定していないことに注意しよう.乗法が可換な環は\textbf{可換環}(commutative ring)と呼ばれ,本ノートの主題である.これ以後\textbf{環の乗法は可換である(すなわち$x*y=y*x$が常に成り立つ)とする.}\footnote{非可換環に対するイデアルの定義などは,必要なら適宜環の入門書を見てほしい.} 環の定義では$R$を記号として使用していたが,以後は可換環のみを取り扱うという気持ちで,またBourbakiに敬意を評して\footnote{7000ページ以上に及ぶ「数学原論」(\'el\'ements de math\'ematique)を出版し,数学の公理的基礎づけを行った.今使われている記号,概念の多く --- 例えば単射,全射 --- はBourbakiが導入したものである.},フランス語で環を表すAnneauの頭文字をとって$A$で表すことにする.

加法の単位元を$0$で表し\textbf{零元}(zero element), $1$を単に\textbf{単位元}という.$0,1,-x$(加法の逆元)は一意的に存在する.零元,単位元については実数などで馴染み深い次の性質がある.

\[\text{任意の }x\in A\text{ に対し }0x=0\text{ が成立する.}\]
\[\text{任意の }x\in A\text{ に対し }(-1)x=-x\text{ が成立する.}\]

$x\in A$が\textbf{単元}(unit)であるとは,ある$x^{-1}\in A$が存在して$x^{-1}x=1$が成立することをいう.~$x^{-1}$を$x$の(乗法)\textbf{逆元}(inverse element)という.単元のことを\textbf{可逆元}(invertible element)ともいう(意味からはこの名称のほうが直感的である). その意味で, $x$が単元であることを\textbf{可逆である}と表現することもある. $A$の単元全体は群になり, $A^\times$とかく.これを\textbf{単元群}(group of units)という.\index{たんげんぐん@単元群}\index{たんげん@単元} どれくらい$A$が単元を持つかというのは大切な問題であって,単元になりえない$0$以外がすべて単元である環を\textbf{体}という.

\begin{defi}[体]\index{たい@体}
	環$A$の0以外の元がすべて単元であるとき, $A$を\textbf{体}(field)という.
\end{defi}

体という日本語はドイツ語のK\uml{o}perに由来する.体はよく$K$でかかれる.体でない環の例として$\Z$, 体の例として$\Q,\R,\Co$を挙げておこう.

環の間の写像の中で代数構造と整合性があるもの,すなわち次の構造を持つものを重要視する.

\begin{defi}[環準同型]\index{かんじゅんどうけい@環準同型}
	$A,B$を環とし,~$f$を$f:A\to B$なる写像とする.以下の条件;
	\begin{defiterm}{RH}
		\item 任意の$x,y\in A$に対し,~$f(x+y)=f(x)+f(y)$が成立する.
		\item 任意の$x,y\in A$に対し,もし$x^{-1}\in A$が存在するならば$f(xy)=f(x)f(y)$が成立する.
		\item 単位元について$f(1)=1$が成立する.
	\end{defiterm}
	
	を満たすとき, $f$を\textbf{環準同型}(ring homomorphism)であるという.
\end{defi}

さらに$f$が全単射であるとき$A$と$B$は\textbf{同型}(isomophic)であるといい,~$A\cong B$で表す. 環準同型写像$f$について;

\begin{sakura}
	\item 零元について$f(0)=0$.
	\item 任意の$x\in A$に対し$f(-x)=-f(x)$である.
	\item 任意の$x\in A$に対し$f(x^{-1})=f(x)^{-1}$である.
\end{sakura}
が成り立つ.

$A$を環とし,集合として$A'\subset A$とする.~$A'$が$A$の単位元を含み,和,積,逆元をとる操作で閉じているとき, $A'$を$A$の\textbf{部分環}(sub ring)であるという.環準同型$f:A\to B$があるとき;
\[\im f=\mkset{f(x)}{x\in A}\]
は$B$の部分環になる.これを$f$の\textbf{像}(image)という.もう1つ部分集合に定義する構造で,部分環と同じくらい(それ以上?)に大切なものに\textbf{イデアル}がある.

\begin{defi}[イデアル]\index{イデアル}
	$A$の部分集合$I$が次の条件;
	\begin{defiterm}{I}
		\item $I$は空でない.
		\item 任意の$x,y\in I$に対し$x+y\in I$である.
		\item 任意の$x\in I$と$a\in A$に対し$ax\in I$である.
	\end{defiterm}
	をみたすとき,~$I$を$A$の\textbf{イデアル}(ideal)という.
\end{defi}

$\{0\},A$は明らかに$A$のイデアルをなす.~$\{0\}$を\textbf{零イデアル}(zero ideal, null ideal)といい,~$A$と零イデアルを$A$の\textbf{自明なイデアル}(trivial ideal)という.\index{じめいないである@自明なイデアル} $A$とことなるイデアルを\textbf{真のイデアル}という.環$A$が自明なイデアルしかイデアルを持たないことと, $A$が体であることは同値である(次に述べる性質から従う).

イデアルの簡単な性質をまとめておく.
\begin{sakura}
	\item $0\in I$である.
	\item $x\in I$ならば$-x\in I$である.
	\item $1\in I$であることと$I=A$であることは同値.特に$I$は可逆元を含むなら$I=A$である.
\end{sakura}

イデアルの中でも,次に与えられる\textbf{有限生成}なイデアルが扱いやすいだけでなく大切な役割を果たす.

\begin{defi}[有限生成イデアル]\index{ゆうげんせいせいいである@有限生成イデアル}
	$x_1,\dots,x_n\in A$について;
	\[(x_1,\dots,x_m)=\mkset{a_1x_1+\dots+a_nx_n}{a_i\in A}\]
	とおくと, $(x_1,\dots, x_n)$は$A$のイデアルになる.
\end{defi}

この記号のもとに自明なイデアルは$(0),(1)$とかかれる.特に1つの元$x$で生成されるイデアル$(x)$を\textbf{単項イデアル}(principal ideal)ないし\textbf{主イデアル}という(主イデアルは古い用語).有限生成イデアルを$Ax_1+\dots+Ax_n$とかくこともある.イデアル$I$について$a_1,\dots,a_n\in I$なら$(a_1,\dots,a_n)\subset I$である.また単項イデアル$(x),(y)$について,  $(x)\subset (y)$であることと,ある$a\in A$が存在して$x=ay$であることは同値.有限生成イデアル$(a_1,\dots,a_n)$について, $u$を$A$の単元とすると$(a_1,\dots,a_r)=(ua_1,\dots,ua_n)$が成り立つ.

いくつかのイデアルについて,和と積を次のように定義できる.
\begin{defi}[イデアルの和,積]
	$I,J$を$A$のイデアルとする.次の定義により,イデアルの和と積が定まる.
	\[I+J=\mkset{x+y}{x\in I,y\in J}\]
	\[IJ=\mkset{\sum_{\text{有限和}}x_iy_i}{x_i\in I,y_j\in I}\]
\end{defi}

また,イデアルの族$\{I_\lambda\}$について,共通部分$\bigcap I_\lambda$もまたイデアルとなる.集合としては;
\[IJ\subset I\cap J\subset I,J\subset I+J\]
となる. $I\cup J$はイデアルとは限らないことに注意しよう. $I+J=A$となるとき, $I,J$は\textbf{互いに素}(coprime)であるという.


\begin{exer}
	これまでに述べたことについて,環$\Z$において例を挙げてみよ.
\end{exer}

いままでの数学において,\textbf{多項式}(polymonial)は身近な対象であったと思う.環$A$の元を係数にもつ多項式全体を考えると,これは環になる.この環はただの例にとどまらず,環論全体において非常に大切な存在である.

\begin{defi}[多項式環]\index{たこうしきかん@多項式環}
	$A$を環とする. $A$の元を係数とする(変数$X$の)多項式;
	\[f(X)=a_nX^n+a_{n-1}X^{n-1}+\dots+a_0\quad(a_n\neq0)\]
	全体のなす集合$A[X]$は自然な演算によって環になる.これを$A$の(1変数\textbf{多項式環})という.
\end{defi}
誤解の恐れがない場合,変数はよく省略されて多項式は単に$f$とかかれる.多項式についての用語をいくつか定義しておこう. $a_i$を$f$の$i$次の\textbf{係数}(coeffieient)といい, $a_0$を\textbf{定数項}(constant term)という. $a_n\neq0$となる最大の$n$を$f$の\textbf{次数}(degree)といい, $\deg f$で表す.多項式$f$について,最高次の係数が1,すなわち$f(X)=X^n+a_{n-1}X^{n-1}+\cdots$となっているとき$f$は\textbf{モニック}(monic)であるという.\index{モニック} 次数については明らかに;
\[\deg (f+g)\leq\max\{(\deg f,\deg g\}\]
が成り立つ.

多変数多項式についても同様に考えることができ, 2変数多項式環$A[X,Y]$とは$A[X]$を係数とする変数$Y$の多項式環,すなわち$A[X,Y]=A[X][Y]$のことである.帰納的に$A[X_1,\dots,X_n]$は1変数多項式環をとる操作を繰り返して得られる.ここから,任意の$f\in A[\nitem{X}]$は単項式$X_1^{k_1}\dots X_n^{k_n} (k_i\geq0)$の$A$係数の線形和である(いままでの数学で考えてきた2変数多項式の全体は$A[X,Y]$とこの定義のもとで一致する).

多項式環の強力さはだんだん学習が進むにつれ身にしみてくることであろうから,次の定義に進もう.通常の計算で当たり前のように用いる$ab=0$なら$a=0$または$b=0$は,一般の環では\textbf{成り立たない.}

\begin{exer}
	その理由を考えよ($\Co$で$ab=0$なら$a=0$または$b=0$であることの証明を考えてみよ).
\end{exer}

一般の環については次のような定義ができ,いわゆる\quo{悪い元}の代表格である.

\begin{defi}[零因子]
	環$A$について,ある$x,y\in A$が存在して$xy=0$かつ$x,y\neq0$となるとき$x,y$を\textbf{零因子}(zero divisor)という.
\end{defi}

逆に,通常の計算のような零因子を持たない環には特別な名前がついている.

\begin{defi}[整域]\index{せいいき@整域}
	可換環$A$が零因子を持たないとき,\textbf{整域}(integral domain)であるという.
\end{defi}

明らかに体は整域である.また整域の部分環も整域である.よく知られた環$\Z$はもちろん整域であり, $\Q,\R,\Co$は体なのでもちろんそうである.

$A$が整域であるとき, $(x)=(y)$であることと, $A$の単元$a$が存在して$x=ay$となることが同値である.一般の環において, $x,y$に対して$A$の単元$a$
が存在して$x=ay$となることは同値関係となる.このことを$x,y$は\textbf{同伴}(associated)であるという.同伴関係と単項イデアルが一致することはすべての環で同値になるわけではないが,整域だけでなく局所環,半局所環で成り立つことが知られている(\ref{defi:半局所環},\ref{prop:半局所環と同伴関係}).

多項式環については次が成り立つ.

\begin{prop}
	$A$が整域なら, $f,g\in A[X]$について;
	\begin{sakura}
		\item $\deg(fg)=\deg f+\deg g$である.
		\item $A[X]$は整域である.
		\item $A[X]$の単元全体は$A^\times$である.
	\end{sakura}
\end{prop}

証明は難しくない. (iii)の整域でない環の反例については, 環$A$がある$n\in\Z_+$によって$a^n=0$となる元$a\in A$を持つ(このような$a$を\textbf{冪零}(nilpotent)であるという)とすると$A[X]$において$(1-aX)(a^{n-1}X^{n-1}+\dots+aX-1)=1$となる.

多項式環,整域といったこれらの概念をより彩らせるために,環をイデアルにより\quo{割る}ということが必要である.次章ではそれを説明することにしよう.

\section{剰余環}

環$A$とそのイデアル$I$を考える. $x,y\in A$に対し$x-y\in I$となることを$x\equiv y \pmod I$とかくと,これは同値関係になる.整数$\Z$における\textbf{合同式}を思い出そう. $n$を法とする合同式は,イデアル$n\Z$による同値関係と対応している.その意味でこれは合同式の一般化である.とすると,次の演算も自然だろう. $x\equiv x',y\equiv y'\pmod I$なら;
\[x+y\equiv x'+y'\pmod I,\quad xy\equiv x'y'\pmod I\]
が成り立つ.

これにより, $A$を$I$が定める同値関係で割った商集合$A/I$に次の演算を入れることができる.

\begin{defi}[剰余環]\index{じょうよかん@剰余環}
	$A/I$の元$\bar{x}$すなわち$x$の同値類を$x+I$と表す.演算を;
	\[(x+I)+(y+I)=(x+y)+I,\quad (x+I)(y+I)=xy+I\]
	と定義すると$A/I$は環になる.これを$A$の$I$による\textbf{剰余環}(residue ring)という.
\end{defi}

$A/(0)=A, A/A=0$であることを簡単に確かめられる.

剰余環を用いると,整域でない簡単な例を与えることができる. $\Z$と$n\in\Z$を考える.剰余環$\Z/n\Z$は, $n$が素数のときに限り整域になり,そうでないとき$\Z/n\Z$は非整域になる.例えば$\Z/4\Z$において$2*2=4=0$である.

\begin{exer}
	$\Z/5\Z,\Z/6\Z$の演算表をかけ.
\end{exer}

$p$が素数のとき$\Z/p\Z$は整域だけでなく体になる.この体を特に$\F_p$とかく.またに$\Z/n\Z$が体であることと$n$が素数であることは同値である.

$\Z/n\Z$の例からわかるように,剰余環$A/I$は$I$の元を0とみなしてできる環である.例えば$\Z/3\Z=\{0,1,2\}$であるが,これは一般の$n\in\Z$が$n=a+3b~(a=0,1,2, b\in\Z)$とかけるので, $\Z/3\Z$では$3b=0$となるからである.

これと同様の考えで,たとえば$\R[X]/(X^2+1)$を考えると, $\R[X]$のなかで$X^2=1=0$とみなす環なので$X^2=-1$すなわち$X$は$i$と同一視できる.よって標準的な同型$\R[X]/(X^2+1)\cong\Co$がある.

次に環準同型について着目していこう. $A,B$の間の環準同型$f:A\to B$を考える.集合;
\[\ker f=\mkset{x\in A}{f(x)=0}\]
を$f$の\textbf{核}(kernel)という. $\ker f$は$A$のイデアルをなすことが簡単に確かめられる.よって,剰余環$A\ker f$が定義できる.これが$\im f$と同型になるというのが\textbf{準同型定理}であり,息をするように用いる.

\begin{thm}[準同型定理]
	$A,B$を環とし,準同型$f:A\to B$を考えると,同型$A/\ker f\cong\im f$が存在する.
\end{thm}

\begin{proof}
	剰余環に誘導される準同型;
	\[f':A/\ker f\to\im f;x+\ker f\mapsto f(x)\]
	が環同型を与える.
	\begin{step}
		\item well-definedness.
		
		$x+\ker f=y+\ker f$とすると$x-y\in\ker f$であるので, $f(x-y)=f(x)-f(y)=0$すなわち$f(x)=f(y)$である.
		
		\item 単射性.
		
		$f'(x+\ker f)=0$とすると, $f(x)=0$すなわち$x\in\ker f$である.よって$x+\ker f=0$である.
		
		\item 全射性.
		
		任意の$y\in \im f$をとると,定義からある$x\in A$が存在して$f(x)=y$である.よって$f'(x+\ker f)=y$である.
	\end{step}
\end{proof}

先に挙げた$\R[X]/(X^2+1)\cong \Co$は,準同型$f:\R[X]\to\Co;f(X)\mapsto f(i)$の核が$(X^2+1)$であることから証明できる.

準同型については次の単射判定法が大切である.
\begin{prop}
	環準同型$f:A\to B$について, $f$が単射であることと$\ker f=0$であることが同値である.
\end{prop}

\begin{exer}
	上の命題を示せ.
\end{exer}

体の特徴づけについて,イデアルが自明なものしかない環ということを述べた.この命題から,体$K$からの準同型$f:K\to A$は必ず単射に限ることがわかる.

もともとの環と剰余環の間にはイデアルの対応関係がある.

\begin{prop}[環の対応定理]\label{prop:環の対応定理}
	$A$を環とし, $I$を$A$のイデアルとする.次の2つの集合;
	\[X=\mkset{J:A\text{ のイデアル}}{I\subset J},\quad Y=\{A/I\text{ のイデアル}\}\]
	の間には1対1の対応がある.
\end{prop}

\begin{proof}
	$\pi:A\to A/I;x\mapsto x+I$を自然な全射とする. $J$を$I$を含む$A$のイデアルとすると, $\pi(J)$は$A/I$のイデアルになることを確かめることができる.よって;
	\[\varphi:X\to Y;J\mapsto\pi(J)\]
	が全単射であることを示せば良い.
	\begin{step}
		\item 単射性.
		
		$\pi(J)=\pi(J')$であるとする.任意の$x\in J$をとる.このとき$\pi(x)\in\pi(J)=\pi(j')$であるので,ある$x'\in J'$が存在して$x-x'\in I$となる.すると$I\subset J'$であるので$x'\in J'$となる.ゆえに$J\subset J'$であり,逆も同様.
		
		\item 全射性.
		
		$A/I$のイデアル$\bar{J}$をとる. $J=\pi^{-1}(\bar{J})$とおく. $\pi^{-1}(0)=I$であるので$I\subset J$は明らか. $J$が$A$のイデアルをなすことも確かめられる. $\pi$が全射なので,これにより$\pi(J)=\bar{J}$となる.
	\end{step}
\end{proof}

特別なイデアルについて,次のような定義をする.

\begin{defi}[素イデアル]\index{そいである@素イデアル}
	環$A$の真のイデアル$P$について,任意の$x,y\in A$に対して$xy\in P$ならば$x\in P$または$y\in P$が成り立つとき$P$を\textbf{素イデアル}(prime ideal)という.
\end{defi}

環$A$の素イデアル全体を$\spec A$とかく.例えば$\spec \Z=\mkset{(p)}{p=0\text{ または }p\text{ は素数}}$である.素イデアルという言葉を用いると,環$A$が整域であることと$(0)$が素イデアルであることは同値である.

\begin{defi}[極大イデアル]\index{きょくだいいである@極大イデアル}
	環$A$の真のイデアル$\ideal{m}$について, $\ideal{m}\subset I\subset A$となる真のイデアル$I$は$\ideal{m}$に限るとき, $\ideal{m}$を\textbf{極大イデアル}(maximal ideal)という.
\end{defi}

環$A$の極大イデアル全体を$\specm A$とかく.

\begin{exer}
	$\specm \Z$を決定せよ.
\end{exer}

環の対応定理により,次の特徴づけが得られる.

\begin{prop}
	$A$を環とする.
	\begin{sakura}
		\item $P$が$A$の素イデアルであることと, $A/P$が整域であることは同値.
		\item $\ideal{m}$が$A$の極大イデアルであることと, $A/\ideal{m}$が体であることは同値.
	\end{sakura}
\end{prop}

\begin{exer}
	対応定理(\ref{prop:環の対応定理})を用いてこれを示せ.
\end{exer}

体は整域であるので,次の系が導かれる.

\begin{cor}
	極大イデアルは素イデアルである.
\end{cor}

環の対応定理は素イデアルに限っても成り立つ.環$A$のイデアル$I$について, $I$を含む素イデアル全体を$V(I)=\mkset{P\in\spec A}{I\subset P}$とかく.

\begin{prop}[素イデアルの対応定理]\label{prop:素イデアルの対応定理}
	$A$を環とし, $I$を$A$のイデアルとする. $V(I)$と$\spec A/I$の間には1対1対応がある.
\end{prop}
\begin{proof}
	対応定理と同様に自然な全射$\pi:A\to A/I$が対応を与える.
	\begin{step}
		\item $\pi(P)$が$A/I$の素イデアルであること.
		
		 $(x+I)(y+I)\in\pi(P)$とする.ある$z\in P$が存在して$xy-z\in I\subset P$であるので, $xy\in P$である. $P$は素イデアルだから$x\in P$または$y\in P$すなわち$x+I\in\pi(P)$または$y+I\in\pi(P)$である.
		
		\item $\pi^{-1}(\bar{P})$が素イデアルであること.
		
		$xy\in\pi^{-1}(\bar{P})$とすると, $\pi(xy)=\pi(x)\pi(y)\in\bar{P}$なので$\pi(x)\in \bar{P}$または$\pi(y)\in \bar{P}$である.すなわち$x\in\pi^{-1}(\bar{P})$または$y\in\pi^{-1}(\bar{P})$となる.
	\end{step}
\end{proof}

この証明と全く同様の方法で,一般の準同型$f:A\to B$について$P\in\spec B$に対して$f^{-1}(P)\in\spec A$を示すことができる.しかし,極大イデアルの引き戻しが極大とは限らない.例えば$\Z\hookrightarrow\Q$の$(0)$の引き戻しを考えよ.

選択公理(Zornの補題)を認めることで,すべてのイデアルに対してそれを含む極大イデアル(素イデアル)の存在を保証できる.

\begin{thm}[Krullの極大イデアル存在定理]\index{#Krullのきょくだいいであるそざ@Krullの極大イデアル存在定理}
	環$A$の真のイデアル$I$について, $I$を含む極大イデアル$\ideal{m}$が存在する.
\end{thm}

\begin{proof}
	次の集合;
	\[\Sigma=\mkset{J:A\text{ の真のイデアル}}{I\subset J}\]
	は$I$を含むので空ではない. $\Sigma$は包含関係で帰納的順序集合をなす.実際$X$を$\Sigma$の全順序部分集合とすると, $\ideal{m}_0=\bigcup_{\ideal{m}\in X}\ideal{m}$が$A$の真のイデアルをなし, $X$の極大元となる.よってZornの補題より$\Sigma$は極大元をもち,それが$I$を含む極大イデアルとなる.
\end{proof}

この定理により,環は少なくとも1つは極大イデアルを持つ.

\begin{defi}[局所環]\index{きょくしょかん@局所環}\index{はんきょくしょかん@半局所環}\label{defi:半局所環}
	環$A$が極大イデアル$\ideal{m}$をただ1つしかもたないとき, $A$を\textbf{局所環}(local ring)という
. 少し広く$A$が極大イデアルを有限個しか持たないとき$A$を\textbf{半局所環}(semi local ring)という.
\end{defi}

局所環は環論全体で重要であり,次の判定条件は有用である.

\begin{prop}\label{prop:local ring equiv}
	$(A,\ideal{m})$が局所環であることと, $A-A^\times$がイデアルであることは同値.特に$\ideal{m}=A-A^\times$となる.
\end{prop}

\begin{proof}
	\begin{eqv}
		\item $\ideal{m}_0=A-A^\times$とおく. $\ideal{m}\subset\ideal{m}_0$は明らかなので,逆を示す.任意の$x\in\ideal{m}_0$をとる.すると, $x\not\in A^\times$より$(x)\subsetneq A$である. $(x)$を含む極大イデアルが存在するが, $A$は局所環なのでそれは$\ideal{m}$に一致する.すなわち$(x)\subset\ideal{m}$となり, $x\in\ideal{m}_0$である.よって$\ideal{m}_0\subset\ideal{m}$である.よって$\ideal{m}_0=\ideal{m}$となる.
		\item 任意の真のイデアル$I$を考えると, $I\cap A^\times=\emptyset$より$I\subset A-A^\times$となり, $A-A^\times$が唯一の極大イデアルである. 
	\end{eqv}
\end{proof}

次に,環の\textbf{標数}と呼ばれる概念について説明しよう.一般の環についても,元$x\in A$の整数倍を$x$を何回足したかどうかによって定義することができる.例えば$3x=x+x+x,-5x=-(x+x+x+x+x)$である.これを用いて次の定義を行う.

\begin{defi}[標数]\index{ひょうすう@標数}
	$A$を環とする.単位元$1$について, $n1=0$となる$n\in\Z_+$が存在するとき, そのような$n$で最小のものを$A$の\textbf{標数}(characteristic number)であるといい, $\Char(A)=n$とかく.そのような$n$が存在しない場合$\Char(A)=0$という.
\end{defi}

例えば$\Char \Z=0,\Char(\Z/n\Z)=n$である.

整域については次のような結果がある.

\begin{prop}\label{prop:整域の標数}
	$A$を整域とすると, $A$の標数は$0$か素数$p$に限る.
\end{prop}

\begin{proof}
	次の準同型;
	\[f:\Z\to A;n\mapsto n1\]
	を考える.準同型定理から$\Z/\ker f$は$A$の部分環$\im f$に同型であり, $A$は整域なので$\Z/\ker f$は整域.よって$\ker f$は$\Z$の素イデアルなので$(0)$か$(p)$でなければならない.標数は$\ker f$の生成元にほかならないので, $0$または素数$p$である.
\end{proof}

任意の環$A$について, $\Z$からの準同型は$n\mapsto n1$に限ることに注意しよう.

先にも少し述べたが,零因子のなかで自分自身のいくつかの積が0になってしまうようなものを\textbf{冪零元}という.

\begin{defi}[被約環]\index{ひやくかん@被約環}
	環$A$が$0$以外の冪零元をもたないとき, $A$を\textbf{被約環}(reduced ring)という.
\end{defi}

整域は明らかに被約である.一般に環$A$の冪零元全体を$\nil A$とおくと,これはイデアルとなる(2項定理を用いる).

\begin{defi}[冪零根基]\index{べきれいこんき@冪零根基}
	環$A$の冪零元全体がなすイデアル$\nil A$を, $A$の\textbf{冪零根基}(nilradical)という.
\end{defi}

環$A$について$A/\nil A$は必ず被約になる.ここで一般的なイデアルの根基について説明する.

\begin{defi}[根基]\index{こんき@根基}
	環$A$のイデアル$I$について;
	\[\sqrt{I}=\mkset{x\in A}{\text{ ある }n\text{ が存在して, }x^n=I}\]
	はイデアルをなす.これをイデアル$I$の\textbf{根基}(radical)という.
\end{defi}

この記法により$\nil (A)=\sqrt{(0)}$とかけることに注意しよう.環$A$の素イデアル全体を$\spec A$, イデアル$I$を含む素イデアル全体を$V(I)$と書くことを思い出そう.一般に次が成り立つ.

\begin{prop}
	環$A$のイデアル$I$について, $\sqrt{I}=\bigcap_{P\in V(I)}P$が成り立つ.
\end{prop}

\begin{cor}
	環$A$について,冪零根基$\nil (A)$は$A$のすべての素イデアルの共通部分である.
\end{cor}

この命題の証明は積閉集合と局所化という考え方を用いた簡明なものがある(\ref{lem:イデアルの根基と素イデアル})ので、そちらを見よ(ここまでの道具で証明する方法もあるが,選択公理(Zornの補題)を用いるスマートでない証明しか筆者は知らない).

冪零根基が素イデアルの共通部分であるので,極大イデアルの共通部分についても考えてみよう.
\begin{defi}[Jacobson根基]\index{#Jacobsonこんき@Jacobson根基}
	環$A$のすべての極大イデアルの共通部分をJacobson\textbf{根基}(Jacobson radical)といい,~$\rad(A)$で表す.
\end{defi}

すなわち$\rad(A)=\bigcap_{\ideal{m}\in\specm A}\ideal{m}$である.極大イデアルは素イデアルであるから,~$\nil(A)\subset\rad(A)$となっている.~Jacobson根基については次の特徴づけがある.
\begin{prop}
	任意の$y\in A$に対し,~$1-xy$が可逆となるような$x$全体の集合を$\mathfrak{A}$とすると,これは$\rad(A)$と一致する.
\end{prop}
\begin{proof}
	\begin{mrkw}
		\item 
		任意の$x\in\mathfrak{A}$を1つとる.背理法で示す.ある極大イデアル$\ideal{m}$が存在して$x\not\in\ideal{m}$であると仮定しよう.すると$\ideal{m}+Ax=A$であるから,ある$m\in\ideal{m}$と$y\in A$とが存在して$m+xy=1$とできる.すると$1-xy=m$となるが,これは可逆でない.よって$x\in\mathfrak{A}$に矛盾する.よって$x$は任意の極大イデアルに含まれるので,~$x\in\rad(A)$である.
		\item 
		任意の$x\in\rad(A)$を1つとる.背理法で示す.ある$y\in A$が存在して,~$1-xy$が非可逆であると仮定しよう.ここで$1-xy\in\ideal{m}$となる極大イデアル$\ideal{m}$が存在する.一方$x\in\rad(A)\subset\ideal{m}$より$xy\in\ideal{m}$である.よって$(1-xy)+xy=1\in\ideal{m}$となり,矛盾する.
	\end{mrkw}
\end{proof}

\section{ED,PID,UFD}
整数環$\Z$の持つ性質を一般の環に抽象化することで, ED,PID,UFDなどの良い性質が得られる.本節ではそれについて説明する.

\begin{defi}[PID]\index{#PID@PID}
	環$A$のすべてのイデアルが単項イデアルであるとき, $A$を\textbf{主環}(principal ring)という.特に$A$が整域のとき,\textbf{単項イデアル整域}(principal ideal domain)という.頭文字をとってPIDと略する.
\end{defi}

$\Z$はPIDである.実際,任意のイデアル$I$について, $n\in I$を$I$の元でもっとも絶対値が小さいものとする.イデアルの定義から$n\in\N$としてよい.このとき,除法の原理から任意の$x\in I$は$x=qn+r~(q\in\Z,0\leq r<n)$と一意にかける.すると$r=x-qn\in I$であり, $n$の最小性から$r=0$となり, $m\in n\Z$である.よって$I=n\Z$である.この証明で本質的なのは\textbf{除法の原理}である.これを一般化することでPIDの判定条件を1つ与えよう.

\begin{defi}[Euclid整域]\index{#Euclid せいいき@Euclid整域,ED}
	$A$を整域とする.写像$\rho:A\to\N$で,次の条件;
	\begin{defiterm}{EF}
		\item 任意の$0$でない$x\in A$について, $\rho(0)<\rho(x)$が成り立つ.
		\item 任意の$x,y\in A$について, $x\neq0$のときある$q,r\in A$が存在して;
		\[y=xq+r,\quad \rho(r)<\rho(x)\]
		となる.
	\end{defiterm}
	を満たすものが存在するとき, $\rho$をEulcid\textbf{関数}(Euclid function), $A$をEuclid\textbf{整域}, ED(Euclid domain)という.
\end{defi}

除法の原理によって,絶対値を与える関数$\Z\to\N$がEuclid関数となり$\Z$はEDである.
\begin{exer}
	体$K$上の多項式環$K[X]$は,次数を与える写像によってEDとなることを示せ.	
\end{exer}

先に述べた通り, EDならばPIDである.
\begin{thm}
	EDはPIDである.
\end{thm}
\begin{proof}
	$A$をEDとし, $\rho:A\to\N$をEuclid関数とする.任意のイデアル$I$をとる. $I-\{0\}$の$\rho$による像;
	\[I'=\mkset{\rho(x)}{0\neq x\in I}\]
	を考える. $I'\subset\N$であり, $\N$は整列集合なので最小元$m'=\min I'$が存在する. $m'\in I'$より,ある$0\neq m\in I$が存在して$\rho(m)=m'$である. $I=Am$を示そう.任意の$x\in I$をとる. $A$はEDなので, $m\neq0$だからある$q,r\in A$が存在して;
	\[x=qm+r,\quad \rho(r)<\rho(m)=m'\]
	とかける.ここで$r=x-qm\in I$であり, $\rho(r)<m'$であるから$\rho(r)\not\in I'$である.ゆえに$r=0$でなければならない.よって$x\in Am$である.ゆえに$I=Am$が示された.
\end{proof}

よって$K[X]$がPIDであることがわかる. PIDは1変数特有の現象で,2変数以上ではPIDにならない(例えば$(X,Y)$が単項ではない).

PIDの素イデアルについては次の命題が知られている.
\begin{prop}
	$A$をPIPとする. $P$を$A$の素イデアルとすると, $P$は$(0)$または極大イデアルである.
\end{prop}
\begin{proof}
	$P=(p)$とする. $p=0$なら示すことはない. $p\neq0$としよう.もし$(p)\subset (x)$となる$x\in A$が存在したとする. $p=ax$とかける.すると$ax\in (p)$より$a\in (p)$または$x\in (p)$である.  $x\in (p)$なら$(x)=(p)$である.また$a\in (p)$とすると, $a=a'p$となる$a'\in A$が存在する.よって$p=aa'p$となり, $p\neq0$より$aa'=1$である.よって$p$と$x$は同伴であり$(p)=(x)$である.以上より$(p)$は極大である.
\end{proof}

PIDは$\Z$における最大公約数の議論を行えるように抽象化した環になっている.もちろん一般の環でも約数,倍数といった概念は定義できるが, PIDでは嬉しい性質があることを見ていこう. 

\begin{defi}[約元,倍元]
	$A$を環とする. $x,y\in A$について,ある$a\in A$が存在して$x=ya$となるとき$x$は$y$の\textbf{約元}(divisor), $y$は$x$の\textbf{倍元}(multiple)であるという.
\end{defi}

$z\in A$が$x,y$の共通の約元であるとき $z$は$x,y$の\textbf{公約元}(common divisor)であるといい, $g$が$x,y$の公約元であって,任意の$x,y$の公約元の倍元であるとき$g$を$x,y$の\textbf{最大公約元}(greatest common divisor)であるという. $\Z$ですら最大公約元は一意に決まらない(例えば$12$と$18$の最大公約数は$6,-6$)し,一般の整域では存在するかもわからない.最大公約元が必ず存在する整域をGCD整域という(\ref{defi:GCDdomain}).

\begin{prop}
	$A$がPIDであるとき,任意の$x,y\in A$について最大公約元$g$が少なくとも1つ存在する.
\end{prop}

\begin{proof}
	イデアル$(x,y)$を考えると, $A$はPIDなので$(x,y)=(g)$となる$g\in A$が存在する.この$g$が$x,y$の最大公約元になる.公約元になることは$x,y\in(g)$からわかる.次に$z$を$x,y$の公約元とすると,定義から$x,y\in (z)$なので$(g)=(x,y)\subset (z)$である.ゆえに$z$は$g$の約元である.
\end{proof}

このアイデアから次の強力な定理を証明できる.
\begin{thm}
	$A$をPIDとし, $a,b\in A$を0でない元とする. $a,b$の最大公約元を$g$とすると,2変数1次方程式;
	\[ax+by=g\]
	の解$(x,y)$が$A$の中に見つかる.
\end{thm}

\begin{proof}
	$g,g'$が$a,b$の最大公約元であるとする.このとき$g,g'$はともに0でなく,お互いの約元である.よって$u,t\in A$が存在して$g=g't,g'=gu$とかける.すると$g(1-ut)=0$であるが, $A$が整域なので$g\neq0$より$ut=1$である.ゆえに$g,g'$は互いの単元倍である.すると, $(a,b)=(d)$となる$d\in A$をとると先の命題より$d$は$a,b$の最大公約元であって, $g$の単元倍であるので$(a,b)=(d)=(g)$が従う.よって$g\in (a,b)$であるから主張が従う.
\end{proof}

約数の一般化として約元を定義したのと同様に,素数の一般化を考えてみよう.

\begin{defi}[既約元]\index{きやくげん@既約元}\label{defi:既約元}
	$A$を環とし, $x\in A$を単元でないとする.ここで$a,b\in A$を用いて$x=ab$とかけるとき,必ず$a$または$b$が単元であるとき, $x$を\textbf{既約}(irreducible)であるという.
\end{defi}

$\Z$の単元は$1,-1$のみであるので, $\Z$の既約元はまさに素数(の単元倍)である. $\Z$においては,単元倍を除いて一意的に素因数分解できる.一般の整域では既約元分解は必ずできるが,一意性は成り立つとは限らない.例えば$\Z[X]$に$\sqrt{-5}$を代入して得られる($\Z[X]/(X^2+5)$と表現してもよい);
\[\Z[\sqrt{-5}]=\mkset{a+b\sqrt{-5}}{a,b\in\Z}\]
を考えると,次の等式;
\[6=2\cdot 3=(1-\sqrt{-5})(1+\sqrt{-5})\]
が成り立つ.また$2,3,1-\sqrt{-5},1+\sqrt{-5}$が$\Z[\sqrt{-5}]$の既約元であることを確かめることができる.

このようなことが起こりえない,すなわち任意の元を一意的に既約元に分解できる環をUFDという.

\begin{defi}[UFD]\index{#UFD@UFD}
	整域$A$の$0$と単元以外のすべての元が既約元に(単元倍を除いて)一意的に分解できるとき, $A$を\textbf{一意分解整域}, UFD(unique factorization domain)という.
\end{defi}

UFDの例を与えるために, UFDの性質をいくつか見ていく必要がある.まずUFDでは$f$が既約元であるかどうかを次のように判定できることを示そう.
\begin{prop}
	$A$をUFDとする.このとき$f\neq0$が既約元であることと$(f)$が素イデアルであることは同値である.
\end{prop}
\begin{proof}
	\begin{eqv}
		\item $xy\in(f)$とする.定義よりある$a\in A$が存在して$xy=af$となる.ここで$x,y,a$を既約分解して$x=ux_1\dots x_n,y=ty_1\dots y_m,a=sa_1\dots a_l~(u,t,s$は$A$の単元)としよう.このとき;
		\[utx_1\dots x_ny_1\dots y_m=sa_1\dots a_lf\]
		において,両辺はともに$xy$(ないし$af$)の既約分解になっているので, $f$は$\nitem{x},\nitem[m]{y}$のどれかの単元倍である.よって$x\in(f)$または$y\in(f)$が成り立つ.
		
		\item $(f)$が素イデアルとすると,もし$xy=f$とかけたなら$x\in (f)$または$y\in(f)$である. $x\in (f)$であるとすると$x=af$とかけるので, $f(1-ay)=0$である. $f\neq0$より$1=ay$であり, $y$は単元である. $y\in (f)$なら同様に$x$が単元となるので, $f$は既約元である.
	\end{eqv}
\end{proof}

$(\Longleftarrow)$は一般の整域でも成り立つ.整域において, $(f)$が素イデアルとなるような$f$のことを\textbf{素元}(prime element)という.実は次の命題が成り立つ.

\begin{prop}
	$A$を整域とする. $x\in A$が素元の2通りの積;
	\[x=up_1\dots p_n=tp_1'\dots p_m'~(u,t\in A^\times)\]
	にかけたとすると, $n=m$であり適当に並べ替えると$(p_i)=(p_i')$が必ず成立する($p_i'$は$p_i$の単元倍である).
\end{prop}

\begin{proof}
	$p_1'\dots p_m'\in (p_1)$であるので, $p_1'\in (p_1)$または$p_2'\dots p_m'\in (p_1)$である.これを続けることで,ある$p_i'$について$p_i'\in (p_1)$とできる.それを並び替えて$p_1'$としよう.ここで$p_1$も$p_1'$も既約であるから,単元$u\in A$が存在して$p_1'=p_1u$である.すなわち$p_2p_3\dots p_n=up_2'\dots p_m'$である.同様の操作を繰り返すことで$n=m$であり, $p_i'$は$p_i$の単元倍すなわち$(p_i')=(p_i)$であることを得る.
\end{proof}

一般に素元は既約元であったから,この命題より素元に分解できることを確かめさえすればよいということになる.これを用いて次を示そう.
\begin{thm}
	PIDはUFDである.
\end{thm}

\begin{proof}
	$A$をPIDとし, $x\neq0$を単元でないとする.このとき真のイデアル$(x)$を含む極大イデアル$\ideal{m}$が存在する(Krullの極大イデアル存在定理を使ってもよいが, PIDはNoether環と呼ばれる環(\ref{defi:Noether環})であることに注意すると,選択公理は不要である). $A$はPIDなので, $\ideal{m}=(p_1)$となる$p_1\in A$が存在する. $x\in(p_1)$であるから$x=a_1p_1$となる$a_1\in A$が存在する. $a_1$が単元ならば$x$は素元であり証明が終了する. $a_1$が単元でないならば,同様の操作により$x=a_2p_1p_2$となる素元$p_2$が見つかる.この操作が有限回で終了することを示そう.この操作が無限回続いたとすると,無限に続く非単元の族$\{a_i\}$で, $(a_i)\subsetneq (a_{i+1})$となるものがとれる.すると;
	\[I=\bigcup_{i=1}^\infty (a_i)\]
	もまた$A$のイデアルになる.すると$A$はPIDなので$I=(a_\infty)$となる$a_\infty\in A$が存在する.構成より$a_\infty\in (a_n)$となる$n$が存在するが,これは$I=(a_n)=(a_{n+1})=\cdots$を導くので矛盾.よってこの操作は有限回で終了する.
\end{proof}

操作が有限回で終了する,というところは本質的にはすべてのイデアルが有限生成である,ということから来ている(Noether性について勉強したときにわかるだろう).

これにより$\Z$や体係数の多項式環$K[X]$はUFDである. EDもあわせて考えると,次のような流れになっている.

\[\text{体}\Longrightarrow\text{ED}\Longrightarrow\text{PID}\Longrightarrow\text{UFD}\Longrightarrow\text{整域}\Longrightarrow\text{環}\]

体上の多変数多項式はPIDではなかったが, UFDであることを示すことができる(\ref{cor:多変数もUFD}). これはUFDだがPIDでない例を与える.

\section{体の拡大}

この節では,体の拡大(例えば$\Q\to\R\to\Co$)について,\textbf{代数的数}や\textbf{超越数}などに触れながら概説しよう.

体$L$の部分環$K$がまた体になっているとき, $K$を$L$の\textbf{部分体}(sub field)であるという.このとき,自然な単射$K\hookrightarrow L$を体の\textbf{拡大}(extension)という.このとき$L/K$とかく.

整域$A$があったとき,ちょうど整数$\Z$から有理数$\Q$を作ることと全く同じように,体を作る自然な方法がある.直積$A\times(A-\{0\})$に次のような同値関係を入れよう;
\[(a,b)\sim(c,d)\Longleftrightarrow ad-bc=0\]
$(a,b)$を$a/b,(c,d)$を$c/d$をみなせば, $a/b=c/d$は$ad=bc$と同値だから,この定義は整数における分数の定義を一般の整域に拡張したものとなる.直積$A\times(A-\{0\})$をこの同値関係で割った商集合を$\Frac (A)$とかく(fraction). $\Frac (A)$は分数と同様の演算で体になる.

\begin{defi}[商体]\index{しょうたい@商体}
	整域$A$について, $\Frac (A)$の元$(x,s)$を$x/s$とかき;
	\[\frac{x}{s}+\frac{y}{t}=\frac{xt+ys}{st},\quad \frac{x}{s}\cdot\frac{y}{t}=\frac{xy}{st}\]
	と定めると体となる.これを$A$の\textbf{商体}(field of quotients), \textbf{分数体}(field of fractions)という.
\end{defi}

次の命題の証明からもわかることだが,整域$A$の商体$\Frac (A)$は$A$を含む最小の体となる(ある体$K$について$A\subset K$ならば$\Frac (A)\subset K$となる).
\begin{prop}
	$K$を体とする. $K$の標数が0であることと, $K$が$\Q$(と同型な体)を部分体として含むことは同値. $K$の標数が$p\neq0$であることは$K$が$\F_p$を部分体として含むことと同値.
\end{prop}

\begin{proof}
	\begin{mrkw}
		\item $\Char K=0$について.
		
		\begin{eqv}
			\item $\Z$からの自然な準同型$\varphi$の核が$0$であるので, $K$は$\Z$と同型な環$\im\varphi$を部分環にもつ.このとき$\Frac(\im\varphi)\cong\Q$である.さて任意の$a,b\in\im\varphi$について$b\neq0$とすると, $b\in K^\times$である.ゆえに$ab^{-1}\in K$となるので, $a/b\in\Frac(\im\varphi)$を$ab^{-1}$と同一視して, $\Frac(\im\varphi)\cong\Q$は$K$の部分体である.
			
			\item $\varphi:\Z\to K;n\to n1$は$\Q$を経由する準同型に分解する.すなわち$\psi:\Z\to\Q$(単射)と自然な単射$\iota:\Q\to K$を考えると, 2つの準同型$\varphi,\iota\circ\psi:\Z\to K$が定まる.しかし, $\Z$からの準同型は$\varphi$に限るので, $\varphi=\iota\circ\psi$となる.よって$\varphi$は単射ある.
		\end{eqv}
	
		\item $\Char K=p$なら$\Z/\ker\varphi=\Z/p\Z=\F_p$が$K$の部分環になる.また, $\F_p$が部分環ならば$\psi:\Z\to\F_p;\iota:\F_p\to K$の合成$\iota\circ\psi:\Z\to K$が定まる. $K$が$\Q$を部分体として含む場合と同様にして, $\iota\circ\psi:\Z\to K$の核が$(p)$であるから$\Char K=p$である.
	\end{mrkw}
\end{proof}

この命題より,すべての体は$\Q, \F_p$のいずれかを含む.

\begin{defi}[素体]\index{そたい@素体}
	体$\Q,\F_p$をそれぞれ標数$0,p$の\textbf{素体}(prime field)という.
\end{defi}

ここから体の拡大について見ていく.例えば体$\Co$について,拡大$\Co/\Q$は$\Co/\R,\R/\Q$に分解する.このように体の拡大$L/K$について,ある体$M$が存在して$L/M,M/K$が体の拡大になっているとき, $M$を拡大$L/K$の\textbf{中間体}(intermediate field)という.\index{ちゅうかんたい@中間体} 一般に$M_1,M_2$が$L/K$の中間体なら体$M_1\cap M_2$も$L/K$の中間体になる.拡大$\Co/\Q$の中間体は$\R$以外にも$\Q(\sqrt{2})=\mkset{a+b\sqrt{2}}{a,b\in\Q}$などがある(これが体であることを示せ).

E. Artin により導入された次の捉え方により,線形代数的な手法で体の拡大について考えることができるようになった.

\begin{prop}
	体の拡大$L/K$について, $L$は$K$上のベクトル空間とみなすことができる.
\end{prop}

\begin{proof}
	$L$はそれ自身の演算によりAbel群である.また,任意の$x\in L$について, $K$の元$y$によるスカラー倍を, $L$の情報によって$xy$と考えることで$L$は$K$ベクトル空間になる.
\end{proof}

例えば,拡大$\Co/\R$について$\Co$は基底$\{1,i\}$をもつ$2$次元の$\R$ベクトル空間になる.

\begin{defi}[拡大次数]\index{かくだいじすう@拡大次数}
	体の拡大$L/K$について, $\dim_K L$を$L/K$の\textbf{拡大次数}(extension degree)といい, $[L:K]$とかく.
\end{defi}

$[L:K]$が有限のとき$L/K$を有限次拡大,無限次拡大という.有限次拡大$L/K$の中間体$M$について, $L/M,M/K$も有限次拡大になる.ここから次の結果が従う(証明は線形代数なので省略する).

\begin{prop}
	$L/K$を有限次拡大とし, $M$をその中間体とする.このとき$[L:K]=[L:M][M:K]$となる.
\end{prop}

次に,体の拡大について\quo{代数的}とはどういうことであるかについて考察する.まず,体に元を\textbf{添加}することについて述べよう.体の拡大$L/K$について, $\alpha\in L$に対して;
\[K(\alpha)=\mkset{\frac{f(\alpha)}{g(\alpha)}\in L}{f,g\in K[X],g(\alpha)\neq0}\]
とおく.これを$K$に$\alpha$を\textbf{添加}(adjunction)した体という. $K(\alpha)$は$L/K$の中間体となり,また$M$が$L/K$の中間体で$\alpha\in M$を満たすなら$K(\alpha)\subset M$となる.さらに$\beta\in L$を$K(\alpha)$に添加した体$(K(\alpha))(\beta)$を$K(\alpha,\beta)$とかく.

一般に$K$に異なる元$\alpha,\beta\in L$を添加したとしても$K(\alpha)=K(\beta)$となることがあるので注意が必要である.例えば$i,1+i$を$\Q$に添加した体を考えてみよ($\Q(i)=\mkset{a+bi}{a,b\in\Q}$である).

いくつか例を見てみよう. $\R(i)=\Co$であり$[\Co:\R]=2$である(基底は$\{1,i\}$). また;
\[\Q(\sqrt{2},\sqrt{3})=\mkset{a+b\sqrt{3}}{a,b\in\Q(\sqrt{2})}=\mkset{x+y\sqrt{2}+z\sqrt{3}+w\sqrt{6}}{x,y,z,w\in\Q}\]
であり$[\Q(\sqrt{2},\sqrt{3}):\Q]=4$である(一次独立を確かめよ).

既約元の定義(\ref{defi:既約元})を思い出そう.体$K$上の定数でない多項式$f\in K[X]$で, $f=gh (g,h\in K[X])$なら$g,h$のどちらかは定数であるようなもの(定数でない多項式の積に分解できないもの)を\textbf{既約多項式}(irreducible polymonial)という.例えば$\R[X]$において$X^2+1$は既約であり, $\Co[X]$では$X^2+1=(X-i)(X+i)$となり既約ではない.このように体を拡大すると多項式が既約でなくなることがある.逆に,ある既約多項式を分解できるように体を拡大することを考えよう. $\Co/\R$の例をみると, $\Co=\R(i)$となっている.これは$\R$に$X^2+1$の\quo{根}を添加するように拡大したと考えることができる.しかし,定義を見ればわかるように根を添加するには, $\R$の拡大体$L$で$f(\alpha)=0$となる$\alpha\in L$が見つかっていなければならない(この例では$\Co$という$X^2+1$の根の居場所がわかっているので問題がない).  そこで$\Co\cong\R[X]/(X^2+1)$であったことを思い出そう.ここから類推できるのは$K[X]$を既約元で割ることで,根を添加したような拡大体が作れないか?ということである.

\begin{prop}
	体$K$上の多項式環$K[X]$について, $f$が既約多項式であることと$K[X]/(f)$が体であることは同値.
\end{prop}

\begin{proof}
	$f$が既約であることと$(f)$が極大イデアルであることが同値なことを示せばよい.
	
	\begin{eqv}
		\item $(f)\subset I$となるイデアル$I$があるとする. $K[X]$はPIDなので, $I=(g)$となる$g\in K[X]$がある.すると$f=hg (h\in K[X])$とかけるが, $f$が既約なので$g,h$のどちらかは定数である.もし$g$が定数なら$(g)=K[X]$であり, $h$が定数なら$f,g$は同伴なので$(f)=(g)$となる.よって$(f)$は極大である.
		\item $f=gh$としよう. $g$が定数でないとすると, $(f)\subset(g)\neq K[X]$であるので$(f)=(g)$となり$f$と$g$は同伴.よって$h$は定数であり$f$は既約.
	\end{eqv}
\end{proof}

よって$f\in K[X]$が既約なら$L=K[X]/(f)$は$K$の拡大体となる.ここで;
\[f(X)=a_nX^n+a_{n-1}X^{n-1}+\dots+a_0\]
とおいておく. $g$の$L$における像を$[g]$とかくことにすると, $[g]=0$であることと$g\in (f)$であることは同値である.すると$L$において;
\[a_n[x]^n+a_{n-1}[x]^{n-1}+\dots+a_0=[a_nX^n+\dots+a_0]=[f]=0\]
であるので, $f$を$L$上の多項式とみなしたとき, $[x]=\alpha$とおくと$f(\alpha)=0$が言えることになる($\R[X]/(X^2+1)\cong\Co$で考えてみよ. $i$は$[X]$に対応する元であった). $L=K[X]/(f)=K(\alpha)$が成り立つことを示そう.

\begin{prop}\label{prop:既約多項式の商による拡大}
	$f\in K[X]$が$K$上既約であり, $\deg f=n$とする. $L=K[X]/(f),\alpha=[X]\in L$とおくと, $[L:K]=n$であり,その基底は$\{1,\alpha,\dots,\alpha^{n-1}\}$である.
\end{prop}

\begin{proof}
	$\{1,\alpha,\dots,\alpha^{n-1}\}$が一次独立であることを示す. $\nitem<0>[n-1]{a}\in K$をとり$a_0+a_1\alpha+\dots+a_{n-1}\alpha_{n-1}=0$と仮定する.これは$g=a_{n-1}X^{n-1}+\dots+a_0$とおくと$g\in (f)$を意味する.よって$g$は$f$の定数倍または定数だが, $\deg g<\deg f$なので$g$は定数でなければならない.よって$a_1=\dots=a_{n-1}=0$となる.すると$a_i$たちの定義から$a_0=0$でなければならず,一次独立であることが従う.
\end{proof}

この系として$L= K(\alpha)$が従う. $f(\alpha)=0$となる$\alpha$を添加することで$K$から拡大体$L$を作ることを見たので,次は$\alpha\in L$について$f(\alpha)=0$となる$f\in K[X]$があるかどうかを考えよう.

\begin{defi}[代数的]\index{だいすうてき@代数的}
	$L/K$を体の拡大とする. $\alpha\in L$で,ある$f\in K[X]$が存在して$f(\alpha)=0$となるとき$\alpha$は$K$上\textbf{代数的}(algebraic)であるという.そのような$f$が存在しないときは$\alpha$は$K$上\textbf{超越的}(transcendental)であるという.
\end{defi}

拡大$\Co/\Q$における代数的な元を\textbf{代数的数},超越的な元を\textbf{超越数}という.有名な超越数では$e,\pi,\pi+e^\pi$がある($e+\pi$が超越的かは未解決).一方で$\sqrt[n]{2}$や$i$などは代数的数である.

\begin{defi}[最小多項式]\index{さいしょうたこうしき@最小多項式}
	$\alpha$を$K$上代数的とする.このとき$f\in K[X]$であって, $f(\alpha)=0$かつ$f$はモニックなものが存在し,かつ次数が最小なものは一意的である.これを$\alpha$の$K$上の\textbf{最小多項式}(minimal polymonial)といい, $F_\alpha$で表す.
\end{defi}

存在し,一意なことを示そう.
\begin{proof}
	\begin{step}
		\item 存在すること.
		
		$\alpha$は代数的なので,ある$f\in K[X]$がとれて$f(\alpha)=0$となる. $f$の最高次の係数が$a_n$であるとすると$1/a_n\cdot f$は$\alpha$を根にもつ$K$上のモニック多項式であるので,存在することがわかる.
		
		\item 次数最小のものが一意的であること.
		
		$f\neq g$が条件を満たすとする.定義から$\deg f=\deg g$で,どちらもモニックなので$f-g\neq0 $は$f$より次数が真に小さくなる.明らかに$f(\alpha)-g(\alpha)=0$なので,これを最高次の係数で割ったものも$\alpha$を根にもつモニック多項式だが,これは次数の最小性に矛盾.
	\end{step}
\end{proof}

体(整域)上の最小多項式は次数の最小性から既約である.また, $f\in K[X]$について$f(\alpha)=0$であることと$f$が$F_\alpha$で割り切れることは同値であり,更に$f$がモニックで既約なら$f=F_\alpha$であることが簡単にわかる(演習問題とする).

体の拡大$L/K$において$\alpha,\beta\in L$はどちらも$K$上代数的であるとする. $F_\alpha=F_\beta$となっているとき, $\alpha$と$\beta$は$K$上\textbf{共役}(conjugate)\index{きょうえき@共役}であるという.この定義は複素数の共役の拡張になっていることを確かめよ($\Co/\R$を考えよ).


\begin{defi}[代数拡大,超越拡大]\index{だいすうかくだい@代数拡大}\index{ちょうえつかくだい@超越拡大}
	$L/K$を体の拡大とする.任意の$\alpha\in L$に対し$\alpha$が$K$上代数的であるとき$L/K$は\textbf{代数拡大}(algebraic extension)といい,そうでないとき\textbf{超越拡大}(transcendental extension)という.
\end{defi}

\begin{prop}
	$L/K$を体の拡大とする. $\alpha\in L$が$K$上代数的であることと, $K(\alpha)/K$が有限次であることは同値.
\end{prop}

\begin{proof}
	\begin{eqv}
		\item $\alpha$の最小多項式$F_\alpha$は既約である. $L=K[X]/(F_\alpha)$は体であり, \ref{prop:既約多項式の商による拡大}より$L/K$は有限次拡大,また$L=K(\alpha)$であるので題意が従う.
		
		\item $[K(\alpha):K]=n$とすると, $\{1,\alpha,\dots,\alpha^n\}$は一次独立ではない.よって;
		\[a_n\alpha^n+\dots+a_1\alpha+a_0=0\]
		となる$a_i\in K$について,少なくとも1つは0でない.よって$f(X)=a_nX^n+\dots+a_1X+a_0$とおけば$f$は$\alpha$を根にもつ$K$上の多項式となる.
	\end{eqv}
\end{proof}

系として$\nitem{\alpha}$が$K$上代数的なら$K(\nitem{\alpha})/K$は代数拡大.また$L/K$が有限次なら代数拡大である.これまでの結果から,代数的数の(有限個の)和,積,商はすべて代数的である.

$K$を係数とする既約多項式の根を$K$に添加することで体を拡大する手法をみたが,これは$K$内で「解けない」多項式の存在を仮定するものだった.具体的には$X^2+1$は$\R$では解けないが, $\Co$では$X^2+1=(X-i)(X+i)$と解くことができる.では, $\Co[X]$には解けない多項式は存在しない(すべての多項式は$1$次式の積に分解できる.代数学の基本定理). このような体を更に代数拡大することはできるのだろうか?答えは No である.

\begin{defi}[代数閉体]\index{だいすうへいたい@代数閉体}
	体$K$であって, $K[X]$の既約多項式がすべて$1$次であるとき$K$を\textbf{代数閉体}(algebraically closed field)であるという.
\end{defi}

$\Co$は代数閉体である.定義から$K$が代数閉体のとき,任意の$f\in K[X]$に対し$\deg f\geq1$ならばある$\alpha\in K$が存在して$f(\alpha)=0$となる.逆に体$K$でこの条件が成り立つとき, $f(X)=(X-\alpha)f_1(X)$と分解され(因数定理),  $f_1$に同様の操作を繰り返すことで$K$が代数閉体であることを示すことができる.よって$K$が代数閉であることと,任意の$1$次以上の多項式が$K$内に根を持つことは同値である.

\begin{prop}
	体$K$が代数閉体であることと, $L/K$が体の代数拡大ならば$L=K$となることは同値.
\end{prop}

\begin{proof}
	\begin{eqv}
		\item $L/K$を代数拡大とする.任意の$\alpha\in L$について$\alpha$は$K$上代数的なので$f(\alpha)=0$となる$f\in K[X]$が存在する.仮定から$f=g_1g_2\dots g_n (\deg g_i=1)$と既約分解できる.このときある$i$について$g_i(\alpha)=0$となる(体は整域).そこで$g_i(X)=aX+B (a,b\in K, a\neq0)$とおくと, $a\alpha+b=0$より$\alpha=-b/a\in K$である.よって$L=K$となる.
	
	\item 任意の$f\in K[X],\deg f\geq1$をとる. $g$を$f$を割り切る既約多項式とすると, $K[X]/(g)$は$K$の拡大体となり,ある$\alpha\in K[X]/(g)$が存在して$K[X]/(g)=K(\alpha),K(\alpha)/K$は代数拡大となる.すると$K(\alpha)=K$なので$\alpha\in K$だから$f$は$K$内に根を持つ.よって$K$は代数閉体である.
	\end{eqv}
\end{proof}

明らかに素体は代数閉ではない.どんな体についても,それを含むような代数閉体が存在する.

\begin{defi}[代数閉包]\index{だいすうへいほう@代数閉包}
	$K$を体とすると,代数閉体$L$が存在して$L/K$が代数拡大となる.このとき$L$を$K$の\textbf{代数閉包}(algebraic closure)という.
\end{defi}

\begin{proof}[\textbf{存在することの証明.}]
	$K$の代数拡大全体のなす族を$\Sigma$とすると, $\Sigma$は包含関係によって帰納的順序集合をなす. Zornの補題により極大元が存在し,それが$K$の代数閉包になる.
\end{proof}